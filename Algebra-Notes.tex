% Preamble
\documentclass{book}
\usepackage[margin=1in]{geometry}

% Standard mathematical typesetting packages
\usepackage{amsfonts, amsthm, amsmath, amssymb}
\usepackage{mathtools}  % Extension to amsmath

% Symbol and utility packages
\usepackage{cancel}
\usepackage[mathscr]{euscript}
% \usepackage[nointegrals]{wasysym}
\usepackage{bbm}
\usepackage[normalem]{ulem}

% Styling
\usepackage{fancyhdr}
\usepackage{environ}
\usepackage{systeme}
\usepackage{graphicx}
\usepackage{adjustbox}

% Extras
\usepackage[italicdiff]{physics}    % Lots of useful shortcuts and macros
\usepackage{tikz-cd}                % For drawing commutative diagrams easily
\usepackage{color, xcolor}          % Add some colour to life
\usepackage{microtype}              % Minature font tweaks
\usepackage{comment}
\usepackage{xparse}
\usepackage{mdframed}
\usepackage[shortlabels]{enumitem}
\usepackage{indentfirst}
\usepackage{hyperref}
\usepackage{bm}

% Vector style
\renewcommand{\vec}[1]{\underline{\ensuremath{#1}}} % underline type vector

% Matrices and Vectors
\newcommand{\twomatrix}[4]{\begin{bmatrix}
    #1 & #2 \\ #3 & #4
\end{bmatrix}}
\newcommand{\twovector}[2]{\begin{bmatrix}
    #1 \\ #2
\end{bmatrix}}
\newcommand{\threevector}[3]{\begin{bmatrix}
    #1 \\ #2 \\ #3
\end{bmatrix}}

% Beautifying some symbols
\renewcommand{\emptyset}{\varnothing}
\renewcommand{\div}{\divisionsymbol}

% Common shortcuts
\newcommand{\mbb}[1]{\mathbb{#1}}
\newcommand{\mfk}[1]{\mathfrak{#1}}
\newcommand{\mca}[1]{\mathcal{#1}}

\newcommand{\bN}{\mbb{N}}
\newcommand{\bC}{\mbb{C}}
\newcommand{\bR}{\mbb{R}}
\newcommand{\bQ}{\mbb{Q}}
\newcommand{\bZ}{\mbb{Z}}
\newcommand{\bF}{\mbb{F}}
\newcommand{\bH}{\mbb{H}}

% Common notation
\DeclareMathOperator{\ord}{ord}         % order (of a group)
\DeclareMathOperator{\lcm}{lcm}         % least common multiple (of two numbers)
\DeclareMathOperator{\sgn}{sgn}         % sign function
\DeclareMathOperator{\stab}{stab}       % stabilizer (of a group)
\DeclareMathOperator{\orb}{orb}         % orbit (of an action)
\DeclareMathOperator{\Centralizer}{C}   % centralizer (of a set)
\DeclareMathOperator{\Centre}{Z}        % centre (of a group)
\DeclareMathOperator{\Normalizer}{N}    % normalizer (of a set)

\DeclareMathOperator{\adj}{adj}         % adjugate (matrix)
\DeclareMathOperator{\id}{id}           % identity mapping

\DeclareMathOperator{\Hom}{Hom}         % set of homomorphisms
\DeclareMathOperator{\End}{End}         % set of endomorphisms
\DeclareMathOperator{\Aut}{Aut}         % set of automorphisms
\DeclareMathOperator{\Ob}{Ob}           % collection of objects
\DeclareMathOperator{\Mor}{Mor}         % collection of morphisms
\DeclareMathOperator{\dirlim}{\displaystyle\lim_{\longrightarrow}}  % direct limit
\DeclareMathOperator{\invlim}{\displaystyle\lim_{\longleftarrow}}   % inverse limit

\DeclareMathOperator{\Span}{span}       % span of vector spaces

\DeclareMathOperator{\Gal}{Gal}         % Galois group

% Sometimes helpful macros
\newcommand{\func}[3]{#1\colon#2\to#3}
\newcommand{\vfunc}[5]{\func{#1}{#2}{#3},\;#4\mapsto#5}
\newcommand{\floor}[1]{\left\lfloor#1\right\rfloor}
\newcommand{\ceil}[1]{\left\lceil#1\right\rceil}
\newcommand{\abrac}[1]{\left\langle#1\right\rangle}

% Some standard theorem definitions
\newcounter{t}
\counterwithin{t}{section}

\theoremstyle{definition}
\newtheorem{theorem}[t]{Theorem}
\newtheorem{proposition}[t]{Proposition}
\newtheorem{lemma}[t]{Lemma}
\newtheorem{corollary}[t]{Corollary}

\newtheorem{definition}[t]{Definition}
\newtheorem{axiom}[t]{Axiom}
\newtheorem{remark}[t]{Remark}

% Define proof environment
\renewenvironment{proof}[1][]{
    \textit{Proof. #1}
}{\medbreak}

\hypersetup{
    colorlinks=true,
    linkcolor=blue,
    filecolor=magenta,
    urlcolor=blue,
}
\pagestyle{fancy}



%%%%%%%%%%%%%%%%%%%%%%%%%%%%%%%%%%%%%%%%%%%%%
%Fill in the appropriate information below
% \lhead{Boris Li}
% \rhead{Algebra}
% \chead{\textbf{Notes}}
%%%%%%%%%%%%%%%%%%%%%%%%%%%%%%%%%%%%%%%%%%%%%

\begin{document}

\frontmatter

\begin{titlepage}
\begin{center}
    \vspace*{1cm}
    {\Huge\textbf{Notes on Algebra}}

    \vspace{5mm}
    {\LARGE at the Undergraduate Level}

    \vspace{15mm}
    {\Large\textbf{Boris Li}}

    \vspace*{4cm}
    \adjustbox{scale=3,center}{%
        \begin{tikzcd}[row sep=6ex, column sep=5ex]
        X \arrow{d}[swap]{\pi} \arrow{r}{\phi} & Y \\
        X/{\sim} \arrow[dashrightarrow]{ru}[swap]{\exists!\bar{\phi}} &{}
        \end{tikzcd}
    }

    \vfill

    {\large Last compiled: \today}
\end{center}
\end{titlepage}

\vspace*{30mm}

\begin{center}
    \textit{For an undergraduate algebra student:}

    \vspace{5mm}

    \textit{You're gonna get better at this.}
\end{center}

\chapter*{Foreword}\markboth{FOREWORD}{}\addcontentsline{toc}{chapter}{Foreword}

\section*{Version 1.0}

This document was written in the summer of 2023
for the purpose of helping myself review the material
in various classes of algebra.
It is based on Prof.\ Vatsal's MATH 322 (groups) \& 323 (rings) in-class notes.
% and Prof.\ Silberman's MATH 412 in-class notes.

While this does not replace any course notes or the textbook,
I have personally found Jacobson's \textit{Basic Algebra}~\cite{jacobson}
to be overly concise,
with too many details hidden within singular paragraphs;
and Herstein's \textit{Topics in Algebra}~\cite{herstein} sometimes overly verbose.
I aim to lift out definitions from large proofs
as to aid my own digestion of the material,
and to ease the process of revisiting ideas from group theory.

Morgan suggests that Rotman's \textit{An Introduction to the Theory of Groups}~\cite{rotman}
to be a useful book to self-learn from,
while I have occasionally seen Justin
pick up a copy of Lang's \textit{Algebra}~\cite{lang} as reference.
Both of these are good textbooks,
but my current pick still goes to Dummit and Foote's \textit{Abstract Algebra}~\cite{dummitfoote},
as I find it as a great reference book,
with neatly numbered theorems, and detailed proofs;
but it is currently out of print,
so I presume the reader knows where to find such copies.
Although I have never been one to absorb knowledge directly from textbooks,
the reader must be way more diligent than I am,
so go ahead and experiment,
go find out the correct textbook for you.

\medskip

To Madeline:
I truly hope that you can find math as beautiful as I thought it is,
and find understanding algebra for the first time
to be less painful than when I understood it for the first time.
I want you to be better at this than I ever was.

To Arsam:
Maybe this one course will help you find your true place in math,
and can help you decide the direction that you wish to pursue.
While we may want rigour for the sake of understanding,
it may sometimes get in the way of clarity;
choose wisely when it comes to writing proofs.

To Morgan and Aryan:
I know the pain of
Jacobson not having a search-able PDF copy online,
so hopefully this document comes in handy.
Or you guys are just too smart
and can memorize all those definitions first try.

% \vspace{5mm}
\medskip

Boris

August 24, 2023

% \section*{Version 2.0}

% This document was continuously worked on
% over the 2023--2024 academic year
% in an attempt to fulfill my grand ambition
% to consolidate all of my knowledge in algebra
% learnt at the undergraduate level.
% The new sections are based on course material taught in
% Prof.\ Vatsal's MATH 323 (modules),
% Prof.\ Silberman's MATH 412 (linear algebra),
% Prof.\ Ramdorai's MATH 422 (fields and Galois theory),
% and Prof.\ Karu's MATH 423 (commutative algebra),
% with bits and pieces coming from Prof.\ Bryan's MATH 426 (topology).
% Newly included reference texts include
% Halmos' \textit{Finite-Dimensional Vector Spaces}~\cite{halmos},
% Roman's \textit{Advanced Linear Algebra}~\cite{roman} (both for linear algebra),
% Atiyah and MacDonald's \textit{Introduction to Commutative Algebra}~\cite{atiyahmacdonald} (for commutative algebra),
% and by Morgan's recommendation,
% Mac Lane's \textit{Categories for the Working Mathematician}~\cite{maclane} (for category theory).
% Numerous proofs are directly referenced from the typeset notes of
% Profs.\ Silberman, Ramdorai, and Williams.


% A small token of thanks to Justin
% suggesting that the section on categories can be shifted earlier
% as to avoid reproving tedious amounts of theorems later on.
% It didn't really reduce the number of proofs,
% but I guess it looks nicer now?
% I would also like to apologize,
% because Lang is indeed superior to Dummit \& Foote in a variety of ways.

% \medskip

% Boris

% Date?

\vspace{5mm}

Please do not redistribute without prior permission.


\tableofcontents

\setcounter{chapter}{-1}

\mainmatter

\section{Prerequisite Material}

\begin{remark}
    There is a huge amount of prerequisite material that is needed,
    and it is never possible to study algebra independently of other subjects.
    We need skills like doing proofs
    by induction, contradiction, and contraposition,
    and we also need a basic understanding of numbers in general,
    and perhaps a good deal of geometric intuition would be nice;
    occasionally some proofs might also involve a non-trivial amount of analysis.
    But we need a starting point,
    and for me that starting point
    is the material that perhaps was never formally covered in Science One math
    and second year linear algebra and calculus,
    or maybe was a topic that was last taught in high school.
\end{remark}

\subsection{Relations}

\begin{definition}
    A relation is a denotion on two elements of a set \(X\).
    More specifically,
    if a relation \(R \subset X \cross X\),
    then we write \(aRb\) if \((a,b) \in R\).
\end{definition}

\begin{definition}
    An equivalence relation \(\sim\)
    is a relation that has three properties:
    \begin{enumerate}[label={(\roman*)}, itemsep=0mm]
        \item reflexive \(\forall x \in X, x \sim x\);
        \item symmetric \(\forall x,y \in X, x \sim y \implies y \sim x\); and
        \item transitive \(\forall x,y,z \in X,
            x \sim y \land y \sim z \implies x \sim z\).
    \end{enumerate}
    We say \(x\) is relatied to \(y\) if \(x \sim y\).
\end{definition}

\begin{definition}
    Suppose we have an equivalence relation \(\sim\).
    The equivalence class of \(x \in X\) is
    \([x] = \{y \in X : y \sim x\}\).
\end{definition}
\begin{lemma}
    An element is in its own equivalence class; \(x \in [x]\).
\end{lemma}
\begin{proof}
    By reflexivity.
\end{proof}
\begin{theorem}
    \(x \sim y \iff [x] = [y]\);
    two related elements must have the same equivalence class.
\end{theorem}
\begin{proof}
    If \(x \sim y\),
    then \([x]\) consists of all elements \(z\) that are related to \(x\),
    so \(z \sim x\), and transitivity gives us \(z \sim y\),
    so \(z \in [y]\), and hence \([x] \subset [y]\).
    Reversing the argument gives us \([y] \subset [x]\),
    so \([x] = [y]\).

    If \([x] = [y]\), then \(y \in [x]\),
    so by definition \(y \sim x\).
\end{proof}
\begin{corollary}
    For all elements \(\{x,y\} \in X\),
    either \([x] = [y]\) or \([x] \cap [y] = \emptyset\).
\end{corollary}
\begin{proof}
    Suppose \([x] \cap [y] \neq \emptyset\).
    Then there exists some element \(z \in [x]\) and \(z \in [y]\).
    But then \(z \sim x\) and \(z \sim y\),
    so \(x \sim y\) by transitivity,
    which by our theorem gives \([x] = [y]\).
\end{proof}

\begin{theorem}\label{thm:equiv-class-partition}
    The set of equivalence classes \(X/{\sim}\) partition \(X\).
\end{theorem}
\begin{proof}
    From the lemma,
    each element must belong in an equivalence class.
    The above corollary tells us equivalence classes do not overlap.
\end{proof}

\subsection{Functions}

\begin{definition}
    A function \(f\) that maps elements in \(A\) to \(B\)
    is denoted \(\func{f}{A}{B}\).
    \(A\) is the domain, and \(B\) is the codomain.
    The range \(f(A)\) is the set of all possible outputs of the function,
    namely \(f(A) = \{b \in B : \exists a \in A, f(a) = b\}\).
\end{definition}

\begin{definition}
    Suppose \(A' \subset A\) and \(B' \subset B\).
    The image of \(A'\) is the set of all possible outputs
    given elements in \(A'\),
    denoted \(f(A') = \{f(a) \in B : a \in A'\}\).
    The preimage of \(B'\) is the set of all possible inputs
    that result in an element in \(B'\),
    denoted \(f^{-1}(B') = \{x \in A : f(x) \in B'\}\).
\end{definition}

\begin{definition}
    A function is 1-to-1 or injective
    when each point in the range (or codomain)
    only corresponds to (at most) one point in the domain.
    That is, \(\forall x,y \in A, f(x) = f(y) \implies x = y\).
\end{definition}
\begin{definition}
    A function is onto or surjective
    when every point in the codomain
    corresponds to at least one point in the domain,
    i.e.\ the codomain is the range.
    That is, \(\forall y \in B, \exists x \in A, f(x) = y\).
\end{definition}
\begin{definition}
    A function is bijective if it is both injective and surjective.
\end{definition}

% \newpage

\begin{theorem}[Pigeonhole Principle]\label{thm:pigeonhole}
    Suppose two finite sets \(A,B\) have the same cardinality
    \(\abs{A} = \abs{B}\).
    Then \(\func{f}{A}{B}\) is injective
    if and only if it is surjective.
\end{theorem}
\begin{proof}
    Suppose \(f\) is injective but not surjective.
    Then if \(\abs{A} = n\),
    then there are \(n\) elements in \(B\) that are in the range.
    Lack of surjectivity implies that
    there exists at least one element that is not in the range.
    Hence \(\abs{B} > n\), which is a contradiction.

    Suppose \(f\) is surjective but not injective.
    Then if \(\abs{B} = n\),
    then there are at least \(n\) elements in \(A\).
    But no injectivity implies there must be some two elements
    \(x,y \in A\) that map to the same point \(f(x) = f(y)\),
    which so forces us to conclude \(\abs{A} \geq n+1\),
    which is a contradiction.
\end{proof}

\begin{theorem}\label{thm:composite-injection}
    Suppose \(f = f_n \circ f_{n-1} \circ \cdots \circ f_2 \circ f_1\).
    If \(f\) is injective, then \(f_1\) is injective.
\end{theorem}
\begin{proof}
    % Without loss of generality, let \(f = f_2 \circ f_1\).
    Suppose, by way of contradiction, that \(f_1\) is not injective.
    Then there exists \(x,y\) in the domain of \(f_1\)
    such that \(f_1(x) = f_1(y)\).
    But that implies \(f(x) = f(y)\),
    which contradicts that \(f\) is injective.
\end{proof}
\begin{theorem}\label{thm:composite-surjective}
    Suppose \(f = f_n \circ f_{n-1} \circ \cdots \circ f_2 \circ f_1\).
    If \(f\) is surjective, then \(f_n\) is surjective.
\end{theorem}
\begin{proof}
    % Without loss of generality, let \(f = f_2 \circ f_1\).
    Suppose, by way of contradiction, that \(f_n\) is not surjective.
    Then there exists \(y\) in the codomain of \(f_n\)
    such that for all \(x\) in the domain of \(f_n\),
    \(f_n(x) \neq y\).
    But then there exists \(y\) in the codomain of \(f\)
    such that \(f(x) \neq y\),
    which contradicts that \(f\) is surjective.
\end{proof}

\subsection{Integers}

\begin{definition}
    Suppose we have integers \(a,b\).
    The greatest common denominator, or gcd
    is the largest integer \(n\) such that \(n \mid a\) and \(n \mid b\).
    This is often denoted \(\gcd(a,b)\) or simply \((a,b)\).
    The least common multiple, or lcm
    is the smallest integer \(m\) such that \(a \mid m\) and \(b \mid m\).
    This is often denoted \(\lcm(a,b)\) or simply \([a,b]\).
\end{definition}

\begin{theorem}[Euclid's Lemma]\label{lem:euclid}
    Suppose we have prime \(p\) and integers \(a,b\).
    If \(p \mid ab\), then \(p \mid a\) or \(p \mid b\).
\end{theorem}
\begin{proof}
    Suppose \(p \nmid a\).
    Since \(p \mid ab\),
    there exists an integer \(q\) such that \(pq = ab\).

    We first prove the base case,
    supposing that \(ab = 2\).
    Then the only prime that divides it is 2,
    and we know that \(a = 1\) and \(b = 2\),
    so clearly \(p \mid b\).

    Now proceeding by induction,
    suppose that all values smaller than \(ab\) are proven.
    If \(p < a\),
    then \(pq - pb = ab - pb\),
    which gives us \(p(q-b) = (a-p)b\),
    i.e.\ \(p \mid (a-p)b\).
    Notice that \(p \nmid a-p\),
    and that \((a-p)b < ab\),
    so Euclid's lemma holds by the induction hypothesis.
    If \(p > a\),
    then \(pb - npq = pb - nab\),
    which gives us \(p(b-nq) = (p-na)b\),
    i.e.\ \(p \mid (p-na)b\).
    Notice that \(p \nmid p-a\),
    and that there exists an \(n\) such that \((p-na)b < ab\),
    which completes our proof.
\end{proof}
\begin{remark}
    This is essentially how the Euclidean algorithm for finding gcd works.
    We know that by subtracting off the other number,
    the gcd does not change,
    which allows us to reduce the problem to a smaller case.
\end{remark}

\begin{theorem}[B\'{e}zout's Identity]\label{thm:bezout}
    Suppose \(\gcd(a,b) = d\).
    Then there exists \(m,n \in \bZ\) such that \(ma + nb = d\).
    Moreover, for any \(p,q \in \bZ\)
    we have \(d \mid pa + qb\).
\end{theorem}
\begin{proof}
    Without loss of generality, let \(a \leq b\).
    If \(a = b\), then clearly \(d = a = b\) which is in our desired form.
    If not, then since we know that we can find some \(b - na \leq a\),
    which reduces the case down to a smaller number
    while always keeping the numbers that we are taking the gcd of
    in the form \(ma + nb\).
    Eventually, \(d\) must equal one of these numbers.

    Then we write \(pa + qb = (p-m)a + (q-n)b + ma + nb
    = (p-m)a + (q-n)b + d\).
    Since \(d \mid a\) and \(d \mid b\),
    we have our desired result.
\end{proof}
\begin{remark}
    One might use B\'{e}zout's identity to prove Euclid's lemma,
    but both proofs essentially come from the Euclidean algorithm.
\end{remark}

\begin{theorem}[Fundamental Theorem of Arithmetic]
    Every integer greater than 1 factors uniquely into a product of primes.
    That is, we can write any integer \(n > 1\) as
    \begin{equation*}
        n = \prod_{i=1}^k p_i^{n_i}
    \end{equation*}
    where \(p_i\) are distinct primes.
\end{theorem}
\begin{proof}
    We first prove such a prime factorization exists.
    We can see that 2 is prime.
    Proceeding by induction, assume all integers between 2 and \(n\)
    have a prime decomposition.
    If \(n\) is prime, we are done.
    If \(n\) is not prime,
    then it must be represented by \(n = ab\),
    where \(a,b\) must both be smaller,
    and by the inductive hypothesis,
    have prime decompositions.

    Now we prove that prime factorization is unique.
    Assume the contrary,
    and let \(n\) be the smallest such integer without unique factorization.
    Then we write \(n = \prod_{i=1}^k p_i = \prod_{i=1}^{k'} q_i\),
    two distinct factorizations.
    By Euclid's lemma we see that \(p_1\) divides some \(q_i\),
    which without loss of generality say this is \(q_1\).
    Then \(p_1 = q_1\).
    So \(n/p_1 = \prod_{i=2}^k p_i = \prod_{i=2}^{k'} q_i\)
    is also an integer without unique factorization,
    which contradicts our assumption that \(n\) is the smallest.
\end{proof}


\part{Constructions}
\chapter{Categories}\label{sec:categories}

\begin{remark}
    This section is entirely extra,
    and is not required in order to understand the rest of the constructions.
\end{remark}

\section{Basic Definitions}

\begin{definition}
    A category \(\mca{C}\) is a collection of objects \(\Ob(\mca{C})\)
    alongisde a collection of morphisms \(\Mor(\mca{C})\) or \(\Hom(\mca{C})\),
    where
    \begin{enumerate}[label={(\roman*)}, itemsep=0mm]
        \item \(\forall f \in \Mor(\mca{C}),\; \exists X,Y \in \Mor(\mca{C}),\; \func{f}{X}{Y}\),
            a morphism is a mapping from a source object to a target object;
            we define the collection of morphisms from \(X\) to \(Y\)
            as \(\Hom(X,Y) = \{f\in\Mor(\mca{C}) \mid \func{f}{X}{Y}\}\);
        \item \(\forall f \in \Hom(X,Y),\; \forall g \in \Hom(Y,Z),\; g \circ f \in \Hom(X,Z)\),
            the composition is a morphism;
        \item \(\forall f,g,h \in \Mor(\mca{C}),\; (f \circ g) \circ h = f \circ (g \circ h)\),
            morphisms are associative; and
        \item \(\forall X \in \Ob(\mca{C}),\; \exists \id_X \in \Hom(X,X),\; \id_X \circ g = g,\; f \circ \id_X = f\)
            the identity morphism exists.
    \end{enumerate}
\end{definition}
\begin{remark}
    \(\Mor\) and \(\Hom\) are often interchangeable in this context,
    since in algebra, the morphisms that we deal with are usually homomorphisms.
    We shall stick to the convention that the morphisms of a category is \(\Mor\),
    while the homomorphisms between objects are \(\Hom\).
\end{remark}

\begin{definition}
    A morphism \(f \in \Hom(X,Y)\) is an isomorphism
    if it has an inverse \(g \in \Hom(Y,X)\)
    such that \(f \circ g = \id_Y\) and \(g \circ f = \id_X\).
\end{definition}
\begin{definition}
    A morphism \(f \in \Hom(X,X)\) that maps to itself is called an endomorphism.
    The collection of endomorphisms is \(\End(X) = \Hom(X,X)\).
\end{definition}
\begin{definition}
    A morphism \(f \in \Aut(X)\) if it is both an isomorphism and an endomorphism.
\end{definition}

\begin{definition}
    A category \(\mca{C}\) is small
    if \(\Ob(\mca{C})\) and \(\Mor(\mca{C})\) both form sets.
    A category is large otherwise.
\end{definition}

\begin{proposition}
    The category \(\mathbf{Set}\) consisting of all sets and set maps form a category.
\end{proposition}
\begin{proposition}
    The category \(\mathbf{Grp}\) consisting of all groups and group homomorphisms
    form a category.
\end{proposition}
\begin{proposition}
    The category \(\mathbf{Ab}\) consisting of all abelian groups
    and group homomorphisms between them form a category.
\end{proposition}
\begin{proposition}
    The category \(\mathbf{Ring}\) consisting of all rings and ring homomorphisms
    form a category.
\end{proposition}
\begin{proof}
    Obvious for the above four.
\end{proof}

\begin{theorem}
    Suppose \(\mca{C}\) is a category, and \(X \in \Ob(\mca{C})\).
    Then \(\Aut(X)\) forms a group.
\end{theorem}
\begin{proof}
    First off, if \(f,g \in \Aut(X)\),
    then \(f \circ g \in \Aut(X)\),
    since \(f \circ g \in \End(X)\) and composition of bijections are bijections.
    Associativity and identity is given for free.
    By definition, the inverse exists since they are isomorphisms.
\end{proof}

\begin{definition}
    A universal property is a property
    that characterizes an object up to isomorphism.
    This is often stated as given some objects and morphisms,
    there exists a (unique) morphism to some desired object that we want to define.
\end{definition}


\section{Duality}

\begin{remark}
    Since morphisms are directed,
    the notion of duality is to reverse such morphisms
    and see whether they make sense.
\end{remark}

\begin{definition}
    A functor \(\func{F}{\mca{C}}{\mca{D}}\) is a mapping between categories,
    and \(F\) maps both objects \(X \in \Ob(\mca{C}) \mapsto F(X) \in \Ob(\mca{D})\)
    and morphisms \(f \in \Mor(\mca{C}) \mapsto F(f) \in \Mor(\mca{D})\).
    Functors respect the identity \(F(\id_\mca{C}) = \id_\mca{D}\).
\end{definition}

\begin{definition}
    We call a functor covariant if it preserves the direction of morphisms
    \(f \in \Hom(X,Y)\) maps to \(F(f) \in \Hom(F(X),F(Y))\),
    and composition \(F(g \circ f) = F(g) \circ F(f)\).
    We call a functor contravariant if it reverses the direction of morphisms
    \(f \in \Hom(X,Y)\) maps to \(F(f) \in \Hom(F(Y),F(X))\),
    and composition \(F(g \circ f) = F(f) \circ F(g)\).
\end{definition}
\begin{remark}
    Contravariant functors are not really functors.
    See the definition of the opposite category below.
\end{remark}

\begin{definition}
    Consider a categeory \(\mca{C}\).
    The opposite category is \(\mca{C}^\text{op}\)
    consists of the same objects, but all morphisms reversed.
\end{definition}
\begin{proposition}
    For all categories, \({(\mca{C}^\text{op})}^\text{op}\).
\end{proposition}
\begin{proof}
    Reversing morphisms twice amounts to keeping them identical.
\end{proof}

\begin{proposition}
    Contravariant functors \(\func{F}{\mca{C}}{\mca{D}}\)
    are covariant functors \(\func{F}{\mca{C}}{\mca{D}^\text{op}}\).
\end{proposition}
\begin{proof}
    Observe that if \(\phi \in \Mor(\mca{C})\), \(\func{\phi}{X}{Y}\)
    then \(\func{F(\phi)}{F(Y)}{F(X)}\),
    and in the opposite category, \(\func{F(\phi)}{F(X)}{F(Y)}\).
\end{proof}

\subsubsection*{Subobject}

\begin{remark}
    We first generalize the notion of a subset to any categorical object.
\end{remark}
\begin{definition}
    Suppose \(X,Y\) are objects.
    We say \(X \subseteq Y\) is (isomorphic to) a subobject
    if there exists an inclusion (injective homomorphism)
    \(\func{\iota}{X}{Y}\).
\end{definition}

% \begin{theorem}[Universal Property of Subobjects]
%     Let \(X \subseteq Y\) be a subobject-superobject pair,
%     and \(Z\) some arbitrary object.
%     Suppose \(\func{\iota}{X}{Y}\) is the subobject inclusion,
%     and \(\func{\phi}{Z}{Y}\) is any injective homomorphism
%     with the image \(\phi(Z) = \iota(X)\).
%     Then there exists a unique homomorphism \(\func{\bar{\phi}}{Z}{X}\)
%     such that \(\phi = \iota\circ\bar{\phi}\).

%     This is represented by the following commutative diagram:
%     \begin{center}
%         \begin{tikzcd}
%             Y & Z \arrow{l}[swap]{\phi}
%             \arrow[dashrightarrow]{ld}{\exists!\bar{\phi}} \\
%             X \arrow{u}[swap]{\iota}
%         \end{tikzcd}
%     \end{center}
% \end{theorem}
% \begin{proof}
%     % We wish to prove uniqueness by assuming existence.
%     % Suppose \(\phi = \iota\circ\bar{\phi} = \iota\circ\bar{\phi}'\).
%     Since \(\phi(Z) = \iota(X)\),
%     there exists a bijection \(Z \to \phi(Z) \to X\).
%     That bijection is our desired map \(\bar{\phi}\).
% \end{proof}

\subsubsection*{Quotient Object}

\begin{remark}
    The dual to a subobject is a quotient object.
\end{remark}
\begin{definition}
    Suppose \(X\) is an object, and \(\sim\) an equivalence relation on \(X\).
    We say \(X/{\sim}\) is a quotient object of \(X\)
    if there exists a projection \(\func{\pi}{X}{X/{\sim}}\),
    where we map any element to its equivalence class.
\end{definition}

% \begin{theorem}[Universal Property of Quotients]
%     Let \(X\) be an object with a relation \(\sim\).
%     Suppose \(\func{\pi}{X}{X/{\sim}}\) is the quotient projection morphism,
%     and \(\func{\phi}{X}{Z}\) is any surjective homomorphism
% \end{theorem}


\section{Products}

\subsubsection*{Product}

\begin{definition}
    Suppose \(X,Y\) are two objects.
    We call \(X \times Y\) the product of objects if
    \begin{enumerate}[label={(\roman*)}, itemsep=0mm]
        \item there exists projections \(\func{\pi_1}{X \times Y}{X}\)
            and \(\func{\pi_2}{X \times Y}{Y}\); and
        \item for any object \(Z\),
            if \(\func{\phi_1}{Z}{X}\) and \(\func{\phi_2}{Z}{Y}\) are homomorphisms,
            then there exists a unique \(\func{\bar{\phi}}{Z}{X \times Y}\)
            such that \(\phi_i = \pi_i\circ\bar{\phi}\) for \(i = 1,2\).
    \end{enumerate}

    This can be represented by the following commutative diagram:
    \begin{center}
        \begin{tikzcd}
            & Z \arrow{ld}[swap]{\phi_1} \arrow{rd}{\phi_2}
            \arrow[dashrightarrow]{d}{\exists!\bar{\phi}} \\
            X & X \times Y \arrow{l}{\pi_1} \arrow{r}[swap]{\pi_2} & Y
        \end{tikzcd}
    \end{center}
\end{definition}

\begin{definition}
    Suppose \({\{X_i\}}_{i \in I}\) is a family of objects.
    We call \(\prod_{i \in I} X_i\) the (infinite) product of objects if
    \begin{enumerate}[label={(\roman*)}, itemsep=0mm]
        \item there exists projections \(\func{\pi_i}{\prod_{i \in I} X_i}{X_i}\); and
        \item for any object \(Z\),
            if \(\func{\phi_i}{Z}{X_i}\) are homomorphisms,
            then there exists a unique \(\func{\bar{\phi}}{Z}{\prod_{i \in I} X_i}\)
            such that \(\phi_i = \pi_i\circ\bar{\phi}\).
    \end{enumerate}

    This can be represented by the logical statement
    \begin{equation*}
        \forall Z \in \mca{C},\;
        \forall i \in I,\;
        \forall \phi_i \in \Hom(X,V_i),\;
        \exists! \phi \in \Hom\pqty{Z,\prod_{i \in I} X_i},\;
        \phi_i = \pi_i\circ\bar{\phi}
    \end{equation*}
    and the following commutative diagram:
    \begin{center}
        \begin{tikzcd}
            & Z \arrow{ld}[swap]{\phi_i}
            \arrow[dashrightarrow]{d}{\exists!\bar{\phi}} \\
            X_i & \prod_{i \in I} X_i \arrow{l}{\pi_i}
        \end{tikzcd}
    \end{center}
\end{definition}

\subsubsection*{Coproduct}

\begin{remark}
    The coproduct is the dual to the product.
\end{remark}

\begin{definition}
    Suppose \(X,Y\) are two objects.
    We call \(X \oplus Y\) the coproduct of objects if
    \begin{enumerate}[label={(\roman*)}, itemsep=0mm]
        \item there exists inclusions \(\func{\iota_1}{X}{X \oplus Y}\)
            and \(\func{\iota_2}{Y}{X \oplus Y}\); and
        \item for any object \(Z\),
            if \(\func{\phi_1}{X}{Z}\) and \(\func{\phi_2}{Y}{Z}\) are homomorphisms,
            then there exists a unique \(\func{\bar{\phi}}{X \oplus Y}{Z}\)
            such that \(\phi_i = \bar{\phi}\circ\iota_i\) for \(i = 1,2\).
    \end{enumerate}

    This can be represented by the following commutative diagram:
    \begin{center}
        \begin{tikzcd}
            & Z \\
            X \arrow{r}[swap]{\iota_1} \arrow{ru}{\phi_1} &
            X \oplus Y \arrow[dashrightarrow]{u}[swap]{\exists!\bar{\phi}} &
            Y \arrow{l}{\iota_2} \arrow{lu}[swap]{\phi_2}
        \end{tikzcd}
    \end{center}
\end{definition}

\begin{definition}
    Suppose \({\{X_i\}}_{i \in I}\) is a family of objects.
    We call \(\bigoplus_{i \in I} X_i\) the (infinite) coproduct of objects if
    \begin{enumerate}[label={(\roman*)}, itemsep=0mm]
        \item there exists inclusions \(\func{\iota_i}{X}{\bigoplus_{i \in I} X_i}\); and
        \item for any object \(Z\),
            if \(\func{\phi_i}{X_i}{Z}\) are homomorphisms,
            then there exists a unique \(\func{\bar{\phi}}{\bigoplus_{i \in I} X_i}{Z}\)
            such that \(\phi_i = \bar{\phi}\circ\iota_i\).
    \end{enumerate}

    This can be represented by the logical statement
    \begin{equation*}
        \forall Z \in \mca{C},\;
        \forall i \in I,\;
        \forall \phi_i \in \Hom(X_i,Z),\;
        \exists! \phi \in \Hom\pqty{\bigoplus_{i \in I} X_i, Z},\;
        \phi_i = \bar{\phi}\circ\iota_i
    \end{equation*}
    and the following commutative diagram:
    \begin{center}
        \begin{tikzcd}
            & Z \\
            X_i \arrow{r}[swap]{\iota_i} \arrow{ru}{\phi_i} &
            \bigoplus_{i \in I} X_i \arrow[dashrightarrow]{u}[swap]{\exists!\bar{\phi}}
        \end{tikzcd}
    \end{center}
\end{definition}

\subsubsection*{Pullback}

\begin{definition}
    Suppose \(W,X,Y\) are three objects.
    We call \(X \times_W Y\) the pullback or fibre product with respect to \(W\) if
    \begin{enumerate}[label={(\roman*)}, itemsep=0mm]
        \item there exists projections \(\func{\pi_X}{X}{W}\) and \(\func{\pi_Y}{Y}{W}\);
        \item there exists projections \(\func{\pi_1}{X \times_W Y}{X}\)
            and \(\func{\pi_2}{X \times_W Y}{Y}\); and
        \item for any object \(Z\),
            if \(\func{\phi_1}{Z}{X}\) and \(\func{\phi_2}{Z}{Y}\) are homomorphisms,
            then there exists a unique \(\func{\bar{\phi}}{Z}{X \times_W Y}\)
            such that the following diagram commutes.
    \end{enumerate}
    \begin{center}
        \begin{tikzcd}
            & Z \arrow{ld}[swap]{\phi_1} \arrow{rd}{\phi_2}
            \arrow[dashrightarrow]{d}{\exists!\bar{\phi}} \\
            X \arrow{rd}[swap]{\pi_X} &
            X \times Y \arrow{l}{\pi_1} \arrow{r}[swap]{\pi_2} &
            Y \arrow{ld}{\pi_Y} \\
            & W
        \end{tikzcd}
    \end{center}
\end{definition}

\subsubsection*{Pushout}

\begin{remark}
    The pushout is the dual to the pullback.
\end{remark}

\begin{definition}
    Suppose \(W,X,Y\) are three objects.
    We call \(X +_W Y\) the pushout or fibre coproduct with respect to \(W\) if
    \begin{enumerate}[label={(\roman*)}, itemsep=0mm]
        \item there exists inclusions \(\func{\iota_X}{W}{X}\) and \(\func{\iota_Y}{W}{Y}\);
        \item there exists inclusions \(\func{\iota_1}{X}{X +_W Y}\)
            and \(\func{\iota_2}{Y}{X +_W Y}\); and
        \item for any object \(Z\),
            if \(\func{\phi_1}{X}{Z}\) and \(\func{\phi_2}{Z}{Y}\) are homomorphisms,
            then there exists a unique \(\func{\bar{\phi}}{X +_W Y}{Z}\)
            such that the following diagram commutes.
    \end{enumerate}
    \begin{center}
        \begin{tikzcd}
            & Z \\
            X \arrow{r}[swap]{\iota_1} \arrow{ru}{\phi_1} &
            X +_W Y \arrow[dashrightarrow]{u}[swap]{\exists!\bar{\phi}} &
            Y \arrow{l}{\iota_2} \arrow{lu}[swap]{\phi_2} \\
            & W \arrow{lu}{\iota_X} \arrow{ru}[swap]{\iota_Y}
        \end{tikzcd}
    \end{center}
\end{definition}


\section{Limits}

\subsubsection*{Direct Limit}

\begin{definition}
    Suppose \({\{X_i\}}_{i \in I}\) is a family of objects,
    and the index set \(I\) is ordered with the relation \(\leq\).
    A direct system over \(I\) is when \(i \leq j\),
    let \(\func{f_{ij}}{X_i}{X_j}\) be a homomorphism such that
    \begin{enumerate}[label={(\roman*)}, itemsep=0mm]
        \item \(f_{ii} = \id_{X_i}\); and
        \item \(f_{ik} = f_{jk} \circ f_{ij}\) for all \(i \leq j \leq k\).
    \end{enumerate}

    This can be represented by the following commutative diagram:
    \begin{center}
        \begin{tikzcd}
            X_i \arrow{r}[swap]{f_{ij}} \arrow[bend left]{rr}{f_{ik}} &
            X_j \arrow{r}[swap]{f_{jk}} & X_k
        \end{tikzcd}
    \end{center}
\end{definition}

\begin{definition}
    Suppose \({(X_i,f_{ij})}_{i,j \in I}\) is a direct system.
    The direct limit of this system is an object \(\dirlim X_i\)
    that satisfies the following universal property:
    \begin{enumerate}[label={(\roman*)}, itemsep=0mm]
        \item there exists inclusions \(\func{\iota_i}{X_i}{\dirlim X_i}\)
            such that for all \(i \leq j\), \(\iota_i = \iota_j \circ f_{ij}\); and
        \item for any object \(Z\), if \(\func{\phi_i}{X_i}{Z}\) are homomorphisms
            such that for all \(i \leq j\), \(\phi_i = \phi_j \circ f_{ij}\),
            then there exists a unique \(\func{\bar{\phi}}{\dirlim X_i}{Z}\)
            such that \(\phi_i = \bar{\phi}\circ\iota_i\).
    \end{enumerate}

    This can be represented by the following commutative diagram:
    \begin{center}
        \begin{tikzcd}
            & Z \\
            & \dirlim X_i \arrow[dashrightarrow]{u}[swap]{\exists!\bar{\phi}} \\
            X_i \arrow{rr}[swap]{f_{ij}} \arrow{ur}[swap]{\iota_i} \arrow[bend left]{uur}{\phi_i} &&
            X_j \arrow{ul}{\iota_j} \arrow[bend right]{uul}[swap]{\phi_j}
        \end{tikzcd}
    \end{center}
\end{definition}

\subsubsection*{Inverse Limit}

\begin{remark}
    The inverse system and inverse limit
    is dual to the direct system and direct limit.
\end{remark}

\begin{definition}
    Suppose \({\{X_i\}}_{i \in I}\) is a family of objects,
    and the index set \(I\) is ordered with the relation \(\leq\).
    An inverse system over \(I\) is when \(i \leq j\),
    let \(\func{f_{ij}}{X_j}{X_i}\) be a homomorphism such that
    \begin{enumerate}[label={(\roman*)}, itemsep=0mm]
        \item \(f_{ii} = \id_{X_i}\); and
        \item \(f_{ik} = f_{ij} \circ f_{jk}\) for all \(i \leq j \leq k\).
    \end{enumerate}

    This can be represented by the following commutative diagram:
    \begin{center}
        \begin{tikzcd}
            X_i & X_j \arrow{l}{f_{ij}} &
            X_k \arrow{l}{f_{jk}} \arrow[bend right]{ll}[swap]{f_{ik}}
        \end{tikzcd}
    \end{center}
\end{definition}

\begin{definition}
    Suppose \({(X_i,f_{ij})}_{i,j \in I}\) is an inverse system.
    The inverse limit of this system is an object \(\invlim X_i\)
    that satisfies the following universal property:
    \begin{enumerate}[label={(\roman*)}, itemsep=0mm]
        \item there exists projections \(\func{\pi_i}{\invlim X_i}{X_i}\)
            such that for all \(i \leq j\), \(\pi_i = f_{ij}\circ\pi_j\); and
        \item for any object \(Z\), if \(\func{\phi_i}{Z}{X_i}\) are homomorphisms
            such that for all \(i \leq j\), \(\phi_i = f_{ij}\circ\phi_j\),
            then there exists a unique \(\func{\bar{\phi}}{Z}{\invlim X_i}\)
            such that \(\phi_i = \pi_i\circ\bar{\phi}\).
    \end{enumerate}

    This can be represented by the following commutative diagram:
    \begin{center}
        \begin{tikzcd}
            & Z \arrow[bend right]{ddl}[swap]{\phi_i} \arrow[bend left]{ddr}{\phi_j}
            \arrow[dashrightarrow]{d}{\exists!\bar{\phi}} \\
            & \invlim X_i \arrow{dl}{\pi_i} \arrow{dr}[swap]{\pi_j} \\
            X_i && X_j \arrow{ll}{f_{ij}}
        \end{tikzcd}
    \end{center}
\end{definition}


% \section{Universal Algebra}


% \section{Categorical Constructions}

\section{Groups}

\subsection{Basic Definitions}

\begin{definition}
    A group is a triple \((G,\cdot,1)\),
    where \(G \neq \emptyset\) is a set
    equipped with a multiplicative operation \(\cdot\)
    with an identity \(1\),
    with these four properties:
    \begin{enumerate}[label={(\roman*)}, itemsep=0mm]
        \item closure \(\func{\cdot}{G \cross G}{G}\);
        \item associativity \(\forall \{a,b,c\} \subset G, (ab)c = a(bc)\);
        \item identity \(\exists 1 \in G, \forall g \in G, 1g = g1 = g\); and
        \item inverse \(\forall g \in G, \exists h \in G, gh = hg = 1\),
            which we usually denote \(g^{-1} = h\).
    \end{enumerate}
\end{definition}
\begin{remark}
    If we think of groups as objects that we use
    to describe actions that preserve symmetry,
    these are exactly the four criteria that we require:
    any composition of two actions must also be an action (closure),
    you must be able to compose actions in any order (associativity),
    the act of doing nothing preserves symmetry (identity),
    and reversing an action also preserves symmetry (inverse).
\end{remark}

\begin{definition}
    A monoid is a triple \((X,\cdot,1)\),
    where \(X \neq \emptyset\) is a set
    equipped with a multiplicative operation \(\cdot\)
    with an identity \(1\),
    with the first three properties of a group,
    that is, closure, associativity, and identity.
\end{definition}
% \begin{definition}
%     A monad is a triple \((X,\cdot,1)\)
%     with the same hypotheses as a monoid,
%     but with only closure and identity.
% \end{definition}
% \begin{definition}
%     A semigroup is a double \((X,\cdot)\)
%     with the same hypotheses,
%     but only closure is guaranteed.
% \end{definition}

\begin{definition}
    Suppose \(G\) is a group.
    The order of an element \(g \in G\)
    is the smallest positive integer \(n\)
    such that \(g^n = 1\).
    If no such \(n\) exists,
    we say the order of \(g\) is infinite.
    This is often denoted \(\abs{g} = \ord{g} = \#g = n\).
\end{definition}
\begin{definition}
    The order of a group \(G\),
    also the cardinality of a group,
    is the number of elements in a group \(G\).
    If \(G\) has infinitely many elements,
    we say the order of \(G\) is infinite.
    This is often denoted \(\abs{G} = \ord{G} = \#G\).
\end{definition}

\begin{definition}
    A commutative group or an abelian group
    is a group \(G\) with a commutative operation \(\cdot\),
    that is, \(gh = hg\) for all \(\{g,h\} \subset G\).
\end{definition}
\begin{remark}
    We sometimes denote abelian groups with additive notation,
    where \((G,+,0)\) is a group.
\end{remark}


\subsection{Transformation Groups}

\begin{definition}
    Suppose we have a set \(X\).
    \(G\) is a group of transformations on \(X\)
    if all elements of \(G\) are bijective functions that map \(X \to X\),
    with multiplication being function composition.
\end{definition}
\begin{remark}
    It is worth remembering that \(X\) can be any set.
    Sometimes \(X\) indeed carries more structure,
    namely being a group \(X\),
    but remember that a group is but a set of elements
    with some additional rules.
\end{remark}

\begin{definition}
    The rotations and reflections on an \(n\)-gon
    trivially form a transformation group.
    These are transformations on \(n\) vertices that preserve adjacency.
    This is often denoted as \(D_n\),
    the dihedral group of an \(n\)-gon.
\end{definition}
\begin{proposition}
    \(D_n\) is a group of order \(\abs{D_n} = 2n\),
    with elements
    \begin{equation*}
        D_n = \{1,\tau,\tau^2,\hdots,\tau^{n-1},
        \sigma,\sigma\tau,\sigma\tau^2,\hdots,\sigma\tau^{n-1}\}
    \end{equation*}
    in which \(\tau^n = \sigma^2 = 1\)
    and \(\sigma\tau = \tau^{-1}\sigma\).
\end{proposition}
\begin{proof}
    We think of \(\tau\) as a rotation by \(2\pi/n\),
    and \(\sigma\) as a reflection.

    We prove that all such elements are distinct.
    Clearly by definition,
    all \(\tau^j\) are distinct when \(j \in [0,n-1]\).
    Moreover, all \(\sigma\tau^j\) are distinct when \(j \in [0,n-1]\)
    because if \(\sigma\tau^i = \sigma\tau^j\),
    then \(\sigma^2\tau^i = \sigma^2\tau^j\),
    which implies \(\tau^i = \tau^j\).
    Lastly, if \(\sigma\tau^i = \tau^j\),
    then \(\sigma\tau^i\tau^{-j} = \tau^j\tau^{-j} = 1\),
    which would imply there is \(\sigma = \tau^{j-i}\),
    a contradiction,
    as reflections are not rotations.

    With this, we also prove that all \(\sigma\tau^j\)
    are reflections.
\end{proof}

\begin{definition}
    A (group) homomorphism or a homomorphic function
    is \(\func{\phi}{G_1}{G_2}\),
    where \(G_1,G_2\) are groups,
    with the properties that \(\phi(gh) = \phi(g)\phi(h)\),
    and \(\phi(1)\) is the identity on \(G_2\).
    \(\phi\) provides a correspondence
    between the multiplication in \(G_1\) and \(G_2\).
    We call the set of all homomorphisms between \(G_1\) and \(G_2\)
    the hom-set \(\Hom(G_1,G_2)\).
\end{definition}
\begin{definition}
    Injective homomorphisms are called monomorphisms.
    Surjective homomorphisms are called epimorphisms.
    And most importantly,
    bijective homomorphisms are called isomorphisms.
    Two groups \(G,H\) that are isomorphic to each other
    are denoted \(G \cong H\).
\end{definition}
\begin{definition}
    Isomorphisms that map \(G \to G\) are called automorphisms;
    the set of all automorphisms of \(G\) is the automorphism group \(\Aut(G)\).
    The homomorphism that map a group \(G \to \{1\}\)
    by mapping all elements to \(1\)
    is called the trivial homomorphism.
\end{definition}

\begin{theorem}[Cayley's Theorem]\label{thm:cayley}
    Suppose \(G\) is a finite group.
    Then we can find a set \(X\)
    such that \(G\) can be represented
    as a transformation group on \(X\).
    More specifically,
    we can find a transformation group \(H\) on \(X\)
    such that \(G \cong H\).
\end{theorem}
\begin{proof}
    We consider \(X = G\),
    i.e.\ \(G\) potentially being isomorphic
    to a group that transforms all elements of itself.

    We first define a function \(\vfunc{\ell_a}{G}{G}{x}{ax}\),
    which we choose some \(a\)
    that multiply some element \(x \in G\) on the left.
    This is clearly a permutation on \(G\),
    i.e.\ it is a bijection \(G \to G\),
    because if \(\ell_a(x) = \ell_a(y)\) for some \(\{x,y\} \subset G\)
    hence \(ax = ay\), then \(a^{-1}ax = a^{-1}ay\), and \(x = y\),
    which proves that this is an injection;
    the pigeonhole principle proves that this is a surjection.

    For every element \(g \in G\),
    we can create a function \(\ell_g\).
    We let \(H = \{\ell_g : g \in G\}\)
    be the set of all such left permutations.
    We claim that \(H\) is a group,
    with function composition as multiplication,
    and \(\ell_1\) being the identity.
    We have closure because \(\ell_a\ell_b\)
    will map \(x \mapsto bx \mapsto abx\),
    which is equivalent to \(\ell_{ab}\).
    Function composition is naturally associative,
    so we get that for free.
    The inverse mapping from \(ax \to x\)
    is achieved by left multiplying by \(a^{-1}\),
    which itself is equivalent to \(\ell_{a^{-1}}\).
    Lastly, the identity works,
    because \(\ell_a\ell_1\) maps \(x \mapsto 1x = x \mapsto ax\)
    which is equivalent to \(\ell_a\),
    and similarly with \(\ell_1\ell_a = \ell_a\).

    We then want to prove that
    the mapping from \(a \to \ell_a\) is an isomorphism.
    Suppose \(\vfunc{\phi}{G}{H}{a}{\ell_a}\).
    This is a homomorphism because
    \begin{equation*}
        \phi(ab) = \ell_{ab} = \ell_a\ell_b = \phi(a)\phi(b)
    \end{equation*}
    And by construction, this is a bijection,
    because if \(\phi(a) = \phi(b)\),
    then \(\ell_a = \ell_b\),
    which implies \(ax = bx\) for all \(x \in G\),
    and hence \(axx^{-1} = bxx^{-1}\),
    so \(a = b\),
    which gives us injectivity;
    the pigeonhole principle again gives us surjectivity.

    We have now found a transformation group \(H\)
    that acts on a set \(G\),
    and that \(G\) is isomorphic to the transformation group \(H\).
\end{proof}
\begin{remark}
    One can repeat the entire proof with right multiplication instead,
    it is equally valid.
\end{remark}


\subsection{Cyclic Groups}

\begin{definition}
    Suppose \(G\) is a group.
    A subgroup \(H\) of \(G\) is a subset \(H \subseteq G\),
    with \(1 \in H\), and properties of a group hold.
    We call the group \(\{1\}\) the trivial subgroup.
\end{definition}

\begin{definition}
    Suppose we have a group \(G\),
    and some non-empty subset \(S \subseteq G\).
    The subgroup generated by \(S\)
    is the group of all elements that can be formed
    by the products of elements in \(S\) and their inverses.
    We denote it as
    \(\langle S \rangle = \{\prod_i s_i : s_i \in S \lor s_i^{-1} \in S\}\).
\end{definition}
\begin{remark}
    One can think of this as almost like how a vector space
    spanned by a set of basis vectors.
\end{remark}

\begin{lemma}\label{lem:intersection-subgroup}
    Suppose \(G\) is a group,
    and \(H_i \subseteq G\) subgroups.
    Then \(\bigcap_i H_i\) is also a subgroup of \(G\).
\end{lemma}
\begin{proof}
    If \(\{a,b\} \subset \bigcap_i H_i\),
    then \(\{a,b\} \subset H_i\) for all \(i\),
    and hence \(ab \in H_i\),
    which gives us closure \(ab \in \bigcap_i H_i\).

    Associativity is inherited from \(G\).

    The identity must be in each of \(H_i\),
    so it must also be in \(\bigcap_i H_i\).

    Lastly, if \(a \in \bigcap_i H_i\),
    then \(a \in H_i\) for all \(i\),
    so \(a^{-1} \in H_i\) for all \(i\),
    so we have inverses as \(a^{-1} \in \bigcap_i H_i\).
\end{proof}

\begin{proposition}\label{prop:subset-generated-subgroup}
    \(\langle S \rangle = \bigcap_i H_i\),
    where \(H_i\) is any subgroup of \(G\) containing \(S\).
\end{proposition}
\begin{proof}
    We first need to prove that
    \(\bigcap_i H_i \subseteq \langle S \rangle\),
    which it is sufficient to prove that \(\langle S \rangle\)
    is any subgroup of \(G\) containing \(S\),
    i.e.\ \(\langle S \rangle\) corresponds to some \(H_i\),
    since the intersection must be contained in any \(H_i\).
    It is trivial to see that \(1 \in \langle S \rangle\)
    by choosing \(s_2 = s_1^{-1}\), which gives us \(1\).
    Closure and associativity is inherited from \(G\),
    so we need not prove anything.
    Inverses exist since it is easy to check that
    \({(s_1 s_2 \hdots s_n)}^{-1} = s_n^{-1} \hdots s_2^{-1} s_1^{-1}\).
    Hence \(\langle S \rangle \subseteq G\) is a subgroup,
    which must contain \(S\) by construction.

    We then see that \(H_i \subseteq G\),
    so every element of \(\langle S \rangle\) is in \(H_i\),
    as it is closed under multiplication and inversion.
    Hence we have \(\langle S \rangle \subseteq \bigcap_i H_i\),
    and subsequently \(\langle S \rangle = \bigcap_i H_i\).
\end{proof}

\begin{definition}
    Suppose we have a group \(G\),
    and some element \(g \in G\).
    The subgroup generated by one element,
    otherwise known as the cyclic group generated by \(g\),
    is denoted \(\langle g \rangle = \{g^n : n \in \bZ\}\).
\end{definition}
\begin{proposition}
    Cyclic groups are abelian.
\end{proposition}
\begin{proof}
    \(g\) will always commute with itself.
\end{proof}

\begin{theorem}
    Cyclic groups of the same order (finite or infinite) are isomorphic.
\end{theorem}
\begin{proof}
    Suppose we have a cyclic group \(\langle g \rangle\).
    We first attempt to construct a homomorphism
    \(\vfunc{\phi}{(\bZ,+,0)}{\langle g \rangle}{n}{g^n}\).
    A quick proof shows us that it is indeed a homomorphism,
    with \(\phi(0) = g^0 = 1\), and
    \begin{equation*}
        \phi(m+n) = g^{m+n} = g^m g^n = \phi(m)\phi(n)
    \end{equation*}
    We also see that this is surjective by construction,
    but not necessarily injective,
    if the order of \(g\) is finite.

    We first consider the case of infinite order of \(g\).
    Then it is clear that no two integers will map to the same element,
    and hence \(\phi\) is injective, and therefore and isomorphism.
    This proves the part where we claim that
    all infinite cyclic groups are isomorphic,
    and specifically with \((\bZ,+,0)\).

    We now consider some \(g\) with order \(\abs{g} = d\).
    Then clearly \(\phi(d) = \phi(0) = g^d = 1\),
    which shows us that \(\phi\) is not injective.
    We can however show that \(\{1,g,g^2,\hdots,g^{d-1}\}\)
    are all distinct,
    which proves that \((\bZ/d\bZ,+,0) \cong \langle g \rangle\).
    We proceed by contradiction,
    assuming that there exists some \(0 \leq i < j < d\)
    such that \(g^i = g^j\).
    But then we know \(g^{j-i} = g^j g^{-i} = g^i g^{-i} = 1\),
    which tells us there exists some number \(j-i < d\)
    such that \(g^{j-i} = 1\),
    contradicting that \(\abs{g} = d\).
    Hence all cyclic groups with order \(d\)
    are isomorphic to \(\bZ/d\bZ,+,0\),
    which in turn are isomorphic to each other.
\end{proof}

\begin{definition}
    \(C_n\) is the cyclic group with order \(\abs{C_n} = n\).
    It is also sometimes denoted \(C_n = Z_n = \bZ/n\bZ = \bZ/(n) = \bZ/n\).
\end{definition}

\begin{theorem}\label{thm:cyclic-subgroup}
    Any subgroup of a cyclic group is cyclic.
    In particular, if \(H \subseteq G\) a finite cyclic group,
    \(\abs{H} \mid \abs{G}\).
\end{theorem}
\begin{proof}
    Suppose \(G = \langle g \rangle\) is generated by this element.
    Then all elements of \(H\) can be written as some power of \(g\).
    We pick the smallest such integer \(s\) such that \(g^s \in H\).
    We claim that \(H = \langle g^s \rangle\).
    No matter what element we pick in \(H\),
    we can write it as \(g^{s'}\).
    Via long division, we write \(s' = qs + r\), with \(0 \leq r < s\),
    so we know our element is
    \(g^{s'} = g^{qs+r} = {(g^s)}^q g^r\).
    Since \(g^s \in H\), by closure we have \({(g^s)}^q \in H\);
    which gives us \(g^r = {\qty({(g^s)}^q)}^{-1} g^{s'} \in H\).
    But if \(r \neq 0\),
    we will have found a smaller integer than \(s\)
    such that \(g^r \in H\),
    which itself is a contradiction.
    We are then forced to conclude that \(r = 0\),
    so all elements in \(H\) must be written as
    \(g^{s'} = g^{qs} = {(g^s)}^q \in \langle g^s \rangle\),
    and hence \(H \subseteq \langle g^s \rangle\).
    But obviously, \(g^s \in H\), so \({(g^s)}^q \in H\),
    and we have \(\langle g^s \rangle \subseteq H\).
    We then have our claim of equality \(H = \langle g^s \rangle\),
    which is generated by a single element.

    For the second part of the theorem,
    it is sufficient to prove that \(\abs{g^s} \mid \abs{g}\).
    Suppose \(\abs{g^s} = d\) and \(\abs{g} = n\).
    Then we know that \(g^{sd} = 1\), which we apply long division on.
    \(sd = qn + r\), with \(0 \leq r < n\),
    so \(g^{sd} = g^{qn}g^r = {(g^n)}^q g^r = g^r = 1\),
    which gives us \(r = 0\),
    because \(r\) is smaller than the order of \(g\).
    This tells us that \(sd \mid n\).
    Now, we know that \(d\) is the smallest number
    that we need to multiply \(s\) by to obtain a multiple of \(n\),
    which tells us \(d\) is the product of
    all the factors that \(n\) has,
    but \(s\) has not (counted with multiplicity).
    A simple decomposition of \(n = \gcd(s,n) \cdot n/\gcd(s,n)\)
    tells us that the gcd must be the factors that \(n\) and \(s\) share,
    so the other part must be the ones \(n\) has, but \(s\) has not,
    meaning \(d = n/\gcd(s,n)\).
    But by definition, these are factors of \(n\),
    so \(d \mid n\).
\end{proof}
\begin{corollary}
    \(\abs{g^s} = \abs{g}/\gcd(s,\abs{g})\).
\end{corollary}
\begin{remark}
    The above theorem gives us the ability to
    find the order of a given subgroup of a cyclic group.
    But to do the reverse, given an order finding a subgroup,
    each order might correspond to multiple generators,
    which means we need to prove the next theorem.
\end{remark}

\begin{theorem}\label{thm:cyclic-subgroup-uniqueness}
    Suppose \(G = \langle g \rangle\), with \(\abs{g} = n\).
    Given that \(d \mid n\), the subgroup of order \(d\) is unique.
\end{theorem}
\begin{proof}
    Let \(s = n/d\).
    We first see that \(H = \langle g^s \rangle\)
    is one such subgroup of order \(d\),
    since the corollary gives us
    \(\abs{H} = n/\gcd(n/d,n) = n/\gcd(n/d) = d\).

    Now suppose we have another subgroup \(H' \subseteq G\)
    with order \(d\).
    Then from the \hyperref[thm:cyclic-subgroup]{above theorem},
    we know that \(H' = \langle g^{s'} \rangle\) is cyclic.
    By the corollary, we get
    \begin{equation*}
        d = \frac{n}{\gcd(n,s')} \implies s = \frac{n}{d} = \gcd(n,s')
    \end{equation*}
    with \(s \mid s'\) by definition of gcd.
    Hence \(s'\) is a multiple of \(s\),
    so \(g^{s'} \in H\), and therefore \(H' \subseteq H\).
    But as the orders are the same,
    the pigeonhole principle gives us \(H' = H\).
\end{proof}

\begin{theorem}
    Suppose \(g\) and \(h\) commute (\(gh = hg\)),
    with \(m = \abs{g}\), \(n = \abs{h}\),
    \(m,n\) coprime (\(\gcd(m,n) = 1\)).
    Then \(\abs{gh} = mn = \abs{g}\abs{h}\).
\end{theorem}
\begin{proof}
    Let \(d = \abs{gh}\).
    We first notice that
    \({(gh)}^{mn} = g^{mn}h^{mn} = {(g^m)}^n{(h^n)}^m = 1\),
    so \(d \mid mn\).

    Now we claim that \(mn\) is minimal.
    Suppose we have a \(0 < d < mn\) such that \({(gh)}^d = g^d h^d = 1\),
    so we know that \(g^d = h^{-d}\).
    Since their orders are coprime,
    unless one of the groups is \(\{1\}\),
    they are not subgroups of each other,
    so we have \(1 = g^d = h^{-d}\),
    and \(m \mid d\) and \(n \mid d\).
    But then \(d\) must be some multiple of \(\lcm(m,n) = mn\),
    which is impossible within the range \(0 < d < mn\).
    Hence \(d = mn\).
\end{proof}
\begin{remark}
    In general, \(g\) and \(h\) will not commute,
    so it is impossble to relate \(\abs{gh}\) to \(\abs{g}\abs{h}\),
    since you would have to understand the groups
    \(\langle g \rangle\) and \(\langle h \rangle\) together.
\end{remark}


\subsection{Permutations}

\begin{definition}
    Let \(X = \bN_n = \{1,2,3,\hdots,n\}\)
    a set (any set) of \(n\) elements.
    We call a bijective function \(\func{\sigma}{X}{X}\) a permutation.
\end{definition}
\begin{proposition}\label{prop:symmetric-group}
    The set of all possible permutations,
    with function composition as multiplication
    and the identity mapping as the identity,
    forms a group of order \(n!\).
\end{proposition}
\begin{proof}
    It is easy to see that the composition of two bijections
    is also a bijection,
    which gives us closure.
    We inherit associativity from function composition,
    and the identity mapping works as intended
    on both the left and the right side.
    Lastly, the inverse of a bijection
    must also be a bijection.

    To count all the ways we can construct such permutations,
    we see that for any given \(\sigma\),
    there are \(n\) possible elements that we can map \(1\) to,
    and after that choice is taken,
    \(n-1\) possible elements that we can map \(2\) to,
    and so forth until there is only one choice
    for what we can map \(n\) to.
    This gives us a total of \(n(n-1)\hdots(2)(1) = n!\) choices.
\end{proof}
\begin{definition}
    It is clear that the groups of permutations
    of any two sets of \(n\) items
    are isomorphic to each other.
    We call this group \(S_n\),
    the symmetric group of \(n\) elements.
\end{definition}

\begin{definition}
    We can write elements of \(S_n\) as disjoint cycles
    \((a_1 a_2 \hdots a_p) \hdots (b_1 b_2 \hdots b_q)\)
    where this permutation will map \(a_1 \mapsto a_2\),
    \(a_2 \mapsto a_3\), and so on until \(a_p \to a_1\),
    repeating this process for every cycle.
    We will often omit cycles of length 1.
\end{definition}
\begin{remark}
    It is worth noting that disjoint cycles commute.
\end{remark}

\begin{definition}
    We call \(\tau = (ab)\) a cycle of two elements
    a flip or a transposition.
\end{definition}
\begin{lemma}[Breaking the cycle]\label{lem:breaking-cycles-sn}
    Suppose \(\rho = (a c_1 \hdots c_m b d_1 \hdots d_n)\)
    is a cycle of length \(m+n+2\).
    Then \(\rho = (ab)(a c_1 \hdots c_m)(b d_1 \hdots d_n)\).
\end{lemma}
\begin{proof}
    This is equivalent to proving that
    \((ab)\rho = (a c_1 \hdots c_m)(b d_1 \hdots d_n)\).
    We can see that \(\rho\) maps the following way,
    so \((ab)\rho\) must simply swap all occurrences of \(a\) with \(b\)
    in the results, and vice versa.
    \begin{equation*}
        \rho = \begin{cases}
            a \mapsto c_1 \\
            c_i \mapsto c_{i+1} & i < m \\
            c_m \mapsto b \\
            b \mapsto d_1 \\
            d_j \mapsto d_{j+1} & j < n \\
            d_n \mapsto a
        \end{cases} \qquad
        (ab)\rho = \begin{cases}
            a \mapsto c_1 \\
            c_i \mapsto c_{i+1} & i < m \\
            c_m \mapsto a \\
            b \mapsto d_1 \\
            d_j \mapsto d_{j+1} & j < n \\
            d_n \mapsto b
        \end{cases}
    \end{equation*}
\end{proof}
\begin{theorem}\label{thm:sn-product-transpositions}
    Any \(\sigma \in S_n\) is a non-unique product of transpositions.
\end{theorem}
\begin{proof}
    We write \(\sigma = \gamma_1\hdots\gamma_n\) as disjoint cycles.
    We now decompose each of \(\gamma_i\).
    First suppose \(\gamma_i\) is a transposition already.
    Then there is no need to do anything.

    Now suppose \(\gamma_i = (abc)\) is of length 3.
    Then by our \hyperref[lem:breaking-cycles-sn]{lemma above},
    \(\gamma_i = (ac)(ab)(c) = (ac)(ab)\)
    and we have a product of transpositions.

    Then suppose \(\gamma_i = (abcd)\) is of length 4.
    By our \hyperref[lem:breaking-cycles-sn]{lemma above},
    \(\gamma_i = (ac)(ab)(cd)\)
    and we have a product of transpositions.

    Applying the induction step,
    assume that \(\gamma_i = (a b c_1 \hdots c_n)\) is a cycle of \(n+2\),
    and the case of cycles of length \(n\) is already proven.
    Then we know by our lemma \(\gamma_i = (ac_1)(ab)(c_1 \hdots c_n)\),
    so we have two transpositions and a cycle of length \(n\),
    which by the inductive hypothesis can be decomposed into transpositions.

    This representation is not unique,
    since if \(\sigma = \tau_1\hdots\tau_n\),
    and \(\tau'\) is another transposition,
    we can write \(\sigma = \tau'{\tau'}^{-1}\tau_1\hdots\tau_n
    = \tau'\tau'\tau_1\hdots\tau_n\)
    a different representation.
\end{proof}

\begin{proposition}\label{prop:sn-conjugation}
    Suppose \(\sigma \in S_n\).
    Then \(\sigma(n_1 n_2 \hdots n_d)\sigma^{-1}
    = (\sigma(n_1) \sigma(n_2) \hdots \sigma(n_d))\).
\end{proposition}
\begin{proof}
    Since all \(\sigma\) can be written a a product of transpositions
    (Theorem~\ref{thm:sn-product-transpositions}),
    it is sufficient to consider \(\tau(n_1 n_2 \hdots n_d)\tau^{-1}\).
    Suppose we have \((ab)\rho(ab)\),
    and \(\rho = (ac_1 \hdots c_m bd_1 \hdots d_n)\)
    Then we have
    \begin{equation*}
        (ab) = \begin{cases}
            a \mapsto b \\
            c_i \mapsto c_i \\
            b \mapsto a \\
            d_j \mapsto d_j
        \end{cases} \qquad
        \rho(ab) = \begin{cases}
            a \mapsto d_1 \\
            c_i \mapsto c_{i+1} & i < m \\
            c_m \mapsto b
            b \mapsto c_1 \\
            d_j \mapsto d_{j+1} & j < n\\
            d_n \mapsto a
        \end{cases}
    \end{equation*}
    which in cycle notation is
    \(\rho(ab) = (ad_1 \hdots d_n)(bc_1 \hdots c_m)\),
    and if we \hyperref[lem:breaking-cycles-sn]{break the cycle}
    we get \((ab)\rho(ab) = (ad_1 \hdots d_n bc_1 \hdots c_m)\)
    which is exactly as desired, swapping the positions of \(a\) and \(b\).
\end{proof}

\begin{definition}
    Suppose \(\sigma = \gamma_1\gamma_2\hdots\gamma_n\)
    is a permutation written in disjoint cycle notation,
    with \(\gamma_i\) having length \(d_i\).
    Then we define the sign of the permutation as
    \begin{equation*}
        \sgn(\sigma) = {(-1)}^{\sum_{i=1}^n (d_i-1)}
        = \prod_{i=1}^n {(-1)}^{(d_i-1)}
    \end{equation*}
    If \(\sgn(\sigma) = 1\), \(\sigma\) is called even;
    conversely, if \(\sgn(\sigma) = -1\), \(\sigma\) is called odd.
\end{definition}
\begin{remark}
    It is worth noting that cycles of odd length are even,
    and cycles of even length are odd.
    Then cycles of length 1 (identity) are even,
    so it does not change sign.
\end{remark}

\pagebreak

\begin{proposition}
    Suppose \(\tau = (ab)\) is a transposition, and \(\sigma \in S_n\).
    Then \(\sgn(\tau\sigma) = -\sgn(\sigma)\).
\end{proposition}
\begin{proof}
    We write \(\sigma = \gamma_1\hdots\gamma_r\) as disjoint cycles,
    inclding 1-cycles.
    We first consider the case that \(a,b\) are in the same cycle,
    so without loss of generality let \(a,b\) be in \(\gamma_1\),
    which has length \(m+n+2\).
    Then we have, by the lemma
    \begin{gather*}
        \tau\sigma
        = (ab)(ac_1 \hdots c_m bd_1 \hdots d_n)\gamma_2\hdots\gamma_r
        = (ac_1 \hdots c_m)(bd_1 \hdots d_n)\gamma_2\hdots\gamma_r \\
        \sgn(\tau\sigma)
        = {(-1)}^m{(-1)}^n \prod_{i=2}^r {(-1)}^{(d_i-1)}
        = {(-1)}^{m+n} \prod_{i=2}^r {(-1)}^{(d_i-1)}
        = -{(-1)}^{m+n+1} \prod_{i=2}^r {(-1)}^{(d_i-1)}
        = -\sgn(\sigma)
    \end{gather*}

    Now consider the case that \(a,b\) are in different cycles,
    so without loss of generality
    let \(a\) be in \(\gamma_1\) of length \(m+1\),
    and \(b\) be in \(\gamma_2\) of length \(n+1\).
    By the lemma again
    \begin{gather*}
        \tau\sigma
        = (ab)(ac_1 \hdots c_m)(bd_1 \hdots d_n)\gamma_3\hdots\gamma_r
        = (ac_1 \hdots c_m bd_1 \hdots d_n)\gamma_3\hdots\gamma_r \\
        \sgn(\tau\sigma)
        = {(-1)}^{m+n+1} \prod_{i=3}^r {(-1)}^{(d_i-1)}
        = -{(-1)}^{m+n} \prod_{i=3}^r {(-1)}^{(d_i-1)}
        = -{(-1)}^m{(-1)}^n \prod_{i=2}^r {(-1)}^{(d_i-1)}
        = -\sgn(\sigma)
    \end{gather*}
\end{proof}
\begin{corollary}\label{cor:sgn-transposition}
    Suppose \(\sigma = \tau_1\tau_2\cdots\tau_k\)
    written as a product of transpositions.
    Then \(\sgn(\sigma) = {(-1)}^k\).
\end{corollary}
\begin{proof}
    \(\sgn(\sigma) = \sgn(\tau_1\tau_2\cdots\tau_k)
    = (-1)\sgn(\tau_2\cdots\tau_k)
    = \cdots = {(-1)}^{k-1}\sgn(\tau_k)
    = {(-1)}^k\)
\end{proof}
\begin{remark}
    Since the sign of a permutation is either odd or even,
    this tells us that the number of transpositions might not be unique,
    but whether there are an odd or even number of them
    is inherent to each permutation.
\end{remark}

\begin{theorem}\label{thm:sgn-mult}
    \(\sgn(\sigma_1\sigma_2) = \sgn(\sigma_1)\sgn(\sigma_2)\).
\end{theorem}
\begin{proof}
    Suppose \(\sigma_1 = \tau_1\hdots\tau_m\),
    and \(\sigma_2 = \tau'_1\hdots\tau'_n\).
    Then by the \hyperref[cor:sgn-transposition]{corollary above},
    \begin{align*}
        \sgn(\sigma_1\sigma_2)
        &= \sgn(\tau_1\hdots\tau_m\tau'_1\hdots\tau'_n)
        = {(-1)}^{m+n}
        = {(-1)}^m{(-1)}^n \\
        &= \sgn(\tau_1\hdots\tau_m)\sgn(\tau'_1\hdots\tau'_m)
        = \sgn(\sigma_1)\sgn(\sigma_2)
    \end{align*}
\end{proof}
\begin{corollary}
    The even permutations form a subgroup of \(S_n\).
\end{corollary}
\begin{proof}
    Suppose \(\sigma_1,\sigma_2\) are both even,
    Then since \(\sgn(\sigma_1) = \sgn(\sigma_2) = 1\),
    we have \(\sgn(\sigma_1\sigma_2) = \sgn(\sigma_1)\sgn(\sigma_2) = 1\).
    We have closure.

    The identity is also even,
    because \(\sgn(\sigma) = \sgn(1\sigma) = \sgn(1)\sgn(\sigma)\),
    which gives us \(\sgn(1) = 1\), an even permutation.

    Lastly, if \(\sigma\) is even,
    then \(\sigma^{-1}\) is too,
    because if \(\sigma = 1\),
    then \(1 = \sgn(1) = \sgn(\sigma\sigma^{-1})
    = \sgn(\sigma)\sgn(\sigma^{-1}) = \sgn(\sigma^{-1})\).
\end{proof}
\begin{definition}
    We call the subgroup of even permutations of \(S_n\)
    the alternating group of \(n\) elements,
    usually denoted as \(A_n\).
\end{definition}


\subsection{Cosets}\label{sec:cosets}

\begin{proposition}
    Suppose \(G\) is a transformation group of some set \(X\).
    For any \(\{x,y\} \subset X\),
    if we define a relation \(x \sim y\) to be
    when there exists some \(g \in G\)
    such that \(g(x) = y\),
    such a relation is an equivalence relation.
\end{proposition}
\begin{proof}
    This is reflexive because \(1 \in G\),
    and \(1(x) = x\), so \(x \sim x\).
    This is symmetric because suppose \(x \sim y\),
    then there exists some \(g \in G\) such that \(g(x) = y\);
    it is easy to see that then \(g^{-1}(y) = x\),
    so we have \(y \sim x\).
    This is transitive because if \(x \sim y\) and \(y \sim z\),
    then there exists \(g(x) = y\) and \(h(y) = z\),
    and since \(hg \in G\), \((hg)(x) = h(y) = z\),
    so \(x \sim z\).
\end{proof}
\begin{definition}
    We call the equivalence class of \(x \in X\) the orbit of \(x\),
    which is often denoted as \(\orb(x) = Gx = \{g(x) : g \in G\}\).
    If this is the entire set,
    i.e.\ there exists some \(x \in X\)
    (and therefore all \(x \in X\) by Theorem~\ref{thm:equiv-class-partition})
    such that \(Gx = X\),
    we call \(G\) a transitive transformation group.
\end{definition}

\begin{proposition}
    Suppose \(G\) is a group, and \(H \subseteq G\) some subgroup.
    For any \(\{g_1,g_2\} \subset G\),
    if we define a relation \(g_1 \sim g_2\) to be
    when there exists some \(h \in H\) such that \(g_2 = g_1h\),
    such a relation is an equivalence relation.
\end{proposition}
\begin{proof}
    This is reflexive because \(1 \in H\),
    and \(g_1 = g_1 1\), so \(g_1 \sim g_1\).
    This is symmetric because suppose \(g_1 \sim g_2\),
    then there exists \(h \in H\) such that \(g_2 = g_1 h\);
    it is easy to see that \(g_1 = g_2 h^{-1}\), so we have \(g_2 \sim g_1\).
    This is transitive because if \(g_1 \sim g_2\) and \(g_2 \sim g_3\),
    then there exists \(h_1\) and \(h_2\) such that
    \(g_2 = g_1 h_1\) and \(g_3 = g_2 h_2\),
    so \(g_3 = g_1 h_1 h_2\),
    and hence \(g_1 \sim g_3\).
\end{proof}
\begin{definition}
    We call the equivalence class of \(g \in G\) a left coset of \(H\) in \(G\),
    which is denoted \(gH = \{gh : h \in H\}\).
    We can similarly define a right coset of \(H\) in \(G\),
    which is denoted \(Hg = \{hg : h \in H\}\).
\end{definition}
\begin{definition}
    The set of all left cosets is denoted \(G/H\),
    while the set of all right cosets is denoted \(H \backslash G\).
\end{definition}
\begin{definition}
    The number of left cosets is called the index of \(H\) in \(G\)
    and is denoted \(\abs{G/H} = [G:H]\).
\end{definition}

\begin{lemma}\label{lem:order-coset}
    Suppose \(G\) a group, and \(H \subseteq G\) some subgroup.
    Then for any \(g \in G\),
    the cosets are the same size,
    and in particular \(\abs{gH} = \abs{H}\).
\end{lemma}
\begin{proof}
    We write the left multiplication function
    \(\vfunc{\ell_g}{H}{gH}{h}{gh}\).
    We can show that it is injective,
    because if \(gh_1 = gh_2\),
    then \(g^{-1}gh_1 = g^{-1}gh_2\),
    so \(h_1 = h_2\).
    We can also show that it is surjective,
    because for every element \(gh \in gH\),
    clearly \(h \mapsto gh\) and \(h \in H\) by definition.
    Hence \(\ell_g\) is a bijection,
    which shows that \(\abs{H} = \abs{gH}\).
\end{proof}
\begin{theorem}[Lagrange's Theorem]\label{thm:lagrange}
    For some finite group \(G\), and \(H \subseteq G\) any subgroup,
    \(\abs{G} = \abs{H}[G:H]\).
\end{theorem}
\begin{proof}
    By definition, there are a total of \([G:H]\) cosets,
    and the subgroup itself is a coset \(H = 1H\).
    Since cosets are equivalence classes,
    they partition \(G\),
    so the order of \(G\) must be the sum of the orders of the cosets.
    But we also know that all the cosets are the of size \(\abs{H}\)
    from the \hyperref[lem:order-coset]{lemma above},
    so we have \(\abs{G} = \abs{H}[G:H]\).
\end{proof}
\begin{corollary}\label{cor:prime-order-subgroup}
    Suppose \(\abs{G} = p\) some prime order.
    Then if \(H \subseteq G\) is a subgroup,
    either \(H = \{1\}\) or \(H = G\),
    and in the second case,
    \(H\) is generated by any element \(g \in G\) when \(g \neq 1\).
\end{corollary}
\begin{proof}
    Since from the \hyperref[thm:lagrange]{theorem above}
    we have \(\abs{H}\mid\abs{G}\),
    \(\abs{H}\) is either 1 or \(p\).
    If \(\abs{H} = 1\), since all groups must have the identity,
    \(H = \{1\}\).

    On the other hand, if \(\abs{H} = p\),
    then it must be the whole group, so \(H = G\).
    Now pick any \(g \in G\).
    We know that \(\langle g \rangle \subseteq G\) is a subgroup,
    and if \(g \neq 1\),
    we cannot have \(\langle g \rangle = \{1\}\),
    so we are forced to conclude otherwise,
    and we have \(\langle g \rangle = G\).
\end{proof}
\begin{corollary}\label{cor:order-element-group}
    Suppose \(G\) is a finite group, with \(g \in G\).
    If \(n = \abs{G}\), then \(g^n = 1\).
\end{corollary}
\begin{proof}
    Suppose \(\abs{g} = m\), so \(g^m = 1\).
    By the \hyperref[thm:lagrange]{theorem above},
    we know that \(m \mid n\),
    so there exists some \(r \in \bN\) such that \(n = mr\).
    Hence \(g^n = g^{mr} = {(g^m)}^r = 1^r = 1\).
\end{proof}
\begin{corollary}
    \(\abs{A_n} = n!/2\).
\end{corollary}
\begin{proof}
    The even and odd permutations form two cosets in \(S_n\),
    since multiplying by even permutations
    (and hence by elements of \(A_n\)) do not change sign
    by Theorem~\ref{thm:sgn-mult}.
    Then we have \(\abs{S_n} = \abs{A_n}[S_n:A_n]\),
    so \(\abs{A_n} = \abs{S_n}/2 = n!/2\).
\end{proof}


\subsection{Normal Subgroups}

\begin{definition}
    Suppose \(G\) is a group, and \(H \subseteq G\) some subgroup.
    \(H\) is a normal subgroup of \(G\)
    if for all \(g \in G\) and \(h \in H\),
    \(ghg^{-1} \in H\).
    We often denote this as \(H \lhd G\).
\end{definition}
\begin{theorem}
    If \(H \lhd G\),
    then the set of left cosets \(G/H\) forms a group,
    and in particular, there exists a epimorphism
    from \(G\) to \(G/H\).
\end{theorem}
\begin{proof}
    We attempt to write a function \(\vfunc{\phi}{G}{G/H}{g}{gH}\)
    that maps every element into its coset.
    By definition, this is a surjective mapping,
    since every coset must have some element,
    and those elements must be in \(G\).

    We also see that every coset must be represented by some element,
    so in the following, let \(g'_i \in G\) represent the coset \(g_i H\);
    by definition of a coset we have \(g_i = g'_i h_i\) for some \(h_i \in H\).
    A simple proof of the homomorphism shows that
    \(\phi(g_1g_2) = (g_1 g_2)H\) which is the coset of \(g_1 g_2\),
    and \(\phi(g_1)\phi(g_2) = (g_1 H)(g_2 H)\)
    which is the coset of the representatives \(g'_1 g'_2\).
    For them to be in the same coset,
    we see \(g_1 g_2 = g'_1 h_1 g'_2 h_2
    = g'_1 g'_2 ({(g'_2)}^{-1} h_1 g'_2) h_2\)
    needs to be written in the form \(gH\),
    so we want \({(g'_2)}^{-1} h_1 g'_2 \in H\)
    for every possible \(g'_2 \in G\) and \(h_1 \in H\).
    But this is exactly the condition that normal subgroups provide,
    and hence there exists an epimorphism from \(G\) to \(G/H\),
    which shows that \(G/H\) forms a group,
    inheriting the identity as the coset of \(1 \in G\),
    and the multiplication itself from \(G\).
\end{proof}
\begin{definition}
    Suppose \(G\) some group, and \(H \lhd G\).
    We call \(G/H\) the quotient group of \(G\) by \(H\).
\end{definition}

\begin{proposition}\label{prop:abelian-subgroup-normal}
    Subgroups of abelian groups must be normal.
\end{proposition}
\begin{proof}
    Suppose \(G\) is our abelian group,
    and \(H \subseteq G\) our subgroup.
    Then for all \(g \in G\) and \(h \in H\),
    we realize that both \(g,h\) are elements of \(G\),
    and hence will commute,
    so we have \(ghg^{-1} = gg^{-1}h = h \in H\),
    and hence \(H \lhd G\).
\end{proof}

\begin{theorem}\label{thm:equal-coset-normal}
    \(H \lhd G\) is a normal subgroup if and only if
    the left and right cosets are identical, that is,
    \(gH = Hg\) for all \(g \in G\).
\end{theorem}
\begin{proof}
    In the forward direction,
    assuming a normal subgroup,
    by definition we have for all \(g \in G\) and \(h \in H\),,
    \(ghg^{-1} \in H\),
    which tells us that there exists some \(h' \in H\)
    such that \(ghg^{-1} = h'\).
    When we right-multiply by \(g\) on both sides,
    we get \(gh = h'g\).
    This tells us that for all elements \(g \in G\) and \(h \in H\)
    \(gh \in Hg\), so we get \(gH \subseteq Hg\).
    Similarly, we can left-multiply by \(g^{-1}\) on both sides,
    and get \(hg^{-1} = g^{-1}h'\),
    which tells us for all \(g \in G\) and \(h \in H\),
    \(hg \in gH\), so we get \(Hg \subseteq gH\).
    Combining these two statements, we get \(gH = Hg\).

    In the reverse direction,
    assuming that left and right cosets are equal,
    we reverse the argument to see that
    for all \(g \in G\) and \(h \in H\),
    \(gh \in Hg\), so there exists some \(h' \in H\)
    such that \(gh = h'g\), which implies \(ghg^{-1} = h' \in H\).
\end{proof}

\begin{remark}
    It is good to remind ourselves that
    \(gH \subseteq G\), cosets are subsets of the whole group;
    but \(gH \in G/H\), these are now elements of the quotient group.
\end{remark}


\subsection{Homomorphisms and Isomorphisms}

% \begin{proposition}
%     Suppose \(\func{\phi}{G_1}{G_2}\) is a homomorphism between groups,
%     not necessarily injective nor surjective.
%     The image of \(G_1\) under \(\phi\) is a subgroup of \(G_2\),
%     that is, \(\phi(G_1) \subseteq G_2\) forms a subgroup.
% \end{proposition}

% \begin{lemma}
%     Suppose we have a homomorphism \(\func{\phi}{G_1}{G_2}\).
%     The preimage of the identity is a normal subgroup,
%     that is, \(\phi^{-1}(1_{G_2}) \lhd G_1\).
% \end{lemma}

\begin{definition}
    Suppose we have a group homomorphism \(\func{\phi}{G}{H}\).
    We call the preimage of the identity \(\phi^{-1}(1)\)
    the kernel of \(\phi\),
    sometimes denoted \(\ker(\phi) = \{g \in G : \phi(g) = 1\}\).
\end{definition}

\begin{theorem}[Universal Property of Quotient Groups]\label{thm:univ-prop-quotient-group}
    Let \(G,H\) be groups,
    and \(N \lhd G\) be a normal subgroup.
    Suppose \(\func{\pi}{G}{G/N}\) is the quotient homomorphism
    and \(\func{\phi}{G}{H}\) is any group homomorphism
    with \(N \subseteq \ker(\phi)\).
    Then there exists a unique group homomorphism
    \(\func{\bar{\phi}}{G/N}{H}\) such that \(\phi = \bar{\phi}\circ\pi\).

    This is represented by the following commutative diagram:
    \begin{center}
        \begin{tikzcd}
            G \arrow{r}{\phi} \arrow{d}{\pi} & H \\
            G/N \arrow{ru}[swap]{\exists! \bar{\phi}}
        \end{tikzcd}
    \end{center}
\end{theorem}
\begin{proof}
    This universal property asserts that
    both the existence and and the uniqueness of \(\bar{\phi}\),
    which needs to be proven separately.
    We will write our proof in reverse order,
    first proving the uniqueness assuming existence,
    and then proving existence without any assumptions.

    Suppose \(\phi = \bar{\phi}\circ\pi = \bar{\phi}'\circ\pi\).
    Since \(N \subseteq \ker(\phi)\),
    all elements \(n \in N\) obey \(\phi(n) = 1\).
    Knowing that all elements in \(G\) belong to some coset \(gN\),
    all \(g' \in gN\) maps to
    \(\phi(g') = \phi(gn) = \phi(g)\phi(n) = \phi(g)\),
    which tells us the image of each coset
    is a single element \(\phi(gN) = \{\phi(g)\}\).
    Now, seeing that \(\pi\) maps \(g' \mapsto gN\),
    its coset, by definition of a quotient mapping
    if \(\bar{\phi} \neq \bar{\phi}'\),
    there must be one such coset \(gN\)
    that \(\bar{\phi}(gN) \neq \bar{\phi}'(gN)\) disagrees on.
    However, this is a contradiction,
    because for all \(g' \in gN \subseteq G\),
    \(\bar{\phi}'(gN) = \bar{\phi}'(\pi(g')) = \phi(g')
    = \bar{\phi}(\pi(g')) = \bar{\phi}(gN)\),
    contradicting with our assumed inequality above.
    Hence we have established uniqueness of \(\bar{\phi}\).

    We will now prove existence by constructing such a homomorphism.
    Let \(\vfunc{\bar{\phi}}{G/N}{H}{gN}{\phi(g)}\),
    mapping all cosets \(gN\) to the function output
    of its coset representative.
    Suppose some arbitrary element \(g' \in gN \subseteq G\)
    in an arbitrary coset.
    Then we know that there exists \(n \in N\) such that \(g' = gn\),
    which allows us to conclude that
    \begin{equation*}
        \phi(g') = \phi(gn) = \phi(g)\phi(n) = \phi(g)
        = \bar{\phi}(gN) = \bar{\phi}(\pi(g'))
    \end{equation*}
\end{proof}
\begin{remark}
    Logically speaking,
    the existence part is proven before the uniqueness part
    in the sense that one requires existence to prove uniqueness.
    However, it is convenient to write universal property proofs
    in the reverse order
    because during the proof of uniqueness,
    we often demonstrate criteria that must be followed
    by our unique function \(\bar{\phi}\),
    which greatly helps us in deciding how to construct \(\bar{\phi}\)
    during the proof of existence.
\end{remark}

\begin{remark}\label{rmk:iso-numbering}
    Jacobson here groups all the statements below into something
    known as the Fundamental Theorem of Homomorphisms of Groups
    and the Isomorphism Theorems.
    % which concerns only with the image of \(G\),
    % which is a subset of \(H\).
    We will group the theorems differently,
    following the convention as given in Rotman and Dummit \& Foote,
    and with the correspondence theorem denoted the Fourth Isomorphism Theorem.
    Nevertheless, armed with the universal property,
    these are much easier to prove now.
\end{remark}
% \begin{corollary}[Fundamental Homomorphism Theorem for Groups]
%     Suppose \(\func{\phi}{G}{H}\) is a group homomorphism,
%     and \(N = \ker(\phi)\).
%     Then there exists a unique isomorphism
%     that makes the image \(\phi(G) \cong G/N\).
% \end{corollary}
% \begin{proof}
%     We want to prove that for all \(g \in \phi^{-1}(x)\)
%     we have \(\phi^{-1}(x) = gH\) being the coset.
%     Suppose we have \(\phi(g_1) = \phi(g_2) = x\)
%     mapping to the same element.
%     Then we see that \(\phi(g_1^{-1} g_2) = \phi(g_1)^{-1}\phi(g_2)
%     = x^{-1}x = 1\).
%     This implies that \(g_1^{-1} g_2 \in H\) inside the kernel,
%     so that there exists \(h \in H\) such that \(h = g_1^{-1}g_2\),
%     and therefore \(g_1 h = g_2\),
%     telling us that they are in the same coset,
%     \(g_2 \in g_1 H\), implying that \(\phi^{-1}(x) \subseteq gH\).
% \end{proof}
\begin{remark}
    The next four isomorphism theorems do also apply for monoids and submonoids.
    The proofs are in fact part of proving the group isomorphism theorems,
    just slightly simpler.
\end{remark}

\begin{theorem}[First Isomorphism Theorem for Groups]\label{thm:iso-1-group}
    Suppose \(\func{\phi}{G}{H}\) is a group homomorphism,
    and \(N = \ker(\phi)\).
    We have:
    \begin{enumerate}[label={(\alph*)}, itemsep=0mm]
        \item \(N \lhd G\), the kernel is a normal subgroup;
        \item \(\phi(G) \subseteq H\), the image is a subgroup; and
        \item \(\phi(G) \cong G/N\),
            the image is uniquely isomorphic to the quotient group.
    \end{enumerate}

    This is represented by the following commutative diagram:
    \begin{center}
        \begin{tikzcd}
            G \arrow{rd}{\phi} \arrow{d}{\pi} & H \\
            G/N \arrow{r}{\cong} & \phi(G) \arrow[dash]{u}
        \end{tikzcd}
    \end{center}
\end{theorem}
\begin{proof}
    We first prove that the preimage of the identity
    forms a subgroup of \(H\).
    Suppose \(\{g_1,g_2\} \subset \phi^{-1}(1)\).
    Then we know \(\phi(g_1) = \phi(g_2) = 1\),
    which gives us closure with \(\phi(g_1 g_2) = \phi(g_1)\phi(g_2) = 1\).
    Associativity is inherited from the multiplication of \(G_1\),
    while the identity by definition must obey \(\phi(1) = 1\).
    Lastly we have \(1 = \phi(1) = \phi(g_1 g_1^{-1})
    = \phi(g_1)\phi(g_1^{-1})\),
    which forces us to conclude that \(g_1^{-1} \in \phi^{-1}(1)\),
    giving us an inverse.
    Hence \(\phi^{-1}(1)\) is a group.

    Now to prove that this is a normal subgroup,
    suppose we have some arbitrary element \(g \in G\),
    and \(h \in \phi^{-1}(1)\).
    We see that \(\phi(ghg^{-1}) = \phi(g)\phi(h)\phi(g^{-1})
    = \phi(g)\phi(g^{-1}) = \phi(gg^{-1}) = \phi(1) = 1\),
    which implies \(ghg^{-1} \in \phi^{-1}(1)\),
    exactly the definition of a normal subgroup,
    which proves statement (a).

    \medskip

    If \(\{\phi(x),\phi(y)\} \subset \phi(G)\),
    then \(\phi(x)\phi(y) = \phi(xy) \in \phi(G)\),
    and thus we have closure.
    Then \(\phi(1_G) \in \phi(G)\)
    which must be the identity by definition of a homomorphism.
    Associativity is obtained for free
    when we equate the multiplicative operation
    between \(G\) and \(\phi(G)\).
    Lastly the inverse must be given by
    \(1_H = \phi(1_G) = \phi(xx^{-1}) = \phi(x)\phi(x^{-1})\).
    Hence \(\phi(G)\) forms a subgroup of \(H\),
    which proves statement (b).

    \medskip

    By the \hyperref[thm:univ-prop-quotient-group]{universal property},
    there exists a unique homomorphism \(\func{\bar{\phi}}{G/N}{\phi(G)}\)
    such that \(\phi = \bar{\phi}\circ\pi\),
    where \(\pi\) is the quotient homomorphism.
    Since by definition, \(\phi\) is a surjective mapping
    from \(G\) to its image \(\phi(G)\),
    \(\bar{\phi}\) must also be a surjection.

    Again taking the same definition for \(\bar{\phi}\)
    as in the universal property,
    \(\vfunc{\bar{\phi}}{G/N}{\phi(G)}{gN}{\phi(g)}\),
    suppose we have two cosets \(g_1 N\) and \(g_2 N\).
    such that \(\bar{\phi}(g_1 N) = \bar{\phi}(g_2 N)\),
    giving us \(\phi(g_1) = \bar{\phi}(\pi(g_1)) = \bar{\phi}(g_1 N)
    = \bar{\phi}(g_2 N) = \bar{\phi}(\pi(g_2)) = \phi(g_2)\)
    But then we have \(1 = \phi(g_1){\phi(g_2)}^{-1} = \phi(g_1 g_2^{-1})\),
    implying that \(g_1 g_2^{-1} \in N\), the kernel.
    Hence we have \(g_1 \in g_2 N\),
    which gives us \(g_1 N \subseteq g_2 N\),
    so without loss of generality, \(g_2 N \subseteq g_1 N\),
    giving us \(g_1 N = g_2 N\), the cosets must be equal.
    This proves injectivity.

    Combining surjection and injection
    gives us our required bijective homomorphism,
    which is an isomorphism,
    proving statement (c).
\end{proof}

\begin{definition}
    Suppose \(G\) is a group,
    and \(S,T \subseteq G\) some subgroups.
    We define the product to be \(ST = \{st : s \in S, t \in T\}\).
\end{definition}

\begin{theorem}[Second Isomorphism Theorem for Groups]\label{thm:iso-2-group}
    Suppose \(G\) is a group,
    \(S \subseteq G\) some subgroup,
    and \(N \lhd G\) a normal subgroup.
    We have:
    \begin{enumerate}[label={(\alph*)}, itemsep=0mm]
        \item \(SN \subseteq G\), the product is a subgroup;
        \item \(N \lhd SN\),
            the normal subgroup is also normal to the product;
        \item \(S \cap N \lhd S\),
            the intersection is normal to the subset; and
        \item \(SN/N \cong S/(S \cap N)\),
            these two quotients are isomorphic.
    \end{enumerate}
    
    This is represented by the following commutative diagram:
    \begin{center}
        \begin{tikzcd}
            G \arrow[dash]{d} \\
            SN \arrow[dash]{d} \arrow{r}{\pi} \arrow[dash]{rd} &
            SN/N \arrow[leftrightarrow]{rd}{\cong} \\
            N \arrow[dash]{rd} & S \arrow{r}{\pi'} \arrow[dash]{d} &
            S/(S \cap N) \\
            & S \cap N
        \end{tikzcd}
    \end{center}
\end{theorem}
\begin{proof}
    We can see that for all \(s_i n_i \in SN\),
    \(s_1 n_1 s_2 n_2 = s_1 s_2 s_2^{-1} n_1 s_2 n_2 \in SN\),
    since \(s_2^{-1} n_1 s_2 \in N\) by definition of normal,
    and we have closure.
    Associativity is inherited from \(G\),
    and the identity \(1\) is in both \(S\) and \(N\).
    Lastly, \({(sn)}^{-1} = n^{-1}s^{-1} = s^{-1}sn^{-1}s^{-1} \in SN\),
    since \(sn^{-1}s^{-1} \in SN\) by definiton of normal,
    which gives us inverse.
    Therefore \(SN \subseteq G\) forms a group,
    giving us statement (a).

    \medskip

    Now, clearly \(N \subseteq SN\),
    because every element \(n \in N\) multiplied by \(1 \in S\)
    gives us an element of \(SN\).
    To prove normality,
    we need to prove that for all \(s \in S\) and \(n_i \in N\),
    \(sn_1 n_2 {(sn_1)}^{-1} \in N\).
    This is easy because \(sn_1 n_2 {(sn_1)}^{-1}
    = s(n_1 n_2 n_1^{-1})s^{-1} \in N\)
    by definition of \(N \lhd G\).
    Hence \(N \lhd SN\), proving statement (b).

    \medskip

    Also, clearly \(S \cap N \subseteq S\),
    because by definition \(g \in S \cap N\)
    implies \(g \in S\) (and \(g \in N\)).
    Now it is sufficient to prove that
    for all \(g \in S \cap N\) and \(s \in S\),
    \(sgs^{-1} \in S \cap N\).
    It is clear that since both \(g,s\) are elements of \(S\),
    \(sgs^{-1} \in S\).
    Then since \(g \in N\), and \(s \in S \subseteq G\),
    we have \(sgs^{-1} \in N\) since \(N \lhd G\).
    Hence \(sgs^{-1} \in S \cap N\),
    so we have \(S \cap N \lhd S\),
    proving statement (c).

    \medskip

    We now attempt to construct a homomorphism
    \(\vfunc{\phi}{S}{SN/N}{s}{sN}\).
    % \(\vfunc{\phi}{SN/N}{S/(S \cap N)}{sN}{s(S \cap N)}\).
    % Obviously the kernel is \(\ker(\phi) = N\),
    % because to map to 1, \(s = 1\),
    % which implies \(sn = n \in N\);
    % conversely, when \(s \neq 1\), then \(sn \notin N\),
    % and \(\phi(sn) = \phi(s) = s \neq 1\).
    We demonstrate that this is a valid homomorphism,
    by showing that
    \begin{equation*}
        \phi(s_1 s_2) = (s_1 s_2)N = (s_1 N)(s_2 N) = \phi(s_1)\phi(s_2)
    \end{equation*}
    since \(s_1,s_2\) are elements of \(G\),
    so same logic as the quotient subgroup epimorphism applies.

    We now want to show that \(\phi\) is surjective.
    The elements \(sn \in SN\) must belong in some coset \(snN\),
    which we can see is equivalent to \(sN\).
    By definition, \(\phi\) maps \(s \mapsto sN\),
    so every coset is covered by \(\phi\),
    and therefore it is an epimorphism.

    We can then demonstrate that \(\ker(\phi) = S \cap N\).
    We can see that if \(g \in S \cap N\),
    then \(g \in N\), so \(\phi(g) = gN = N\),
    which gives us \(S \cap N \subseteq \ker(\phi)\).
    On the other hand, if \(g \in \ker(\phi)\),
    then \(\phi(g) = gN \subseteq N\),
    which requires \(g \in N\),
    giving us \(\ker(\phi) \subseteq S \cap N\).
    Hence \(\ker(\phi) = S \cap N\).

    Lastly, we can apply the
    \hyperref[thm:iso-1-group]{first isomorphism theorem},
    and prove that there exists a unique isomorphism
    between \(S/(S \cap N) \cong SN/N\).
\end{proof}

\begin{theorem}[Third Isomorphism Theorem for Groups]\label{thm:iso-3-group}
    Suppose \(G\) is a group, \(N \lhd G\) a normal subgroup.
    Then:
    % Then we have:
    \begin{enumerate}[label={(\alph*)}, itemsep=0mm]
        \item if \(K\) is a subgroup such that \(N \subseteq K \subseteq G\),
            then \(K/N \subseteq G/N\) is a subgroup;
        \item a subgroup of \(G/N\) must be of the form \(K/N\)
            such that \(K\) is a subgroup with \(N \subseteq K \subseteq G\);
        \item if \(K\) is a normal subgroup such that \(N \subseteq K \subseteq G\),
            then \(K/N \lhd G/N\) is a normal subgroup;
        \item a normal subgroup of \(G/N\) must be of the form \(K/N\)
            where \(K \lhd G\) is a normal subgroup with \(N \subseteq K \subseteq G\); and
        \item if \(K \lhd G\) is a normal subgroup such that \(N \subseteq K \subseteq G\),
            then \((G/N)/(K/N) \cong G/K\).
    \end{enumerate}

    This is represented by the following commutative diagrams:
    \begin{center}
        \begin{tikzcd}
            G \arrow[dash]{d} \arrow{r} & G/N \arrow[dash]{d} \\
            % & G \arrow{d}{\pi'} \arrow{r}{\pi} & G/N \arrow{d}{\pi'} \\
            K \arrow{r} & K/N % & K \arrow{r}{\pi} & K/N
        \end{tikzcd} \qquad
        \begin{tikzcd}
            G \arrow{r} \arrow{d} \arrow{rd} &
            G/K \arrow[leftrightarrow]{rd}{\cong} \\
            K \arrow{rd} & G/N \arrow{r} \arrow{d} & (G/N)/(K/N) \\
            & K/N
        \end{tikzcd}
    \end{center}
\end{theorem}
\begin{proof}
    We first have to prove that \(N \lhd K\),
    which is obvious because by definition of normality,
    for all \(n \in N\) and \(g \in G\),
    \(gng^{-1} \in N\), which can be restricted to \(g \in K \subseteq G\).
    % We will first demonstrate that \(N \lhd K\).
    % For some \(k \in K \subseteq G\) and \(n \in N\),
    % \(knk^{-1} \in N\) by normality of \(N\) in \(G\),
    % since we can treat \(k\) as elements of \(G\).
    % Hence all three \(G/N, K/N, G/K\) are valid quotient groups.

    It is now easy to see that
    with the quotient homomorphism \(\func{\pi}{G}{G/N}\),
    the image of the subgroup \(\pi(K) = K/N\),
    so by the \hyperref[thm:iso-1-group]{first isomorphism theorem}
    \(K/N\) forms a subgroup.
    This proves statement (a).

    % Secondly, we can see that \(K/N \lhd G/N\),
    % because for some arbitrary cosets \(kN \in K/N\) and \(gN \in G/N\),
    % there are \(n_i \in N\)
    % such that the coset representatives \(k' = kn_1\) and \(g' = gn_2\),
    % \((gN)(kN){(gN)}^{-1} = (NgkN)\)

    \medskip

    Suppose \(K' \subseteq G/N\) is a subgroup.
    We can look at the preimage \(\pi^{-1}(K')\),
    which since \(1 \in K'\), we have \(K' \supseteq \ker(\pi) = N\).
    Notice that the preimage of a group is still a group:
    suppose \(\func{\phi}{G'}{H'}\) is a homomorphism,
    since \(\{x,y\}\subseteq\phi^{-1}(H')\)
    implies \(\phi(xy) = \phi(x)\phi(y) \in H'\) (closure),
    \(1 \in \ker(\phi) \subseteq \phi^{-1}(H')\) (identity),
    associativity inherited from \(G'\),
    and \(x \in \pi^{-1}(H')\)
    implies \({\pi(x)}^{-1} = \pi(x^{-1}) \in H'\) (inverse).
    This proves statement (b).

    \medskip

    It is now sufficient to prove the normality condition.
    % If we have \(k \in K\) and \(g \in G\) such that \(gkg^{-1} \in K\),
    % then under the quotient homomorphism \(\func{\pi}{G}{G/N}\)
    % \(\pi(gkg^{-1}) = \pi(g)\pi(k)\pi(g)^{-1}\)
    \(K \lhd G\) tells us that for all \(g \in G\), \(gKg^{-1} \subseteq K\),
    which under mapping is \(\pi(gKg^{-1}) = \pi(g)\pi(K){\pi(g)}^{-1}\).
    Notice that by Theorem~\ref{thm:equal-coset-normal},
    \(\pi(g)\pi(K) = \pi(K)\pi(g)\),
    which allows us to cancel out the \(g\) and its inverse,
    letting us conclude that \(\pi(gKg^{-1}) \subseteq \pi(K)\),
    giving us \(\pi(K) = K/N \lhd G/N\).
    This proves statement (c).

    \medskip

    Similarly, assume \(K' \lhd G/N\);
    then for all \(x \in G/N\), \(xK'x^{-1} \subseteq K'\).
    No matter what element \(g \in G\), \(k \in \pi^{-1}(K')\) we choose,
    we have \(\pi(gkg^{-1}) = \pi(g)\pi(k){\pi(g)}^{-1} \in K'\)
    by definition of normality in \(G/N\),
    so \(ghg^{-1} \in \pi^{-1}(K')\),
    and the preimage is normal.
    This proves statement (d).

    \medskip
    
    We can now attempt to construct a homomorphism
    \(\vfunc{\phi}{G/N}{G/K}{gN}{gK}\).
    % We demonstrate that this is a valid homomorphism, as
    % first by showing that all elements in the same \(N\)-coset
    % will be mapped to the same \(K\)-coset;
    % observe that if \(g' = gn\),
    % then \(\phi(g'N) = g'K = gnK = gK = \phi(gN)\)
    % since \(n \in N \subseteq K\).
    % Moreover, we have
    % \begin{equation*}
    %     \phi(g_1 g_2 N) = g_1 g_2 K = (g_1 g_2 K)K = (K g_1 g_2)K
    %     = (g_1 K)(g_2 K) = \phi(g_1 N)\phi(g_2 N)
    % \end{equation*}
    This is valid because
    by the \hyperref[thm:univ-prop-quotient-group]{universal property},
    we have \(\func{\pi}{G}{G/N}\) and \(\func{\eta}{G}{G/K}\),
    so there is a unique homomorphism that makes \(\eta = \phi\circ\pi\).
    We can also show that this is surjective,
    since the \(K\)-cosets partition \(G\),
    so each \(K\)-coset must have some element \(gK\)
    that represents it,
    and clearly this element \(gN\) in the \(N\)-cosets
    must get sent to it,
    alongside all other elements in that \(N\)-coset.

    We claim the kernel is \(\ker(\phi) = K/N\).
    % To see this, first observe \(kN \in K/N\)
    % must get mapped to \(\phi(kN) = kK = K\),
    % which is the coset of the identity,
    % giving us \(K/N \in \ker(\phi)\).
    % For the reverse argument,
    % for the image to be the identity coset \(K\),
    % the element \(g\) must belong in \(K\),
    % which means the kernel must be the cosets of the elements in \(K\),
    % otherwise written as \(\ker(\phi) \subseteq K/N\).
    % This proves our claim.
    Observe that \(\ker(\eta) = K\) and \(\ker(\pi) = N\),
    so \(\phi\) must map all the \(N\)-cosets
    that are represented by elements of \(K\) into \(1\).

    Lastly, we invoke the \hyperref[thm:iso-1-group]{first isomorphism theorem},
    which gives us \(\ker(\phi) = K/N \lhd G/N\),
    making our quotient \((G/N)/(K/N)\) valid;
    and also that \((G/N)/\ker(\phi) = (G/N)/(K/N) \cong G/K\),
    proving statement (e).
\end{proof}

\begin{theorem}[Fourth Isomorphism Theorem for Groups]\label{thm:iso-4-group}
    Suppose \(G\) is a group,
    \(N \lhd G\) some normal subgroup,
    and \(\vfunc{\pi}{G}{G/N}{g}{gN}\) the quotient homomorphism.
    Then \(\pi\) is a bijection
    between the subgroups of \(G/N\)
    and the subgroups of \(G\) containing \(N\);
    and is also a bijection between normal subgroups of \(G/N\)
    and normal subgroups of \(G\) containing \(N\).
\end{theorem}
\begin{proof}
    % We can split this theorem into 4 statments:
    % first that subgroups of \(G\)
    % First suppose \(KN \subseteq H \subseteq G\) for some subgroup \(H\);
    % then clearly \(\pi(H) \subseteq G/N\) is a subgroup.
    % Now suppose \(H' \subseteq G/N\) is a subgroup;
    % since \(1 \in H'\), the preimage \(\pi^{-1}(H')\) must contain \(N\),
    % and notice that the preimage of a group under homomorphism is still a group,
    % elements of \(H'\) must be cosets \(hN\),
    % where \(h \in H \subseteq G\) forms a group
    This is merely a corollary of
    the \hyperref[thm:iso-3-group]{third isomorphism theorem}.
    Statements (a) and (b) prove the correspondence between subgroups,
    while statements (c) and (d)
    prove the correspondence between normal subgroups.

    % Then suppose \(N \lhd H \lhd G\) for some normal subgroup \(H\);
    % this tells us that for all \(g \in G\), \(gHg^{-1} \subseteq H\),
    % which under mapping is \(\pi(gHg^{-1}) = \pi(g)\pi(H){\pi(g)}^{-1}\).
    % Notice that \(\pi(g)\pi(H) = \pi(H)\pi(g)\),
    % which allows us to cancel out the \(g\) and its inverse,
    % letting us conclude that \(\pi(gHg^{-1}) \subseteq \pi(H)\),
    % giving us \(\pi(H) \lhd G/N\).
    % Now in the other direction, 

    % Lastly, we wish to prove that \(G/H \cong (G/N)/\pi(H)\).
    % But this is exactly the conclusion of
    % the \hyperref[thm:iso-3-group]{third isomorphism theorem}.
\end{proof}


\subsection{Action on Sets}

\begin{definition}
    Suppose \(X\) is a set of \(n\) elements.
    The group of all possible permutations
    (guaranteed by Proposition~\ref{prop:symmetric-group})
    is denoted \(S_X\),
    which is of course isomorphic to \(S_n\).
\end{definition}

\begin{definition}
    Suppose \(G\) a group, and \(X\) some set.
    The action of \(G\) on \(X\) is a homomorphism \(\func{\phi}{G}{S_X}\).
\end{definition}
\begin{remark}
    Notice that by \hyperref[thm:cayley]{Cayley's theorem},
    every finite group is a transformation group on itself.
    With this knowledge,
    we can formally say that we can let \(G\) act on itself
    by left multiplication,
    where the homomorphism is \(\vfunc{\phi}{G}{S_G}{g}{\ell_g}\),
    and \(\vfunc{\ell_g}{G}{G}{\alpha}{g\alpha}\).
    Beware that when considering actions,
    the homomorphism \(\phi(g)\) outputs a function
    on the set \(X\) that we are acting on,
    so in general \(\phi(g)(x) = \phi_g(x)\) are both valid notation,
    the former focusing on that we have a homomorphism \(\phi(g)\),
    and the latter focusing on that we have a function \(\phi_g\).
\end{remark}

\begin{proposition}[Action by Conjugation]
    Suppose \(G\) is a group.
    \(G\) can act on itself by conjugation,
    that is, \(\vfunc{\phi}{G}{S_G}{g}{\gamma_g}\)
    where \(\vfunc{\gamma_g}{G}{G}{\alpha}{g \alpha g^{-1}}\).
\end{proposition}
\begin{proof}
    We need to first prove that \(\gamma_g\) is a bijection.
    Suppose \(\{\alpha,\beta\} \subset G\),
    \(g \alpha g^{-1} = g \beta g^{-1}\);
    then \(\alpha = g^{-1}g \alpha g^{-1}g
    = g^{-1}g \beta g^{-1}g = \beta\),
    which implies \(\gamma_g\) is an injection.
    Now, clearly, for every element \(x \in G\),
    we can map \(\gamma_g(g^{-1}xg) = gg^{-1}xgg^{-1} = x\),
    which implies \(\gamma_g\) is a surjection.
    Hence \(\gamma_g\) is a bijection.

    We now then prove that \(\phi\) is a homomorphism.
    Clearly \(\gamma_1\) is the identity mapping
    \(\alpha \mapsto 1\alpha 1^{-1} = \alpha\).
    \begin{equation*}
        \gamma_{gh}(\alpha) = (gh)\alpha{(gh)}^{-1}
        = g h\alpha h^{-1}g^{-1} = \gamma_g(h \alpha h^{-1})
        = \gamma_g(\gamma_h(\alpha))
    \end{equation*}
    We have showed that
    \(\phi(gh) = \gamma_{gh} = \gamma_g \circ \gamma_h = \phi(g)\phi(h)\),
    which proves homomorphism.
\end{proof}

\begin{definition}
    Suppose \(G\) is a group that acts on some set \(X\)
    via the homomorphism \(\func{\phi}{G}{S_X}\).
    The stabilizer of some element \(x \in X\)
    is the set of all actions that fix \(x\),
    \(\stab(x) = \{g \in G : \phi_g(x) = x\}\).
\end{definition}
\begin{proposition}\label{prop:stabilizer-subgroup}
    The stabilizer \(\stab(x)\) is a subgroup of \(G\).
\end{proposition}
\begin{proof}
    Suppose our action is \(\vfunc{\phi}{G}{S_G}{g}{\phi_g}\),
    where \(\func{\phi_g}{G}{G}\) is a permutation.
    Then the stabilizer is a set of all \(g\)
    such that \(\phi_g(x) = x\).
    Suppose \(\phi_g(x) = \phi_h(x) = x\);
    then \(\phi_{gh}(x) = \phi_g(x)\phi_h(x) = x\),
    so the stabilizer is closed under multiplication.
    Associativity is given to us for free with function composition,
    and \(\phi_1(x) = x\) is trivially the identity mapping.
    Lastly if \(\phi_g(x) = x\), we know that
    \(x = \phi_1(x) = \phi_g(x)\phi_{g^{-1}}(x) = \phi_{g^{-1}}(x)\)
    which gives us an inverse.
\end{proof}

\begin{definition}
    Suppose \(G\) is a group
    and \(x \in G\) some element in the group.
    The centralizer of the element
    is the set of all elements in \(G\) that commute with it,
    \(\Centralizer(x) = \{g \in G : gx = xg\}\).
    Similarly, the centralizer of a subset \(S \subseteq G\)
    is the set of all elements in \(G\)
    that commute with every element in \(S\),
    \(\Centralizer(S) = \Centralizer_G(S)
    = \{g \in G : \forall x \in S, gx = xg\}\).
\end{definition}
\begin{proposition}\label{prop:centralizer-stabilizer}
    Suppose \(G\) acts on itself by conjugation.
    Then for any \(x \in G\), \(\Centralizer(x) = \stab(x)\).
\end{proposition}
\begin{proof}
    \(\stab(x) = \{g \in G : \gamma_g(x) = x\}
    = \{g \in G : gxg^{-1} = x\} = \{g \in G : gx = xg\}
    = \Centralizer(x)\).
\end{proof}
\begin{corollary}\label{cor:centralizer-subgroup}
    The centralizer \(\Centralizer(S) \subseteq G\) is a subgroup.
\end{corollary}
\begin{proof}
    We first observe that
    \(\Centralizer(S) = \bigcap_{s \in S} \Centralizer(s)\),
    because it should include all elements
    that commute with every element of \(S\).
    Then we know from the
    \hyperref[prop:centralizer-stabilizer]{proposition above}
    that \(\Centralizer(s) = \stab(s)\),
    which are subgroups by Proposition~\ref{prop:stabilizer-subgroup}.
    Lastly, from Lemma~\ref{lem:intersection-subgroup} we know
    the intersection of subgroups is a subgroup.
\end{proof}

\begin{definition}
    We sometimes call the centralizer of the entire group \(\Centralizer(G)\)
    the center of the group \(\Centre(G)\).
    It consists of the elements that commute with everything in the group.
\end{definition}
\begin{proposition}z
    Suppose \(G\) acts on itself by conjugation.
    Then the kernel of the homomorphism is the centralizer of the group,
    \(\ker(\phi) = \Centre(G)\).
\end{proposition}
\begin{proof}
    \(\ker(\phi) = \{g \in G : \phi(g) = 1\}
    = \{g \in G : \gamma_g = \gamma_1\}
    = \{g \in G : \forall x \in G, gxg^{-1} = x\}
    = \Centre(G)\).
\end{proof}
\begin{remark}
    The centralizer tells us what and how many elements of \(G\)
    commute with everything,
    so in a sense, self-action by conjugation
    allows us to `measure' the noncommutativity of \(G\).
\end{remark}

\begin{definition}
    Suppose \(G\) is a group
    and \(S \subseteq G\) some subset of the group.
    The normalizer of the subset
    is the set of elements that fix \(S\) under conjugation,
    i.e.\ conjugation by the normalizer sends elements of \(S\)
    to (potentially other) elements of \(S\).
    This is denoted \(\Normalizer(S) = \Normalizer_G(S)
    = \{g \in G : gSg^{-1} = S\}\).
\end{definition}
\begin{proposition}\label{prop:normalizer-subgroup}
    The normalizer \(\Normalizer(S) \subseteq G\) is a subgroup.
\end{proposition}
\begin{proof}
    We clearly have closure with
    \((gh)S{(gh)}^{-1} = ghSh^{-1}g^{-1} = gSg^{-1} = S\).
    Associativity and identity is given to us,
    as per usual, for free.
    Since \(gSg^{-1} = S\),
    then \(S = g^{-1}gSg^{-1}g = g^{-1}Sg\),
    and we get inverse.
\end{proof}

\begin{remark}
    Recall from the definition of cosets and orbits
    in Section~\ref{sec:cosets}
    that the actions of \(G\) form orbits,
    which are equivalence classes demonstrating that
    there exists an action \(g \in G\)
    such that \(\phi_g(x) = y\)
    for all \(\{x,y\} \subset X\) in the same orbit.
    If we allows \(G\) to act on itself,
    these now define equivalence classes on \(G\) itself,
    which allow us to deduce facts about subgroups of \(G\).
    In particular, under self-action,
    the set of orbits \(G/{\sim}\) partition \(G\).
\end{remark}
\begin{definition}
    Suppose \(G\) acts on a set \(X\) by conjugation.
    The orbits of \(X\) are equivalence classes,
    which we call conjugacy classes,
    sometimes denoted \([x] = \{\gamma_g(x) = gxg^{-1} : g \in G\}\).
\end{definition}

\begin{definition}
    Suppose \(G\) acts on two sets \(X\) and \(Y\)
    via \(\vfunc{\phi}{G}{S_X}{g}{\phi_g}\)
    and \(\vfunc{\psi}{G}{S_Y}{g}{\psi_g}\) respectively.
    We say \(X\) is equivalent to \(Y\) under \(G\)-action
    if there exists a bijection \(\func{f}{X}{Y}\)
    such that \(f(\phi_g(x)) = \psi_g(f(x))\)
    for all \(x \in X\) and \(g \in G\).

    This can be summarized by the following commutative diagram:
    \begin{center}
        \begin{tikzcd}
            X \arrow{r}{f} \arrow{d}{\phi_g} & Y \arrow{d}{\psi_g} \\
            X \arrow{r}{f} & Y
        \end{tikzcd}
    \end{center}
\end{definition}
\begin{lemma}\label{lem:transitive-orbit-subgroup}
    Suppose \(G\) acts on \(X\) transitively (there is only one orbit).
    Then there exists a subgroup \(H \subseteq G\)
    such that \(X\) is equivalent to \(G/H\).
\end{lemma}
\begin{proof}
    Suppose the action of \(G\) on \(X\)
    is \(\vfunc{\phi}{G}{S_X}{g}{\phi_g}\).
    To prove such an equivalence,
    first we have to construct \(H\).
    Choosing some arbitrary \(x \in X\),
    let us find its stabilizer \(H = \stab(x) = \{g \in G : \phi_g(x) = x\}\),
    which by Proposition~\ref{prop:stabilizer-subgroup} we know is a subgroup.
    % We claim that \(H \lhd G\).
    % If \(\phi_h(x) = x\),
    % we see that \(\phi_{ghg^{-1}} = \phi_g\phi_h\phi_{g^{-1}}\),
    % so we have \(\phi_{ghg^{-1}}(\phi_g(x)) = \phi_g(x)\),
    % which stabilizes some arbitrary element \(\phi_g(x)\),
    % and hence fixes all elements \(x \in X\) as \(\phi_g\) is a bijection,
    % giving us normality.

    Now, suppose \(G\) acts on \(G/H\) by left multiplication,
    so that we have \(\vfunc{\psi}{G}{S_{G/H}}{g}{\ell_g}\),
    where \(\vfunc{\ell_g}{G/H}{G/H}{rH}{grH}\).
    Note that \(G/H\) here is not a quotient group,
    but rather the set of all left cosets.
    We claim that the function \(\vfunc{f}{X}{G/H}{\phi_g(x)}{gH}\)
    forms a bijection.
    We first see that this is a completely valid definition,
    because the action is transitive, so \(\phi_g(x)\) covers all of \(X\).
    We then have to check that if \(\phi_{g_1}(x) = \phi_{g_2}(x)\),
    \(g_1H = g_2H\);
    this is obvious because \(\phi_{g_2}^{-1}\circ\phi_{g_1}(x) = x\),
    and \(g_2^{-1} g_1 \in H\), so \(g_2 \in g_1 H\),
    and without loss of generality we also have \(g_1 \in g_2 H\).
    \(f\) is injective because if \(g_1 H = g_2 H\),
    then \((g_2^{-1}g_1)H = H\),
    so \(\phi_{g_2^{-1}g_1}(x) = \phi_{g_2}^{-1}\circ\phi_{g_1}(x) = x\),
    and hence \(\phi_{g_1}(x) = \phi_{g_2}(x)\).
    \(f\) is surjective also because for every \(gH\),
    at least \(gx\) will map to it.
    Therefore, \(f\) is a bijection.

    We will now check the equivalence under multiplication.
    For some arbitrary \(r \in G\),
    and hence some arbitrary \(\phi_r(x) \in X\),
    \begin{equation*}
        \ell_g(f(\phi_r(x))) = \ell_g(rH) = grH
        = f(\phi_{gr}(x)) = f(\phi_g(\phi_r(x)))
    \end{equation*}
    We can conclude that there is equivalency between \(X\) and \(G/H\).
\end{proof}
\begin{corollary}[Orbit-Stabilizer Theorem]\label{cor:orbit-stabilizer}
    Suppose \(G\) is a group that acts on a finite set \(X\).
    Let \(x \in X\) denote an arbitrary element.
    Then we have an equivalence between the size of the orbit of \(x\)
    and the index of the stabilizer in \(G\),
    that is, \(\abs{\orb(x)} = [G:\stab(x)]\).
\end{corollary}
\begin{proof}
    We shall restrict the action on \(G\) on \(X\)
    to only acting onto the subset \(\orb(x) \subseteq X\).
    This new action is now by definition transitive.
    The \hyperref[lem:transitive-orbit-subgroup]{lemma above}
    asserts an equivalence between \(\orb(x)\) and \(G/\stab(x)\),
    that is, a bijection between elements of these two sets.
    Hence we can equate the cardinality of these two sets.
\end{proof}
\begin{corollary}\label{cor:transitive-cardinality}
    If \(G\) acts on \(X\) transitively,
    then \(\abs{X} \mid \abs{G}\).
\end{corollary}
\begin{proof}
    From the \hyperref[lem:transitive-orbit-subgroup]{lemma above},
    since there is a bijection between \(X\) and \(G/H\)
    we know that \(\abs{X} = \abs{G/H}\).
    But \hyperref[thm:lagrange]{Lagrange's theorem}
    guarantees that the number of cosets is \(\abs{G/H} = [G:H] \mid \abs{G}\).
\end{proof}

\begin{theorem}[Class Equation]\label{thm:class-equation}
    Suppose \(G\) acts on \(X\).
    Then \(\abs{X} = \sum_i \abs{X_i} = \sum_i \abs{G/H_i}\),
    where \(H_i \subseteq G\) subgroups,
    and hence \(\abs{G/H_i} \mid \abs{G}\).
\end{theorem}
\begin{proof}
    Suppose the orbits of \(X\) are disjoint sets \(X_i\).
    Then if we restrict to any subset \(X_i \subseteq X\),
    \(G\) acts on \(X_i\) transitively.
    By the \hyperref[lem:transitive-orbit-subgroup]{lemma above},
    \(X_i\) is equivalent to \(G/H_i\),
    so \(\abs{X_i} = \abs{G/H_i}\).
    Now, since orbits partition \(X\),
    \(\abs{X} = \sum \abs{X_i}\).
\end{proof}
\begin{corollary}
    \(\abs{G} = \abs{\Centre(G)} + \sum_i n_i\)
    where \(n_i \mid \abs{G}\).
\end{corollary}
\begin{proof}
    Let \(G\) act on itself by conjugation.
    For the elements \(x \in G\) that get fixed by conjugation,
    i.e.\ they commute with all \(g \in G\),
    their orbits (conjugacy classes) are a single element
    \([x] = \{\gamma_g(x) = gxg^{-1} : g \in G\} = [x] = \{x\}\).
    Together, all these elements form the center \(\Centre(G)\).

    Then all other conjugacy classes (with more than one element)
    are equivalent to \(G/H_i\)
    by the \hyperref[thm:class-equation]{class equation},
    which are divisors of \(\abs{G}\).
\end{proof}

\begin{theorem}\label{thm:pr-nontrivial-center}
    Suppose \(G\) is a finite group,
    with order \(\abs{G} = p^r\) where \(p\) is prime.
    Then \(G\) has a nontrivial center,
    that is, \(\Centre(G) \neq \{1\}\).
\end{theorem}
\begin{proof}
    Suppose we have a trivial center.
    Then the \hyperref[thm:class-equation]{class equation} tells us that
    \(p^r = 1 + \sum_i p^{r_i}\),
    since the order is \(\abs{G} = p^r\), \(\abs{\Centre(G)} = 1\),
    and all divisors of \(p^r\) must be some power of \(p\),
    with \(r_i \neq 0\), as those orbits cannot be trivial.
    But clearly the left side is a multiple of \(p\),
    and the right side has a multiple of \(p\) plus 1,
    which is a contradiction.
\end{proof}
\begin{corollary}\label{cor:p2-abelian}
    Suppose \(G\) is a group with order \(\abs{G} = p^2\),
    where \(p\) is prime.
    Then \(G\) is abelian.
\end{corollary}
\begin{proof}
    We know from the \hyperref[thm:class-equation]{theorem above}
    that it has a nontrivial center (\(\abs{\Centre(G)} \neq 1\)),
    so if the order of some element is \(p^2\),
    then it generates the entire group, and it must be cyclic (abelian).
    % But since the center is a subgroup
    % (Corollary~\ref{cor:centralizer-subgroup}),
    % its order must divide \(\abs{G} = p^2\) (Theorem~\ref{thm:lagrange}),
    % i.e.\ \(\abs{\Centre(G)} = p\) or \(p^2\).

    % Suppose, by way of contradiction, that \(\abs{\Centre(G)} = p\).
    % Then we can clearly see that all elements in their orbits
    Now suppose there are no elements of order \(p^2\);
    then all elements other than the identity must have order \(p\),
    which implies the center must also be a subgroup
    (Corollary~\ref{cor:centralizer-subgroup})
    that is cyclic with order \(p\) (Theorem~\ref{thm:lagrange}).
    % Let \(H = \Centre(G)\).
    % We write down the number of cosets that we get,
    % that is, \([G:H] = \abs{G}/\abs{H} = p^2/p = p\).
    Let \(\Centre(G) = \langle g \rangle\), where \(\abs{g} = p\).
    Clearly there is another element \(k \in G\), but \(k \neq g^r\)
    such that \(\abs{k} = p\).
    If \(gk = kg\), then a simple counting argument gives us
    \(\langle g,k \rangle\) a group of order \(p^2\),
    since every element can be written in the form \(g^i k^j\),
    \(i,j\) each having \(p\) choices each.
    If on the other hand, \(gk \neq kg\),
    then \(\langle g,k \rangle\) must include
    all previously mentioned \(p^2\) elements, and also \(kg\),
    which implies \(\abs{\langle g,k \rangle} > p^2\),
    and contradicts the fact that \(\langle g,k \rangle \subseteq G\).
    Hence \(gk = kg\),
    and any power of \(g\) commutes with any power of \(k\),
    so all elements of \(G\) commute with each other.
\end{proof}

\begin{theorem}
    Conjugacy classes in \(S_n\) are the cycle types;
    same cycle types are in the same conjugacy class,
    and different cycle types are in different conjugacy classes.
\end{theorem}
\begin{proof}
    Suppose we have a \(d\)-cycle \((n_1 n_2 \hdots n_d)\).
    We know that for all \(g \in S_n\), \(n_i \in \{1,\hdots,n\}\),
    any \(g(n_1 n_2 \hdots n_d)g^{-1} = (g(n_1) g(n_2) \hdots g(n_d))\)
    by Proposition~\ref{prop:sn-conjugation}.
    This is also a \(d\)-cycle.
    As all elements of \(S_n\) are disjoint cycles,
    all such disjoint cycle types must only ever map to their own cycle type
    under conjugation.
\end{proof}
\begin{remark}
    In general it is difficult to determine
    exactly the conjugacy classes of a group.
\end{remark}


\subsection{Sylow's Theorems}

\begin{theorem}[Cauchy's Theorem]\label{thm:cauchy}
    Suppose \(G\) is a finite abelian group,
    and there is some prime \(p \mid \abs{G}\)
    that divides the order of the group.
    Then there exists an element \(g \in G\) of order \(\abs{g} = p\).
\end{theorem}
\begin{proof}
    We first consider the base case where \(\abs{G} = p\).
    By Corollary~\ref{cor:prime-order-subgroup},
    we know that there exists a subgroup \(\langle g \rangle \subseteq G\)
    where \(g \in G\) is any arbitrary element \(g \neq 1\).

    We now proceed by strong induction on the order of the group.
    Suppose \(\abs{G} = mp\), where \(m \in \bN\),
    and that for all \(k < m\) Cauchy's Theorem for order \(kp\) is proven.
    Let us arbitrarily pick some \(x \neq 1\), \(x \in G\).
    Suppose the order of this element is some \(\abs{x} = n\).
    If \(p \mid n\), then Theorem~\ref{thm:cyclic-subgroup-uniqueness}
    guarantees us that there exists a subgroup
    \(\langle x^m \rangle \subseteq \langle x \rangle\)
    with order \(\abs{x^m} = p\).
    If \(p \nmid n\), then we can construct a subgroup \(G/\langle x \rangle\)
    (\(\langle x \rangle\) guaranteed to be normal
    by Proposition~\ref{prop:abelian-subgroup-normal}),
    which by \hyperref[thm:lagrange]{Lagrange's theorem}
    has order \(\abs{G/\langle x \rangle} = \abs{G}/\abs{x} = mp/n\),
    and \hyperref[lem:euclid]{Euclid's lemma} tells us \(p \mid mp/n\).
    Then clearly \(m/n \in \bN\), specifically \(m/n < m\),
    so the induction hypothesis applies.
\end{proof}

\begin{theorem}[Sylow's First Theorem]\label{thm:sylow-1}
    Suppose \(G\) a finite group, \(p\) a prime number,
    and that the order is \(\abs{G} = Np^s\),
    with \(\gcd(N,p) = 1\) coprime
    (that is, \(p^s\) is the largest power of \(p\) that divides \(\abs{G}\)).
    Then for all \(0 \leq t \leq s\),
    \(G\) contains a subgroup of order \(p^t\).
\end{theorem}
\begin{proof}
    We first consider the case where \(G\) is abelian.
    % The most simple case is when \(\abs{G} = Np\);
    % there trivially exists a subgroup \(\{1\}\) of order 1,
    % and by \hyperref[thm:cauchy]{Cauchy's theorem}
    % there exists a subgroup \(\langle g \rangle\) of order \(p\).
    Trivially, \(G\) itself is a subgroup of order \(\abs{G} = Np^s\),
    which we will use as our base case.

    We now suppose, by way of weak induction,
    that we have a subgroup \(H_k\) of order \(\abs{H_k} = Np^k\)
    (for \(k \geq 1\)).
    We wish to prove that a subgroup \(H_{k-1}\)
    of order \(\abs{H_{k-1}} = Np^{k-1}\) exists.
    By \hyperref[thm:cauchy]{Cauchy's theorem},
    we can find an \(x_k \in H_k\) of order \(\abs{x_k} = p\).
    Proposition~\ref{prop:abelian-subgroup-normal}
    tells us that \(\langle x_k \rangle \lhd H_k\),
    so \(H_k/\langle x_k \rangle\) is a subgroup,
    with \hyperref[thm:lagrange]{Lagrange's theorem} telling us
    its order is \(\abs{H_k/\langle x_k \rangle}
    = \abs{H_k}/\abs{x_k} = Np^k/p = Np^{k-1}\).
    This allows us to conclude that there are subgroups of \(G\)
    with orders \(Np^r\) where \(0 \leq r \leq s\).

    Again invoking Proposition~\ref{prop:abelian-subgroup-normal},
    \(H_{s-t} \lhd G\) for all \(0 \leq t \leq s\).
    Hence \(G/H_{s-t}\) is a subgroup,
    with its order determined by \hyperref[thm:lagrange]{Lagrange's theorem}
    to be \(\abs{G/H_{s-t}} = \abs{G}/\abs{H_{s-t}}
    = (Np^s)/(Np^{s-t}) = p^t\).

    \medskip

    Now consider the case where \(G\) is not abelian.
    Then clearly there are elements that do not commute,
    so \(\Centre(G) \subsetneq G\).
    We can write down the \hyperref[thm:class-equation]{class equation}
    \(\abs{G} = \abs{\Centre(G)} + \sum_i [G:H_i]\),
    where there is at least one \(i\) in the summation.

    We can first prove the simplest case when \(\abs{G} = p^s\).
    It suffices to consider \(G\) not abelian,
    since the abelian case is already proven above.
    The center of the group \(\Centre(G)\)
    is a subgroup (Corollary~\ref{cor:centralizer-subgroup}),
    so its order must divide the order of the whole group
    (\(\abs{\Centre(G)} \mid \abs{G} = p^s\),
    \hyperref[thm:lagrange]{Lagrange's theorem}),
    which implies \(p \mid \abs{\Centre(G)}\).
    Then \hyperref[thm:cauchy]{Cauchy's theorem} tells us that
    we can find an element \(x \in \Centre(G) \subsetneq G\)
    such that \(\abs{x} = p\).
    Proceeding similarly to the induction in the abelian case,
    we see that \(\langle x \rangle \lhd \Centre(G)\)
    (Proposition~\ref{prop:abelian-subgroup-normal}),
    and in fact \(\langle x \rangle \lhd G\),
    simply because \(g \in \Centre(G)\),
    and will commute with every element of \(G\)
    (\(gx^r g^{-1} = x^r gg^{-1} = x^r \in \langle x \rangle\)).
    Hence \(G/\langle x \rangle\) is a subgroup
    with order \(\abs{G}/\abs{x} = p^{s-1}\)
    (\hyperref[thm:lagrange]{Lagrange}).
    Recursively applying this process on the quotient group
    generates subgroups of order \(1,p,p^2,\hdots,p^s\).

    Let us then go back to the general case of \(\abs{G} = Np^s\).
    We shall additionally assume, by way of strong induction,
    that any group of order \(mp^s\), where \(m < N\)
    has subgroups of order \(1,p,p^2,\hdots,p^s\).
    Suppose there exists a subgroup \(H_i\) in the class equation
    such that \(p^s \mid \abs{H_i}\).
    Then clearly \(\abs{H_i} = mp^s\), \(m < N\),
    so the induction hypothesis immediately gives our desired result.

    Suppose otherwise, when \(p^s \nmid \abs{H_i}\);
    we can then see that \(p \mid [G:H_i]\) for all \(i\),
    since \hyperref[thm:lagrange]{Lagrange's theorem}
    guarantees \(Np^s = \abs{G} = \abs{H_i}[G:H_i]\),
    and our supposition tells us that the prime factorization of \(\abs{H_i}\)
    includes at most \(p^{s-1}\),
    and therefore \([G:H_i]\) prime factorizes to include at least one \(p\).
    Then every term in the summation inside the class equation
    is divisible by \(p\),
    and the left side is \(Np^s\), which also is divisible by \(p\),
    so the remaining term,
    the centralizer must also be \(p \mid \abs{\Centre(G)}\).
    But this condition means we can follow the proof
    as in the previous paragraph;
    given some group \(H_k\) of order \(Np^k\),
    we can find an element \(x_k\) of order \(p\)
    such that \(H_{k-1} = H_k/\langle x_k \rangle\)
    is a group of order \(\abs{H_{k-1}} = Np^{k-1}\).
    The natural quotient homomorphism \(\func{\pi_k}{H_k}{H_{k-1}}\)
    has a kernel of order \(p\).
    It is then possible to construct a chain of homomorphisms
    \(\pi_{s-r}\circ\pi_{s-r+1}\circ\cdots\circ\pi_s\),
    which has a kernel of order \(p^r\);
    and as we know
    from the \hyperref[thm:iso-1-group]{first isomorphism theorem}
    the kernel is a group,
    which gets us our desired result.
\end{proof}
\begin{remark}
    This theorem does not hold for order \(d^s\) when \(d\) is not prime.
\end{remark}
\begin{corollary}[Cauchy's Theorem]
    Suppose \(G\) is a finite group, abelian or not,
    and there is a some prime \(p \mid \abs{G}\)
    that divides the order of the group.
    Then there exists an element \(g \in G\) of order \(\abs{g} = p\).
\end{corollary}
\begin{proof}
    Special case of the \hyperref[thm:sylow-1]{theorem above},
    with \(t = 1\).
\end{proof}

\begin{definition}
    A group with prime power order,
    that is, \(\abs{G} = p^r\) for some \(r \in \bN\),
    is called a \(p\)-group.
\end{definition}
\begin{corollary}
    There exists subgroups of all possible orders
    that divide the order of a \(p\)-group.
\end{corollary}
\begin{proof}
    The order of a \(p\)-group is \(p^s\),
    so the \hyperref[thm:sylow-1]{theorem above} applies.
\end{proof}

\begin{definition}
    Suppose \(G\) is a finite group with order \(Np^s\), \(\gcd(N,p) = 1\).
    A subgroup of order \(p^s\) (maximal prime power order)
    is often called a Sylow \(p\)-subgroup of \(G\),
    and less often called a \(p\)-Sylow subgroup of \(G\).
\end{definition}
\begin{lemma}\label{lem:prime-power-sylow-subgroup}
    Suppose \(G\) is a finite group with \(\abs{G} = Np^s\), \(\gcd(N,p) = 1\),
    with \(P \subseteq G\) a Sylow \(p\)-subgroup.
    If \(K \subseteq \Normalizer(P)\) a subgroup, and \(\abs{K} = p^t\),
    then \(K \subseteq P \subseteq \Normalizer(P)\).
\end{lemma}
\begin{proof}
    By definition of a normalizer, if \(g \in \Normalizer(P)\),
    we have \(gPg^{-1} \subseteq P\), so \(P \lhd \Normalizer(P)\).

    Since \(P \subseteq \Normalizer(P) \subseteq G\),
    \hyperref[thm:lagrange]{Lagrange's theorem} gives us
    \(\Normalizer(P) = mp^s\), \(m \mid N\).
    We form a quotient group \(\Normalizer(P)/P\),
    which has an order \(mp^s/p^s = m\), which is coprime to \(p\)
    since \(N\) is already coprime to \(p\).
    The natural quotient homomorphism is
    \(\func{\pi}{\Normalizer(P)}{\Normalizer(P)/P}\).
    We now inspect the image \(\pi(K)\),
    which by the \hyperref[thm:iso-1-group]{first isomorphism theorem}
    is a subgroup of \(\Normalizer(P)/P\).

    Now, as elements of \(k \in K\) have \(k^{p^t} = 1\)
    by Corollary~\ref{cor:order-element-group},
    under the homomorphism we should have \({\pi(k)}^{p^t} = 1\),
    so the order \(\abs{\pi(k)} \mid p^t\).
    But the order also \(\abs{\pi(k)} \mid m\),
    because it should divide the order of the codomain group;
    since \(m,p\) coprime, \(m,p^t\) also coprime,
    so the order must be \(\abs{\pi(k)} = 1\),
    and therefore the image must be \(\pi(K) = \{1\}\).
    Hence we can conclude that \(K \subseteq \ker(\pi) = P\).
\end{proof}
\begin{theorem}[Sylow's Second Theorem]\label{thm:sylow-2}
    Suppose \(G\) is a finite group, \(p\) a prime number,
    and that the order is \(\abs{G} = Np^s\), with \(\gcd(N,p) = 1\).
    Let us denote \(P\) and \(P'\) as arbitrary Sylow \(p\)-subgroups.
    Then the following will hold:
    \begin{enumerate}[label={(\alph*)}, itemsep=0mm]
        \item Any two Sylow \(p\)-subgroups are conjugate,
            that is, if \(\abs{P} = \abs{P'} = p^s\),
            then there exists a \(g \in G\) such that \(gPg^{-1} = P'\);
        \item Let the number of Sylow \(p\)-subgroups be \(n\),
            then \(n \mid [G:P] = N\) and \(n \equiv 1 \pmod{p}\); and
        \item If \(K \subseteq G\) is a subgroup
            with order \(\abs{K} = p^t\), \(0 \leq t \leq s\),
            then \(K \subseteq P\) a subgroup of a Sylow \(p\)-subgroup.
    \end{enumerate}
\end{theorem}
\begin{proof}
    Let \(\Pi\) be the set of all Sylow \(p\)-subgroups,
    that is, the set of all subgroups of order \(p^s\).
    By \hyperref[thm:sylow-1]{Sylow's first theorem},
    this set is not empty \(\Pi \neq \emptyset\).
    Allow us to let \(G\) act on \(\Pi\) by conjugation.

    % To prove the first statement,
    Let us fix a singular Sylow subgroup \(P \in \Pi\).
    We can clearly see that the stabilizer here is also the normalizer,
    since \(\Normalizer(P) = \stab(P) = \{g \in G : gPg^{-1} = P\}\).
    Then the \hyperref[cor:orbit-stabilizer]{orbit-stabilizer theorem}
    gives us that \(\abs{\orb_G(P)} = \abs{G}/\abs{\Normalizer(P)}\).
    By Proposition~\ref{prop:normalizer-subgroup}
    and Lemma~\ref{lem:prime-power-sylow-subgroup}
    we know that \(P \subseteq \Normalizer(P) \subseteq G\),
    so \hyperref[thm:lagrange]{Lagrange's theorem} tells us
    \(p^s \mid \abs{\Normalizer(P)}\),
    and \(\abs{\Normalizer(P)} \mid Np^s\),
    and hence we can write \(\abs{\Normalizer(P)} = mp^s\)
    where \(m \mid N\).
    If we now were to assume that statement (a) is true,
    then \(G\)-conjugation on \(\Pi\) is transitive,
    so \(\orb_G(P) = X\), and we have \(\abs{X} = (Np^s)/(mp^s) = N/m\),
    and therefore the number of Sylow \(p\)-subgroups
    is \(\abs{X} \mid N\),
    proving the first part of statement (b).

    The orbit of \(P\) under \(G\)-conjugation is a set of groups,
    which we will denote \(\Sigma = \orb_G(P) \subseteq \Pi\) below.
    Restricting our action such that \(P\) acts on \(\Sigma\) by conjugation,
    by closure of group multiplication,
    we can see that any \(g \in P\) gives us \(gPg^{-1} = P\),
    so the orbit of \(P\) under \(P\)-conjugation is
    \(\orb_P(P) = \{P\} \subseteq \Sigma\),
    which must only contain itself.
    The \hyperref[thm:class-equation]{class equation} now tells us that
    \(\abs{\Sigma} = \abs{\orb_P(P)} + \sum_i \abs{\orb_P(X_i)}\),
    where \(X_i \subseteq \Sigma\),
    and \(\abs{\orb_P(P)} = 1\) as seen above.

    From our assumptions, we know that the terms in the summation
    must divide the order \(\abs{P}\), so they must be powers of \(p\),
    that is, \(\orb_P(X_i) = p^r\), \(r \geq 0\).
    We claim that the order of those are at least \(p\),
    that is, \(\orb_P(X_i) = p^r\) where \(r \geq 1\).
    Suppose, by way of contradiction,
    that there exists some other \(P' \in \Sigma\), \(P' \neq P\)
    such that \(\orb_P(P') = \{P'\}\).
    Then going back to the definition of an orbit,
    we have for all \(g \in P\), \(gP'g^{-1} = P'\),
    which implies \(g \in \Normalizer(P')\),
    i.e.\ \(P \subseteq \Normalizer(P')\).
    As \(\abs{P} = p^s\),
    we can invoke the \hyperref[lem:prime-power-sylow-subgroup]{above lemma}
    and conclude that \(P \subseteq P' \subseteq \Normalizer(P')\).
    In particular, since \(\abs{P} = \abs{P'}\),
    we can see that \(P = P'\),
    which contradicts our premise,
    implying that the orbits \(\abs{\orb_P(X_i)} > 1\).
    This allows us to say that all terms in the summation
    must be divisible by \(p\),
    allowing us to write the class equation as
    \(\abs{\Sigma} = 1 + \sum_i r_i p \equiv 1 \pmod{p}\);
    again, if we assume statement (a) is true,
    then \(\abs{\Pi} = \abs{\Sigma}\),
    which gives us \(\abs{\Pi} \equiv 1 \pmod{p}\),
    proving the second part of statement (b).

    Going back to statement (a),
    it is sufficient to prove that
    there exists only one orbit under conjugation.
    Suppose, by way of contradiction,
    that there is some Sylow \(p\)-subgroup \(Q \in (\Pi-\Sigma)\).
    Then we will restrict the action by conjugation
    such that \(Q\) acts on \(\Sigma\).
    Following a similar argument as \(P'\) in the previous paragraph,
    we know each individual \(Q\)-orbit has order \(p^r\), \(r \geq 0\).
    We again claim that each orbit is at least order \(p\).
    If not, there exists some \(Q' \in \Sigma\)
    such that \(\orb_{Q}(Q') = \{Q'\}\),
    which implies for all \(g \in Q\), \(gQ'g^{-1} = Q'\),
    giving us \(g \in \Normalizer(Q')\),
    and hence \(Q \subseteq \Normalizer(Q')\).
    Invoking the \hyperref[lem:prime-power-sylow-subgroup]{above lemma},
    we get \(Q \subseteq Q' \subseteq \Normalizer(Q')\),
    which since \(\abs{Q} = \abs{Q'}\), we have \(Q = Q'\),
    contradicting the fact that \(Q' \in \Sigma\),
    but \(Q \notin \Sigma\).
    Hence we are forced to conclude that all \(Q\)-orbits
    have cardinalities that are multiples of \(p\),
    implying \(p \mid \abs{\Sigma}\).
    But that itself contradicts the class equation
    \(\abs{\Sigma} = 1 + \sum_i r_i p \equiv 1 \pmod{p}\),
    which tells us \(p \nmid \abs{\Sigma}\).
    Therefore we are forced to conclude that \(Q\) does not exist,
    and \(\Pi-\Sigma = \emptyset\).
    There is only one \(G\)-orbit,
    so \(G\) acts by conjugation
    on the set of Sylow \(p\)-subgroups transitively,
    which implies we can send any two \(\{P,P'\} \subset \Pi\)
    to each other via conjugation, \(gPg^{-1} = P'\) for some \(g \in G\).
    This proves statement (a).

    Armed with statements (a) and (b),
    we now let \(K\) act on \(\Pi\) via conjugation.
    The \hyperref[thm:class-equation]{class equation} tells us that
    \(\abs{\Pi} = \sum_i \abs{X_i}\),
    where the orbits \(X_i\)
    must have orders \(\abs{X_i} \mid \abs{K} = p^t\),
    so all orbits are powers of \(p\).
    But statement (b) tells us that \(\abs{\Pi} \equiv 1 \pmod{p}\),
    so there must be at least one \(K\)-orbit that is of size 1.
    Suppose \(P \in \Pi\) has \(\abs{\orb_K(P)} = 1\).
    Then we see that for all \(k \in K\), \(kPk^{-1} = P\),
    so \(k \in \Normalizer(P)\), and \(K \subseteq \Normalizer(P)\).
    We can use the \hyperref[lem:prime-power-sylow-subgroup]{lemma above}
    one last time
    to see that \(K \subseteq P \subseteq \Normalizer(P)\).
    This proves statement (c).
\end{proof}

\section{Rings}

\subsection{Basic Definitions}

\begin{definition}
    A ring is a quintuple \((R,+,\cdot,0,1)\),
    where \(R \neq \emptyset\) is a set
    equipped with two operations \(+,\cdot\),
    each with their identity \(0,1\) respectively,
    with the following three properties:
    \begin{enumerate}[label={(\roman*)}, itemsep=0mm]
        \item \((R,+,0)\) forms an abelian group;
        \item \((R^\star,\cdot,1)\),
            where \(R^\star = R - \{0\}\) forms a monoid; and
        \item \(\forall \{a,x,y\} \subset R\),
            the distributive laws \(a(x+y) = ax + ay\)
            and \((x+y)a = xa + ya\) hold.
    \end{enumerate}
\end{definition}
\begin{proposition}
    In a nondegenerate ring \(R\), where \(R \neq \{0\}\), \(0 \neq 1\).
\end{proposition}
\begin{proof}
    For all \(x \in R\), we know \(0 = x + (-x)\).
    Suppose, by way of contradiction, that \(0 = 1\).
    Then we have \(x = 1x = (x+(-x))x = x^2 + (-x^2) = 0\),
    implying that \(R = \{0\}\).
\end{proof}

\begin{definition}
    Suppose \((R,+,\cdot,0,1)\) is a ring.
    Then \(S \subseteq R\) is a subring if:
    \begin{enumerate}[label={(\roman*)}, itemsep=0mm]
        \item \((S,+,0) \subseteq (R,+,0)\) is a subgroup; and
        \item \((S^\star,\cdot,1) \subseteq (R^\star,\cdot,1)\) is a submonoid.
    \end{enumerate}
\end{definition}

\begin{lemma}\label{lem:intersection-subring}
    Suppose \(R\) is a ring, and \(S_i \subseteq R\) subrings.
    Then \(\bigcap_i S_i\) is also a subring of \(R\)
\end{lemma}
\begin{proof}
    Lemma~\ref{lem:intersection-subgroup} gives us that
    the intersection of additive groups are subgroups,
    and the intersection of multiplicative monoids are submonoids.
\end{proof}

\begin{definition}
    Suppose \(R\) is a ring, and \(X \subseteq R\) is a subset.
    The subring generated by \(X\)
    is the intersection of all subrings \(R'\) that contain \(X\),
    that is, \(S = \bigcap_{R \supset R' \supset X} R'\).
\end{definition}
\begin{proposition}
    The subring generated by a subset
    is the ring of all possible sums and products of elements of the subset.
\end{proposition}
\begin{proof}
    Apply Proposition~\ref{prop:subset-generated-subgroup} twice,
    to both the additive group and the multiplicative monoid.
\end{proof}

\begin{definition}
    A ring homomorphism is a function \(\func{f}{R_1}{R_2}\)
    where \(R_1,R_2\) are rings,
    with the properties that:
    \begin{enumerate}[label={(\roman*)}, itemsep=0mm]
        \item \(\func{f}{(R_1,+,0)}{(R_2,+,0)}\) is a group homomorphism; and
        \item \(\func{f}{(R_1^\star,\cdot,1)}{(R_2^\star,\cdot,1)}\)
            is a monoid homomorphism.
    \end{enumerate}
\end{definition}
\begin{proposition}
    For every ring \(R\),
    there exists a unique homomorphism \(\func{\zeta}{\bZ}{R}\).
\end{proposition}
\begin{proof}
    We attempt to define \(\zeta\) starting with the simplest requirements.
    We know that \(\zeta\) must map \(0 \mapsto 0_R\) and \(1 \mapsto 1_R\),
    By addition, and inversion,
    we can clearly define for any \(n \in \bZ\),
    \begin{equation*}
        \zeta(n) = \zeta(\underbrace{1+1+\cdots+1}_\textrm{\(n\) times})
    = \underbrace{1_R + 1_R + \cdots + 1_R}_\textrm{\(n\) times} = n_R
    \end{equation*}

    It is now sufficient to prove existence
    by checking that this homomorphism is valid under multiplication.
    \begin{equation*}
        \zeta(m)\zeta(n)
        = \underbrace{(1_R + 1_R + \cdots + 1_R)}_\textrm{\(m\) times}
        \underbrace{(1_R + 1_R + \cdots + 1_R)}_\textrm{\(n\) times}
        = \underbrace{1_R + 1_R + \cdots + 1_R}_\textrm{\(mn\) times}
        = \zeta(mn)
    \end{equation*}

    Now suppose, by way of contradiction,
    that there exists another homomorphism \(\func{\phi}{\bZ}{R}\).
    Then for them to be different,
    there exists some \(n \in \bZ\) such that \(\phi(n) \neq \zeta(n)\).
    But clearly, by definition of \(\zeta\),
    \begin{equation*}
        \phi(n) = \phi(\underbrace{1+1+\cdots+1}_\textrm{\(n\) times})
        = \underbrace{\phi(1)+\phi(1)+\cdots+\phi(1)}_\textrm{\(n\) times}
        = \underbrace{1_R + 1_R + \cdots + 1_R}_\textrm{\(n\) times}
        = \zeta(n)
    \end{equation*}
    which is a contradiction,
    since \(\phi(1) = 1_R\) by definition of ring homomorphism.
\end{proof}
\begin{definition}
    The homomorphism \(\func{\zeta}{\bZ}{R}\)
    is the canonical homomorphism.
\end{definition}
\begin{remark}
    In notation for rings,
    we might sometimes write some number \(n\)
    and an element \(x \in R\) as \(nx \in R\),
    which by strict definition means `adding up \(x\), \(n\) times',
    which makes \(nx = \zeta(n)x\).
\end{remark}

\begin{definition}
    If the multiplicative monoid \((R^\star,\cdot,1)\) is a group,
    i.e.\ there exists multiplicative inverses for every element,
    then \(R\) is a division ring.
\end{definition}
\begin{definition}
    If the multiplicative monoid \((R^\star,\cdot,1)\) is an abelian group,
    i.e.\ there exists multiplicative inverses,
    and multiplication is commutative,
    then \(R\) is a field.
\end{definition}
\begin{definition}
    If the multiplicative monoid \((R^\star,\cdot,1)\) is commutative,
    then \(R\) is a commutative ring.
\end{definition}
\begin{proposition}[Binomial Theorem]
    In any commutative ring (and therefore a field),
    \({(x+y)}^n = \sum_{r=0}^n \binom{n}{r} x^r y^{n-r}\),
    where \(\binom{n}{r}\) is the usual binomial coefficient
    given by \(\binom{n}{r} = (n!)/(r!(n-r)!)\).
\end{proposition}
\begin{proof}
    % This is obvious and we will only give a sketch of the proof.
    % Clearly because of commutativity,
    % we can group all the terms with the same number of \(x\) and \(y\),
    % which add up to the binomial coefficient.
    % Use a combinatorics argument
    % for the definition of the coefficient itself.
    \sout{Simple proof by induction.}
    \textit{Fine, I'll give you the proof.}

    Clearly \(x+y = \binom{1}{0}x + \binom{1}{1}y\),
    as \(\binom{1}{0} = 1 = \binom{1}{1}\) by simple computation.

    Now, suppose the binomial theorem holds for case \(k\).
    Then we have
    \begin{align*}
        {(x+y)}^{k+1} &= x{(x+y)}^k + y{(x+y)}^k
        = \sum_{r=0}^k \binom{k}{r} x^{r+1} y^{k-r}
        + \sum_{r=0}^k \binom{k}{r} x^r y^{k-r+1} \\
        &= \sum_{r=1}^{k+1} \binom{k}{r} x^r y^{k-r+1}
        + \sum_{r=0}^k \binom{k}{r} x^r y^{k-r+1}
        = \sum_{r=0}^{k+1} \qty[\binom{k}{r}+\binom{k}{r-1}]x^r y^{k-r+1}
    \end{align*}
    since \(\binom{k}{r-1} = \binom{k}{k+1} = 0\) by usual definition.
    It is now sufficient to prove the sum of binomial coefficients.
    \begin{equation*}
        \binom{k}{r}+\binom{k}{r-1}
        = \frac{k!}{r!(k-r)!} + \frac{k!}{(r-1)!(k-r+1)!}
        = \frac{k!(k-r+1) + k!r}{r!(k-r+1)!}
        = \frac{k!(k+1)}{r!(k+1-r)!}
        = \frac{(k+1)!}{r!(k+1-r)!}
    \end{equation*}
\end{proof}

\begin{definition}
    Suppose \(R\) is a ring, and \(x \in R\) some element.
    \(x\) is a left zero divisor if there exists some \(y \neq 0\)
    such that \(xy = 0\);
    \(x\) is a right zero divisor if there exists some \(y \neq 0\)
    such that \(yx = 0\).
\end{definition}
\begin{definition}
    \(R\) is an integral domain if \(xy = 0\)
    implies \(x = 0\) or \(y = 0\) for all \(\{x,y\} \subset R\).
    That is, the only zero divisor is \(0 \in R\).
\end{definition}

\begin{definition}
    Suppose \(R\) is a ring, and \(x \in R\) some element.
    \(x\) is a unit if there exists \(y \in R\)
    such that \(xy = yx = 1_R\),
    i.e.\ there exists a multiplicative inverse.
    This implies that we can use cancellation when dealing with units,
    since we can multiply on both the left and the right
    its multiplicative inverse \(x^{-1} = y\).
\end{definition}
\begin{remark}
    The word `unit' here is significantly different
    as the word `unit' in analysis;
    units need not have a norm equal to 1 here,
    for example when in a field,
    every nonzero element has a multiplicative inverse by definition,
    and hence is a `unit' in algebra,
    but don't you dare write in your analysis homework
    that \(2\) is a unit.
\end{remark}
\begin{proposition}
    The set of units in \(R\) is a subgroup
    of the multiplicative monoid \((R^\star,\cdot,1)\).
\end{proposition}
\begin{proof}
    Suppose \(x,y\) are units; then \(x^{-1},y^{-1}\) exist.
    By definition of a unit, \(0\) is not a unit,
    so all units are in \(R^\star\).
    There is closure, since \({(xy)}^{-1} = y^{-1}x^{-1}\).
    Associativity is inherited,
    and \(1\) is its own inverse by definition.
    Lastly, it is obvious that \({(x^{-1})}^{-1} = x\),
    just by definition, so inverses exist.
\end{proof}


\subsection{Matrix Rings}\label{sec:matrix-rings}

\begin{remark}
    We shall establish the following convention in this section:
    \begin{enumerate}[label={(\roman*)}, itemsep=0mm]
        \item \(a\) a lowercase symbol denotes a generic element in a set;
        \item \(\vec{a}\) an underlined symbol denotes a vector; and
        \item \(\vb{A}\) a bold symbol (usually uppercase)
            denotes a matrix quantity.
    \end{enumerate}
    Note that sometimes these definitions are blurry,
    as such quantities can be expressed in different ways;
    in that case, these notations are merely the way the author thinks of them.
\end{remark}

\pagebreak

\begin{proposition}
    Suppose \(R\) is a ring.
    Then \(R^{n \times n}\),
    the set of \(n \times n\) matrices with entries in \(R\),
    forms a ring with entry-wise addition
    and the usual matrix multiplication.
\end{proposition}
\begin{proof}
    Check the ring axioms yourself.
\end{proof}
\begin{remark}
    We shall use bold fonts to represent matrices in this document.
    In particular, \(\vb{I} = \vb{I}_n\)
    is the \(n \times n\) identity matrix,
    and \(\vb{0} = \vb{0}_n\) is the \(n \times n\) zero matrix.
    \(\vb{e}_{ij}\) will represent a matrix
    with 1 at the \(i\)th row and \(j\)th column,
    with zeroes everywhere else;
    so we can write \((a_{ij}) = \sum_{ij} a_{ij}\vb{e}_{ij}\).
\end{remark}
\begin{remark}\label{rem:matrix-mult}
    For ease of reference,
    we shall write down the rules of matrix multiplication.
    Suppose \(\vb{A} = (a_{ij}) \in R^{m \times n}\)
    and \(\vb{B} = (b_{ij}) \in R^{n \times p}\),
    then \(\vb{C} = \vb{A}\vb{B} = (c_{ij}) \in R^{m \times p}\)
    is given by
    \begin{equation*}
        c_{ij} = \sum_{k=1}^n a_{ik}b_{kj}
    \end{equation*}
\end{remark}

\begin{definition}
    Suppose \(R\) is a commutative ring.
    A function \(\func{\det}{R^{n \times n}}{R}\) is called a determinant
    if it is defined by the following three rules:
    \begin{enumerate}[label={(\alph*)}, itemsep=0mm]
        \item \(\det\vb{I} = 1\);
        \item If \(\sigma \in S_n\) is a row permutation,
            and \(\vb{A} \in R^{n \times n}\),
            then \(\det(\sigma\vb{A}) = \sgn(\sigma)\det(\vb{A})\); and
        \item \(\det\) is an \(n\)-linear function with respect to rows,
            that is, for some arbitrary \(i\)th row,
            assuming all other rows are the same,
            \begin{equation*}
                \det\threevector{\vdots}{au + bv}{\vdots}
                = a\det\threevector{\vdots}{u}{\vdots}
                + b\det\threevector{\vdots}{v}{\vdots}
            \end{equation*}
    \end{enumerate}
\end{definition}
\begin{proposition}[Uniqueness of the determinant]
    The determinant exists and is unique.
\end{proposition}
\begin{proof}
    We claim that if \(\vb{A} = (a_{ij})\), then
    \begin{equation*}
        \det(\vb{A}) = \sum_{\sigma \in S_n}\sgn(\sigma)
        \prod_{i=1}^n a_{i\sigma(i)}
        = \sum_{\sigma \in S_n}\sgn(\sigma)
        a_{1\sigma(1)}a_{2\sigma(2)} \hdots a_{n\sigma(n)}
    \end{equation*}
    is a determinant.

    We can first check that \(\det(\vb{I}) = 1\)
    since \(a_{ij} = 0\) if and only if \(i \neq j\),
    so any \(\sigma\) that is not the identity
    inside the sum produces 0.

    Now suppose we permute the rows by \(\tau \in S_n\).
    Then we are simply sending all \(a_{ij} \mapsto a_{\tau(i)j}\),
    so via a change of variables \(\upsilon\tau = \sigma\),
    and as the sum over the \(\tau(i)\) is the same as sum over \(i\),
    \begin{align*}
        \det(\tau\vb{A}) &= \sum_{\sigma \in S_n} \sgn(\sigma)
        \prod_{i=1}^n a_{\tau(i)\sigma(i)}
        = \sum_{\upsilon \in S_n} \sgn(\upsilon\tau)
        \prod_{i=1}^n a_{\tau(i)\upsilon\tau(i)} \\
        &= \sgn(\tau) \sum_{\upsilon \in S_n} \sgn(\upsilon)
        \prod_{i=1}^n a_{i\upsilon(i)}
        = \sgn(\tau) \det(\vb{A})
    \end{align*}

    The \(n\)-linear condition is obvious
    from the definition of the determinant,
    since every summand \(\prod a_{i\sigma(i)}\)
    contains exactly one term from the the \(i\)th row.
    Hence, when all other rows are kept the same,
    the determinant is a sum of a constant coefficient
    times the \(i\)th row term,
    which is by definition, linear.
    Combined together this proves existence.

    \medskip

    To prove uniqueness, suppose \(\vb{A} = (a_{ij})\) again.
    Then the \(i\)th row is a vector
    \(\vec{a}_{i\star} = a_{i1}\vec{e}_1 + \cdots + a_{in}\vec{e}_n\),
    where \(\vec{e}_j\) is a basis row vector for the \(j\)th coordinate.
    Then the determinant must be
    \begin{equation*}
        \det(\vb{A}) = \det\begin{bmatrix}
            \sum_t a_{1t}\vec{e}_t \\
            \sum_t a_{2t}\vec{e}_t \\
            \vdots \\
            \sum_t a_{nt}\vec{e}_t
        \end{bmatrix}
    \end{equation*}
    which by the \(n\)-linear condition means
    the determinant must be a linear combination of
    \(\det({\mqty[\vec{e}_{\alpha_1} & \hdots & \vec{e}_{\alpha_n}]}^\intercal)\),
    where \(\alpha_i\) is a choice between 1 to \(n\).
    Since we have to accomodate all permutations (by condition (b)),
    all combinations must be in there,
    and we need to sign of the permutation to be part of the coefficient.
    Lastly, since the identity must return 1,
    we are forced to conclude that we cannot have another factor in there,
    so the coefficient must only be the sign.
    % TODO: make this proof better
\end{proof}
\begin{remark}
    We shall discuss the determinant,
    especially with how it is algorithmically defined,
    further in Section~\ref{sec:linear-algebra} for linear algebra.
\end{remark}

\begin{definition}
    Suppose \(\vb{A} \in R^{n \times n}\).
    We denote the submatrix formed
    by removing the \(i\)th row and the \(j\)th column
    as \(\vb{M}_{ij}^{\vb{A}}\) or simply \(\vb{M}_{ij}\).
    The determinant of this submatrix is called a minor,
    or sometimes a first minor
    \(m_{ij} = m_{ij}^{\vb{A}} = \det\vb{M}_{ij}^{\vb{A}}\).
    We can similarly define a second, third, or \(k\)th minor
    by removing two, three, or \(k\) rows and columns from \(\vb{A}\).
\end{definition}
\begin{theorem}[Laplace Expansion for the Determinant]\label{thm:laplace-expansion-det}
    Suppose \(\vb{A} = (a_{ij}) \in R^{n \times n}\).
    Then
    \begin{equation*}
        \det\vb{A} = \sum_{j=1}^n {(-1)}^{i+j} a_{ij} m_{ij}
    \end{equation*}
    for any \(1 \leq i \leq n\).
\end{theorem}
\begin{proof}
    We shall first prove the case of expansion along the first row.
    We denote \(\tau_k = (1k)\) and \(tau_1 = 1 \in S_n\)
    as swapping the first row with the others,
    and \(S_{n-1}\) as the permutation of all rows except the first.
    Then we can clearly write the determinant as
    \begin{equation*}
        \det\vb{A} = \sum_{k=1}^n \sum_{\sigma \in S_{n-1}} \sgn(\tau_k \sigma)
        \prod_{i=1}^n a_{i\tau_k\sigma(i)}
        = \sum_{k=1}^n \sgn(\tau_k) a_{1k} \sum_{\sigma \in S_{n-1}} \sgn(\sigma)
        \prod_{i=2}^n a_{i\tau_k\sigma(i)}
    \end{equation*}
    But the second summation is effectively the definition of the determinant,
    but with the first row, and the \(k\)th column removed.
    Notice that there is a sign change
    since the effective odd/even column numbering has changed
    for the first \(k-1\) rows,
    which we know to be a \((k-1)\)-cycle
    that has a sign \({(-1)}^{k-2} = {(-1)}^k\).
    This completes the first part of the proof as
    \begin{equation*}
        \det\vb{A} = a_{11}m_{11}
        + \sum_{k=2}^n \sgn(\tau_k) a_{1k} {(-1)}^k m_{1k}
        = \sum_{k=1}^n {(-1)}^{1+k} a_{1k} m_{1k}
    \end{equation*}
    
    The expansion along the \(i\)th row is simply
    an expansion along the first row after shuffling the first \(i\) rows,
    which is a permutation of sign \(i-1\).
    Hence we multiply by \({(-1)}^{i-1}\) by the definition of the determinant
    and that yields our final answer.
    \begin{equation*}
        \det\vb{A} = {(-1)}^{i-1} \sum_{k=1}^n {(-1)}^{1+k} a_{ik} m_{ik}
        = \sum_{k=1}^n {(-1)}^{i+k} a_{ik} m_{ik}
    \end{equation*}
\end{proof}

\begin{definition}
    Suppose \(R\) is a commutative ring,
    and \(\vb{A} \in R^{n \times n}\) is a square matrix.
    We call the matrix \(\adj\vb{A} \in R^{n \times n}\)
    that makes \((\adj\vb{A})\vb{A} = \vb{A}(\adj\vb{A}) = (\det\vb{A})\vb{I}\)
    the adjugate matrix, or the classical adjoint.
    This is not to be confused with the more common notion of an adjoint,
    which is the conjugate transpose of a matrix.
\end{definition}
\begin{proposition}
    The adjugate is given by the transpose of the cofactor matrix \(\vb{C}\).
    More specifically, if \(\vb{C} = (c_{ij})\),
    then its entries are \(c_{ij} = {(-1)}^{i+j}m_{ij}\).
\end{proposition}
\begin{proof}
    Clearly \(\adj\vb{A} = (d_{ij})\)
    where \(d_{ij} = {(-1)}^{i+j}m_{ji}\).
    If \(\vb{A} = (a_{ij})\),
    then by definition of matrix multiplication we have
    \(\vb{A}(\adj\vb{A}) = (b_{ij})\), with
    \begin{equation*}
        b_{ij} = \sum_{k=1}^n a_{ik}d_{kj}
        = \sum_{k=1}^n a_{ik} {(-1)}^{k+j}m_{jk}
        = \sum_{k=1}^n {(-1)}^{j+k} a_{ik}m_{jk}
    \end{equation*}
    Notice that if \(i=j\),
    then this is exactly the definition of the determinant
    by \hyperref[thm:laplace-expansion-det]{Laplace's expansion}.
    Hence the entries on the main diagonal must be equal to \(\det(\vb{A})\).
    
    Now it suffices to prove that all off-diagonals are 0.
    Since \(i \neq j\), we can replace row \(j\) with row \(i\) in \(\vb{A}\),
    which we will denote this new matrix a \(\vb{A'}\);
    as the minor \(m_{jk}\) removes the \(j\)th row anyways,
    \(m_{jk}^{\vb{A}} = m_{jk}^{\vb{A'}}\).
    But since the \(i\)th row of the old matrix \(\vb{A}\)
    is the \(j\)th row of the new matrix \(\vb{A'}\),
    \(a_{ik} = a'_{jk}\), so \(b_{ij}\) is again,
    by \hyperref[thm:laplace-expansion-det]{Laplace's expansion},
    the determinant of the new matrix \(\det\vb{A'}\).
    But clearly \(\det\vb{A'} = 0\),
    as it has a duplicate row.

    Left multiplication by the adjugate yields the same result
    by simple computation.
\end{proof}

\begin{theorem}
    Suppose \(R\) is a commutative ring,
    and \(\vb{A} \in R^{n \times n}\).
    Then \(\vb{A}^{-1} \in R^{n \times n}\) exists
    if and only if \({(\det\vb{A})}^{-1} \in R\) exists.
\end{theorem}
\begin{proof}
    By the definition of the adjugate,
    \({(\det\vb{A})}^{-1} \adj\vb{A} = \vb{A}^{-1}\).
    It is now clear that one cannot exist without the other.
\end{proof}


\subsection{Quaternions}

\begin{lemma}
    There exists solutions to polynomial equations with real-valued coefficients
    that do not reside in the reals.
    That is, \(\bR\) is not an algebraically closed field.
\end{lemma}
\begin{proof}
    The counterexample is \(x^2 + 1 = 0\),
    requiring \(\pm\sqrt{-1}\) as solutions.
\end{proof}
\begin{remark}
    This is historically the motivation
    as to constructing the complex numbers \(\bC\).
    However, it does not seem possible to construct
    even larger fields via this method.
\end{remark}
\begin{theorem}[Fundamental Theorem of Algebra]
    For a polynomial equation of degree \(n\) with complex-valued coefficients,
    there exists exactly \(n\) solutions in \(\bC\) counted with multiplicity.
    That is, \(\bC\) is algebraically closed.
\end{theorem}
\begin{proof}
    % TODO: provide algebraic proof
    We shall first formalize the above statement.
    Suppose we have a polynomial \(p(z) = \sum_{i=0}^n a_i z^i\).
    We wish to prove that this has \(n\) solutions,
    but it is sufficient to prove that it has one solution,
    because by factoring out that solution,
    one would obtain a polynomial one degree lower,
    and by induction that would complete our argument.

    We can assume a monic polynomial \(a_n = 1\),
    since multiplication by a factor does not change the roots.
    Let \(\mu = \inf_{z\in\bC}\abs{p(z)}\).
    We wish to show that \(\mu = 0\),
    and that the infimum is attained at some \(z_0 \in \bC\).
    For some arbitrary \(\abs{z} = R\) on a circle,
    \begin{equation*}
        \abs{p(z)}
        = \abs{z}^n\abs{1 + \frac{a_{n-1}}{z} + \cdots + \frac{a_0}{z^n}}
        \geq \abs{z}^n\qty(1 - \frac{\abs{a_{n-1}}}{\abs{z}}
        - \cdots - \frac{\abs{a_0}}{\abs{z}^n})
        = R^n \qty(1 - \frac{\abs{a_{n-1}}}{\abs{z}}
        - \cdots - \frac{\abs{a_0}}{\abs{z}^n})
    \end{equation*}
    which tends to infinity as \(R \to \infty\).
    Hence we can conclude there exists some \(R_0 > 0\)
    such that \(\abs{p(z)} \geq \mu + 1\).
    Focusing on inside this circle of radius \(R_0\),
    \(\abs{p}\) is continuous, and \(\{z : \abs{z} \leq R_0\}\) is compact,
    so analysis tells us the infimum of \(\abs{p}\)
    inside this region is attained;
    tacked on the fact that the infimum outside of the circle must be larger,
    the global infimum \(\mu\) must be the same as the one inside the circle.
    
    Now suppose, by way of contradiction, that \(\mu > 0\).
    Let \(q(z) = p(z_0 + z)/p(z_0)\), and \(q(0) = 1\).
    As we know \(q(z)\) is also a polynomial of degree \(n\),
    we can write \(q(z) = 1 + b_k z^k + \cdots + b_n z^n\),
    where \(b_k\) is the first nonzero coefficient.
    We see that \(\abs{q(z)} = \abs{p(z_0+z)}/\mu \geq 1\);
    but on the other hand,
    \(\abs{q(z)} = \abs{1 + b_k z^k + \cdots + b_n z^n}
    \leq \abs{1 + b_k z^k} + \sum_{m=k+1}^n \abs{b_m}\abs{z}^m\).
    Writing \(b_k z^k = \abs{b_k}\frac{b_k}{\abs{b_k}}z^k\),
    where \(\frac{b_k}{\abs{b_k}} = e^{it}\) for some \(t\),
    and \(z = re^{i\theta}\),
    this yields us \(b_k z^k = \abs{b_k}r^k e^{i(t+k\theta)}\).
    We can choose a value \(\theta = (\pi-t)/k\),
    which evaluates to \(b_k z^k = \abs{b_k}r^k e^{i\pi} = -\abs{b_k}r^k\),
    which will be strictly within \(-1\) and \(0\) for \(r\) small enough.
    Then we have
    \begin{equation*}
        \abs{q(z)} = \abs{q(re^{i\theta})}
        \leq 1 - \abs{b_k}r^k + \sum_{m=k+1}^n \abs{b_k}r^m
        = 1 - r^k\qty(\abs{b_k} - \sum_{m=k+1}^n \abs{b_k}r^m) < 1
    \end{equation*}
    since the quantity inside the brackets is positive for \(r\) small enough.
    This is a contradiction,
    so \(\mu = 0\), and there must be a root.
\end{proof}
\begin{remark}
    Despite the name, the Fundamemtal Theorem of Algebra
    is not a fundamental theorem in the study of algebra,
    but rather a theorem in analysis.
    A purely analytic proof is always provided
    in any analysis textbook at the undergraduate level;
    however, a partially algebraic proof is possible
    if we assume some facts from analysis,
    such as continuity and the intermediate value theorem.
\end{remark}

\begin{definition}
    The quaternions are a noncommutative division ring
    \(\bH = \{\smqty[a&b \\ -\bar{b}&\bar{a}] : a,b \in \bC\}
    \subseteq \bC^{2 \times 2}\).
\end{definition}
\begin{remark}
    This might not be the usual way people define quaternions,
    but it is equivalent, and lends itself to obviously being a matrix ring.
    The inverse is given by \({\smqty[a&b \\ -\bar{b}&a]}^{-1}
    = \frac{1}{\abs{a}^2+\abs{b}^2}\smqty[\bar{a}&-b \\ \bar{b}&a]\).
\end{remark}
\begin{proposition}
    The quaternions \(\bH\) can also be thought of
    as a 4-dimensional vector space over the reals \(\bR^4\)
    with the unit vectors \(1,i,j,k\).
\end{proposition}
\begin{proof}
    We shall write a quaternion as \(\smqty[a+ib&c+id \\ -c+id&a-ib]\),
    with \(a,b,c,d \in \bR\);
    then by simple verification one can find
    \(a + bi + cj + dk\) with the multiplication table \(i^2=j^2=k^2=ijk=-1\),
    since we can represent the following:
    \begin{equation*}
        i = \twomatrix{\sqrt{-1}}{0}{0}{\sqrt{-1}} \qquad
        j = \twomatrix{0}{1}{-1}{0} \qquad
        k = \twomatrix{0}{\sqrt{-1}}{\sqrt{-1}}{0}
    \end{equation*}
\end{proof}

\begin{definition}
    Suppose \(q = \smqty[a&b \\ -\bar{b}&a] \in \bH\).
    The quaternion norm is defined as
    \(\Normalizer(q) = \abs{a}^2 + \abs{b}^2\).
\end{definition}


\subsection{Ideals}

\begin{definition}
    A ring homomorphism is a function \(\func{\phi}{R}{S}\),
    where \(R,S\) are rings,
    with the properties
    \begin{enumerate}[label={(\roman*)}, itemsep=0mm]
        \item \(\func{\phi}{(R,+,0)}{(S,+,0)}\) is a group homomorphism; and
        \item \(\func{\phi}{(R^\star,\cdot,1)}{(S^\star,\cdot,1)}\)
            is a monoid homomorphism.
    \end{enumerate}
\end{definition}
\begin{definition}
    The kernel of a ring homomorphism
    is the preimage of \(0\),
    that is, \(\ker(\phi) = \phi^{-1}(0)\).
    Note that this is a subgroup of \((R,+,0)\).
\end{definition}

\begin{definition}
    Suppose \(R\) is a ring, and \(I \subseteq R\).
    \(I\) is an ideal if it fits the following criteria:
    \begin{enumerate}[label={(\roman*)}, itemsep=0mm]
        \item \(I \subseteq (R,+,0)\) is a subgroup; and
        \item \(\forall x \in I, \forall y \in R, \{xy,yx\} \subset I\).
    \end{enumerate}
    Note that an ideal does not necessarily contain the unit \(1\),
    and therefore is not always a ring.
    A left ideal is defined with only \(yx \in I\),
    and a right ideal is defined with only \(xy \in I\).
    We shall sometimes denote this as \(I \lhd R\),
    as one shall soon see its similarity to normal subgroups.
\end{definition}
% \begin{proposition}
%     The kernel of any ring homomorphism is an ideal.
% \end{proposition}
% \begin{proof}
% \end{proof}

\begin{definition}
    Suppose \(R\) is a ring, and \(r \in R\) some arbitrary element.
    The left principal ideal is \(Rr = \{yr : y \in R\}\),
    and the right principal ideal is \(rR = \{ry : y \in R\}\).
    These two definitions coincide with \(R\) is a commutative ring,
    and we call it the principal ideal.
\end{definition}

\begin{lemma}
    Suppose \(R\) is a commutative ring, and \(r \in R\).
    % and \(I \subseteq R\) is a left (or right) principal ideal
    % generated by \(R\).
    If \(\vfunc{f}{R}{R}{x}{xr}\) (or \(rx\)) is a mapping,
    then \(f\) is surjective if and only if \(r\) is a unit.
\end{lemma}
\begin{proof}
    Suppose \(r\) is a unit.
    Then for some \(y \in R\),
    clearly \(yr^{-1} \mapsto yr^{-1}r = y\).
    This is therefore surjective.

    Suppose \(f\) is surjective.
    Then for all \(y \in R\),
    there exists an element \(x \in R\) such that \(xr = y\).
    In particular, let \(y = 1\) and we see that there is an inverse.
\end{proof}
\begin{remark}
    Ideals are typically generated by a set of generators \(x_i \in S\),
    so elements are of the form
    \(I = \{\sum_i r_i x_i : r_i \in R, x_i \in S\}\).
\end{remark}
\begin{definition}
    A Noetherian ring is a ring with every ideal being finitely generated.
\end{definition}

\begin{lemma}
    Suppose \(R\) is a ring, and \(I \lhd R\) is an ideal.
    Then \(R/I\) forms a ring via \((x+I)(y+I) = (xy+I)\).
\end{lemma}
\begin{proof}
    Clearly \((R,+,0)/I\) readily forms an additive group,
    since \(I\) is an additive subgroup,
    and Proposition~\ref{prop:abelian-subgroup-normal}
    informs us that it is normal.

    Now suppose \(\{r,r'\} \subset I\).
    Clearly \((x+r)(y+r') = xy + xr' + ry + rr' \in I\),
    since the last three terms are all in \(I\);
    we are guaranteed closure.
    Next up the multiplicative unit is \(1+I\),
    which since \((1+I)(x+I) = 1x+I = x+I = x1+I = (x+I)(1+I)\)
    gives us the identity.
    Associativity is inherited.
\end{proof}
\begin{corollary}
    \(\vfunc{\pi}{R}{R/I}{x}{x+I}\) is the quotient homomorphism.
\end{corollary}
\begin{proof}
    All properties are inherited, and it forms a ring.
\end{proof}

\begin{proposition}\label{prop:ideal-operations}
    Suppoose \(R\) is a commutative ring,
    and \(I\) and \(J\) are ideals.
    The \(I \cap J\), \(I + J = \{x+y : x \in I, y \in J\}\),
    and \(IJ = \{\sum_i x_i y_i : x_i \in I, y_i \in J\}
    \subseteq I \cap J\) are all ideals.
\end{proposition}
\begin{proof}
    Suppose \(\{x,y\} \subset I \cap J\). Then clearly
    \hyperref[lem:intersection-subgroup]{the intersection is a subgroup}.
    Moreover, by closure of each of \(I\) and \(J\),
    \(xy\) is in both \(I\) and \(J\).
    This proves \(I \cap J\) is an ideal.

    Suppose \(\{x,x'\} \subset I\), and \(\{y,y'\} \subset J\).
    Any \((x+y)+(x'+y') = (x+x')+(y+y') \in I+J\),
    associativity is given by \(R\),
    \(0 \in I\) and \(0 \in J\) so \(0 = 0+0 \in I+J\),
    and inverse is clearly \(-x-y \in I+J\);
    \(I+J\) is therefore an additive subgroup.
    Moreover, \((x+y)(x'+y') = xx' + xy' + yx' + yy' \in I+J\),
    since \(xx' + xy' \in I\) and \(yx' + yy' \in J\).
    This proves \(I+J\) is an ideal.

    Lastly, again suppose \(x \in I\) and \(y \in J\).
    We have closure simply by the summation definition,
    associativity is inherited from \(R\),
    the empty sum is \(0 \in IJ\),
    and the inverse is just \(\sum_i -x_i y_i\),
    which since \(-x_i \in I\), gives us the inverse;
    \(IJ\) is therefore an additive subgroup.
    It is also clear that this is in \(I\) and in \(J\).
    Moreover, \((\sum_i x_i y_i)(\sum_j x_j y_j)
    = \sum_{ij} x_i y_i x_j y_j\)
    and since \(y_i x_j y_j \in J\),
    we have closure under multiplication.
    This proves \(IJ\) is an ideal.
\end{proof}
\begin{proposition}\label{prop:nested-ideals}
    Suppose \(I_1 \subseteq I_2 \subseteq \cdots \subseteq
    I_n \subseteq \cdots\) is an ascending chain of ideals.
    Then \(I = \bigcup_{i=1}^\infty I_i\) is an ideal.
\end{proposition}
\begin{proof}
    Suppose \(\{x,y\} \subset I\).
    Then \(x \in I_m\) and \(y \in I_n\) for some \(m,n\).
    Without loss of generality let \(I_m \subseteq I_n\),
    then \(\{x,y\} \subset I_n\),
    so we have closure inside \(I_n\), and hence inside \(I\).
    Associativity is as per usual inherited from \(R\).
    The identity \(0 \in I_1 \subseteq I\).
    And lastly, if \(x \in I\), then \(x \in I_n\) for some \(n\),
    so \(-x \in I_n \subseteq I\).
    This proves that \(I\) is an additive subgroup.
    
    Now suppose \(x \in I\), and \(r \in R\).
    Clearly \(x \in I_n\) for some \(n\)
    and therefore \(xr \in I_n \subseteq I\).
\end{proof}

\begin{definition}
    Suppose \(R\) is a ring, and \(S \subseteq R\) a subset.
    The ideal generated by \(S\) is
    \((S) = \{\sum_i r_i s_i r_i' : s_i \in S, \{r_i,r_i'\} \subset R\}\).
    We call an ideal \(I\) finitely generated
    if \(I = (S)\) for some finite \(S\).
\end{definition}
\begin{definition}
    Suppose \(R\) is a ring, and \(I \lhd R\) an ideal.
    We call \(I\) maximal (in \(R\))
    if there are no ideals \(J \lhd R\) such that \(I \subseteq J \subseteq R\).
\end{definition}

\begin{proposition}\label{prop:field-no-proper-ideals}
    Suppose \(R\) is a commutative ring.
    Then \(R\) is a field if and only if
    \(R\) has no nonzero proper ideals.
\end{proposition}
\begin{proof}
    Suppose \(R\) is a field and \(I\) an ideal.
    If there exists an element \(x \in I\), \(x \neq 0\),
    by the definition of a field we have \(x^{-1} \in R\).
    Then by definition of an ideal \(1 = xx^{-1} \in I\),
    and again by closure, any element \(r \in R\)
    woud therefore be \(r = 1r \in I\).
    Hence \(I = R\) is not a proper ideal.

    Now suppose \(R\) has no proper ideals.
    Then if \(x \in R\), then \(xR\) is an ideal,
    so if \(xR \neq \{0\}\), \(xR = R\),
    so there exists \(y \in R\) such that \(xy = 1\).
    We have found an inverse \(x^{-1} = y\).
\end{proof}


\subsection{Homomorphisms and Isomorphisms}

\begin{remark}
    The isomorphism theorems work in a lot of algebraic structures
    (formally, it works in any universal algebra).
    We will number them the same way we did as for groups.
\end{remark}

\begin{theorem}[Universal Property of Quotient Rings]\label{thm:univ-prop-quotient-ring}
    Let \(R,S\) be commutative rings,
    and \(I \lhd R\) be an ideal.
    Suppose \(\func{\pi}{R}{R/I}\) is the quotient homomorphism
    and \(\func{\phi}{R}{S}\) is any ring homomorphism
    with \(I \subseteq \ker(\phi)\).
    Then there exists a unique ring homomorphism
    \(\func{\bar{\phi}}{R/I}{S}\) such that \(\phi = \bar{\phi}\circ\pi\).

    This is represented by the following commutative diagram:
    \begin{center}
        \begin{tikzcd}
            R \arrow{r}{\phi} \arrow{d}{\pi} & S \\
            R/I \arrow{ru}[swap]{\exists! \bar{\phi}}
        \end{tikzcd}
    \end{center}
\end{theorem}
\begin{proof}
    We wish to prove uniqueness by assuming existence.
    Suppose \(\phi = \bar{\phi}\circ\pi = \bar{\phi}'\circ\pi\).
    Since \(I \subseteq \ker(\phi)\),
    all elements \(x \in I\) obey \(\phi(x) = 0\).
    Knowing that all elements in \(R\) belong in some coset \(r+I\),
    all \(r' \in r+I\) maps to \(\phi(r') = \phi(r+x) = \phi(r)+\phi(x) = \phi(r)\),
    which tells us the image of each coset
    is a single element \(\phi(r+I) = \{\phi(r)\}\).
    Now, seeing that \(\pi \) maps \(r' \mapsto r+I\), its coset,
    by definition of a quotient mapping,
    if \(\bar{\phi} \neq \bar{\phi}'\),
    there must be one such coset \(r+I\)
    thata \(\bar{\phi}(r+I) \neq \bar{\phi}'(r+I)\) disagrees on.
    However, this is a contradiction,
    because for all \(r' \in r+I \subseteq R\),
    \(\bar{\phi}'(r+I) = \bar{\phi}'(\pi(r')) = \phi(r')
    = \bar{\phi}(\pi(r')) = \bar{\phi}(r+I)\),
    contradicting with our assumed inequality above.
    Hence we have established uniqueness of \(\bar{\phi}\).

    We will now prove existence by constructing such a homomorphism.
    Let \(\vfunc{\bar{\phi}}{R/I}{S}{r+I}{\phi(r)}\),
    mapping all cosets \(r+I\) to the function output of its coset representative.
    Suppose some arbitrary element \(r' \in r+I \subseteq R\)
    in an arbitrary coset.
    Then we know that there exists \(x \in I\) such that \(r' = r+x\),
    which allows us to conclude that
    \begin{equation*}
        \phi(r') = \phi(r+x) = \phi(r) + \phi(x)
        = \phi(r) = \bar{\phi}(r+I) = \bar{\phi}(\pi(r'))
    \end{equation*}
\end{proof}

\begin{theorem}[First Isomorphism Theorem for Rings]\label{thm:iso-1-ring}
    Suppose \(\func{\phi}{R}{S}\) is a ring homomorphism,
    and \(I = \ker(\phi)\).
    We have:
    \begin{enumerate}[label={(\alph*)}, itemsep=0mm]
        \item \(I \lhd R\), the kernel is an ideal;
        \item \(\phi(R) \subseteq S\), the image is a subring; and
        \item \(\phi(R) \cong R/I\),
            the image is uniquely isomorphic to the quotient ring.
    \end{enumerate}
\end{theorem}
\begin{proof}
    By the simple fact that \(\phi\) is also an additive group homomorphism,
    the \hyperref[thm:iso-1-group]{first group isomorphism theorem}
    gives us that the kernel is an additive subgroup.

    Now suppose some arbitrary \(x \in \ker(\phi)\), and \(y \in R\).
    Then clearly \(\phi(xy) = \phi(x)\phi(y) = 0\phi(y) = 0= 
    \phi(y)0 = \phi(y)\phi(x) = \phi(yx)\).
    This proves statement (a).

    \medskip

    Secondly, the homomorphism inherits all its properties from \(R\),
    so its image must form a subring.
    In particular,
    the \hyperref[thm:iso-1-group]{first group (monoid) isomorphism theorem}
    guarantees us that the image will be an additive subgroup,
    and a multiplicative monoid.
    This proves statement (b).

    \medskip

    The \hyperref[thm:univ-prop-quotient-ring]{universal property}
    guarantees a unique homomorphism \(\func{\bar{\phi}}{R/I}{\phi(R)}\)
    such that \(\phi = \bar{\phi}\circ\pi\),
    \(\pi\) being the quotient homomorphism.
    By the \hyperref[thm:iso-1-group]{first group isomorphism theorem},
    when applied to the additive group,
    we see that it is an isomorphism.
    This proves statement (c).
\end{proof}

\begin{theorem}[Second Isomorphism Theorem for Rings]\label{thm:iso-2-ring}
    Suppose \(R\) is a ring, \(S \subseteq R\) some subring,
    and \(I \lhd R\) an ideal.
    We have:
    \begin{enumerate}[label={(\alph*)}, itemsep=0mm]
        \item \(S + I = \{s+x : x \in S, x \in I\} \subseteq R\),
            the sum is a subring;
        \item \(I \lhd S + I\), the ideal is also and ideal of the sum;
        \item \(S \cap I \lhd S\),
            the intersection is an ideal of the subring; and
        \item \((S+I)/I \cong S/(S \cap I)\),
            these two quotients are isomorphic.
    \end{enumerate}
\end{theorem}
\begin{proof}
    By the \hyperref[thm:iso-2-group]{second group isomorphism theorem},
    \(S+I\) forms an additive subgroup.
    For \(\{s,s'\} \subset S\) and \(\{x,x'\} \subset I\),
    we see that \((s+x)(s'+x') = ss' + xs' + sx' + xx' \in S+I\),
    since \(ss' \in S\) and \(xs' + sx' + xx' \in I\),
    which gives us closure.
    Associativity is inherited from \(R\),
    and the identity is \(0+1 \in S+I\).
    This proves statement (a).

    \medskip

    \(I\) already forms a group,
    and clearly \(I = 0+I \subseteq S+I\),
    so it is already a subgroup.
    By definition of an ideal,
    for all \(x \in I\) and \(y \in R\), \(\{xy,yx\} \subset I\),
    so more specifically this holds for \(y \in S+I \subseteq R\).
    This proves statement (b).

    \medskip

    Suppose \(\{x,x'\} \in S \cap I\).
    Then \(x+x' \in S \cap I\),
    since there is closure for both \(S\) as a group,
    and \(I\) as an ideal (which is also a group).
    Associativity is always inherited,
    and the identity is \(0 = \in S \cap I\)
    as \(0 \in S\) and \(0 \in I\) by definition.
    Lastly, additive inverses always exist in a subgroups \(S,I\),
    so \(-x \in S \cap I\).
    Hence \(S \cap I \subseteq S\) is a subgroup.

    Now, we need to check the multiplicative condition.
    Suppose \(x \in S \cap I\), and \(y \in S\).
    We see that by the definition of a subring, \(\{xy,yx\} \subset S\);
    but also \(\{xy,yx\} \subset I\) since \(I \lhd R\) is an ideal,
    so it also works for elements \(y \in S \subseteq R\).
    This proves statement (c).

    \medskip
    \pagebreak

    By statements (c) \& (d), these two quotients are well-formed.
    We now attempt to construct a homomorphism
    \(\vfunc{\phi}{S}{(S+I)/I}{s}{s+I}\).
    We demonstrate that this a valid homomorphism,
    by showing that
    \begin{gather*}
        \phi(s_1 s_2) = (s_1 s_2)+I = (s_1 + I)(s_2 + I) = \phi(s_1)\phi(s_2) \\
        \phi(s_1+s_2) = s_1+s_2+I = (s_1+I) + (s_2+I) = \phi(s_1) + \phi(s_2)
    \end{gather*}
    since \(s_1, s_2\) are elements of \(R\),
    so same logic as the quotient ring epimorphism applies.

    We now want to show that \(\phi\) is surjective.
    The elements \(s+x \in S+I\) must belong in some coset \(s+x+I\),
    which we now see is equivalent to \(s+I\).
    By definition, \(\phi\) maps \(s \mapsto s+I\),
    so every coset is covered by \(\phi\),
    and therefore it is an epimorphism.

    We can then demonstrate that \(\ker(\phi) = S \cap I\).
    We can see that if \(x \in S \cap I\),
    then \(x \in I\), so \(\phi(x) = x+I = I\),
    which gives us \(S \cap I \subseteq \ker(\phi)\).
    On the other hand, if \(x \in \ker(\phi)\),
    then \(\phi(x) = x+I \subseteq I\),
    which requires \(x \in I\), giving us \(\ker(\phi) \subseteq S \cap I\).
    Hence \(\ker(\phi) = S \cap I\).

    Lastly, we apply the \hyperref[thm:iso-2-ring]{first isomorphism theorem},
    and prove that there exists a unique homomorphism
    between \(S/(S \cap I) \cong (S+I)/I\).
\end{proof}

\begin{theorem}[Third Isomorphism Theorem for Rings]\label{thm:iso-3-ring}
    Suppose \(R\) is a ring, \(I \lhd R\) an ideal. Then:
    \begin{enumerate}[label={(\alph*)}, itemsep=0mm]
        \item if \(S\) is a subring such that \(I \subseteq S \subseteq R\),
            then \(S/I \subseteq R/I\) is a subring;
        \item a subring of \(R/I\) must be of the form \(S/I\)
            such that \(S\) is a subring with \(I \subseteq S \subseteq R\);
        \item if \(J\) is an ideal such that \(I \subseteq J \subseteq R\),
            then \(J/I \lhd R/I\) is an ideal;
        \item an ideal of \(R/I\) must be of the form \(J/I\)
            such that \(J \lhd R\) is an ideal
            with \(I \subseteq J \subseteq R\); and
        \item if \(J \lhd R\) is an ideal such that \(I \subseteq J \subseteq R\),
            then \((R/I)/(J/I) \cong R/J\).
    \end{enumerate}
\end{theorem}
\begin{proof}
    We first have to prove that \(I \lhd S\),
    which is obvious because it is already an additive subgroup,
    and \(\{xy,yx\} \in I\) for all \(x \in I\) and \(y \in R\)
    can be restricted to \(y \in S \subseteq R\).

    It is now easy to see that
    with the quotient homomorphism \(\func{\pi}{R}{R/I}\),
    the image of the subring \(\pi(S) = S/I\),
    so by the \hyperref[thm:iso-1-group]{first isomorphism theorem}
    \(S/I\) forms a subring.
    This proves statement (a).

    \medskip

    Suppose \(S' \subseteq R/I\) is a subring.
    We can look at the preimage \(\pi^{-1}(S')\),
    which since \(0 \in S'\), we have \(S' \supseteq \ker(\pi) = I\).
    Notice that the preimage of a ring is still a ring,
    because the correspondence given by
    the \hyperref[thm:iso-4-group]{fourth group (monoid) isomorphism theorem}
    when applied to the additive group and the multiplicative monoid
    forms a ring.
    This proves statement (b).

    \medskip

    We now look at the quotient homomorphism \(\func{\pi}{R}{R/I}\),
    especially the image of the larger ideal \(\pi(J) = J/I\).
    By the \hyperref[thm:iso-1-group]{first group isomorphism theorem},
    \(\pi(J)\) forms an additive group,
    and if \(x \in J\) and \(r \in R\),
    \(\pi(x)\pi(r) = \pi(xr) \in \pi(J) = J/I\),
    and similarly this holds for \(\pi(r)\pi(x)\).
    Hence \(J/I\) is an ideal, proving statement (c).

    \medskip

    Suppose \(J' \lhd R/I\) is an ideal.
    We can look at the preimage \(\pi^{-1}(J')\),
    which since \(0 \in J'\), we have \(J' \supseteq \ker(\pi) = I\).
    Notice that the preimage \(\pi^{-1}(J') = J' + I\) is an ideal,
    which is given by Proposition~\ref{prop:ideal-operations}.
    Hence we have the preimage being an ideal \(J\),
    so \(J' = \pi(J) = J/I\),
    proving statement (d).

    \medskip

    We can now attempt to construct a homomorphism
    \(\vfunc{\phi}{R/I}{R/J}{r+I}{r+J}\).
    This is valid because
    by the \hyperref[thm:univ-prop-quotient-ring]{universal property},
    we have \(\func{\pi}{R}{R/I}\) and \(\func{\eta}{R}{R/I}\),
    so there is a unique homomorphism that makes \(\eta = \phi\circ\pi\).
    We can show that this is surjective,
    since the \(J\)-cosets partition \(R\),
    so each \(J\)-coset must have some element \(r+J\) that represents it,
    and clearly this element \(r+I\) in the \(I\)-cosets must get sent to it,
    alongside all other elements in that \(I\)-coset.

    We claim the kernel is \(\ker(\phi) = J/I\).
    Observe that \(\ker(\eta) = J\) and \(\ker(\pi) = I\),
    so \(\phi\) must map all the \(I\)-cosets
    that are represented by elements of \(J\) into \(0\).

    Lastly, we invoke the \hyperref[thm:iso-1-ring]{first isomorphism theorem},
    which gives us \(\ker(\phi) = J/I \lhd R/I\),
    making our quotient \((R/I)/(J/I)\) valid,
    and also that \((R/I)/\ker(\phi) = (R/I)/(J/I) \cong R/J\),
    proving statement (e).
\end{proof}

\begin{theorem}[Fourth Isomorphism Theorem for Rings]\label{thm:iso-4-ring}
    Suppose \(R\) is a ring, \(I \lhd R\) some ideal,
    and \(\vfunc{\pi}{R}{R/I}{x}{x+I}\) the quotient homomorphism.
    Then \(\pi\) is a bijection between the subrings of \(R/I\)
    and the subrings of \(R\) containing \(I\);
    and is also a bijection between ideals of \(R/I\)
    and ideals (and subrings) of \(R\) containing \(I\).
\end{theorem}
\begin{proof}
    This is merely a corollary of
    the \hyperref[thm:iso-3-ring]{third isomorphism theorem}.
    Statements (a) and (b) prove the correspondence between subrings,
    while statements (c) and (d) prove the correspondence between ideals.
\end{proof}


\subsection{Field of Fractions}

\begin{definition}
    Suppose \(R\) is a ring,
    and \(\func{\zeta}{\bZ}{R}\) the canonical homomorphism.
    If \(\zeta\) is injective, % then the kernel is \(\ker(\zeta) = \{0\}\),
    % so \(\zeta(\bZ) \subseteq R\), and
    we call \(R\) having characteristic 0.
    If on the other hand \(\zeta\) is not injective,
    then the kernel clearly must be some ideal \(\ker(\zeta) = n\bZ\),
    % so \(R \cong \bZ/n\bZ\),
    and we call \(R\) having characteristic \(n\).
\end{definition}
\begin{proposition}
    Suppose \(p\) prime.
    Then the ring \(\bZ/p\bZ\) must be a field.
\end{proposition}
\begin{proof}
    We first prove that \(p\bZ\) is maximal in \(\bZ\),
    i.e.\ there does not exist an ideal \(n\bZ\),
    such that \(p\bZ \subsetneq n\bZ \subsetneq \bZ\).
    Clearly if \(p\bZ \subseteq n\bZ\),
    then \(n \mid p\), which tells us \(n = 1\) or \(p\),
    which is either \(\bZ\) or \(p\bZ\).
    Hence the only ideals of \(\bZ\) that contain \(p\bZ\)
    are \(\bZ\) and \(p\bZ\).

    By (the contraposition of)
    the \hyperref[thm:iso-3-ring]{third isomorphism theorem},
    then the only ideals of \(\bZ/p\bZ\) are itself and \(p\bZ/p\bZ = \{0\}\),
    so there does not exists proper ideals of \(\bZ/p\bZ\).
    Then by Proposition~\ref{prop:field-no-proper-ideals},
    \(\bZ/p\bZ\) is a field.
\end{proof}
\begin{definition}
    For some prime \(p\),
    we call \(\bZ/p\bZ = \mathbb{F}_p\)
    the finite field of \(p\) elements.
\end{definition}

\begin{lemma}\label{lem:field-no-zero-divisors}
    Fields do not have nonzero zero divisors.
\end{lemma}
\begin{proof}
    If \(x,y \neq 0\) and \(xy = 0\),
    \(x^{-1}\) exists, so \(x^{-1}xy = y = 0x^{-1} = 0\)
    which imply \(y = 0\), which is a contradiction in itself.
\end{proof}
\begin{lemma}\label{lem:field-prime-kernel}
    Suppose \(F\) is a field,
    and \(\func{\zeta}{\bZ}{F}\) is the canonical homomorphism.
    Then either \(\zeta\) is injective,
    or \(\ker(\zeta) = p\bZ\) for some prime \(p\).
\end{lemma}
\begin{proof}
    If \(\zeta\) is not injective,
    then clearly the kernel is some \(\ker(\zeta) = n\bZ\).
    Suppose, by way of contradiction, that \(n = ab\) such that \(1 < a,b < n\).
    Then \(0 = \zeta(n) = \zeta(ab) = \zeta(a)\zeta(b)\),
    and since \(a,b\) are not elements of \(n\bZ\),
    \(\zeta(a),\zeta(b)\) are nonzero,
    and in particular, they are zero divisors.
    But that contradicts the \hyperref[lem:field-no-zero-divisors]{lemma above}.
    Hence \(n\) must be prime.
\end{proof}
\begin{theorem}\label{thm:field-unique-prime-char}
    The characteristic of a field is unique,
    and it is either 0 or a prime \(p\).
\end{theorem}
\begin{proof}
    The \hyperref[lem:field-prime-kernel]{lemma above}
    already proves that the characteristic is either 0 or prime.
    It is now sufficient to prove uniqueness.
    Suppose \(\mathbb{F}_p, \mathbb{F}_q \subseteq F\) some field.
    then \(\mathbb{F}_p \subseteq F\) implies \(p1 = 0\),
    while \(\mathbb{F}_q \subseteq F\) implies \(q1 = 0\),
    But \(\gcd(p,q) = 1\), which implies \(1 = 0\),
    which is considered a trivial ring and not a field.
\end{proof}

\begin{definition}[Universal Property of Field of Fractions]
    Suppose \(R\) is a commutative domain.
    We call \(F\) its field of fractions or its quotient field if:
    \begin{enumerate}[label={(\roman*)}, itemsep=0mm]
        \item there exists an monomorphism \(\func{\iota}{R}{F}\); and
        \item for any field \(S\), if \(\func{\phi}{R}{S}\) is a monomorphism,
            then there exists a unique \(\func{\bar{\phi}}{F}{S}\)
            such that \(\phi = \bar{\phi}\circ\iota\).
    \end{enumerate}
    This is represented by the following commutative diagram:
    \begin{center}
        \begin{tikzcd}
            R \arrow{r}{\phi} \arrow{d}{\iota} & S \\
            F \arrow{ru}[swap]{\exists! \bar{\phi}}
        \end{tikzcd}
    \end{center}
\end{definition}
\begin{theorem}[Existence and Uniqueness of Field of Fractions]
    For any commutative domain \(R\),
    its field of fractions \(F\) exists and is unique up to isomorphism.
\end{theorem}
\begin{proof}
    Again, as with all universal properties,
    we first prove uniqueness assuming existence.
    Suppose, by way of contradiction,
    that \(F\) and \(F'\) are both fields of fractions.
    Then by definition there exists \(\func{\iota}{R}{F}\) injective,
    and for any field \(S\), in particular \(F'\),
    \(\func{\iota'}{R}{F'}\) injective
    such that there exists a unique \(\func{\bar{\phi}}{F}{F'}\)
    where \(\iota' = \bar{\phi}\circ\iota\).
    But same can be said for \(\func{\iota'}{R}{F'}\) injective,
    for any field \(S\), in particular \(F'\),
    \(\func{\iota}{R}{F}\) injective
    such that there exists a unique \(\func{\bar{\phi}'}{F'}{F}\)
    where \(\iota = \bar{\phi}'\circ\iota'\).
    \begin{center}
        \begin{tikzcd}
            R \arrow{r}{\iota'} \arrow{d}[swap]{\iota} &
            F' \arrow[rightharpoondown, shift right=0.25ex]{ld}%
                [xshift=.5ex, yshift=-.5ex, swap]{\bar{\phi}'} \\
            F \arrow[rightharpoondown, shift right=0.25ex]{ru}%
                [xshift=-.5ex, yshift=.5ex, swap]{\bar{\phi}}
        \end{tikzcd}
    \end{center}

    Then following the definitions,
    we come to these two conclusions:
    \begin{align*}
        \bar{\phi}'\circ\bar{\phi}\circ\iota = \bar{\phi}'\circ\iota' = \iota
        &\implies \bar{\phi}'\circ\bar{\phi} = \id_F \\
        \bar{\phi}\circ\bar{\phi}'\circ\iota' = \bar{\phi}\circ\iota = \iota'
        &\implies \bar{\phi}\circ\bar{\phi}' = \id_{F'}
    \end{align*}
    which by definition means \(\bar{\phi}' = \bar{\phi}^{-1}\).
    Hence \(\bar{\phi}\) is a bijection,
    telling us that \(F \cong F'\) are isomorphic.

    Now we prove existence.
    Suppose we have tuples \((x,y) \in R \times R^\star\),
    i.e.\ the standard condition that the denominator does not equal 0.
    We shall define an equivalence relation
    \((a,b) \sim (c,d)\) if \(ad = bc\).
    We check the three conditions:
    \((a,b) \sim (a,b)\) since \(ab = ab\) (reflexive);
    \((a,b) \sim (c,d)\) implies \(ad = bc\)
    implies \(cb = da\) implies \((c,d) \sim (a,b)\) (symmetric);
    and \((a,b) \sim (c,d)\) and \((c,d) \sim (e,f)\)
    implies \(ad = bc\) and \(cf = de\),
    so \(adcf = bcde\), when cancelling out \(cd\) we have \(af = be\),
    which implies \((a,f) \sim (b,e)\) (transitive).

    We define \(F = (R \times R^\star)/{\sim}\),
    i.e.\ elements are \((x,y) \in R \times R^\star\)
    without unique representation.
    We shall denote elements of \(F\) as \(\overline{(x,y)} \in F\).
    This forms a ring with addition and multiplication defined as
    \begin{equation*}
        \overline{(a,b)} + \overline{(c,d)} = \overline{(ad+bc,bd)} \qquad
        \overline{(a,b)} \cdot \overline{(c,d)} = \overline{(ac,bd)}
    \end{equation*}
    The ring axioms are obvious,
    and essentially encompass the operations on \(\bQ\).

    Lastly, we check that it satisfies the morphisms.
    % We can start again by proving uniqueness first,
    % but we presume that the operations on \(\bQ\) are familiar enough
    % that we shall simply construct the morphisms.
    We define \(\vfunc{\iota}{R}{F}{x}{\overline{(x,1)}}\),
    and for some arbitrary \(\func{\phi}{R}{S}\),
    let \(\vfunc{\bar{\phi}}{F}{S}{\overline{(x,y)}}{\phi(x){\phi(y)}^{-1}}\).
    We check that for arbitrary \((x,y) \in R \times R^\star\),
    \begin{equation*}
        \bar{\phi}(\iota(x)){\bar{\phi}(\iota(y))}^{-1}
        = \bar{\phi}\overline{(x,1)}{\bar{\phi}\overline{(y,1)}}^{-1}
        = \phi(x){\phi(1)}^{-1}{\phi(y)}^{-1}\phi(1)
        = \phi(x){\phi(y)}^{-1}
    \end{equation*}
    so \(\phi = \bar{\phi}\circ\iota\),
    and our definition is one such set of morphisms, proving existence.
    %
    % Now suppose, by way of contradiction,
    % there exists another \(\bar{\phi}'\)
    % such that it satisfies the commutative diagram,
    % i.e.\ \(\bar{\phi}'\overline{(x,y)} \neq \phi(x){\phi(y)}^{-1}\)
    % for some \(x,y\).
    % But then \(\phi(x){\phi(y)}^{-1}
    % \neq \bar{\phi}'(\iota(x)){\bar{\phi}'(\iota(y))}^{-1}\),
    % which implies either \(\phi(x) \neq \bar{\phi}'(\iota(x))\)
    % or \(\phi(y) \neq \bar{\phi}'(\iota(y))\),
    % either of which would lead to a contradiction.
\end{proof}
\begin{remark}
    Notice that this time we use a universal property
    as a definition for an algebraic structure.
    This is a way to generalize constructions
    in terms of morphisms instead of elements,
    which allow us to define things mostly up to isomorphism.
\end{remark}
\begin{remark}
    Because of how we define \(\bar{\phi}\),
    it is common to write \(\overline{(a,b)} = ab^{-1} = a/b\),
    which is the common notation for fractions.
\end{remark}


\subsection{Polynomial Rings}

\begin{definition}
    For some commutative ring \(R\),
    we call the polynomials with coefficients in \(R\)
    the polynomial ring \(R[x] = \{\sum_{i=0}^n a_i x^i : a_i \in R, n \geq 0\}\).
    This can be represented as an infinite sequence of coefficients
    with finitely many nonzero terms (with finite support),
    i.e.\ \(R[x] \cong \bigoplus_{i=0}^\infty R\).
\end{definition}
\begin{remark}
    Notice that the indeterminate \(x\) is not restricted to any set,
    and can be not in \(R\);
    an analogy is that although we often look at polynomials \(\bR[x]\),
    the solutions (and thereofre the indeterminates \(x\) that we investigate)
    are sometimes in the bigger field \(\bC\).
\end{remark}
\begin{remark}
    We shall again explore more of the direct sum
    in Section~\ref{sec:linear-algebra} for linear algebra.
\end{remark}
\begin{proposition}
    \(R[x]\) forms a ring with addition
    \(\sum a_i x_i + \sum b_i x_i = \sum (a_i+b_i) x_i\)
    and multiplication
    \((\sum a_i x_i)(\sum b_j x_j) = \sum_k (\sum_{i+j=k} a_i b_j) x_k\).
\end{proposition}
\begin{proof}
    Addition is term-wise and inherits all properties from \(R\).
    Multiplication is clearly closed, associative, commutative,
    and has an identity \(1 \in R \subseteq R[x]\).
\end{proof}

\begin{theorem}[Universal Property of Polynomial Rings]\label{thm:univ-prop-polynomial}
    Suppose \(R,S\) are commutative rings,
    and \(R[x]\) is a univariate polynomial ring.
    For all homomorphisms \(\func{\phi}{R}{S}\),
    and for all \(u \in S\),
    there exists a unique homomorphism \(\vfunc{\eta}{R[x]}{S}{x}{u}\),
    such that \(\phi = \eta\circ\iota_k\) for all \(k \in \bZ_0^+\).

    This can be represented by the logical statement
    \begin{equation*}
        \forall S \in \mathbf{Ring},\;\forall \phi \in \Hom(R,S),\;
        \forall u \in S,\;\exists!\vfunc{\eta}{R[x]}{S}{x}{u},\;
        \phi = \eta\circ\iota_k
    \end{equation*}
    and the following commutative diagram:
    \begin{center}
        \begin{tikzcd}
            R \arrow{r}{\phi} \arrow{d}{\iota_k} & S \\
            R[x] \arrow{ru}[swap]{\exists! \eta}
        \end{tikzcd}
    \end{center}
\end{theorem}
\begin{proof}
    We prove uniqueness first.
    Suppose, by way of contradiction,
    that there exists \(\eta \neq \eta'\) that maps \(R[x] \to S\).
    Since we know \(R[x] \cong \bigoplus_k R\),
    any element \(p \in R[x]\) can be written as \(p = \sum_k \iota_k(a_k)\)
    for some \(a_k \in R\).
    Assume \(\eta\) and \(\eta'\) differ at \(p\).
    Then clearly \(\eta(p) = \sum_k (\eta\circ\iota_k)(a_k)\),
    and same goes for \(\eta'\),
    but that implies \(\eta'\circ\iota_k \neq \eta\circ\iota_k = \phi\)
    at at least one \(a_k \in R\),
    which is a contradiction.

    Now we prove existence.
    Let \(\vfunc{\iota_k}{R}{R[x]}{a}{a_k x^k}\),
    and \(\vfunc{\eta}{R[x]}{S}{b}{\phi(b)},\; x \mapsto u\),
    i.e.\ evaluating the polynomial at \(x = u\).
    We see that the morphisms follow as expected:
    \begin{equation*}
        \sum_k \eta(\iota_k(a_k)) = \sum_k \eta(a_k x^k)
        = \sum_k \eta(a_k){\eta(x)}^k = \sum_k \phi(a_k)u^k
    \end{equation*}
\end{proof}

% TODO: construction of C from R[x]/(x^2+1)?

\begin{definition}
    Suppose \(R\) is a commutative ring.
    We can inductively define the multivariate polynomial ring
    as \(R[x_1,x_2,\hdots,x_n] = R[x_1,x_2,\hdots,x_{n-1}][x_n]\).
\end{definition}
\begin{theorem}\label{thm:polynomial-ring-isomorphism}
    Suppose \(R\) is a commutative ring,
    \(R[x_1,x_2,\hdots,x_n]\) a polynomial ring in \(n\) variables
    and for some permutation of variables
    \(\sigma \in S_n\), \(y_i = x_{\sigma(i)}\),
    \(R[y_1,y_2,\hdots,y_n]\) another polynomial ring in \(n\) variables.
    Then \(R[x_1,x_2,\hdots,x_n] \cong R[y_1,y_2,\hdots,y_n]\),
    that is, polynomial rings with the same number of variables
    are unique up to isomorphism.
\end{theorem}
\begin{proof}
    By the \hyperref[thm:univ-prop-polynomial]{universal property},
    univariate polynomial rings are unique up to isomorphism.
    Hence we have \(R[x_1] \cong R[y_1]\).
    Then by induction,
    suppose \(R[x_1,x_2,\hdots,x_k] \cong R[y_1,y_2,\hdots,y_k]\).
    Then clearly we can write a bijection \(\beta\) between these two,
    and suppose \(a_i \in R[x_1,x_2,\hdots,x_k]\).
    Elements of \(R[y_1,y_2,\hdots,y_{k+1}]\)
    can now be written as \(\sum_i \beta(a_i) y_{k+1}^i\),
    so we have \(R[x_1,x_2,\hdots,x_k][y_{k+1}] \cong R[y_1,y_2,\hdots,y_{k+1}]\).
    By the \hyperref[thm:univ-prop-polynomial]{universal property} again, we have
    \(R[x_1,x_2,\hdots,x_{k+1}] \cong R[x_1,x_2,\hdots,x_k][y_{k+1}]\),
    so \(R[x_1,x_2,\hdots,x_{k+1}] \cong R[y_1,y_2,\hdots,y_{k+1}]\).
\end{proof}

\begin{definition}
    Suppose we have \(R \subseteq S\) two rings,
    and a polynomial ring \(R[x]\).
    Choosing an arbitrary element \(u \in S\),
    we can write the evaluation morphism as
    \(\vfunc{\eta_u}{R[x]}{S}{x}{u}\).
    We call the image the subring generated by \(u\) over \(R\),
    commonly written as \(\eta_u(R[x]) = R[u] \subseteq S\).
\end{definition}
\begin{proposition}
    \(R[u] \cong R[x]/I\), where \(I = \{p(x) \in R[x] : p(u) = 0\}\).
\end{proposition}
\begin{proof}
    A direct consequence of
    the \hyperref[thm:iso-1-ring]{first isomorphism theorem}.
\end{proof}

\begin{definition}
    Suppose \(R \subseteq S\) both fields,
    and \(R[x]\) is a polynomial ring.
    For some arbitrary \(u \in S\),
    consider the evaluation morphism \(\vfunc{\eta_u}{R[x]}{S}{x}{u}\).
    If the kernel is \(\ker(\eta_u) = \{0\}\),
    then \(R[u] \cong R[x]\), and we call \(u\) transcendental;
    if on the other hand, the kernel \(I = \ker(\eta_u) \neq \{0\}\),
    then \(R[u] \cong R[x]/I\), and we call \(u\) algebraic,
    in the sense that \(u\) satisfies polynomials,
    in particular the ones in \(I\).
\end{definition}

\begin{theorem}
    Suppose \(R \subseteq S\) both fields,
    and \(R[x]\) a polynomial ring.
    Suppose \(I \subseteq R[x]\) is any ideal.
    If \(I\) contains any constant \(r \in R\), \(r \neq 0\)
    then it is not a substitution kernel,
    i.e.\ it cannot be the kernel for \(\eta_u\) for any \(u \in S\).
\end{theorem}
\begin{proof}
    By way of contradiction, suppose \(r \in I\) and \(r \in R\), \(r \neq 0\),
    and that there exists some \(u \in S\)
    such that \(R[u] \cong R[x]/I\).
    Without loss of generality, let us choose the smallest \(r\).
    Now, this would imply that \(r = 0\) in \(S\).
    Hence \(S\) must have characteristic \(r\).
    Then this tells us \(R\) must also have characteristic \(r\),
    since \(R \subseteq S\).
    But this is a contradiction,
    as this would make \(R \cong S \cong \bZ/r\bZ\);
    as \(R \subseteq R[u] \subseteq R[x] \subseteq S\),
    then the kernel must be \(\{0\}\).
\end{proof}

\begin{definition}
    We define a function \(\func{\deg}{R[x]}{\bZ_0^+}\)
    that gives the degree of a polynomial,
    which is the largest \(n\) such that the coefficient \(a_n \neq 0\).
\end{definition}
\begin{lemma}\label{lem:degree-arithmetic}
    Suppose \(f(x) = \sum_i a_i x^i\) and \(g(x) = \sum_i b_i x^i\)
    both elements of \(R[x]\). Then
    \begin{enumerate}[label={(\alph*)}, itemsep=0mm]
        \item \(\deg (f+g) \leq \max\{\deg f, \deg g\}\); and
        \item \(\deg (fg) = \deg f + \deg g\).
    \end{enumerate}
\end{lemma}
\begin{proof}
    If \(\deg f \neq \deg g\), then clearly under addition,
    the leading term remains
    the leading term of the polynomial with the larger degree.
    If \(\deg f = \deg g\),
    if the two leading terms do not cancel out,
    the degree remains the same;
    if they do cancel out,
    then the degree must be smaller.
    This proves statement (a).

    The product of the leading terms \(a_n x^n\) and \(b_m x^m\)
    must be \(a_n b_m x^{n+m}\),
    so the degree is the sum of the leading terms;
    notice all other terms will multiply to a smaller power of \(x\).
    This proves statement (b).
\end{proof}

% \begin{definition}
% \end{definition}
\begin{proposition}\label{prop:ring-polynomial-long-div}
    Suppose \(R\) a commutative ring,
    and \(R[x]\) its polynomial ring.
    Let \(f,g \in R[x]\),
    and we have \(f(x) = \sum_{i=0}^n a_i x^i\),
    \(g(x) = \sum_{i=0}^m b_i x^i\).
    We can define long division on polynomials as
    \(b_m^k f(x) = q(x)g(x) + r(x)\)
    for some arbitrary \(k \in \bN\), and \(\{q,r\} \in R[x]\).
    % The above long division algorithm,
    Along with the degree function,
    this allows for the Euclidean algorithm to apply,
    with the criteria that \(\deg r < \deg g\) or \(r = 0\).
\end{proposition}
\begin{proof}
    We first consider the case that \(\deg f < \deg g\).
    Then let \(b_m^0 = 1\), and \(q(x) = 0\),
    so \(r(x) = f(x)\), and immediately \(\deg r = \deg f < \deg g\).

    We now consider the case that \(\deg f \geq \deg g\).
    We can construct the first term of \(q(x)\) as \(q_1(x) = a_n x^{n-m}\),
    so that
    \begin{equation*}
        f_1(x) = b_m f(x) - a_n x^{n-m} g(x)
        = \sum_{i=0}^n a_i b_m x^i - \sum_{i=n-m}^n a_n b_{i-n+m} x^i
        = \sum_{i=0}^{n-1} a_i b_m x^i - \sum_{i=n-m}^{n-1} a_n b_{i-n+m} x^i
    \end{equation*}
    leaving us with an \(f_1\) that has at least one degree lower than \(f\).
    We can recursively repeat this process
    \begin{equation*}
        f_{i+1}(x) = b_m f_i(x) - q_{i+1}(x)g(x)
        = b_m f_i(x) - a_{\deg f_i} x^{\deg f_i - m} g(x)
    \end{equation*}
    until the \(\deg f_k\) drops below \(\deg g\),
    which allows us to write down \(r(x) = f_k(x)\) at that point,
    and \(q(x) = \sum_{i=1}^k q_i(x)\).
    \begin{equation*}
        q(x)g(x) + r(x) = \sum_{i=1}^k q_i(x)g(x) + f_k(x)
        = \sum_{i=1}^{k-1} q_i(x)g(x) + b_m f_{k-1}(x)
        = \cdots = b_m^k f(x)
    \end{equation*}
    There is at most \(n-m+1\) steps to this induction,
    so \(k \leq n-m+1\).
\end{proof}
\begin{remark}
    Notice that this long division is not unique,
    as we can make \(k\) arbitrarily larger than the choice above,
    and we merely have to multiply \(q(x)\) by that difference.
\end{remark}
\begin{remark}
    We shall discuss more of Euclidean domains
    in Section~\ref{sec:factorial-rings} for unique factorization domains.
\end{remark}

% \begin{definition}
% \end{definition}
\begin{proposition}\label{prop:field-polynomial-long-div}
    Suppose \(F\) a field, and \(F[x]\) its polynomial ring.
    Let \(f,g \in F[x]\),
    and we have \(f(x) = \sum_{i=0}^n a_i x^i\),
    \(g(x) = \sum_{i=0}^m b_i x^i\).
    We can define long division on polynomials as
    \(f(x) = q(x)g(x) + r(x)\), with \(\{q,r\} \in F[x]\).
    Along with the degree function,
    this allows for the Euclidean algorithm to apply,
    with the criteria that \(\deg r < \deg g\) or \(r = 0\).
\end{proposition}
\begin{proof}
    We have the same base case as \hyperref[prop:ring-polynomial-long-div]{above},
    where if \(\deg f < \deg g\), let \(q(x) = 0\),
    so \(r(x) = f(x)\) and \(\deg r = \deg f < \deg g\).

    Now the inductive case is much easier.
    Let \(f(x) = f_0(x)\).
    Using the same process, alternately calculate
    \begin{gather*}
        q_i(x) = \frac{a_{\deg f_{i-1}}}{b_m} x^{\deg f_{i-1} - m} \\
        f_{i+1}(x) = f_i(x) - q_{i+1}(x)g(x)
        = f_i(x) - \frac{a_{\deg f_{i-1}}}{b_m} x^{\deg f_i - m}
        \sum_{i=0}^m b_i x^i
    \end{gather*}
    so that the leading term gets cancelled out every time.
    Repeat this process inductively until \(\deg f_k < \deg g\),
    which is always possible because the degree drops by at least one every step.
    Again let \(f(x) = f_k(x)\) and \(q(x) = \sum_{i=1}^k q_i(x)\).
    \begin{equation*}
        q(x)g(x) + r(x) = \sum_{i=1}^k q_i g(x) + f_k(x)
        = \sum_{i=1}^{k-1} q_i(x)g(x) + f_{k-1}(x)
        = \cdots = f_0(x) = f(x)
    \end{equation*}
\end{proof}
\begin{theorem}
    Long division in fields is unique.
\end{theorem}
\begin{proof}
    Suppose there are \(q \neq q'\) and \(r \neq r'\)
    that both satisfy the long division process.
    Then \(f(x) = q(x)g(x) + r(x) = q'(x)g(x) + r'(x)\).
    Grouping terms together gives us \((q-q')g(x) = r'(x) - r(x)\),
    so this tells us that \(\deg ((q-q')g) = \deg (r'-r)\).
    But the definition of long division requires \(\deg r < \deg g\),
    so Lemma~\ref{lem:degree-arithmetic} gives us
    \(\deg(r'-r) \leq \deg g < \deg(q-q') + \deg g = \deg ((q-q')g)\),
    which contradicts our equality above.
\end{proof}
\begin{corollary}\label{cor:field-polynomial-pid}
    Suppose \(F\) is a field,
    and \(I \subseteq F[x]\) any ideal.
    Then there exists some element \(f \in I\)
    such that it is generated by that element, \(I = (f(x))\),
    i.e.\ all elements of \(I\) are multiples of \(f(x)\).
\end{corollary}
\begin{proof}
    Pick any \(f(x) \in I\) with the lowest degree.
    If \(g \in I\), by \hyperref[prop:field-polynomial-long-div]{long division},
    \(g(x) = q(x)f(x) + r(x)\), which implies \(r(x) = g(x) - q(x)f(x) \in I\).
    But we also know that \(\deg r < \deg f\),
    which tells us that \(r(x) = 0\),
    giving us \(g(x) = q(x)f(x) \in (f(x))\), and hence \(I \subseteq (f(x))\).
    By definition of ideals, \((f(x)) \subseteq I\).
    Therefore \((f(x)) = I\).
\end{proof}

\begin{definition}
    Suppose \(R\) is a commutative domain.
    We call \(R\) a principal ideal domain (PID)
    if every ideal is generated by a single element.
\end{definition}
\begin{remark}
    In fact, what we have just proven \hyperref[cor:field-polynomial-pid]{above}
    is that for any field \(F\), \(F[x]\) is a PID.\@
    More will be discussed on PIDs
    in Section~\ref{sec:factorial-rings} for unique factorization domains.
\end{remark}

\begin{proposition}\label{prop:polynomial-domain}
    Suppose \(R\) is an integral domain.
    Then \(R[x]\) is also an integral domain.
\end{proposition}
\begin{proof}
    Suppose we have nonzero \(\{f(x),g(x)\} \subset R[x]\)
    Then we know \(\deg f = m > 0\), \(\deg g = n > 0\).
    Then \(\deg(fg) = m+n > 0\),
    so \(f(x)g(x) \neq 0\).
\end{proof}

\begin{theorem}
    Suppose \(F\) is a field, and \(f(x) \in F[x]\).
    For some \(u \in F\), if \(f(u) = 0\), then \((x-u) \mid f(x)\);
    and there exists \(r\) distinct elements \({\{u_i\}}_{i=1}^r \subset F\)
    such that \(\prod_{i=1}^r (x-u_i) \mid f(x)\),
    where \(r \leq \deg f\).
\end{theorem}
\begin{proof}
    Let \(\eta_u\) be the evaluation morphism.
    If \(f(u) = 0\), then \(f(x) \in \ker(\eta_u)\),
    which is an ideal.
    % We use the fact that
    % \hyperref[thm:pid-ufd]{elements in a PID are uniquely factorable}
    % (we will prove this later),
    Since \(F[x]\) is a PID by Corollary~\ref{cor:field-polynomial-pid},
    these ideals are generated by a single element,
    and we claim that it is \((x-u)\).
    Clearly \(u-u = 0\) also \((x-u) \in \ker(\eta_u)\),
    and it is the lowest degree.

    We can repeat this process until we cannot find elements in a kernel anymore.
    As every time we divide by \((x-u_i)\),
    the degree decreases by one,
    so we can only do this at most \(\deg f\) times.
\end{proof}
\begin{remark}
    Consider the rational polynomials \(f(x) \in \bQ[x]\).
    We can extend \(\bQ\) to a larger field \(K\)
    by adding the solutions of \(f(x)\).
    Let \(G = \Aut(K)\) be the automorphism group over \(K\).
    The study of Galois theory is that there exists a formula
    for the roots of \(f\) if and only if \(G\) is a solvable group.
\end{remark}

\begin{definition}
    If \(F\) is a field, and \(f(x) \in F[x]\) is a polynomial,
    we call \(f(x)\) irreducible
    if \(f(x) = g(x)h(x)\) implies either \(g\) or \(h\) is constant,
    i.e.\ \(f(x)\) cannot decompose into something with a lower degree.
\end{definition}
\begin{theorem}\label{thm:ideal-divisibility}
    Let \(I = (f(x))\) and \(J = (g(x))\) be two ideals in \(F[x]\).
    Then:
    \begin{enumerate}[label={(\alph*)}, itemsep=0mm]
        \item \(f(x)\) is irreducible if and only if \(I\) is maximal; and
        \item \(I \subseteq J\) if and only if \(g(x) \mid f(x)\).
    \end{enumerate}
\end{theorem}
\begin{proof}
    We first prove statement (b).
    Suppose \(g(x) \mid f(x)\).
    Then there exists \(h(x) \in F[x]\) such that \(f(x) = g(x)h(x)\),
    and hence \(f(x) \in J = (g(x))\),
    and any multiple of \(f(x)\), that is elements of \((f(x))\) is in \(J\);
    therefore \(I \subseteq J\).

    Suppose \(I \subseteq J\).
    As elements of \(J\) are multiples of \(g(x)\),
    we know that in particular elements of \(I \subseteq J\)
    are multiples of \(g(x)\);
    specifically, as \(f(x) \in I \subseteq J\),
    \(g(x) \mid f(x)\).
    This proves statement (b).

    \medskip

    Now suppose \(I\) is reducible.
    Then \(f(x) = g(x)h(x)\) some product of polynomials.
    In particular, \(g(x) \mid f(x)\), \(\deg g < \deg f\),
    so there exists some \(J = (g(x))\) such that \(I \subsetneq J\),
    as it is obvious that \(I\) cannot include elements of degree \(\deg g\).
    But also \(\deg g \geq 1\), so \(J \subsetneq F[x]\),
    so we have found an intermediate ideal,
    and \(I\) not maximal.

    For the reverse direction,
    suppose \(I\) is not maximal,
    and there exists some \(J = (g(x))\)
    such that \(I \subsetneq J \subsetneq F[x]\).
    Then \(g(x) \mid f(x)\),
    and as there are some elements in \(J\) but not in \(I\),
    there exists multiples of \(g(x)\) that is not a multiple of \(f(x)\).
    Hence \(\deg g < \deg f\),
    because otherwise constant multiplies would work if \(\deg g = \deg f\),
    and therefore \(f(x) = g(x)h(x)\),
    where \(\deg h \geq 1\).
    This proves statement (a).
\end{proof}

\begin{remark}
    From \hyperref[thm:ideal-divisibility]{above},
    if \(f(x)\) is irreducible, then \(I\) is maximal in \(F[x]\),
    which by the \hyperref[thm:iso-4-ring]{fourth isomorphism theorem},
    the only ideals of \(F[x]/I\) are itself and \(\{0\}\),
    and Proposition~\ref{prop:field-no-proper-ideals}
    tells us \(F[x]/I \supset F\) is now a field.
    In a sense we are adding the roots of \(f(x)\) to \(F\) to form \(F[x]/I\).

    Suppose \(f(x) = \prod_i g_i(x)\) factors into irreducible polynomials.
    If \(g_i(x) = x-a\) is linear, then \(a \in F\) is a root;
    if \(g_i(x)\) is nonlinear, then there is no root in \(F\),
    since otherwise it would be able to factor this into linear factors.
    Hence we only need to consider irreducible polynomials
    when looking for roots.

    Suppose \(K = F[x]/I\), where \(I = (f(x))\) for some irreducible \(f\).
    Notice that \(x+I \in I\) is always a solution of \(f(x)\) in \(K\),
    as \(x-(x+I) = I\) is in the ideal by definition,
    which is equivalent to \(0 \in K\).
    So if \(\alpha = x+I \in K\),
    we can merely factor out \(x-\alpha\) from \(f(x)\),
    and recursively use this construction to build all the roots of \(f(x)\).
\end{remark}

\subsubsection*{Polynomial Functions}

\begin{definition}
    Given some polynomial \(f(x) \in F[x]\),
    we can write a function \(\vfunc{f}{F}{F}{u}{f(u)}\),
    which is the familiar notion of polynomials being treated as functions.
    Again, the roots of the polynomial is the preimage of zero.
\end{definition}
\begin{remark}
    In general, two different polynomials can give the same function,
    such as \(x = x^2\) when considering \(\bZ/2\bZ\).
\end{remark}
\begin{theorem}\label{thm:polynomial-distinct-func}
    In infinite fields, distinct polynomials are distinct as functions.
\end{theorem}
\begin{proof}
    It is again sufficient to consider monic polynomials,
    since we know that polynomials that are off by a multiple
    have the same roots,
    and are clearly distinct from each other everywhere else.

    We prove the contrapositive.
    Suppose \(\{f(x),g(x)\} \subset F[x]\).
    If \(f = g\) as functions, then \(f(u) = g(u)\) for all \(u \in F\),
    then \(f(u) - g(u) = 0\), so \(u\) is a root of \(f - g\).
    But since \(F\) has infinitely many roots,
    which implies \(f - g = \prod_{u \in F} a(x-u)\) is not a valid polynomial,
    unless \(a = 0\).
    Hence \(f(x) = g(x)\) as polynomials.
\end{proof}
\begin{corollary}
    If \(F\) is an infinite field,
    and \(f(x) \in F[x]\) some nonzero polynomial,
    then there exists some \(u \in F\) such that \(f(u) \neq 0\).
\end{corollary}
\begin{proof}
    Special case of the \hyperref[thm:polynomial-distinct-func]{theorem above}
    that all polynomials are distinct from 0.
\end{proof}
\begin{corollary}
    If \(F\) is an infinite field,
    and \(f(x_1,x_2,\hdots,x_n) \in F[x_1,x_2,\hdots,x_n]\)
    some nonzero polynomial in \(n\) variables,
    then there exists some \((u_1,u_2,\hdots,u_n) \in F^n\)
    such that \(f(u_1,u_2,\hdots,u_n) \neq 0\).
\end{corollary}
\begin{proof}
    We can perform induction on the number of variables,
    and use the \hyperref[thm:polynomial-distinct-func]{theorem above}
    as our base case.
    Suppose, by way of induction, that our corollary holds for \(k\) variables.
    We attempt to prove the case for \(k+1\) variables.
    Suppose \(f(x_1,\hdots,x_{k+1}) \in F[x_1,\hdots,x_{k+1}]\)
    is a nonzero polynomial in \(k+1\) variables.
    Without loss of generality, assume \(x_{k+1}\) is in the expression,
    since otherwise it is merely a polynomial in \(k\) variables,
    and the inductive hypothesis holds.
    Then picking some \((u_1,\hdots,u_k) \in F^k\), \(u_i \neq 0\),
    \(f(u_1,\hdots,u_k,x_{k+1}) \in F[x_{k+1}]\)
    is a nonzero polynomial in one variable,
    so the \hyperref[thm:polynomial-distinct-func]{theorem above} holds.
\end{proof}
\begin{remark}
    In general, finding nonzeros is easy, but finding zeros is hard.
    \hyperref[thm:nullstellensatz]{Hilbert's Nullstellensatz}
    gives us a criteria for where zeros can be found.
\end{remark}

\subsubsection*{Symmetric Polynomials}

\begin{definition}
    Suppose \(R\) is a commutative ring (and more often, a field).
    We call a polynomial \(f \in R[x_1,x_2,\hdots,x_n]\) in \(n\) variables
    symmetric if \(f(x_1,x_2,\hdots,x_n)
    = f(x_{\sigma(1)},x_{\sigma(2)},\hdots,x_{\sigma(n)})\)
    for any \(\sigma \in S_n\),
    that is, the polynomial remains unchanged under permutation of variables.
    We denote the set of symmetric polynomials \({R[x_1,x_2,\hdots,x_n]}^{S_n}\).
\end{definition}
\begin{definition}
    We can write down a set of symmetric polynomials called
    the elementary symmetric polynomials \(p_k\), \(1 \leq k \leq n\).
    They include all possible combinations of
    non-repeating products of \(k\) indeterminates.
    \begin{equation*}
        p_k = \sum_{1 \leq i_1 < i_2 < \cdots < i_k \leq n}
        x_{i_1}x_{i_2} \hdots x_{i_k}
        = \sum_{1 \leq i_1 < i_2 < \cdots < i_k \leq n}
        \prod_{j=1}^k x_{i_j}
    \end{equation*}
    In particular we see that \(p_1 = \sum_{i=1}^n x_i\)
    and \(p_n = \prod_{j=1}^n x_j\).
\end{definition}
\begin{proposition}
    The symmetric polynomials \({R[x_1,x_2,\hdots,x_n]}^{S_n}\)
    forms a commutative ring.
    For \(n \geq 2\),
    this is a proper subring \({R[x_1,x_2,\hdots,x_n]}^{S_n}
    \subsetneq R[x_1,x_2,\hdots,x_n]\).
\end{proposition}
\begin{proof}
    This will become obvious once we prove
    the \hyperref[thm:fundamental-sym-polynomial]{following theorem}.
    In the meantime,
    you can convince yourself by checking the axioms.
    The second fact is easy,
    as all univariate polynomials are by definition symmetric,
    and \(f(x_1,\hdots,x_n) = x_1\) is an asymmetric polynomial
    when \(n \geq 2\).
\end{proof}

\begin{definition}
    The total degree of a monomial \(a\prod_{i=1}^n x_i^{k_i}\)
    is defined as \(\sum_{i=1}^n k_i\).
    We call a polynomial \(f \in R[x_1,x_2,\hdots,x_n]\) homogeneous
    if every monomial it contains has the same total degree.
\end{definition}
\begin{lemma}\label{lem:homogeneous-sym-polynomial}
    Any \(f \in R[x_1,x_2,\hdots,x_n]\) is a sum of homogeneous parts.
    \(f\) is symmetric if and only if each homogeneous part is symmetric.
\end{lemma}
\begin{proof}
    The first statement is clear
    because every monomial has a total degree,
    and you can simply group them by total degree.

    The second statement is also easy,
    because under permutation of variables,
    the total degree of a monomial never changes,
    so it preserves each homogeneous part by themselves.
\end{proof}

\begin{definition}
    We define an order on the monomials
    by saying \(x_1^{a_1}x_2^{a_2} \hdots x_n^{a_n}
    < x_1^{b_1}x_2^{b_2} \hdots x_n^{b_n}\)
    when \((a_1,a_2,\hdots,a_n) < (b_1,b_2,\hdots,b_n)\) lexicographically,
    that is, the earliest index \(i\) where \(a_i \neq b_i\),
    and then \(a_i < b_i\).
\end{definition}
\begin{definition}
    For homogeneous polynomials,
    we call the monomial with the largest such order
    the leading term.
\end{definition}

\begin{lemma}\label{lem:leading-term-sym-polynomial}
    \(p_1^{d_1}p_2^{d_2} \hdots p_n^{d_n}\) has a leading term
    \(x_1^{d_1+d_2+\cdots+d_n}x_2^{d_2+\cdots+d_n} \hdots x_n^{d_n}\).
\end{lemma}
\begin{proof}
    We first see that for each \(i\),
    \(p_i^{d_i}\) has a leading term \(x_1^{d_i} \hdots x_i^{d_i}\).
    We now claim that the leading term of a product of monomials
    is the product of the leading terms of the monomials.
    To see that, if \(x_1^{\alpha_1} \hdots x_n^{\alpha_n}
    > x_1^{\beta_1} \hdots x_n^{\beta_n}\)
    and \(x_1^{\gamma_1} \hdots x_n^{\gamma_n}
    > x_1^{\delta_1} \hdots x_n^{\delta_n}\),
    then \(x_1^{\alpha_1+\gamma_1} \hdots x_n^{\alpha_n+\delta_n}
    > x_1^{\beta_1+\delta_1} \hdots x_n^{\beta_n+\delta_n}\),
    since without loss of generality let \(i \leq j\),
    for the first indices where
    \(\alpha_i \neq \beta_i\) and \(\gamma_j \neq \delta_j\),
    and \(\alpha_i > \beta_i\) implies \(\alpha_i+\gamma_i > \beta_i+\delta_i\).
    We have proven that multiplication preserves the order,
    so if the leading terms are the largest out of all monomials,
    the product will still be the largest,
    and hence we can merely multiply the leading terms.
\end{proof}

\begin{theorem}[Fundamental Theorem of Symmetric Polynomials]\label{thm:fundamental-sym-polynomial}
    Every symmetric polynomial \(f \in {R[x_1,x_2,\hdots,x_n]}^{S_n}\)
    can be uniquely represented by a polynomial \(g \in R[p_1,p_2,\hdots,p_n]\)
    with variates being the elementary symmetric polynomials.
\end{theorem}
\begin{proof}
    We provide an algorithm to convert symmetric polynomials
    into polynomials over \(p_i\).
    By the \hyperref[lem:homogeneous-sym-polynomial]{first lemma above},
    it is sufficient to consider \(f\) homogeneous symmetric.
    With the above definitions,
    we can perform steps that are similar to polynomial long division
    by choosing our polynomials correctly.

    We first claim that we can always write the leading term of \(f\)
    as \(ax_1^{k_1}x_2^{k_2} \hdots x_n^{k_n}\)
    where \(k_1 \geq k_2 \geq \cdots \geq k_n\).
    We observe that since \(f\) is symmetric,
    any permutation of the variates for our leading term also exists,
    so we can simply sort the \(k_i\) for our leading monomial
    into a descending sequence,
    % choose the largest possible order,
    % and make that term our leading monomial
    % by permuting all of \(f\) according the sort
    % that we just performed on the leading term.
    and we realize that there exists that permutation of variables
    already in \(f\),
    and we can choose that as our leading term.

    \pagebreak

    We then construct \(g = ap_1^{k_1-k_2}p_2^{k_2-k_3} \hdots
    p_{n-1}^{k_{n-1}-k_n}p_n^{k_n}\),
    which by the \hyperref[lem:leading-term-sym-polynomial]{second lemma above}
    has a leading term \(ax_1^{k_1}x_2^{k_2} \hdots x_n^{k_n}\),
    as all the sums of exponents are telescoping.
    We can subtract \(f-g = r\) to get a remainder,
    which has a lower leading term,
    allowing us to repeat this process on \(r\).

    We know this process terminates in finite steps,
    because there are only finitely many ways to add \(n\) non-negative numbers
    up to a certain total degree,
    so there are finitely many \(n\)-tuples \((k_1,k_2,\hdots,k_n)\)
    that we need to eliminate.
    This proves the existence of a \(R[p_1,p_2,\hdots,p_n]\) representation.

    \medskip

    Now it suffices to prove uniqueness of such a representation.
    By way of contradiction,
    if there exists an \(f\) with two different representations,
    then we can subtract the two different representations,
    and say that there exists some \(\sum_i a_i \prod_{j=1}^n p_j^{d_{ij}} = 0\).
    In other words, we are proving algebraic independence.

    But the \hyperref[lem:leading-term-sym-polynomial]{second lemma above}
    allows us to see that there is a bijection between product of \(p_j\)
    and the leading terms in \(x_j\).
    Since the \(x_j\) are algebraically independent by definition,
    it is impossible to cancel out the leading term
    of two different products of \(p_j\),
    which imply that \(a_i\) must all be 0.
\end{proof}
\begin{remark}
    It is a good exercise to program yourself
    a symmetric polynomial decomposition calculator,
    simply by following the algorithm as described in the theorem.
\end{remark}
\begin{corollary}
    \({R[x_1,x_2,\hdots,x_n]}^{S_n} \cong R[x_1,x_2,\hdots,x_n]\).
\end{corollary}
\begin{proof}
    The \hyperref[thm:fundamental-sym-polynomial]{theorem above}
    essentially proves \({R[x_1,x_2,\hdots,x_n]}^{S_n} \cong R[p_1,p_2,\hdots,p_n]\).
    But Proposition~\ref{thm:polynomial-ring-isomorphism}
    tells us that \(R[p_1,p_2,\hdots,p_n] \cong R[x_1,x_2,\hdots,x_n]\).
\end{proof}


\subsection{Factorial Rings}\label{sec:factorial-rings}

\subsubsection*{Unique Factorization Domains}

\begin{definition}
    Suppose \(R\) is a commutative integral domain.
    For some \(\alpha \in R^\star\),
    if there exists some \(\{\beta,\gamma\} \subset R\) not units
    such that \(\alpha = \beta\gamma\),
    we say \(\alpha\) factors in \(R\).
    Notice the definition excludes units \(u\)
    since it is always possible to write \(\alpha = u(u^{-1}\alpha)\),
    which we do not consider factorization.
    We say in this case that \(\beta\) and \(\gamma\) properly divides \(\alpha\).
\end{definition}
\begin{definition}
    We call \(\alpha\) irreducible if \(\alpha\) is not factorable.
\end{definition}
\begin{proposition}
    Suppose we can factor \(\alpha = \beta\gamma\).
    Then \((\alpha) \subsetneq (\beta)\) and \((\alpha) \subsetneq (\gamma)\).
\end{proposition}
\begin{proof}
    It is sufficient to prove only the case of \(\beta\).
    Clearly \(\alpha \in (\beta)\), so \((\alpha) \subseteq (\beta)\).
    On the other hand, if \((\alpha) = (\beta)\),
    then there exists some unit \(u\) where \(\alpha = \beta u\),
    which tells us in our case, \((\alpha) \subsetneq (\beta)\).
\end{proof}
\begin{remark}
    Because of this relationship between divisibility and ideals,
    every single statement below
    can be translated between these two languages.
    Sometimes it is easy to think of statements in one language or another.
\end{remark}

\begin{definition}
    Suppose \(R\) is a commutative integral domain.
    \(R\) is a unique factorization domain (UFD)
    if all \(\alpha \in R^\star\) can be factorized into
    \(\alpha = \prod_{i=1}^n p_i\) where \(p_i\) are irreducible elements,
    and this factorization is unique up to reordering of elements,
    and multiplication by units.
\end{definition}
\begin{theorem}\label{thm:ufd-conditions}
    \(R\) is a UFD if the following two conditions hold:
    \begin{enumerate}[label={(\roman*)}, itemsep=0mm]
        \item \textit{Divisor chain.}
            Given any \(a_1 \in R^\star\),
            there exists no infinite proper divisor chain
            such that \(a_{i+1}\) properly divides \(a_i\); and
        \item \textit{\hyperref[lem:euclid]{Primeness.}}
            If \(p\) is irreducible, and \(p \mid ab\),
            then either \(p \mid a\) or \(p \mid b\).
    \end{enumerate}
\end{theorem}
\begin{proof}
    Factorization exists simply based on condition (i),
    since that allows our recursive factorization to terminate.

    Now we prove uniqueness.
    Suppose that we have two factorizations
    \(\alpha = p_1 \hdots p_r = p_1' \hdots p_s'\).
    Since \(p_1 \mid p_1 \hdots p_r\),
    it should also divide \(p_1 \mid p_1' \hdots p_s'\),
    which by condition (ii) it should divide one of \(p_i'\).
    Without loss of generality, let this be \(p_1'\).
    But as \(p_1'\) is irreducible,
    we conclude that \(p_1' = up_1\) for some unit \(u\).
    As we only require factorization up to multiplication by units,
    we can ignore \(u\) in this case,
    and look at the rest of the expression \(p_2 \hdots p_r = p_2' \hdots p_s'\).
    But this is no different from the case of \(p_1\) and \(p_1'\),
    so we recursively repeat that process,
    and see that all terms match up to unit multiplication.
\end{proof}
\begin{remark}
    It is also possible to rephrase condition (i)
    as having no infinite chain of proper ideals.
\end{remark}

\begin{definition}
    Suppose \(R\) is a commutative domain, and \(p \in R^\star\).
    We call \(p\) prime if \(p \mid ab\)
    implies either \(p \mid a\) or \(p \mid b\).
    In a similar vein, suppose \(P \subsetneq R\) is an ideal,
    we call \(P\) a prime ideal if the product of ideals \(IJ \subseteq P\)
    implies either \(I \subseteq P\) or \(J \subseteq P\).
\end{definition}
\begin{proposition}\label{prop:prime-quotient-domain}
    Suppose \(R\) is a commutative ring, and \(I\) an ideal.
    \(I\) is a prime ideal if and only if \(R/I\) is an integral domain.
\end{proposition}
\begin{proof}
    For the foward direction, suppose \(I\) is a prime ideal.
    By way of contradiction, let \(R/I\) not be a domain,
    so there exists nonzero \(x+I, y+I\) such that \((x+I)(y+I) = xy+I = I\).
    But since \(xy \in I\),
    the product of the two generated ideals \((x)(y) \subseteq I\),
    so either \((x) \subseteq I\), which implies \(x \in I\),
    or \((y) \subseteq I\), which implies \(y \in I\),
    contradicting our assumption of nonzero \(x,y\).

    For the reverse direction, suppose \(R/I\) is an integral domain.
    Then for all \(\{x,y\} \subset R\), \((x+I)(y+I) = xy+I = I\),
    \(xy \in I\) implies either \(x \in I\) or \(y \in I\).
    % By way of contradiction, let \(I\) not be prime,
    % so there exists two ideals \(J,K\) such that \(JK \subseteq I\),
    % but neither \(J,K\) are subideals of \(I\).
    For any two ideals \(JK \subseteq I\),
    every \(jk \in I\) for \(j \in J\) and \(k \in K\).
    Without loss of generality, suppose \(J\) not a subideal of \(I\),
    so there exists some \(j_0 \in J\) such that \(j \notin I\).
    Then clearly since \(j_0 k \in I\) for all \(k \in K\),
    \(k \in I\), so \(K \subseteq I\).
    Hence \(I\) is prime.
\end{proof}

\begin{lemma}
    In a UFD, elements are prime if and only if they are irreducible.
\end{lemma}
\begin{proof}
    The reverse direction is given by Theorem~\ref{thm:ufd-conditions}.
    For the forward direction,
    if \(p = ab\) is prime, clearly \(p \mid ab\),
    so \(p \mid a\) or \(p \mid b\).
    But \(a \mid p\) and \(b \mid p\),
    so either \(a\) or \(b\) must be a unit.
\end{proof}

\begin{definition}
    Suppose \(R\) is a commutative domain, and \(\{a,b\} \subset R^\star\).
    \(d \in R\) is a greatest common denominator of \(a\) and \(b\),
    denoted \(d = \gcd(a,b)\),
    if \(d \mid a\) and \(d \mid b\),
    and for all \(c \in R\), \(c \mid a\) and \(c \mid b\) implies \(c \mid d\).
    Similarly, \(m \in R\) is a least common multiple of \(a\) and \(b\),
    denoted \(m = \lcm(a,b)\),
    if \(a \mid m\) and \(b \mid m\),
    and for all \(n \in R\), \(a \mid n\) and \(b \mid n\) implies \(m \mid n\).
\end{definition}
% \begin{proposition}
%     For any two elements \(\{a,b\} \in R\) a UFD, \(\gcd(a,b)\) exists.
% \end{proposition}
% \begin{proof}
% \end{proof}
\begin{theorem}
    The primeness condition is equivalent to the following condition:
    \begin{enumerate}[itemsep=0mm]
        \item[(ii)] \textit{GCD.}
        Any two elements have a GCD.\@
    \end{enumerate}
\end{theorem}
\begin{proof}
    Primeness implies GCD is simple.
    Write \(a = p_1^{i_1} \hdots p_r^{i_r}\)
    and \(b = p_1^{j_1} \hdots p_r^{j_r}\).
    Then \(\gcd(a,b) = p_1^{\min(i_1,j_1)} \hdots p_r^{\min(i_r,j_r)}\).

    \medskip

    GCD is commutative and associative
    as in \(\gcd(a,b)\) is also a \(\gcd(b,a)\) simply by definition;
    \(d_n = \gcd(a_1,a_2,\hdots,a_n)\) is defined recursively as
    \(d_1 = a_1\), \(d_{i+1} = \gcd(d_i,a_{i+1})\),
    and clearly that is by definition \(d_n \mid a_i\) for all \(i\),
    and \(c \mid a_i\) implies \(c \mid d_n\).

    We claim that \(c\gcd(a,b)\) is also a \(\gcd(ca,cb)\).
    Suppose \(d = \gcd(a,b)\).
    Then \(d \mid a\) and \(d \mid b\),
    and hence \(cd \mid ca\) and \(cd \mid cb\),
    and \(cd \mid \gcd(ca,cb)\).
    There exists some \(x \in R\) such that \(cdx = \gcd(ca,cb)\),
    and there exists some \(y \in R\) with \(cdxy = y\gcd(ca,cb) = ca\),
    which imply \(cdx = ca\), and hence \(dx \mid a\).
    Without loss of generality \(dx \mid b\) too, so \(dx \mid d\),
    and we know \(x\) is a unit, so \(cd\) is a GCD of \(ca\) and \(cb\).

    From the above we can now prove that if \(a\) coprime with \(b\)
    and \(a\) coprime with \(c\),
    then \(a\) coprime with \(bc\).
    We can show this with \(1 = \gcd(a,c) = \gcd(a,c\gcd(a,b))
    = \gcd(a,ca,cb) = \gcd(a,cb)\).

    GCD implies primeness because if \(p\) is irreducible
    and \(p \nmid a\) and \(p \nmid b\),
    then \(p\) coprime \(a\) and \(p\) coprime \(b\),
    so \(p\) coprime \(ab\),
    and hence \(p \nmid ab\).
\end{proof}

\subsubsection*{Principal Ideal Domains}

\begin{remark}
    Recall from earlier that principal ideal domains (PID)
    are commutative domains with every ideal generated by a single element.
\end{remark}
\begin{theorem}\label{thm:pid-ufd}
    If \(R\) is a PID, then \(R\) is a UFD.\@
\end{theorem}
\begin{proof}
    We first prove that it satisfies the divisor chain condition
    via the ideal chain condition.
    Suppose we have an infinite ideal chain
    \((a_1) \subseteq (a_2) \subseteq \cdots \subseteq (a_r) \subseteq \cdots\).
    Then by Proposition~\ref{prop:nested-ideals},
    \(I = \bigcup_{i=1}^\infty (a_i)\) is an ideal.
    Notice that \(I = (b)\) must be generated by some element,
    and since \(b \in I\), there exists some \(r\) that \(b \in (a_r)\).
    However, by definition of the infinite union,
    \((b) \subseteq (a_r) \subseteq (a_{r_1}) \subseteq \cdots \subseteq I = (b)\).
    Hence we are forced to conclude that \((b) = (a_r) = (a_{r+1}) = \cdots = I\),
    and every infinite chain of ideals are only proper up to a finite \(r\).
    There are no infinite chain of proper ideals.

    \pagebreak

    Now we prove the primeness condition.
    Suppose \(p\) is irreducible, \(p \mid ab\) and \(p \nmid a\).
    \(p\) irreducible means \((p)\) is maximal
    (similar to Theorem~\ref{thm:ideal-divisibility}).
    As \(p \nmid a\), then \(a \notin (p)\),
    so the ideal generated by two elements \((p,a) \supsetneq (p)\),
    and hence \((p,a) = R = (1)\).
    By \hyperref[thm:bezout]{B\'{e}zout's Identity},
    there exists \(u,v\) such that \(up + va = 1\),
    and hence \(upb + vab = b\).
    As \(p \mid upb\) and \(p \mid v(ab)\), \(p \mid b\).

    Then by Theorem~\ref{thm:ufd-conditions}, \(R\) is a UFD.\@
\end{proof}
\begin{remark}
    In general it is difficult to determine whether any ring is a PID.\@
    The Krull dimension is one way to determine PIDs
    via the length of prime ideals.
\end{remark}

\begin{definition}
    Suppose \(R\) is a commutative domain.
    We call \(R\) a Euclidean domain if there exists a long division algorithm
    \(\func{\delta}{R}{\bZ}\),
    such that for all \(a,b \in R^\star\),
    we can find \(q,r \in R\) with \(a = qb + r\),
    where \(\delta(r) < \delta(b)\),
    and \(\delta(x) = 0\) if and only if \(x = 0\).
\end{definition}
\begin{theorem}\label{thm:euclidean-pid}
    If \(R\) is a Euclidean domain,
    then \(R\) is a PID.\@
\end{theorem}
\begin{proof}
    We first look at the ideal of a single element, \(I = \{0\}\);
    clearly this is \(I = (0)\).

    Now suppose \(I\) has more than one element.
    Choose any nonzero element \(x \in I\)
    such that \(\delta(x) \leq \delta(y)\) for all nonzero \(y \in I\);
    in other words, \(\delta(x)\) is minimal.
    Now suppose we have some nonzero \(y \in I\).
    We factor it with \(y = qx + r\),
    which we know \(r = y - qx \in I\);
    and according to the Euclidean function \(\delta(r) < \delta(x)\),
    as \(\delta(x)\) is minimal, \(\delta(r) = 0\), so \(r = 0\).
    Hence all \(y \in I\) is a multiple of \(x\),
    and \(I = (x)\).
\end{proof}
\begin{corollary}\label{cor:euclidean-ufd}
    If \(R\) is a Euclidean domain,
    then \(R\) is a UFD.\@
\end{corollary}
\begin{proof}
    By the \hyperref[thm:euclidean-pid]{theorem above}
    and Theorem~\ref{thm:pid-ufd}.
\end{proof}

\subsubsection*{Polynomials over UFDs}

\begin{definition}
    Suppose \(R\) is a UFD, and \(f(x) \in R[x]\) a polynomial.
    We define the content of \(f\), a function \(\func{c}{R[x]}{R}\)
    to be the GCD of all nonzero coefficients.
    We call \(f\) primitive if \(c(f)\) is a unit.
    Hence we can write \(f = c(f)f_1\), where \(f_1\) is primitive.
\end{definition}
\begin{lemma}\label{lem:polynomial-primitive-decomp}
    Suppose \(R\) is a UFD, and \(F\) its field of fractions.
    If \(f(x) \in F[x]\),
    there exists a unique factorization up to multiplication of units in \(R\)
    for \(f(x) = \gamma f_1(x)\)
    where \(f_1\) is primitive in \(R[x]\), and \(\gamma \in F\).
\end{lemma}
\begin{proof}
    Since all coefficients of \(f\) are fractions,
    it is sufficient to look at the product of all denominators,
    and multiply by that to obtain some \(g(x) \in R[x]\).
    Hence there exists some \(a \in R\) such that \(af(x) = g(x) \in R[x]\).
    We can then decompose \(g(x) = bg_1(x)\)
    where \(b = c(g)\) and \(g_1\) is primitive.
    Therefore \(af(x) = bg_1(x)\),
    so \(f(x) = \frac{b}{a}g_1(x)\), and there exists \(\gamma = \frac{b}{a}\).

    We now prove uniqueness.
    Suppose \(f(x) = \gamma' g_1'(x)\) for another such combination.
    Then since \(\gamma' \in F\), we write \(\gamma' = b'/a'\),
    and we have \(f(x) = \frac{b'}{a'}g_1'(x) = \frac{b}{a}g_1(x)\).
    So we have \(aa'f(x) = ab'g_1'(x) = a'b g_1(x) \in R[x]\),
    but that tells us \(c(aa'f) = ab' = a'b\) up to unit multiplication.
    Hence \(ab' = ua'b\) for some unit \(u\),
    and \(ub/a = b'/a'\).
\end{proof}
\begin{corollary}\label{cor:polynomial-unit-mult}
    Suppose \(\{f(x),g(x)\} \subset R[x]\) are both primitive.
    If there exists some \(\gamma \in F\) such that \(\gamma f = g\),
    then \(\gamma\) is a unit in \(R\).
\end{corollary}
\begin{proof}
    By the proof of \hyperref[lem:polynomial-primitive-decomp]{lemma above},
    \(\gamma'/\gamma = u\) is a unit.
\end{proof}

\begin{lemma}[Gauss' Lemma]\label{lem:gauss}
    Suppose \(R\) is a UFD.\@
    If \(\{f,g\} \subset R[x]\) are both primitive,
    then \(fg\) is also primitive.
\end{lemma}
\begin{proof}
    We prove the contrapositive.
    Suppose \(fg\) is not primitive.
    Then there exists some prime \(p\) which divides every coefficient of \(fg\).
    Considering \(T = R/(p)\), since \(p\) prime,
    Proposition~\ref{prop:prime-quotient-domain} shows that \(T\) is a domain.
    By Proposition~\ref{prop:polynomial-domain},
    since \(T\) is a domain, \(T[x]\) is also a domain.
    We can write a quotient map \(\vfunc{\pi}{R}{T}{r}{r+(p)}\) the obvious way,
    and extend it to \(\func{\pi}{R[x]}{T[x]}\) by sending \(x \mapsto x\).
    Hence \(\pi(fg) = \pi(f)\pi(g) = (p)\),
    as all the coefficients, and hence all the terms are divisible by \(p\).
    But as \(T[x]\) is a domain, either \(\pi(f) = (p)\) or \(\pi(g) = (p)\),
    which says either \(f\) or \(g\) is not primitive.
\end{proof}
\begin{lemma}\label{lem:polynomial-primitive-irreducible}
    Suppose \(R\) is a UFD, and \(F\) its field of fractions.
    If \(f(x) \in R[x]\) is primitive and irreducible,
    then \(f(x) \in F[x]\) is irreducible.
\end{lemma}
\begin{proof}
    By way of contradiction, suppose \(f(x) = f_1(x)f_2(x)\)
    can be factored in \(F[x]\).
    Then by Lemma~\ref{lem:polynomial-primitive-decomp},
    there exists \(\gamma_i \in F\) and primitive \(g_i \in R[x]\)
    such that \(f(x) = f_1(x)f_2(x) = \gamma_1\gamma_2 g_1(x)g_2(x)\).
    But considering \(f,g_1,g_2\) as primitive elements of \(R[x]\),
    we know that \(\gamma_1\gamma_2\) is a unit in \(R\)
    by Corollary~\ref{cor:polynomial-unit-mult}.
    Then that implies \(f_1,f_2\) are also elements of \(R[x]\),
    so \(f\) factors in \(R[x]\) too,
    which is a contradiction.
\end{proof}
\begin{theorem}
    Suppose \(R\) is a UFD.\@
    Then \(R[x]\) is also a UFD.\@
\end{theorem}
\begin{proof}
    We first prove that a factorization of polynomials exist.
    Suppose \(f(x) \in R[x]\) is nonzero and not a unit.
    Then we can write \(f = c(f)f_1\), where \(f_1\) primitive.
    Clearly \(c(f) \in R\) can factor into irreducible components.
    Now consider \(f_1\).
    If \(f_1\) is irreducible we are done.
    However, if \(f_1\) can be factored such that \(f_1 = g_1 g_2\),
    then we see that \(0 < \deg g_i < \deg f_1\),
    so the degree gets reduced,
    and hence if we perform factorization recursively,
    there are at most \(\deg f_1\) recursions,
    and factorization terminates.
    Hence the factorization of \(f\)
    is the product of the factorization of \(c(f)\)
    and the factorization of \(f_1\).

    \medskip

    We then prove the factorization is unique up to multiplication by units.
    We first consider \(f(x) \in R[x]\) primitive.
    Then \(f(x) = q_1(x)q_2(x) \hdots q_n(x)\) factors into irreducibles,
    with \(\deg q_i > 0\),
    since Corollary~\ref{cor:polynomial-unit-mult} tells us that
    if we have \(q_i = 0\), then \(q_i\) is a unit,
    and by definition we only perform factorization up to units,
    so \(q_i\) should not be in the factorization.

    Suppose we also have another factorization into irreducibles
    \(f(x) = q_1' q_2' \hdots q_m'\), \(\deg q_i' > 0\).
    Since \(q_i\) and \(q_i'\) are all primitive,
    by Lemma~\ref{lem:polynomial-primitive-irreducible}
    they are also irreducible in \(F[x]\).
    We have shown earlier in Corollary~\ref{cor:field-polynomial-pid}
    that \(F[x]\) is a PID,
    which by Theorem~\ref{thm:pid-ufd} \(F[x]\) is a UFD.\@
    Then we know that the factorization is equal up to units in \(F[x]\),
    and without loss of generality we pair up the factors
    \(q_i = \gamma q_i'\) for some \(\gamma \in F\).
    But Corollary~\ref{cor:polynomial-unit-mult} once again tells us that
    \(\gamma \in R\) is a unit,
    which is exactly what we desire.

    Lastly we consider \(f(x)\) not primitive.
    Then \(f(x) = c(f)f_1(x)\) where \(f_1(x)\) is primitive.
    \(c(f) = p_1 \hdots p_m\) uniquely factors up to unit multiplication in \(R\),
    and the previous paragraph tells us \(f_1(x)\) does so too.
    This completes the proof for uniqueness.
\end{proof}

\begin{theorem}[Eisenstein's criterion]
    Suppose we have a polynomial with integer coefficients
    \(f(x) = a_n x^n + a_{n-1}x^{n-1} + \cdots + a_1 x + a_0 \in \bZ[x]\).
    If there exists a prime \(p\) such that
    \begin{enumerate}[label={(\alph*)}, itemsep=0mm]
        \item \(p \mid a_i\) for all \(0 \leq i \leq n-1\),
            \(p\) dividing all coefficients but the leading term;
        \item \(p \nmid a_n\) does not divide the leading term; and
        \item \(p^2 \nmid a_0\) the square does not divide the constant term,
    \end{enumerate}
    then \(f(x)\) is irreducible in \(\bQ[x]\).
\end{theorem}
\begin{proof}
    Suppose, by way of contradiction,
    that \(f(x)\) is reducible in \(\bQ[x]\).
    Then we can write \(f(x) = g(x)h(x)\),
    and we shall denote
    \begin{equation*}
        g(x) = c_k x^k + c_{k-1}x^{k-1} + \cdots + c_1 x + c_0 \qquad
        h(x) = d_\ell x^\ell + d_{\ell-1}x^{\ell-1} + \cdots + d_1 x + d_0
    \end{equation*}
    By condition (a) \& (c),
    only one of \(c_0\) and \(d_0\) are divisible by \(p\),
    so without loss of generality let \(p \mid c_0\), \(p \nmid d_0\).

    We claim that this implies \(p \mid c_i\) for all \(0 \leq i \leq k\).
    We already have our base case \(p \mid c_0\) from above,
    so by way of strong induction, suppose \(p \mid c_0,c_1,\hdots,c_{i-1}\).
    Since \(a_i = \sum_{j+j'=i} c_j d_{j'} = c_i d_0 + c_{i-1}d_1 + \cdots + c_0 d_i\),
    and every term in the sum except for the first term is divisible by \(p\),
    \(p \nmid d_0\), so \(p \mid c_i\).

    But \(p \mid c_i\) implies \(p \mid a_n = c_k d_\ell\),
    so this contradicts condition (b).
    Hence \(f(x)\) is irreducible in \(\bQ[x]\).
\end{proof}

\section{Vector Spaces}\label{sec:linear-algebra}\label{sec:vsp}

\begin{remark}
    We shall resume the following convention as in Section~\ref{sec:matrix-rings}:
    \begin{enumerate}[label={(\roman*)}, itemsep=0mm]
        \item \(a\) a lowercase symbol denotes a generic element in a set;
        \item \(\vec{a}\) an underlined symbol denotes a vector; and
        \item \(\vb{A}\) a bold symbol (usually uppercase)
            denotes a matrix or tensorial quantity.
    \end{enumerate}
\end{remark}

\subsection{Basic Definitions}

\begin{definition}
    A left module over a ring \(R\) is a quadruple \((M,+,\cdot,\vec{0})\)
    where \(\vec{0} \in M\) is a set equipped with addition \(+\)
    with an identity \(\vec{0}\)
    and scalar multiplication,
    with the following four properties:
    \begin{enumerate}[label={(\roman*)}, itemsep=0mm]
        \item \((M,+,\vec{0})\) forms an abelian group;
        \item scalar multiplication
            \(\vfunc{\cdot}{R \times M}{M}{(\alpha,\vec{m})}{\alpha\vec{m}}\),
            and in particular \(1\vec{m} = \vec{m}\);
        \item \(\forall\{\alpha,\beta\} \in R,\, \forall\{\vec{m},\vec{n}\} \subset M\),
            the distributive laws \((\alpha+\beta)(\vec{m}+\vec{n})
            = \alpha\vec{m}+\alpha\vec{n}+\beta\vec{m}+\beta\vec{n}\) hold; and
        \item associativity
            \(\forall\{\alpha,\beta\} \subset R,\,\forall\vec{m} \in M\),
            \((\alpha\beta)\vec{m} = \alpha(\beta\vec{m})\).
    \end{enumerate}
\end{definition}
\begin{definition}
    A vector space over a field \(F\) is similarly defined with \(R = F\).
    It is a quadruple \((V,+,\cdot,\vec{0})\)
    where \(\vec{0} \in V\) is a set equipped with addition \(+\)
    with an identity \(\vec{0}\)
    and scalar multiplication,
    with the following four properties:
    \begin{enumerate}[label={(\roman*)}, itemsep=0mm]
        \item \((V,+,\vec{0})\) forms an abelian group;
        \item scalar multiplication
            \(\vfunc{\cdot}{F \times V}{M}{(\alpha,\vec{v})}{\alpha\vec{v}}\),
            and in particular \(1\vec{v} = \vec{v}\);
        \item \(\forall\{\alpha,\beta\} \in R,\, \forall\{\vec{u},\vec{v}\} \subset M\),
            the distributive laws \((\alpha+\beta)(\vec{u}+\vec{v})
            = \alpha\vec{u}+\alpha\vec{v}+\beta\vec{u}+\beta\vec{v}\) hold; and
        \item associativity
            \(\forall\{\alpha,\beta\} \subset R,\,\forall\vec{v} \in M\),
            \((\alpha\beta)\vec{v} = \alpha(\beta\vec{v})\).
    \end{enumerate}
\end{definition}
\begin{proposition}
    Suppose \(F\) is a field,
    and \(F^X = \{\func{f}{X}{F}\}\) the set of all functions from \(X\) to \(F\).
    When equipped with pointwise addition and multiplication,
    this forms a vector space over \(F\).
\end{proposition}
\begin{proof}
    Since addition and multiplication is pointwise,
    all proofs essentially start with `for all \(x \in X\) \(f(x) \in F\)',
    so distributive laws and associativity laws are inherited from \(F\).
\end{proof}
\begin{corollary}
    \(F^n\) forms a vector space.
\end{corollary}
\begin{proof}
    Simply consider \(X = \{0,1,\hdots,n-1\}\) an \(n\)-element set.
\end{proof}

\begin{definition}
    Suppose \((V,+,\cdot,0,1)\) is a vector space.
    Then \(W\) is a subspace if \(W \subseteq V\)
    and its additive and multiplicative groups form subgroups.
\end{definition}

\begin{definition}
    For any sequence of vectors \({\{\vec{v}_i\}}_{i=1}^r\),
    finite summation is recursively defined as
    \begin{equation*}
        \sum_{i=1}^0 \vec{v}_i = \vec{0} \qquad
        \sum_{i=1}^{r+1} \vec{v}_i = \vec{v}_{r+i} + \sum_{i=1}^r \vec{v}_i \qquad
        \sum_{i=m}^n \vec{v}_i = \sum_{i=1}^n \vec{v}_i - \sum_{i=1}^{m-1} \vec{v}_i
    \end{equation*}
\end{definition}

\begin{definition}
    Suppose \(V\) is a vector space over \(F\)
    (not necessarily finite-dimensional),
    and \(S \subseteq V\) some subset.
    Then
    \begin{enumerate}[label={(\roman*)}, itemsep=0mm]
        \item a vector \(\vec{v} \in V\) (linearly) depends on \(S\)
            if there are finite sets \(\exists {\{\vec{w}_i\}}_{i=1}^r \subset V\),
            \(\exists {\{a_i\}}_{i=1}^r \subset F\),
            such that it can written as a linear combination
            \(\sum_{i=1}^r a_i \vec{w}_i\);
        \item the span (or the \(F\)-span), denoted \(\Span(S)\) or \(\Span_F(S)\),
            is the set of all vectors in \(V\) that depend on \(S\);
        \item \(S\) is (linearly) dependent if \(\exists \vec{v} \in S\)
            such that \(\vec{v}\) depends on \(S \setminus \{\vec{v}\}\),
            and it is (linearly) independent otherwise; and
        \item \(S\) forms a basis of \(V\)
            if \(S\) is (linearly) independent and \(V = \Span(S)\).
    \end{enumerate}
\end{definition}
\begin{lemma}\label{lem:intersection-subspace}
    Suppose \(V\) a vector space, and \(W_i \subseteq V\) subspaces.
    Then \(\bigcap_i W_i\) is also a subspace of \(V\).
\end{lemma}
\begin{proof}
    Apply Lemma~\ref{lem:intersection-subgroup} twice
    exactly like Lemma~\ref{lem:intersection-subring}.
\end{proof}
\begin{proposition}\label{prop:subset-generated-subspace}
    \(\Span(S) = \bigcap_{S \subset W \subset V} W\)
    where \(W\) is a subspace of \(V\).
\end{proposition}
\begin{proof}
    Apply Proposition~\ref{prop:subset-generated-subgroup} twice
    exactly like Proposition~\ref{prop:subset-generated-subring}.
\end{proof}

\begin{lemma}
    \(S\) is dependent if and only if 
    there are distinct \({\{\vec{v}_i\}}_{i=1}^r \subset S\)
    and coefficients \({\{a_i\}}_{i=1}^r \subset F\) that are not all 0
    such that \(\sum_{i=1}^r a_i \vec{v}_i = \vec{0}\).
\end{lemma}
\begin{proof}
    \begin{equation*}
        \vec{v} = \sum_{i=1}^r a_i \vec{v}_i
        \iff \vec{0} = -\vec{v} + \sum_{i=1}^r a_i \vec{v}_i
    \end{equation*}
\end{proof}

\begin{definition}
    Suppose \(U,V\) are vector spaces, and \(\func{f}{U}{V}\) a function.
    \(f\) is linear if \(f(a\vec{u}+\vec{v}) = af(\vec{u}) + f(\vec{v})\),
    i.e.\ it preserves both operations.
\end{definition}
\begin{definition}
    Vector space homomorphisms are linear maps.
\end{definition}
\begin{proposition}
    \(\End(V) = \Hom(V,V)\) endomorphisms of a vector space
    has both a ring structure and a vector space structure.
\end{proposition}
\begin{proof}
    Let \(\func{0}{V}{V}\) map every vector to \(\vec{0}\),
    and \(\func{1}{V}{V}\) be the identity map.
    Clearly if \(\func{f,g}{V}{V}\),
    \(af(\vec{x}) = f(a\vec{x})\) by the fact that it is a linear map,
    and addition is defined the obvious way.
    If we define ring multiplication as function composition,
    then by linearity we have distributivity.
\end{proof}


\subsection{Direct Sum}

\begin{definition}[Universal Property of Direct Sum]
    Suppose \({\{V_i\}}_{i \in I}\) is a family of vector spaces.
    The direct sum of vector spaces \(\bigoplus_{i \in I} V_i\)
    is the categorical coproduct of vector spaces, that is,
    \begin{enumerate}[label={(\roman*)}, itemsep=0mm]
        \item there exist embeddings \(\func{\iota_i}{V_i}{\bigoplus_{i \in I} V_i}\); and
        \item for any vector space \(X\),
            if \(\func{\phi_i}{V_i}{X}\) are homomorphisms,
            then there exists a unique \(\func{\bar{\phi}}{\bigoplus_{i \in I} V_i}{X}\)
            such that \(\phi_i = \bar{\phi}\circ\iota_i\).
    \end{enumerate}

    This can be represented by the logical statement
    \begin{equation*}
        \forall X\in\mathbf{Vect},\;
        \forall i \in I,\; \forall \phi_i\in\Hom(V_i,X),\;
        \exists! \phi\in\Hom\qty(\bigoplus_{i \in I} V_i,X),\;
        \phi_i = \bar{\phi}\circ\iota_i
    \end{equation*}
    and the following commutative diagram:
    \begin{center}
        \begin{tikzcd}
            V_i \arrow{r}{\phi_i} \arrow{d}{\iota_i} & X \\
            \bigoplus_{i \in I} V_i \arrow[dashrightarrow]{ru}[swap]{\exists! \bar{\phi}}
        \end{tikzcd}
    \end{center}
\end{definition}

\begin{definition}
    Suppose \(U,V\) are \(F\)-spaces.
    The external direct sum \(U \oplus V\)
    is a vector space \(U \times V\)
    equipped with coordinate-wise operations.
\end{definition}
\begin{lemma}
    \(U \oplus V\) indeed forms a vector space.
\end{lemma}
\begin{proof}
    The additive groups is a direct sum of (abelian) groups,
    which is (abelian) group.
    Check scalar multiplication by
    \begin{equation*}
        (ab)(\vec{u},\vec{v}) = ((ab)\vec{u},(ab)\vec{v})
        = (a(b\vec{u}),a(b\vec{v})) = a(b\vec{u},b\vec{v})
        = a(b(\vec{u},\vec{v}))
    \end{equation*}
\end{proof}
\begin{lemma}
    \(\dim_F(U \oplus V) = \dim_F U + \dim_F V\).
\end{lemma}
\begin{proof}
    Suppose \(B_U \subset U\) and \(B_V \subset V\) be bases.
    Let \(B = {\{(\vec{u},\vec{0})\}}_{\vec{u} \in B_U} \sqcup
    {\{(\vec{0},\vec{v})\}}_{\vec{v} \in B_V}\) be a disjoint union.
    We wish to prove that this forms a basis.
    \begin{equation*}
        \sum_i a_i(\vec{u}_i,\vec{0}) + \sum_j b_j(\vec{0},\vec{v}_j)
        = (\vec{0},\vec{0})
    \end{equation*}
    invokes linear independence by each of \(U\) and \(V\),
    so all \(a_i\) and \(b_j\) are zero.
    Then clearly \(B\) spans \((\vec{u},\vec{v})\)
    simply by span of \(B_U\) and \(B_V\).
\end{proof}

\begin{theorem}[Uniqueness of Direct Sum]
    The direct sum of vector spaces is associative and commutative up to isomorphism.
\end{theorem}
\begin{proof}
    \begin{center}
        \begin{tikzcd}
            V_i \arrow{r}{\iota_i'} \arrow{d}{\iota_i} &
            \bigoplus_{i \in I} V_i' \arrow[rightharpoondown, shift right=0.25ex]{ld}%
                [xshift=.5ex, yshift=-.5ex, swap]{\bar{\phi}'} \\
            \bigoplus_{i \in I} V_i \arrow[rightharpoondown, shift right=0.25ex]{ru}%
                [xshift=-.5ex, yshift=.5ex, swap]{\bar{\phi}}
            % F' \arrow[rightharpoondown, shift right=0.25ex]{ld}%
            %     [xshift=.5ex, yshift=-.5ex, swap]{\bar{\phi}'} \\
            % F \arrow[rightharpoondown, shift right=0.25ex]{ru}%
            %     [xshift=-.5ex, yshift=.5ex, swap]{\bar{\phi}}
        \end{tikzcd}
    \end{center}
    It is obvious from the universal property that
    the unique inclusions do not get affected by order or bracketing.
    Hence \(\bar{\phi}^{-1} = \bar{\phi}'\),
    and we have isomorphism.
\end{proof}


\subsection{Direct Product}

\begin{definition}[Universal Property of Direct Product]
    Suppose \({\{V_i\}}_{i \in I}\) is a family of vector spaces.
    The direct product of vector spaces \(\prod_{i \in I} V_i\)
    is the categorical product of vector spaces, that is,
    \begin{enumerate}[label={(\roman*)}, itemsep=0mm]
        \item there exists projections \(\func{\pi_i}{\prod_{i \in I} V_i}{V_i}\); and
        \item for any vector space \(X\), if \(\func{\phi_i}{X}{V_i}\) are homomorphisms,
            then there exists a unique \(\func{\bar{\phi}}{X}{\prod_{i \in I} V_i}\)
            such that \(\phi_i = \pi_i\circ\bar{\phi}\).
    \end{enumerate}

    This can be represented by the logical statement
    \begin{equation*}
        \forall X \in \mathbf{Vect},\;
        \forall i \in I,\;
        \forall \phi_i \in \Hom(X,V_i),\;
        \exists! \phi \in \Hom\qty(X,\prod_{i \in I} V_i),\;
        \phi_i = \pi_i \circ \bar{\phi}
    \end{equation*}
    and the following commutative diagram:
    \begin{center}
        \begin{tikzcd}
            V_i & X \arrow{l}[swap]{\phi_i}%
                \arrow[dashrightarrow]{ld}{\exists! \bar{\phi}} \\
            \prod_{i \in I} V_i \arrow{u}[swap]{\pi_i}
                % \arrow[dashrightarrow]{ru}[swap]{\exists! \bar{\phi}}
        \end{tikzcd}
    \end{center}
\end{definition}

\begin{theorem}[Uniqueness of Direct Product]
    The direct product of vector spaces is associative and commutative up to isomorphism.
\end{theorem}
\begin{proof}
    \begin{center}
        \begin{tikzcd}
            V_i & %
                % \arrow[dashrightarrow]{ld}{\exists! \bar{\phi}} \\
            % \prod_{i \in I} V_i \
            \prod_{i \in I} V_i' \arrow[rightharpoondown, shift right=0.25ex]{ld}%
                [xshift=.5ex, yshift=-.5ex, swap]{\bar{\phi}'} \arrow{l}[swap]{\pi_i'} \\
            \prod_{i \in I} V_i \arrow[rightharpoondown, shift right=0.25ex]{ru}%
                [xshift=-.5ex, yshift=.5ex, swap]{\bar{\phi}} \arrow{u}[swap]{\pi_i}
        \end{tikzcd}
    \end{center}
    Again, from the universal property,
    the unique projections do not get affected by order or bracketing.
    Hence we have an isomorphism between direct products.
\end{proof}

\begin{theorem}
    Suppose \({\{V_i\}}_{i=1}^n\) is a finite family of vector spaces.
    Then the direct sum and direct product of finitely many vector spaces are isomorphic.
    \begin{equation*}
        \bigoplus_{i=1}^n V_i \cong \prod_{i=1}^n V_i
    \end{equation*}
\end{theorem}
\begin{proof}
    The following projections and inclusions clearly give a surjection
    from the direct product to the direct sum.
    \begin{center}
        \begin{tikzcd}
            \prod_{i=1}^n V_i \arrow{r}{\pi_i} &
            V_i \arrow{r}{\iota_i} &
            \bigoplus_{i=1}^n V_i
        \end{tikzcd}
    \end{center}
    It suffices to prove that this is also injective.
    But the \hyperref[thm:pigeonhole]{pigeonhole principle}
    gives us that a surjection of finitely many vector spaces
    must also be an injection.
\end{proof}

\begin{remark}
    In infinite indices,
    the direct sum is a smaller space than the direct product.
    Choosing canonical coordinates,
    the direct sum can only have finite support with respect to its basis,
    but the direct product can have infinite support.
\end{remark}


\subsection{Quotient}

\begin{proposition}[Universal Property of Quotients]
    Suppose \(V\) is a vector space, and \(U \subseteq V\) is a subspace,
    and \(\func{\pi}{V}{U}\) is the quotient map as groups.
    \begin{enumerate}[label={(\alph*)}, itemsep=0mm]
        \item \(\pi\) is a linear map, and there exists a unique vector space structure on \(V/U\);
        \item For any vector space \(Z\), if we have \(f \in \Hom(V,Z)\) and \(\ker(f) \supseteq U\),
            then there exists a unique linear map \(\bar{f} \in \Hom(V/U,Z)\)
            such that \(\bar{f} \circ g = f\).
    \end{enumerate}

    This can be represented by the following commutative diagram:
    \begin{center}
        \begin{tikzcd}
            V \arrow{d}{\pi} \arrow{r}{f} & Z \\
            V/U \arrow[dashrightarrow]{ru}[swap]{\exists! \bar{f}}
        \end{tikzcd}
    \end{center}
\end{proposition}
\begin{proof}
    Suppose \(\vec{x} \in V/U\).
    If \(\vec{x} = \vec{v} + U + \vec{v}' + U\),
    then we can see that \(a\vec{v}-a\vec{v}' = a(\vec{v}-\vec{v}') \in U\),
    which allows us to conclude that \(\pi(a\vec{v}) = \pi(a\vec{v}') = a\pi(\vec{v})\)
    is well-defined.
    The additive axioms are clearly satisfied,
    so it suffices to check distributivity.
    \begin{equation*}
        \pi(a\vec{v}+a\vec{w}) = a\pi(\vec{v}) + a\pi(\vec{w})
        = a\pi(\vec{v} +\vec{w})
    \end{equation*}
\end{proof}

\begin{theorem}[First Isomorphism Theorem for Vector Spaces]\label{thm:iso-1-vsp}
    Suppose \(\func{\phi}{V}{W}\) is a vector space homomorphism,
    and \(U = \ker(\phi)\).
    We have:
    \begin{enumerate}[label={(\alph*)}, itemsep=0mm]
        \item \(U \subseteq V\), the kernel is a subspace;
        \item \(\phi(V) \subseteq W\), the image is a subspace; and
        \item \(\phi(V) \cong V/W\), the image is uniquely isomorphic to the quotient subspace.
    \end{enumerate}
\end{theorem}
\begin{theorem}[Third and Fourth Isomorphism Theorems for Vector Spaces]\label{thm:iso-3-vsp}\label{thm:iso-4-vsp}
    Suppose \(V\) is a vector space, \(U \subseteq V\) some subspace,
    and \(\func{\pi}{V}{V/U}\) is the quotient homomorphism.
    Then \(\pi\) is a bijection between the subspaces of \(V/U\)
    and the subspaces of \(V\) containing \(U\).
    In particular, if \(W\) is one such intermediate subspace,
    then \((V/U)/(W/U) \cong V/W\).
\end{theorem}
\begin{remark}
    Observe that vector spaces are modules over fields,
    so these isomorphism theorems are direct consequences of
    \hyperref[thm:iso-1-module]{isomorphism theorems on modules},
    which we will give a treatment in Section~\ref{sec:categories}.
    However, as the reader might be observed
    from the proofs of the isomorphism theorems
    for \hyperref[thm:iso-1-group]{groups} and \hyperref[thm:iso-1-ring]{rings},
    the direct proof of these theorems follow the exact same line of reasoning.
\end{remark}
% EDIT WHEN UNIVERSAL ALGEBRA WRITTEN


\subsection{Duality}

\begin{definition}
    Suppose \(V\) is an \(F\)-vector space.
    The dual vector space is \(V' = \Hom_F(V,F)\)
    the set of all homomorphisms into the base field.
    In some other fields of math the dual is sometimes denoted \(V^\ast\).
\end{definition}
\begin{lemma}
    Suppose \(W\) and \(V\) are \(F\)-vector spaces. Then
    \begin{enumerate}[label={(\alph*)}, itemsep=0mm]
        \item \(W^V\) is a vector space under coordinate-wise operations; and
        \item \(\Hom_F(W,V) \subseteq W^V\) is a subspace.
    \end{enumerate}
\end{lemma}
\begin{proof}
    \(W^V = \prod_{\vb{v} \in V} W\),
    which by the universal property of the direct product is a vector space.

    The homomorphisms are clearly a subset,
    so it suffices to prove that it is a vector space.
    The zero morphism is trivially linear,
    so we show that if \(f,g\) are linear,
    \begin{align*}
        (af+g)(b\vb{u}+\vb{v}) &= af(b\vb{u}+\vb{v}) + g(b\vb{u}+\vb{v})
        = baf(\vb{u}) + af(\vb{v}) + bg(\vb{u}) + g(\vb{v}) \\
        &= b(af(\vb{u}) + g(\vb{u})) + af(\vb{v}) + g(\vb{v})
        = b(af+g)(\vb{u}) + (af+g)(\vb{v})
    \end{align*}
\end{proof}
\begin{proposition}
    \({(V/U)}' = \{\phi \in V' : \phi\vert_U = \vb{0}\}\).
\end{proposition}
\begin{proof}
    Suppose \(\phi \in {(V/U)}'\).
    Then we can extend this to \(\phi\circ\pi\),
    which since it factors through \(V/U\),
    the kernel contains \(U\),
    so \({(V/U)}' \subseteq \{\phi \in V' : \phi\vert_U = \vb{0}\}\).
    On the other hand, if \(\phi\vert_U = \vb{0}\),
    by the universal property of quotients,
    we can find a corresponding morphism \(V/U \to F\),
    i.e.\ in \({(V/U)}'\).
\end{proof}
\section{Modules}


\part{Theory of Equations}
\section{Fields}

\begin{remark}
    We begin by recalling the definition of a field.
    Fields are quintuples \((F,+,\cdot,0,1)\) where
    \((F,+,0)\) and \((F^\star,\cdot,1)\) are both abelian groups,
    and multiplication distributes over addition.
\end{remark}
\begin{remark}
    Fields are commutative rings,
    so theorems in Section~\ref{sec:rings} will entirely hold,
    and we will be using those theorems throughout.
\end{remark}
\begin{remark}
    Much of our foundation will come from extending any integral domain \(R\)
    (in particular \(\bZ\))
    to its field of fractions \(F\) (in particular \(\bQ\)).
\end{remark}


\subsection{Field Extensions}

\begin{definition}
    Suppose \(F\) is a field, and \(F[x]\) its polynomial ring.
    Corollary~\ref{cor:field-polynomial-pid} tells us that
    \(F[x]\) is a domain,
    and hence we can construct its field of fractions,
    which we denote \(F(X) = \{\frac{f(x)}{g(x)} : f,g \in F[x], g \neq 0\}\).
    This is often called the transcendental extension of \(F\),
    or the univariate function field over \(F\).
\end{definition}

\begin{remark}
    Recall that if \(R\) is a commutative ring,
    \(P\) a prime ideal, and \(M\) a maximal ideal,
    then \(R/P\) \hyperref[prop:prime-quotient-domain]{is an integral domain}
    and \(R/M\) \hyperref[cor:maximal-quotient-field]{is a field}.
    But when \(F\) a field,
    \hyperref[prop:pid-maximal-is-prime]{prime ideals and maximal ideals are the same thing}.
\end{remark}

\begin{definition}
    Suppose \(F\) is a field.
    A field extension \(K/F\) is a field \(K \supseteq F\).
    We say \(F\) is a subfield of \(K\),
    where \(F\) is a base field, and \(K\) is an over field of \(F\).

    In commutative diagrams, we often denote this as
    \begin{center}
        \begin{tikzcd}
            K \arrow[dash]{d} \\ F
        \end{tikzcd}
    \end{center}
\end{definition}

\begin{proposition}\label{prop:field-extension-vsp}
    Suppose \(K/F\) a field extension.
    \(K\) can be characterized as a vector space over \(F\).
\end{proposition}
\begin{proof}
    Let \(k,\ell \in K\), and \(\lambda \in F\).
    Trivially by field addition \(k + \ell \in K\),
    and \(\lambda \in F \subseteq K\),
    so by field multiplication we have \(\lambda k \in K\).
    Moreover, distributivity holds because of
    \(\lambda(k+\ell) = \lambda k + \lambda \ell\)
    due to distributivity in \(K\).
\end{proof}

\begin{definition}
    Suppose \(K/F\) field extension.
    We call \(K/F\) finite or \(K\) is a finite extension of \(F\)
    if \(K\) is a finite-dimensional vector space over \(F\).
    If \(K/F\) finite, the degree of \(K/F\),
    denoted \([K:F] = \dim_F K\)
    is the dimension of \(K\) as an \(F\)-space.
\end{definition}
\begin{theorem}\label{thm:finite-extension-stack}
    Suppose \(K/F\) and \(L/K\) are finite field extensions.
    Then \(L/F\) is a finite extension,
    and in particular, \([L:F] = [L:K][K:F]\).
\end{theorem}
\begin{proof}
    Since \(K/F\) finite, let \({\{a_i\}}_{i=1}^m\) be a basis for \(K/F\),
    so \([K:F] = m\).
    For every \(a \in K\), we can write \(a = \sum_{i=1}^m \lambda_i a_i\)
    for some \(\lambda_i \in F\).
    Similarly since \(L/K\) finite,
    let \({\{b_j\}}_{j=1}^n\) be a basis for \(L/K\), so \([L:K] = n\).
    For every \(b \in L\), we can write \(b = \sum_{j=1}^n \mu_j b_j\)
    for some \(\mu_j \in K\).
    But then we have a linear combination of \(mn\) elements
    \begin{equation*}
        b = \sum_{j=1}^n \mu_j b_j
        = \sum_{j=1}^n \qty(\sum_{i=1}^m \lambda_{ij} a_i) b_j
        = \sum_{i,j=1}^{m,n} \lambda_{ij} (a_i b_j)
    \end{equation*}
    It suffices to check that \({{\{a_i b_j\}}_{i=1}^m}_{j=1}^n\) forms a basis.
    Suppose that
    \begin{equation*}
        0 = \sum_{i,j=1}^{m,n} \lambda_{ij} (a_i b_j)
        = \sum_{j=1}^n \qty(\sum_{i=1}^m \lambda_{ij} a_i) b_j
    \end{equation*}
    But by linear independence of \(b_j\),
    for each individual \(j\), \(\sum_{i=1}^m \lambda_{ij} a_i = 0\),
    and by linear independence of \(a_i\),
    for all \(i\), \(\lambda_{ij} = 0\).
    This proves linear independence.
\end{proof}
\begin{corollary}
    Finiteness is a property that is stackable,
    and degree of extension is multiplicative.
\end{corollary}
\begin{proof}
    Merely proceed by induction on the tower of field extensions,
    applying the \hyperref[thm:finite-extension-stack]{theorem above}.
\end{proof}

\begin{theorem}\label{thm:field-hom-injective}
    Every field homomorphism is injective.
\end{theorem}
\begin{proof}
    Suppose \(\func{\phi}{F}{K}\) is a field homomorphism.
    By way of contradiction, suppose \(\phi\) is not injective.
    Then there exists \(a,b \in F\) such that \(\phi(a) = \phi(b)\) and \(a \neq b\).
    But that tells us \(f(a-b) = 0 = f(0)\), and in particular,
    since \(a-b \neq 0\), \(f(a-b)f({(a-b)}^{-1}) = f(1) = 1\).
    But that implies \(0f({(a-b)}^{-1}) = 1\) which is a contradiction.
\end{proof}


\subsection{Algebraic Extensions}

\begin{definition}
    Suppose \(K/F\) is a field extension, and \(a \in K\) some element.
    We call \(a\) algebraic over \(F\)
    if there exists a (monic) polynomial \(f(x) \in F[x]\) such that \(f(a) = 0\).
    If \(a \in K\) is not algebraic over \(F\),
    we call \(a\) transcendental over \(F\).
\end{definition}

\begin{definition}
    Suppose \(a \in K\) is algebraic over \(F\).
    The minimal polynomial of \(a\) over \(F\),
    either denoted \(\min_F(a)\) or \(\min(a;F)\)
    is the irreducible polynomial \(f(x) \in F[x]\)
    with the lowest degree such that \(f(a) = 0\).
\end{definition}
\begin{proposition}
    A monic minimal polynomial exists and is unique for every algebraic \(a \in K\).
\end{proposition}
\begin{proof}
    By definition there must be some polynomial that it satisfies.
    If it is reducible, reduce it into irreducible components,
    and \(a\) must satisfy one of those irreducible components.
    It is unique since if another monic polynomial \(g(x)\) satisfies our conditions,
    \((g-f)(x)\) also satisfies our conditions,
    and since they are the same degree, \(\deg(g-f) < \deg f\) violates minimality.
\end{proof}
\begin{definition}
    Suppose \(a \in K\) is algebraic over \(F\).
    We define the degree of \(a\) to be the degree of its minimal polynomial,
    \(\deg a = \deg \min_F(a)\).
\end{definition}

\begin{definition}
    Suppose \(K/F\) is a field extension.
    If all elements \(a \in K\) are algebraic over \(F\),
    we call \(K/F\) an algebraic extension.
\end{definition}
\begin{definition}
    Suppose \(K/F\) is an extension, and \(a \in K\) some element.
    \(F(a) \subseteq K\) is the smallest subfield that contains \(F\)
    and all elements \(\lambda = \sum_i \ell_i a^i\) for some \(\ell_i \in F\).
\end{definition}
\begin{proposition}
    \(F(a)\) is the field of fractions of \(F[a]\).
\end{proposition}
\begin{proof}
    By definition of a field of fractions,
    \(F(a)\) is the smallest field that contains \(F[a]\).
\end{proof}
\begin{theorem}\label{thm:algebraic-finite-extension}
    Suppose \(K/F\) is a field extension, and \(a \in K\) some element.
    The following are equivalent:
    \begin{enumerate}[label={(\alph*)}, itemsep=0mm]
        \item \(a\) is algebraic over \(F\);
        \item \(F(a)\) is a finite extension of \(F\); and
        \item \(F[a] = F(a)\).
    \end{enumerate}
\end{theorem}
\begin{proof}
    We first prove that (a) implies (c).
    Since \(a\) is algebraic, there is some minimal polynomial
    \(a^n + \lambda_1 a^{n-1} + \cdots + \lambda_n = 0\) for \(\lambda_i \in F\).
    But we can rewrite the above as
    \begin{equation*}
        -\lambda_n = a^n + \lambda_1 a^{n-1} + \cdots + \lambda_{n-1}a
        = a(a^{n-1} + \lambda_1 a^{n-2} + \cdots + \lambda_{n-1})
    \end{equation*}
    and clearly \(\lambda_n \neq 0\),
    since otherwise \(a^{n-1} + \lambda_1 a^{n-2} + \cdots + \lambda_{n-1}\)
    would be a minimal polynomial of degree less than \(n\),
    violating minimality of the original polynomial.
    Hence we have a polynomial expression for the inverse
    \begin{equation*}
        a^{-1} = -\frac{1}{\lambda_n}
        (a^{n-1} + \lambda_1 a^{n-2} + \cdots + \lambda_{n-1})
    \end{equation*}
    so \(F(a) \subseteq F[a]\).
    \(F[a] \subseteq F(a)\) by definition
    and we have equality.
    
    We now prove that (c) implies (b).
    For any \(i \geq 0\), \(a^{n+i}\) is in the \(F\)-span of \({\{a^j\}}_{j=0}^{n-1}\).
    and since we write elements in \(F(a) = F[a]\)
    as linear combinations of powers of \(a\),
    the degree of those elements must be at most \(n\),
    hence \([F(a):F]\) is finite.

    Lastly we prove that (b) implies (a).
    Suppose \(F(a)\) is a finite extension.
    Then \({\{a^i\}}_{i=0}^\infty \subseteq F(a)\) is a linearly dependent set.
    We shall find the smallest \(ell\) such that
    \({\{a^i\}}_{i=0}^\ell\) is a linearly dependent set.
    By linear dependence we will have
    \(\lambda_0 + \lambda_1 a + \cdots + \lambda_\ell a^\ell = 0\)
    for some \(\lambda_i \in F\), \(\lambda_\ell \neq 0\).
    Hence we have found a degree \(\ell\) polynomial that \(a\) satisfies,
    and \(a\) must be algebraic.
\end{proof}
\begin{corollary}\label{cor:algebraic-finite-extension}
    Suppose \(K/F\) is a finite extension.
    Then \(K\) is algebraic over \(F\).
\end{corollary}
\begin{proof}
    Suppose \(a \in K\).
    Since we have \(F \subseteq F(a) \subseteq K\),
    \(F(a)/F\) must be finite,
    which by the \hyperref[thm:algebraic-finite-extension]{theorem above}
    must be algebraic.
\end{proof}
\begin{remark}
    We have shown that finite extensions are algebraic,
    but algebraic extensions are not necessarily finite.
\end{remark}

\begin{theorem}
    Suppose \(K/F\) is a field extension.
    The set of all \(F\)-algebraic elements in \(K\)
    form a subfield of \(K\) containing \(F\).
\end{theorem}
\begin{proof}
    Let \(a,b \in K\) be algebraic over \(F\).
    It is sufficient to prove that \(a \pm b\), \(ab\),
    and \(a/b\) when \(b \neq 0\) are all algebraic.
    Observe that \([F(a):F],[F(b):F]\) are finite by the
    \hyperref[thm:algebraic-finite-extension]{previous theorem}.
    Let \(F_1 = F(a) \subseteq K\).
    Then again, \([F_1(b):F]\) is also finite by Theorem~\ref{thm:finite-extension-stack}.
    This process is repeatable for finitely many elements.
    From this we can see that \(F(a)(b) = F(a,b)\)
    by minimality of the extension by an element.
    But then this tells us that \(a,b \in F(a,b)\) is a field,
    so \(a \pm b\), \(ab\), and \(a/b\) are all in \(F(a,b)\).
    and finite implies algebraic
    by the \hyperref[thm:algebraic-finite-extension]{theorem above}.
    % It is now sufficient to prove that \(\)
    % Now suppose \(S \subseteq K\) is a subset,
    % and let \(F(S)\) be the smallest subfield of \(K\) containing \(F\) and \(S\).
\end{proof}
\begin{definition}
    Suppose \(K/F\) is an extension.
    Then the subfield of all \(F\)-algebraic elements in \(K\)
    is usually denoted \(F^\text{alg} \subseteq K\).
    \(K/F^\text{alg}\) has no algebraic elements over \(F\),
    while \(F^\text{alg}/F\) is an algebraic extension.
\end{definition}

\begin{definition}
    Suppose \(K/F\) is an extension.
    \(K/F\) is finitely generated if there exists a finite set \(S \subseteq K\)
    such that \(K = F(S)\).
\end{definition}
\begin{theorem}\label{thm:finitely-gen-algebraic}
    Suppose \(K/F\) is finitely generated by \(S\).
    If all elements of \(S\) are algebraic, then \(K/F\) is algebraic.
\end{theorem}
\begin{proof}
    If \(S = {\{s_i\}}_{i=1}^n\),
    then merely apply Theorem~\ref{thm:algebraic-finite-extension}
    to see that \(F(S)/F\) is finite,
    so Corollary~\ref{cor:algebraic-finite-extension}
    tells us that it is algebraic.
\end{proof}
\begin{corollary}\label{cor:algebraic-extension-stack}
    Suppose \(K/F\) and \(L/K\) are algebraic extensions.
    Then \(L/F\) is algebraic.
\end{corollary}
\begin{proof}
    Suppose \(\lambda \in L\).
    Then there exists \(f(x) \in K[x]\) such that \(f(\lambda) = 0\).
    \(f(x) = x^n + b_1 x^{n-1} + \cdots + b_0\) where \(b_i \in K\).
    But notice that \(b_i \in K\) are algebraic over \(F\),
    so \(F(b_1,\hdots,b_n)/F \subseteq K\) is a finite extension
    by the \hyperref[thm:finitely-gen-algebraic]{theorem above}.
    We can see that \(f(x)\) is also a polynomial in \(F(b_1,\hdots,b_n)[x]\),
    which tells us that \(F(b_1,\hdots,b_n)(\lambda)/F(b_1,\hdots,b_n)\) is finite.
    But these two finite extensions combined gives us that
    \(F(b_1,\hdots,b_n)(\lambda)/F\) is finite by Theorem~\ref{thm:finite-extension-stack}.
    Since \(F(\lambda) \subseteq F(b_1,\hdots,b_n)\),
    \(F(\lambda)/F\) must also be finite,
    which implies \(\lambda\) is algebraic over \(F\)
    by Theorem~\ref{thm:algebraic-finite-extension}.
\end{proof}


\subsection{Splitting Fields and Algebraic Closures}

\begin{theorem}\label{thm:field-extension-gain-root}
    Suppose \(f(x) \in F[x]\) is an irreducible polynomial of degree at least 2,
    There exists an extension \(K/F\) such that \(f(x)\) has roots in \(K\).
\end{theorem}
\begin{proof}
    Let \(I = (f(x)) \subseteq F[x]\) be a maximal ideal,
    since \(f(x)\) is irreducible by Theorem~\ref{thm:ideal-divisibility}.
    Then we have a field \(K = F[x]/I \supseteq F\)
    by Corollary~\ref{cor:maximal-quotient-field},
    with degree \([K:F] = \deg f\) which we will say is \(n\).
    This induces a homomorphism \(\func{\pi\circ\iota}{F}{K}\)
    where \(\func{\iota}{F}{F[x]}\) is the inclusion of \(F\) into its polynomial ring,
    and \(\func{\pi}{F[x]}{K}\) is the quotient by \(I\),
    both of them ring homomorphisms.
    We can see that \(\pi\circ\iota\) is a field homomorphism,
    since elements in \(F\) within \(F[x]\) will not get collapsed by \(I\)
    as their degrees are always 1.
    Now see that \(\pi(f(x)) = \overline{f(x)} = 0\) in \(K\),
    so \(f(\bar{x}) = 0\), and \(\pi\) maps \(x \mapsto \bar{x}\),
    so \(\bar{x}\) is a root in \(K\),
    and \(K = \langle 1, x, \hdots, x^{n-1} \rangle\).
\end{proof}
\begin{corollary}
    If \(f(x) \in F[x]\) has degree \(n\),
    then there exists a field extension \(K/F\)
    with \([K:F] \leq n\) such that \(f(x)\) acquires a root.
\end{corollary}
\begin{proof}
    Split \(f(x)\) into irreducible components
    and then apply the \hyperref[thm:field-extension-gain-root]{theorem above}.
\end{proof}

\begin{definition}
    Suppose \(L_1/F\) and \(L_2/F\) are field extensions,
    where \(L_1,L_2 \subseteq L\).
    The composite extension or the compositum,
    denoted \(L_1 \cdot L_2\) or \(L_1 L_2\),
    is the smallest field extension of \(F\) in \(L\)
    that contains both \(L_1\) and \(L_2\).
\end{definition}

\begin{definition}
    A polynomial \(f(x) \in F[x]\) splits over an extension \(K/F\)
    if \(f(x) \in K[x]\) can be written as a product of linear polynomials,
    i.e.\ all roots exist in \(K\).
\end{definition}

\begin{definition}
    Suppose \(K/F\) is an extension.
    An automorphism of \(K\) is a field automorphism \(\sigma \in \Aut(L)\)
    that is bijective;
    an \(F\)-automorphism is a field automorphism \(\sigma \in \Aut_F(L)\),
    with the additional condition that it is the identity when restricted to \(F\),
    \(\sigma\vert_F = \id_F\).
\end{definition}
\begin{lemma}
    Suppose \(K/F\) is an extension,
    and \(\sigma \in \Aut_F(K)\) an \(F\)-automorphism.
    If \(a \in K\) is an element with a minimal polynomial \(\min_F(a) = f(x)\),
    then \(\sigma(a)\) is also a root of \(f(x)\).
\end{lemma}
\begin{proof}
    By definition \(a\) is a root of \(f(x)\), so \(f(a) = 0\).
    But \(\sigma(f(a)) = \sigma(0) = 0\),
    and notice that \(\sigma(f(x)) = f(x)\),
    since the coefficients are in \(F\) so they remain unchanged under \(\sigma\),
    so \(f(\sigma(a)) = 0\).
\end{proof}

\begin{definition}
    The splitting field of a polynomial \(f(x) \in F[X]\) over \(F\)
    is a field \(K\) such that
    \begin{enumerate}[label={(\roman*)}, itemsep=0mm]
        \item \(f(x)\) splits completely over \(K\);
        \item \(K \supseteq F\); and
        \item if \(K'\) satisfies the previous two conditions,
            then \(K' \supseteq K\).
    \end{enumerate}
\end{definition}
\begin{proposition}[Existence of Splitting Fields]\label{prop:splitting-field-existence}
    For any \(f(x) \in F[x]\), its splitting field exists.
\end{proposition}
\begin{proof}
    Without loss of generality suppose \(f(x)\) irreducible and monic.
    By Theorem~\ref{thm:field-extension-gain-root},
    we can find a field extension \(F_1/F\)
    such that \(f(x) = (x-a_1)f_1(x) \in F_1[x]\).
    Iterate on \(f_i(x)\) until \(f(x)\) is comprised of linear factors.
\end{proof}
\begin{corollary}
    Let \(f(x) \in F[x]\) be a polynomial of degree \(n\).
    Then there exists an extension \(K/F\) with degree \([K:F] \leq n!\)
    such that \(F\) splits completely over \(K\).
\end{corollary}
\begin{proof}
    Same proof as \hyperref[prop:splitting-field-existence]{proposition above},
    but notice that \([F_1:F] \leq n\),
    and then induct on \(n\), since \(\deg f_1(x) \leq n-1\).
\end{proof}

% \begin{definition}
%     Suppose \(F\) is a field.
%     An algebraic closure \(\overline{F}\) is a field that
%     every polynomial \(f(x) \in F[x]\) splits completely over \(\overline{F}\),
%     and every element is algebraic over \(F\).
% \end{definition}
\begin{lemma}
    Suppose \(F\) is a field.
    Then the following are equivalent:
    \begin{enumerate}[label={(\alph*)}, itemsep=0mm]
        \item The only algebraic extension of \(F\) is itself;
        \item The only finite extension of \(F\) is itself;
        \item If \(K/F\) is an extension,
            then \(F\) is the set of \(F\)-algebraic elements in \(K\);
        \item Every \(f(x) \in F[x]\) splits over \(F\); and
        \item Every \(f(x) \in F[x]\) has a root in \(F\).
    \end{enumerate}
\end{lemma}
\begin{proof}
    (a) implies (b) is the contrapositive of Corollary~\ref{cor:algebraic-finite-extension}.

    (b) implies (c) because if there exists another \(F\)-algebraic element \(a \in K-F\),
    then there is a finite extension \(F(a)/F\).

    (c) implies (d) since if \(f(x) \in F[x]\) and \(a\) is a root,
    then \(a\) is algebraic over \(F\), and must be in \(F\).
    Induct on order of \(f\).

    (d) implies (e) trivially true.

    (e) implies (d) simply by induction on degree of \(f\).

    (d) implies (a) since if there are other algebraic extensions,
    then that means there is some polynomial that does not have a root in \(F\).
\end{proof}
\begin{definition}
    \(F\) is algebraically closed if it satisfies any of the conditions above.
\end{definition}
% \begin{proposition}
%     Algebraic closures are algebraically closed.
% \end{proposition}
% \begin{proof}
%     It is sufficient to prove that \(\overline{\overline{F}} = \overline{F}\).
%     Suppose \(p(x) \in \overline{F}[x]\) is a polynomial
%     and let \(a\) be a root of \(p(x)\).
%     Let \(f(x) = \min_{\overline{F}}(a)\) be the minimal polynomial.
% \end{proof}

\begin{definition}
    If \(K\) is an algebraic extension of \(F\) and is algebraically closed,
    then \(K\) is an algebraic closure of \(F\).
\end{definition}
\begin{theorem}[Existence of Algebraic Closures]
    Suppose \(F\) is a field.
    Then there exists an algebraically closed field \(K \supseteq F\).
\end{theorem}
\begin{proof}
    We first attempt to construct an extension \(F_1/F\)
    such that every polynomial \(f(x) \in F[x]\) of degree at least 1
    has at least one root.
    For every \(f(x) \in F[x]\) let us assign an arbitrary letter
    (some unique indeterminate) \(X_f\),
    and let \(S\) be the set of all \(X_f\).
    We can get a bijection between \(S\) and \(F[x] - F\).
    We now get a multivariate polynomial ring \(F[S]\).

    We now wish to show that the ideal \(I\) generated by all \(f(X_f)\)
    is not the unit ideal \((1) = F[S]\).
    Suppose, by way of contradiction, that \(I = (1) = F[S]\).
    % TODO: see alg closed in 422
\end{proof}
\begin{corollary}
    Suppose \(F\) is a field,
    then there exists an algebraic extension \(F^\text{alg}/F\)
    that is algebraically closed.
\end{corollary}
\begin{proof}
    % TODO: see alg closed in 422
\end{proof}

\begin{proposition}
    Suppose \(\func{\sigma}{F}{L}\) is a field homomorphism,
    where \(L = \overline{L}\) iss algebraically closed.
    Let \(\alpha\) be algebraic over \(F\),
    \(p(x) = \min_F(a) = \sum_{i=0}^n a_i x^i\) a monic minimal polynomial,
    and \(\sigma(p)(x) = \sum_{i=0}^n \sigma(a_i)x^i\) its image.
    Then given a homomorphism \(\func{\tau}{F(\alpha)}{L}\)
    that is an extension of \(\sigma\) (\(\tau\vert_F = \sigma\)),
    the evaluation map \(\tau \mapsto \tau(\alpha)\)
    is a bijection between homomorphisms extending \(\sigma\)
    (\(\{\tau\in\Hom(F(\alpha),L): \tau\vert_F = \sigma\}\))
    and the roots of the image polynomial in \(L\)
    (\(\{\beta \in L : \sigma(p)(\beta) = 0\}\)).
\end{proposition}
\begin{proof}
    Let \(\func{\tau}{F(\alpha)}{L}\) extending \(\sigma\).
    Then \(\tau(p(\alpha)) = \sigma(p)(\tau(\alpha)) = 0\)
    since \(p(\alpha) = 0\).
    % TODO
\end{proof}

\begin{theorem}[Uniqueness of Splitting Fields]\label{thm:uniqueness-splitting-field}
    Suppose \(p(x) \in E[x]\) an irreducible polynomial,
    and \(K\) is a splitting field of \(p(x) \in E[x]\).
    Suppose \(\func{\phi}{E}{F}\) is a field isomorphism,
    and \(L\) is a splitting field of \(\phi(p)(x) \in F[x]\).
    Then \(\phi\) extends to an isomorphism \(\func{\sigma}{K}{L}\)
    where \(\sigma\vert_E = \phi\).
    In particular, splitting fields of \(p(x) \in F[x]\) over \(F\)
    are \(F\)-isomorphic.
\end{theorem}


\subsection{Separability}

\begin{definition}
    An (irreducible) polynomial \(f(x) \in F[x]\) of degree \(n\)
    is separable if \(f(x)\) has \(n\) distinct roots in its splitting field,
    or equivalently, if its irreducible factors are distinct and separable.
\end{definition}
\begin{proposition}\label{prop:derivative-inseparability}
    \(f(x) \in F[x]\) is inseparable
    if and only if \(f\) and \(f'\) share a root.
\end{proposition}
\begin{proof}
    Suppose \(f(x)\) is separable.
    Then we can write, in its splitting field,
    that \(f(x) = \prod_i (x-r_i)\) for all distinct \(r_i\).
    By the product rule,
    \(f'(x)\) has one term that is missing \(x-r_i\),
    and all other terms contain \(x-r_i\),
    Let us inspect \(f'(r_i)\).
    All the terms that contain \(r_i\) will evaluate to 0,
    while the term that has \(r_i\) missing will evaluate to nonzero.
    Hence \(0 = f(r_i) \neq f'(r_i)\).

    Now suppose \(f(x)\) is inseparable.
    Then there is some root \(r\) that has multiplicity \(m > 1\),
    so we write in a splitting field \(f(x) = {(x-r)}^m g(x)\).
    The formal derivative is then \(f'(x) = m{(x-r)}^{m-1} g(x) + {(x-r)}^m g'(x)\).
    Hence \(f(r) = f'(r) = 0\).
\end{proof}

\begin{definition}
    Suppose \(f(x) \in F[x]\) is a (monic) polynomial,
    and over a splitting field is written as
    \(f(x) = \prod_{i=1}^k {(x-r_i)}^{m_i}\)
    where \(r_i\) are all distinct and \(m_i \geq 1\).
    The multiplicity of a root \(r_i\) is \(m_i\).
    We call \(r_i\) a simple root if \(m_i = 1\),
    and we say \(f(x)\) has repeated roots if there exists \(i\)
    such that \(m_i > 1\).
\end{definition}

\subsubsection*{Resultant and Determinant}

\begin{remark}
    Historically the resultant is very important in determining separability,
    but in modern times the discriminant is more often used.
    Nevertheless, we shall include it here for sake of completeness.
\end{remark}
\begin{definition}
    Suppose \(f(x),g(x) \in F[x]\) are polynomials;
    \(f(x) = \sum_{i=0}^n a_i x^{n-i}\) and \(g(x) = \sum_{j=0}^m b_j x^{m-j}\),
    and the leading terms are nonzero.
    The resultant of \(f\) and \(g\) is
    \begin{equation*}
        \mathcal{R}(f,g) = a_0^m b_0^n \prod_{i,j} (\alpha_i - \beta_j)
    \end{equation*}
    where \(\alpha_i\) are the roots of \(f\),
    and \(\beta_j\) are the roots of \(g\).
\end{definition}
\begin{lemma}\label{lem:resultant-zero}
    \(\mathcal{R}(f,g) = 0\) if and only if \(f\) and \(g\) share a root.
\end{lemma}
\begin{proof}
    If they share a root then one of the terms in the product is 0.
\end{proof}
\begin{corollary}
    \(f\) is inseparable if and only if \(\mathcal{R}(f,f') = 0\).
\end{corollary}
\begin{proof}
    \(f\) and \(f'\) sharing a root is equivalent to
    the resultant being zero by the \hyperref[lem:resultant-zero]{lemma above}.
    Then apply Proposition~\ref{prop:derivative-inseparability}.
\end{proof}
\begin{remark}
    Notice that the resultant is closely related to the Vandermonde determinant.
    \begin{equation*}
        \mathcal{R}(f,g) = \det
        \begin{bmatrix}
            a_0 & a_1 & \cdots & a_n \\
            & a_0 & a_1 & \cdots & a_n \\
            & & \ddots & & & \ddots \\
            & & & a_0 & a_1 & \cdots & a_n \\
            b_0 & b_1 & \cdots & b_m \\
            & b_0 & b_1 & \cdots & b_m \\
            & & \ddots & & & \ddots \\
            & & & b_0 & b_1 & \cdots & b_m
        \end{bmatrix}
    \end{equation*}
    where you have \(m\) rows of \(a_i\) and \(n\) rows of \(b_i\),
    resulting in a \((m+n)\times(m+n)\) matrix.
\end{remark}

\begin{proposition}\label{prop:resultant-eval}
    Suppose \(f(x) = \sum_{i=0}^n a_i x^{n-i}\) and \(g(x) = \sum_{j=0}^m b_j x^{m-j}\).
    If \(f\) has roots \({\{\alpha_i\}}_{i=1}^n\)
    and \(g\) has roots \({\{\beta_j\}}_{j=1}^m\), then
    \begin{equation*}
        \mca{R}_{m,n}(f,g) = a_0^m \prod_{i=1}^n g(\alpha_i)
        = {(-1)}^{mn} b_0^n \prod_{j=1}^m f(\beta_j)
    \end{equation*}
\end{proposition}
\begin{proof}
    First assume \(\deg f = n\) and \(\deg g = m\).
    The base case of \(n = m = 0\) is obvious and there is nothing to prove.
    Let us proceed by way of strong induction,
    and assume this formula holds for all values smaller than \(n+m\).
    Observe from the Vandermonde determinant
    that we are simply performing row operations,
    so we have \(\mca{R}(f,g) = \mca{R}(f,r) = a_0^m \prod_{i=1}^n r(\alpha_i)\)
    as desired,
    treating \(r\) as a \(m\)-degree polynomial.
\end{proof}
\begin{remark}
    Note that the first part of the formula works when \(\deg g \leq m\),
    and similarly, the second part of the formula works when \(\deg f \leq n\).
\end{remark}

\begin{definition}
    Suppose \(f(x) \in F[x]\) is a polynomial.
    The discriminant is
    \begin{equation*}
        \mathcal{D}(f) = a_0^{2n-2} {(-1)}^{n(n-1)/2}
        \prod_{i \neq j} (\alpha_i - \alpha_j)
        = a_0^{2n-2} \prod_{i<j} {(\alpha_i - \alpha_j)}^2
    \end{equation*}
    where \(\alpha_i\) are roots.
\end{definition}
\begin{proposition}
    \(\mathcal{R}(f,f') = {(-1)}^{n(n-1)/2} a_0^{-1} \mathcal{D}(f)\).
\end{proposition}
\begin{proof}
    In some splitting field of \(f\),
    suppose we have the roots \({\{\alpha_i\}}_{i=1}^n\).
    Then we have \(f(x) = a_0 \prod_{i=1}^n (x-\alpha_i)\),
    and thus \(f'(x) = a_0 \sum_{i=1}^n \prod_{j \neq i} (x-\alpha_j)\)
    by the product rule.
    Hence \(f'(\alpha_i) = a_0 \prod_{j \neq i} (\alpha_j - \alpha_i)\),
    since all other terms disappear due to some \(\alpha_i - \alpha_i\) inside.
    By Proposition~\ref{prop:resultant-eval},
    \begin{equation*}
        \mca{R}(f,f') = a_0^{n-1} \prod_{i=1}^n f'(\alpha_i)
        = a_0^{n+(n-1)} \prod_{i=1}^n \prod_{j \neq i} (\alpha_j - \alpha_i)
        = a_0^{2n-1} \prod_{j \neq i} (\alpha_j - \alpha_i)
        = a_0^{-1} {(-1)}^{n(n-1)/2} \mca{D}(f)
    \end{equation*}
\end{proof}

\begin{proposition}
    The discriminant is invariant under translation.
\end{proposition}
\begin{proof}
    Since the discriminant is defined in terms of the difference between roots,
    if all roots are translated by the same amount,
    the differences remain the same.
\end{proof}

\subsubsection*{Separable Extensions}

\begin{definition}
    Suppose \(F\) is a field, and \(a\) algebraic over \(F\).
    \(a\) is a separable element if \(\min_F(a)\) is separable.
\end{definition}
\begin{definition}
    Suppose \(E/F\) is an algebraic extension.
    \(E/F\) is a separable extension if every \(a \in E\) is separable.
\end{definition}

\begin{lemma}
    Let \(f(x) = \min_F(a) \in F[x]\).
    \(f(x)\) is separable if and only if \((f,f') = 1\) are coprime.
\end{lemma}
\begin{proof}
    Let \(f(x)\) is separable.
    By way of contradiction,
    suppose \(g(x) \in F[x]\) is a monic irreducible polynomial of degree \(d\),
    where \(g \mid f\) and \(g \mid f'\).
    Let us write \(f(x) = g(x)h(x)\)
    Without loss of generality we may assume \((g,h) = 1\) are also coprime,
    since if they share factors, then send all those factors to \(g\),
    and that does not change irreducibility.
    But then \(f'(x) = g'(x)h(x) + h'(x)g(x)\),
    which implies that \(g(x) \mid g'(x)\)
    since \(g \mid h'g\), \(g \mid g'h\), and \(g \nmid h\).
    But that is not possible, since \(g'\) must have a lower degree that \(g\).
    Hence \((f,f') = 1\) are coprime.

    Now suppose \((f,f') = 1\).
    Notice they cannot share a root in the splitting field in this case,
    which by Proposition~\ref{prop:derivative-inseparability}
    means \(f(x)\) is separable.
\end{proof}

\begin{definition}
    Suppose \(E/F\) is an extension,
    and \(\func{\sigma}{F}{L}\) an embedding of fields.
    We denote the set of all homomorphisms \(\func{\sigma^\ast}{E}{L}\)
    extending \(\sigma\), that is, \(\sigma^\ast\vert_F = \sigma\) as
    \(S_\sigma = \{\sigma^\ast\in\Hom(E,L) : \sigma^\ast\vert_F = \sigma\}\).
\end{definition}
\begin{theorem}\label{thm:simple-extension-hom-count}
    Suppose \(E = F(\alpha)\) is an algebraic extension of \(F\),
    \(L = \overline{L}\) is an algebraically closed field extension of \(F\),
    % \(E = F(\alpha)\) is an algebraic extension of \(F\),
    and \(p(x) = \min_F(\alpha) \in F[x]\) is the minimal polynomial.
    If \(\func{\sigma}{F}{L}\) is an embedding,
    % that is, \(\sigma\vert_F = \id\).
    then \(\abs{S_\sigma}\),
    the number of embeddings of \(E\) into \(L\) extending \(\sigma\)
    is the number of distinct roots of \(\sigma(p)(x)\).
    In particular, \(\abs{S_\sigma} = [E:F]\)
    if and only if \(E/F\) is separable.
\end{theorem}
\begin{proof}
    % TODO: separability 1
\end{proof}

\begin{proposition}\label{prop:algebraic-extension-hom}
    Suppose \(E/F\) is an algebraic extension,
    \(L = \overline{L}\) algebraically closed,
    and \(\func{\sigma}{F}{L}\) an embedding of fields,
    Then there exists an embedding \(\func{\tau}{E}{L}\) that extends \(\sigma\),
    i.e.\ \(\tau\vert_F = \sigma\).
    If \(E\) is an algebraic closure of \(F\),
    and \(L\) is the algebraic closure of \(\sigma(F)\),
    then \(\tau\) is an isomorphism.
\end{proposition}
\begin{proof}
    % TODO: separability 1
\end{proof}
\begin{proposition}\label{prop:algebraic-extension-hom-count-bound}
    Suppose \(E/F\) is a finite extension,
    \(L = \overline{L}\) is algebraically closed,
    and \(\func{\sigma}{F}{L}\) an embedding of fields.
    % If \(E/F\) is a finite extension,
    There are at most \([E:F]\) different embeddings extending \(\sigma\),
    % extending \(\func{\sigma}{F}{L}\),
    % for \(L = \overline{L}\) algebraically closed.
    % That is, if \(S_\sigma = \{\sigma^\ast\in\Hom(E,L) : \sigma^\ast\vert_F = \sigma\}\),
    that is, \(\abs{S_\sigma} \leq [E:F]\).
\end{proposition}
\begin{proof}
    % TODO: separability 2
\end{proof}
\begin{proposition}\label{prop:intermediate-field-separable}
    Suppose \(F \subseteq E \subseteq L\) is a tower of field extensions.
    If \(L/F\) is separable, then \(L/E\) and \(E/F\) are separable.
\end{proposition}
\begin{proof}
    % TODO: separability 2
\end{proof}
\begin{theorem}\label{thm:algebraic-extension-hom-count}
    Suppose \(E/F\) is a finite extension,
    \(L = \overline{L}\) algebraically closed,
    and \(\func{\sigma}{F}{L}\) an embedding of fields.
    Then \(\abs{S_\sigma}\),
    the number of distinct embeddings of \(E\) into \(L\) extending \(\sigma\)
    is exactly \([E:F]\) if and only if \(E/F\) is separable.
\end{theorem}
\begin{proof}
    % TODO: separability 2
\end{proof}

\begin{theorem}[Transitivity of Separable Extensions]\label{thm:separable-transitive}
    Suppose \(F \subseteq E \subseteq L\) is a tower of field extensions.
    If \(E/F\) and \(L/E\) are both separable,
    then \(L/F\) is separable.
\end{theorem}
\begin{proof}
    % TODO: separability 2
\end{proof}

\subsubsection*{Separable Degree}

\begin{definition}
    Suppose \(E/F\) is a finite extension,
    \(L = \overline{L}\) algebraically closed,
    and \(\func{\sigma}{F}{L}\) an embedding of fields.
    We call the number of embeddings of \(E\) to \(L\) extending \(\sigma\)
    the separable degree of \(E/F\),
    denoted \({[E:F]}_s = \abs{S_\sigma}\).
\end{definition}
\begin{corollary}
    Suppose \(E/F\) is a finite extension.
    Then \({[E:F]}_s \leq [E:F]\).
\end{corollary}
\begin{proof}
    Restatement of Proposition~\ref{thm:algebraic-extension-hom-count}.
\end{proof}

\begin{definition}
    Suppose \(E/F\) an extension.
    The separable closure of \(F\) in \(E\) is \(F_s\),
    the field of all elements in \(E\) that are separable over \(F\).
\end{definition}
\begin{proposition}\label{prop:separable-degree}
    Suppose \(E/F\) an extension.
    The separable degree is \([F_s:F] = {[E:F]}_s\).
\end{proposition}
\begin{proof}
    % TODO: inseparability
\end{proof}
\begin{corollary}
    Suppose \(E/F\) a finite extension.
    Then \({[E:F]}_s \mid [E:F]\).
\end{corollary}
\begin{proof}
    Since \(F \subseteq F_s \subseteq E\) a tower of field extensions,
    \([E:F] = [E:F_s][F_s:F]\) by Theorem~\ref{thm:finite-extension-stack}.
    By \hyperref[prop:separable-degree]{proposition above}
    \([F_s:F] = {[E:F]}_s\), which divides \([E:F]\).
\end{proof}

\begin{definition}
    Suppose \(E/F\) a finite extension,
    and \(F_s\) the separable closure.
    Then the degree of inseparability is \({[E:F]}_{pi} = [E:F_s]\).
    Hence by definition \([E:F] = {[E:F]}_{pi}{[E:F]}_s\).
\end{definition}
\begin{proposition}
    Suppose \(F \subseteq L \subseteq E\) is a tower of finite extensions.
    Then \({[E:F]}_{pi} = {[E:L]}_{pi}{[L:F]}_{pi}\)
    and \({[E:F]}_s = {[E:L]}_s{[L:F]}_s\).
\end{proposition}
\begin{proof}
    % TODO: inseparability
\end{proof}

\subsubsection*{Simple Extensions}

\begin{definition}
    Suppose \(E/F\) is an extension.
    \(E/F\) is simple (algebraic)
    if \(E = F(\alpha)\) (for \(\alpha\) algebraic over \(F\)).
    We call \(\alpha\) a primitive element of \(E/F\).
\end{definition}
\begin{theorem}[Primitive Element Theorem]\label{thm:primitive-element}
    Suppose \(E/F\) is a finite separable extension.
    Then \(E/F\) is simple.
\end{theorem}
\begin{proof}
    % TODO: wikipedia
\end{proof}


\subsection{Inseparability}

\begin{remark}
    Recall from Theorem~\ref{thm:field-unique-prime-char}
    that fields come in 2 flavours,
    either characteristic 0 or prime \(p\).
\end{remark}

\begin{proposition}\label{prop:finite-extension-finite-field}
    Let \(F = \bF_p\).
    Suppose \(E/F\) is a finite extension where \([E:F] = n\).
    Then \(\abs{E} = p^n\).
\end{proposition}
\begin{proof}
    Conclusion is obvious when we view \(E\) as an \(F\)-space.
\end{proof}

\begin{definition}
    Suppose \(F\) is a field with characteristic \(p\).
    The Frobenius homomorphism is the map \(\vfunc{\Phi}{F}{F}{x}{x^p}\).
\end{definition}
\begin{proposition}\label{prop:frobenius-hom}
    The Frobenius homomorphism is an endomorphism,
    and is an automorphism if the field is finite.
\end{proposition}
\begin{proof}
    \begin{gather*}
        \Phi(x+y) = {(x+y)}^p = x^p + y^p = \Phi(x) + \Phi(y)
        \Phi(xy) = {(xy)}^p = x^p y^p = \Phi(x)\Phi(y)
    \end{gather*}
    This proves endomorphism.
    But field homomorphisms are always injective
    by Theorem~\ref{thm:field-hom-injective},
    and so by the \hyperref[thm:pigeonhole]{pigeonhole principle}
    it is also surjective,
    which proves automorphism.
\end{proof}
\begin{corollary}
    Any nonzero element \(a \in \bF_p^\star\) is a \(p\)th power.
\end{corollary}
\begin{proof}
    Simple restatement of the \hyperref[prop:frobenius-hom]{proposition above}.
\end{proof}

\begin{proposition}\label{prop:separable-power-polynomial}
    Suppose \(F\) is a field of characteristic \(p\),
    and \(f(x) \in F[x]\) an irreducible polynomial.
    Then there exists some \(g(x) \in F[x]\) irreducible and separable
    such that \(f(x) = g(x^{p^m})\) for some unique \(m \geq 0\).
\end{proposition}
\begin{proof}
    % TODO: inseparability
\end{proof}
\begin{definition}
    We call the degree of \(g\) the inseparable degree of \(f(x)\),
    and \(p^m\) the separable degree of \(f(x)\).
\end{definition}
\begin{proposition}
    Suppose \(E = F(\alpha)\) is an extension of \(F\),
    and \(f(x) = \min_F(\alpha)\).
    Then the definitions of separable and inseparable degrees
    for the polynomial and the field coincide,
    that is, \({[E:F]}_s\) is the separable degree of \(f(x)\),
    and \({[E:F]}_{pi}\) is the inseparable degree of \(f(x)\).
\end{proposition}
\begin{proof}
    % TODO: inseparable
\end{proof}

\begin{proposition}
    Suppose \(F\) is a field.
    Then all roots of \(f(x)\) have the same multiplicity.
    In particular,
    \begin{enumerate}[label={(\alph*)}, itemsep=0mm]
        \item if \(F\) has characteristic 0, the multiplicity is 1; and
        \item if \(F\) has characteristic \(p\),
            the multiplicity is \(p^n\) for some \(n \in \bN\)
    \end{enumerate}
\end{proposition}
\begin{proof}
    If \(F\) has characteristic 0,
    we know that the only irreducible polynomials are linear ones.
    They clearly have one root of multiplicity 1.

    Now consider the case of characteristic \(p\).
    Let \(\{\alpha,\beta\}\subset\overline{F}\) be roots of \(f(x)\).
    Consider the extensions \(F(\alpha)\) and \(F(\beta)\),
    and the homomorphism \(\vfunc{\sigma}{F(\alpha)}{F(\beta)}{\alpha}{\beta}\)
    where \(\sigma\vert_F = \id\).
    By Proposition~\ref{prop:algebraic-extension-hom},
    we can extend \(\sigma\) to \(\tau\in\Aut(\overline{F})\).
    Let \(f(x) = {(x-\alpha)}^m h(x) \in \overline{F}[x]\), \(h(\alpha) \neq 0\).
    Then \(f(x) = \tau(f)(x) = {(x-\beta)}^m \tau(h)(x)\).
    Therefore multiplicity of \(\alpha\) at most the multiplicity of \(\beta\).
    But this argument is symmetric,
    so the multiplicity of \(\alpha\) is also at least multiplicity of \(\beta\).
    Hence the multiplicities are equal.

    Now suppose \(f(x)\) irreducible.
    Note that from Proposition~\ref{prop:separable-power-polynomial}
    we can find \(g(x) \in F[x]\) such that \(f(x) = g(x^{p^m})\).
    If \(a\) is a root of \(f(x)\), then \(a^{p^m}\) is a root of \(g(x)\),
    and hence \(a\) has multiplicity \(p^m\)
    since \(x^{p^m} - a^{p^m} = {(x-a)}^{p^m}\).
\end{proof}

\begin{definition}
    Suppose \(F\) is a field with characteristic \(p\),
    and \(E/F\) an algebraic extension.
    An element \(\alpha \in E\) is purely inseparable
    if \(\min_F(\alpha)\) has only one distinct root.
    % if there exists some \(k \geq 1\) such that \(a^{p^k} \in F\).
\end{definition}
\begin{definition}
    Suppose \(E/F\) an algebraic extension.
    \(E/F\) is purely inseparable
    if all elements in \(E\) are purely inseparable over \(F\).
\end{definition}
\begin{lemma}\label{lem:purely-inseparable-power}
    Suppose \(F\) is a field with characteristic \(p\).
    If \(\alpha\) is algebraic over \(F\),
    then \(\alpha\) is purely inseparable over \(F\)
    if and only if \(\alpha^{p^n} \in F\) for some \(n\).
    In this case, \(\min_F(\alpha) = {(x-\alpha)}^{p^n}\).
\end{lemma}
\begin{proof}
    % TODO: inseparable
\end{proof}

\begin{theorem}
    Suppose \(K/F\) is an algebraic extension.
    If \(\alpha \in K\) is both separable and purely inseparable,
    then \(\alpha = F\).
\end{theorem}
\begin{proof}
    % TODO: inseparable
\end{proof}
\begin{theorem}
    If \(K/F\) is purely inseparable, then \(K/F\) is normal.
    If \(K/F\) is purely inseparable, finite, and \(F\) has characteristic \(p\),
    then \([K:F] = p^n\) for some \(n\).
\end{theorem}
\begin{proof}
    % TODO
\end{proof}
\begin{theorem}
    Suppose \(X\) a set, and \(K = F(X)\),
    with alll \(\alpha \in X\) purely inseparable.
    Then \(K\) is purely inseparable.
\end{theorem}
\begin{proof}
    % TODO
\end{proof}
\begin{theorem}
    Suppose \(F \subseteq L \subseteq K\) are a tower of extensions.
    \(K/F\) is purely inseparable if and only if
    \(K/L\) and \(L/F\) are purely inseparable.
\end{theorem}
\begin{proof}
    % TODO
\end{proof}

% TODO: separable and inseparable closures


\subsection{Perfect Fields and Normal Extensions}

\subsubsection*{Perfect Fields}

\begin{definition}
    Suppose \(F\) is a field.
    \(F\) is perfect if every algebraic extension of \(F\) is separable.
\end{definition}
\begin{proposition}
    Suppose \(F\) is a field with characteristic \(p\).
    \(F\) is perfect if and only if \(F^p = F\),
    i.e.\ the Frobenius endomorphism \(\vfunc{\Phi_p}{F}{F}{x}{x^p}\) is bijective.
\end{proposition}
\begin{proof}
    % TODO
\end{proof}

\begin{proposition}
    Algebraically closed fields are perfect.
\end{proposition}
\begin{proof}
    % TODO
\end{proof}
\begin{theorem}
    Finite fields are perfect.
\end{theorem}
\begin{proof}
    % TODO
\end{proof}

\begin{lemma}
    Suppose \(F\) is a field with characteristic \(p\).
    If \(a \in F - F^p\), then \(x^p - a\) is irreducible and inseparable.
\end{lemma}
\begin{proof}
    % TODO
\end{proof}
\begin{lemma}
    Suppose \(F\) is a field with characteristic \(p\).
    If \(a \in F^p\), then \(x^p - a\) is reducible.
\end{lemma}
\begin{proof}
    % TODO
\end{proof}
\begin{proposition}
    Suppose \(p\) is a prime,
    and \(x^p - 1\) splits completely over a field \(K\).
    Let \(L/K\) be an extension, with \(a \in L\) an algebraic element over \(K\),
    and the prime \(p \nmid [K(a):K]\) does not divide the degree of the extension.
    Then \(K(a) = K(a^p)\).
\end{proposition}
\begin{proof}
    % TODO
\end{proof}
\begin{theorem}
    A field \(K\) is perfect
    if and only if its characteristic is 0,
    or if its characteristic is \(p\) and \(K^p = K\).
\end{theorem}
\begin{proof}
    % TODO
\end{proof}

\subsubsection*{Normal Extensions}

\begin{definition}
    Suppose \(K/F\) is an algebraic extension.
    \(K/F\) is normal if \(f(x) \in F[x]\) irreducible over \(F\)
    and has a root in \(K\) implies \(f(x)\) splits completely in \(K\).
\end{definition}
\begin{remark}
    In general, normality is not a transitive property,
    meaning that it does not stack.
\end{remark}

\begin{theorem}
    Suppose \(F \subseteq K \subseteq L\) is a tower of extensions.
    If \(L/F\) is normal, then \(L/K\) is normal.
\end{theorem}
\begin{proof}
    % TODO
\end{proof}

\begin{lemma}
    Suppose \(F \subseteq E_i \subseteq E\) are two towers of extensions with \(i \in \{1,2\}\).
    If \(E_1/F\) and \(E_2/F\) are normal,
    then \((E_1 \cdot E_2)/F\) and \((E_1 \cap E_2)/F\) are normal.
\end{lemma}
\begin{proof}
    % TODO
\end{proof}
\begin{lemma}
    Suppose \(F\) is a field, and \(K/F\) is an algebraic extension.
    Let \(I\) be some index set, and \(i \in I\),
    with \(F \subseteq E_i \subseteq K\) are intermediate extensions
    such that \(E_i/F\) is normal.
    Then \(\bigcap_{i \in I} E_i\) is a normal extension of \(F\).
\end{lemma}
\begin{proof}
    % TODO
\end{proof}
\begin{lemma}
    Suppose \(K/F\) is a normal extension,
    and \(K_\text{sep}\) denotes the separable closure of \(F\) in \(K\).
    Then \(F \subseteq K_\text{sep} \subseteq K\) is a tower of normal extensions.
\end{lemma}
\begin{proof}
    % TODO
\end{proof}

\begin{proposition}
    Let \(F \subseteq K \subseteq L\) be a tower of field extensions.
    \begin{enumerate}[label={(\alph*)}, itemsep=0mm]
        \item If \(K/F\) is normal, and \(\tau \in \Aut_F(L)\) is an \(F\)-automorphism,
            then \(\tau\vert_K \in \Aut_F(K)\).
        \item If \(L/F\) is normal, then any \(\sigma \in \Hom_F(K,L)\)
            extends to and \(F\)-automorphism over \(L\).
    \end{enumerate}
\end{proposition}
\begin{proof}
    % TODO
\end{proof}

\begin{theorem}\label{thm:normal-extension-equivalence}
    Suppose \(E/F\) is an algebraic extension, with \(E \subseteq \overline{F}\).
    The following are equivalent:
    \begin{enumerate}[label={(\alph*)}, itemsep=0mm]
        \item Every \(\sigma\in\Hom_F(E,\overline{F})\) is also \(\sigma\in\Aut_F(E)\);
        \item \(E\) is a splitting field of a family of polynomials in \(F[x]\); and
        \item \(E/F\) is normal.
    \end{enumerate}
\end{theorem}
\begin{proof}
    % TODO
\end{proof}


\subsection{Finite Fields}

\begin{proposition}[Existence of Finite Fields]\label{prop:existence-finite-fields}
    Any finite field must be of order \(p^n\) for some prime \(p\),
    and such fields exists for all \(n \in \bN\).
\end{proposition}
\begin{proof}
    Suppose \(F\) is a finite field of characteristic \(p\).
    Treating \(F\) as an additive group, \(1 \in F\), \(p \mid \abs{F}\).
    Suppose, by way of contradiction,
    that there exists another prime \(q \neq p\) such that \(q \mid \abs{F}\).
    Then by \hyperref[thm:cayley]{Cauchy's theorem},
    there exists \(x \in F\) with order \(q\) so \(qx = 0\).
    By \hyperref[thm:bezout]{B\'ezout's identity},
    there exists some \(a,b\) such that \(ap + bq = 1\).
    Hence \(x = (ap+bq)x = a(px) + b(qx) = 0\),
    which is contradiction as \(0\) does not have order \(q\).

    We can then apply Proposition~\ref{prop:finite-extension-finite-field}
    and simply have \(\bF_p(x)\), with \(x^n \in 1\) to be our finite field.
\end{proof}

\begin{lemma}\label{lem:finite-mult-group-cyclic}
    Suppose \(K\) is a field,
    and \(G \subseteq K^\ast\) a finite subgroup of the multiplicative group.
    Then \(G\) is cyclic.
\end{lemma}
\begin{proof}
    Let \(\abs{G} = n\), and \(m = \exp(G)\).
    By \hyperref[cor:order-element-group]{Lagrange's theorem}, \(m \mid n\).
    If \(g \in G\), then \(g^m = 1\), so each element of \(G\) is a root of \(x^m - 1\).
    This polynomial has at most \(m\) roots on \(K\),
    which implies all the elements of \(G\) are roots of \(x^m - 1\), so \(n \leq m\).
    Hence we have \(\exp(G) = \abs{G}\) so \(G\) is cyclic.
\end{proof}
\begin{corollary}
    If \(K/F\) is an extension of finite fields,
    then \(K\) is a simple extension of \(F\).
\end{corollary}
\begin{proof}
    Consider \(\abs{K} = p^n\), so \(\abs{K^\ast} = p^n - 1\).
    If \(a \in K^\ast\), then \(a^{p^{n-1}} = 1\).
    Hence \(a^{p^n} = a\) for all \(a \in K\).
    \(K\) is then the splitting field of \(x^{p^n}-x\) over \(\bF_p\),
    and is normal over \(\bF_p\) by Theorem~\ref{thm:normal-extension-equivalence}.
    The \hyperref[prop:derivative-inseparability]{derivative test} yields separability.

    Conlcude from the \hyperref[thm:primitive-element]{primitive element theorem}
    and the \hyperref[lem:finite-mult-group-cyclic]{lemma above},
    since cyclic groups have a single generator.
\end{proof}
\begin{remark}
    In fact the primitive element
    does not have to be a generator of the multiplicative group.
\end{remark}
\begin{corollary}[Uniqueness of Finite Fields]\label{cor:uniqueness-finite-field}
    Finite fields of the same cardinality are unique up to isomorphism.
\end{corollary}
\begin{proof}
    Any two fields of order \(p^n\) are splitting fields over \(\bF_p\) of \(x^{p^n} - x\).
    Conclude by \hyperref[thm:uniqueness-splitting-field]{uniqueness of splitting fields}.
\end{proof}

\begin{lemma}\label{lem:finite-field-quotient}
    The finite field \(\bF_{p^n}\) is expressible as \(\bF_p[x]/(f)\)
    for some \(f(x) \in \bF_p[x]\) of degree \(n\).
\end{lemma}
\begin{proof}
    Since Lemma~\ref{lem:finite-mult-group-cyclic}
    tells us that \(\bF_{p^n}^\ast\) is cyclic,
    we pick \(\theta \in \bF_{p^n}\) to generate it.
    Hence \(\theta\) generates \(\bF_{p^n}\) over \(\bF_p\),
    and it will satisfy some irreducible polynomial \(f \in \bF_p[x]\).
    Then the evaluation map \(\vfunc{\phi}{\bF_p[x]}{\bF_{p^n}}{x}{\theta}\)
    induces an isomorphism \(\bF_p[x]/(f) \cong \bF_{p^n}\),
    since \(\deg f = n\).
\end{proof}
\begin{corollary}
    Suppose \(F\) is a finite field,
    and \(f(x) \in F[x]\) is a monic irreducible polynomial of degree \(n\).
    Then if \(a\) is a root of \(f\) in an extension of \(F\),
    then \(F(a)\) is the splitting field of \(f\).
    Consequently, if \(K\) is a splitting field of \(f\),
    then \([K:F] = n\).
    If \(\abs{F} = q\), then the roots are \(a^{q^r}\) for \(r \geq 1\).
\end{corollary}
\begin{proof}
    Let \(K\) be a splitting field of \(f\).
    If \(a \in K\) is a root,
    then \(F(a)\) is an extension of \(F\) of degree \(n\).
    By irreducibility \(f(x) = \min_F(a)\),
    and the minimal polynomial splits in \(F(a)\).
    Hence \(K = F(a)\) is degree \(n\).

    Lastly, we see that \(\min_F(a) = {(x-a)}^{p^n}\)
    by Lemma~\ref{lem:purely-inseparable-power},
    and clearly \(a^q - a = 0\).
\end{proof}

\begin{proposition}
    For some prime \(p\) and \(n,m \in \bN\),
    \(m \mid n \iff \bF_{p^m} \subseteq \bF_{p^n}\).
\end{proposition}
\begin{proof}
    \(\bF_{p^m}\) is the set of roots of \(x^{p^m} - x\)
    by the \hyperref[cor:uniqueness-finite-field]{uniqueness of finite fields}
    in some algebraic closure \(\overline{\bF}_p\).
    Note that if \(m \mid n\), then \(n = dm\), so
    \begin{equation*}
        x^{p^n} = x^{p^{dm}} = {(x^{p^{(d-1)m}})}^{p^m} = x^{p^{(d-1)m}}
        = \cdots = x^{p^m} = x
    \end{equation*}
    so \(x\) is a root of \(x^{p^n} - x\), and \(x \in \bF_{p^n}\).

    Conversely, suppose \(\bF_{p^m} \subseteq \bF_{p^n}\).
    Then \(\bF_{p^n}\) can be treated as a \(\bF_{p^m}\)-vector space of dimension \(d\),
    by Proposition~\ref{prop:field-extension-vsp}.
    Hence \(p^n = \abs{\bF_{p^n}} = \abs{\bF_{p^m}}^d = {(p^m)}^d = p^{md}\),
    and \(m \mid n\).
\end{proof}

\begin{proposition}
    Let \(n \in \bN\).
    Then \(x^{p^n} - x\) factors over \(\bF_p\)
    into the product of all monic irreducible polynomials over \(\bF_p\)
    of degree \(m\), where \(m \mid n\).
\end{proposition}
\begin{proof}
    % TODO: finite fields
\end{proof}

\begin{lemma}
    Let \(f \in \bF_q[x]\) be an irreducible polynomial over some finite field \(\bF_q\),
    and let \(\alpha\) be a root of \(f\) in some extension.
    Then, for a polynomial \(h \in \bF_q[x]\),
    \(h(\alpha) = 0\) if and only if \(f \mid h\).
\end{lemma}
\begin{proof}
    % TODO
\end{proof}
\begin{lemma}
    Let \(f \in \bF_q[x]\) be an irreducible polynomial of degree \(m\).
    Then \(f \mid x^{q^n} - x\) if and only if \(m \mid n\).
\end{lemma}
\begin{proof}
    % TODO
\end{proof}
\begin{corollary}
    If \(N_q(d)\) is the number of monic irreducible polynomials
    in \(\bF_q[x]\) of degree \(d\), then
    \begin{equation*}
        q^n = \sum_{d \mid n} d N_q(d) \qquad \forall n \in \bN
    \end{equation*}
    where we are summing over all positive \(d \mid n\).
\end{corollary}
\begin{proof}
    % TODO
\end{proof}

\section{Galois Theory}\label{sec:galois}

\subsection{Galois Extensions}

\begin{definition}
    A field extension \(K/F\) is a Galois extension
    if it is algebraic, separable and normal.
\end{definition}
\begin{definition}
    The group of automorphisms that fix \(F\)
    is the the Galois group \(\Gal(K/F) = \Aut_F(K)\).
\end{definition}

\begin{lemma}
    Suppose \(K = F(S)\) is an extension generated by a set.
    If \(\sigma,\tau \in \Gal(K/F)\) and \(\sigma\vert_S = \tau\vert_S\),
    then \(\sigma  = \tau\).
\end{lemma}
\begin{proof}
    Write element of \(K\) as linear combination,
    apply linearity of automorphism.
\end{proof}
\begin{lemma}\label{lem:gal-permute-roots}
    Let \(\tau \in \Hom_F(K,L)\), and \(\alpha \in K\) an algebraic element.
    If \(f(x) \in F[x]\) and \(f(\alpha) = 0\),
    then \(f(\tau(\alpha)) = 0\).
    Therefore \(\tau\) permutes the roots of \(\min_F(\alpha)\),
    and \(\min_F(\alpha) = \min_F(\tau(\alpha))\).
\end{lemma}
\begin{proof}
    Simply by definition of an \(F\)-homomorphism,
    \(0 = f(\alpha) = \tau(f(\alpha)) = f(\tau(\alpha))\).
\end{proof}
\begin{corollary}
    \([K:F]\) finite implies \(\abs{\Gal(K/F)}\) finite.
\end{corollary}
\begin{proof}
    Suppose \(K = F(\alpha_1,\hdots,\alpha_n)\).
    Then \(\sigma \in \Gal(K/F)\) determined completely by \(\sigma(\alpha_i)\),
    and each one of these have finitely many possible images
    by Lemma~\ref{lem:gal-permute-roots}.
\end{proof}

\begin{definition}
    Suppose \(K\) a field, and \(G\) some group.
    A linear \(K\)-character on \(G\) is a homomorphism \(\func{\chi}{G}{L^\ast}\)
    where \(\chi(gh) = \chi(g)\chi(h)\).
\end{definition}
\begin{theorem}[Linear Independence of Characters]\label{thm:linear-independent-characters}
    Suppose \(K\) a field, and \(G\) some group.
    Let \({\{\chi_i\}}_{i=1}^n\) be distinct characters.
    Then \({\{\chi_i\}}_{i=1}^n\) is linearly independent.
\end{theorem}
\begin{proof}
    Suppose, by way of contradiction,
    that they are linearly dependent.
    Then there exists some minimal number \(m\) of nonzero coefficients \({\{a_i\}}_{i=1}^m\)
    such that
    \begin{equation*}
        a_1\chi_1 + a_2\chi_2 + \cdots + a_m\chi_m = 0
    \end{equation*}
    since without loss of generality we may assume by reordering the characters,
    the first \(m\) characters form our minimal dependent set.
    Then for any \(g \in G\), we have
    \begin{equation*}
        a_1\chi_1(g) + a_2\chi_2(g) + \cdots + a_m\chi_m(g) = 0
    \end{equation*}
    Let \(g_0 \in G\) be an element such that \(\chi_1(g_0) \neq \chi_m(g_0)\).
    Then we have
    \begin{align*}
        a_1\chi_1(g_0g) + a_2\chi_2(g_0g) + \cdots + a_m\chi_m(g_0g) &= 0 \\
        a_1\chi_1(g_0)\chi_1(g) + a_2\chi_2(g_0)\chi_2(g) + \cdots + a_m\chi_m(g_0)\chi_m(g) &= 0
    \end{align*}
    But we can also multiply by \(\chi_m(g_0)\) to get
    \begin{equation*}
        a_1\chi_m(g_0)\chi_1(g) + a_2\chi_m(g_0)\chi_2(g) + \cdots + a_m\chi_m(g_0)\chi_m(g) = 0
    \end{equation*}
    Subtracting the above two equations, we have
    \begin{equation*}
        (\chi_m(g_0)-\chi_1(g_0))a_1\chi_1(g) + (\chi_m(g_0)-\chi_2(g_0))a_2\chi_2(g)
        + \cdots + (\chi_m(g_0)-\chi_{m-1}(g_0))a_m\chi_m(g) = 0
    \end{equation*}
    We have now found a set of \(m-1\) characters that is linearly dependent,
    which contradicts minimality as assumed.
\end{proof}
\begin{corollary}[Dedekind's Lemma]\label{cor:dedekind}\label{lem:dedekind}
    Suppose \(K\) is a field, and \({\{\sigma_i\}}_{i=1}^n \subset \Aut K\).
    Then there does not exist \({\{a_i\}}_{i=1}^n \subset K\) not all zero
    such that \(a_1\sigma_1(u) + \cdots a_n\sigma_n(u) = 0\) for all \(u \in K\).
\end{corollary}
\begin{proof}
    Direct application of the \hyperref[thm:linear-independent-characters]{theorem above}.
\end{proof}

\begin{theorem}
    \([K:F]\) finite implies \(\abs{\Gal(K/F)} \leq [K:F]\).
\end{theorem}
\begin{proof}
    Let \(n = [K:F]\), and \({\{u_i\}}_{i=1}^n\) be an \(F\)-basis of \(K\).
    Suppose \({\{\sigma_i\}}_{i=1}^{n+1} \subset \Gal(K/F)\) are distinct.
    Then consider a system of \(n\) homogeneous linear equations in \(n+1\) unknowns.
    \begin{align*}
        \sigma_1(u_1)x_1 + \sigma_2(u_1)x_2  + \cdots + \sigma_{n+1}(u_1)x_{n+1} &= 0 \\
        \sigma_1(u_2)x_1 + \sigma_2(u_2)x_2  + \cdots + \sigma_{n+1}(u_2)x_{n+1} &= 0 \\
        &\;\;\vdots \\
        \sigma_1(u_n)x_1 + \sigma_2(u_n)x_2  + \cdots + \sigma_{n+1}(u_n)x_{n+1} &= 0
    \end{align*}
    This system has a nontrivial solution.
    But that exactly contradicts \hyperref[lem:dedekind]{Dedekind's lemma}.
\end{proof}

\begin{definition}
    Suppose \(K/F\) some extension, and \(G \subseteq \Aut_F(K)\) some subset.
    The fixed field of \(T\) is
    \(K^T =\{a \in K: \sigma(u) = u,\; \forall \sigma \in T\}\).
\end{definition}
\begin{proposition}
    Suppose \(F\) is a field of characteristic 0,
    \(K/F\) some finite extension, \(G = \Aut_F(K)\).
    Then \(K/F\) is normal if and only if \(K^G = F\).
\end{proposition}
\begin{proof}
    % TODO: galois extension
\end{proof}
\begin{proposition}
    Suppose \(L/K\) is a finite extension,
    \(\func{\phi}{K}{K'}\) a ring homomorphism,
    and \(L'/K'\) is normal.
    Then there is at least 1, and at most \([L:K]\) ring homomorphisms
    \(\func{\psi}{L}{L'}\) that extend \(\phi\),
    with equality if and only if \(L/K\) separable.
\end{proposition}
\begin{proof}
    % TODO
\end{proof}
\begin{proposition}
    Let \(E/F\) be a finite extension.
    Then \(\abs{\Aut_F(E)} \leq {[E:F]}_s\),
    with equality if and only if \(E/F\) normal.
\end{proposition}
\begin{proof}
    % TODO
\end{proof}

\begin{theorem}
    Suppose \(F\) is a field of characteristic 0,
    and \(K\) a (finite) normal extension of \(F\),
    \(H \subseteq \Gal(K/F)\) some subgroup.
    Let \(K^H\) be the fixed field of \(H\).
    Then \([K:K^H] = \abs{H}\), \(H = \Gal(K/K^H)\),
    and in particular, if \(H = \Gal(K/F)\) then \([K:F] = \Gal(K/F)\).
\end{theorem}
\begin{proof}
    % TODO
\end{proof}

\begin{theorem}
    Suppose \(K/F\) is a Galois extension, \(G = \Gal(K/F)\) is a Galois group.
    Then \(F = K^G\).
    If \(F \subseteq E \subseteq K\), then \(K/E\) is Galois.
    The map \(E \mapsto \Gal(K/E)\) from the set of intermediate field extensions
    to the set of subgroups of \(G\) is injective.
\end{theorem}
\begin{proof}
    % TODO
\end{proof}

\begin{lemma}
    Suppose \(K/F\) is Galois. Then the following hold:
    \begin{enumerate}[label={(\alph*)}, itemsep=0mm]
        \item If \(L_1 \subseteq L_2\) are subfields of \(K\),
            then \(\Gal(K/L_2) \subseteq \Gal(K/L_1)\).
        \item If \(L\) is a subfield of \(K\),
            then \(L \subseteq K^G\) where \(G = \Gal(K/L)\).
        \item If \(S_1 \subseteq S_2\) are subsets of \(Aut K\),
            then \(K^{S_2} \subseteq K^{S_1}\).
        \item If \(S\) is a subset of \(\Aut K\),
            then \(S \subseteq \Gal(K/K^S)\).
        \item If \(L = K^S\) for some \(S \subseteq \Aut K\),
            then \(L = K^G\) where \(G = \Gal(K/L)\).
        \item If \(H = \Gal(K/L)\) for some subfield \(L\),
            then \(H = \Gal(K/K^H)\).
    \end{enumerate}
\end{lemma}
\begin{proof}
    % TODO
\end{proof}
\begin{corollary}
    Suppose \(K/F\) is Galois, and \(G = \Gal(K/F)\).
    Then every subgroup \(H\) corresponds to some subfield \(L\)
    such that \(F \subseteq L \subseteq K\).
\end{corollary}
\begin{proof}
    % TODO
\end{proof}

\begin{theorem}
    Suppose \(K/F\) Galois and \(G = \Gal(K/F)\).
    Suppose \(L\) is a subfield with \(F \subseteq L \subseteq K\) and \(H = \Gal(K/L)\).
    Then \(L\) is normal over \(F\) if and only if \(H \lhd G\).
    If \(L\) is indeed normal over \(F\),
    then restricting \(\Gal(K/F) \to \Gal(L/F)\) via \(\sigma \mapsto \sigma\vert_L\)
    is a surjective homomorphism with kernel \(H\).
    Hence \(\Gal(L/F) \cong G/H\).
\end{theorem}
\begin{proof}
    % TODO
\end{proof}

\begin{definition}
    A Galois extension \(K/F\) is abelian/cyclic
    if \(\Gal(K/F)\) is abelian/cyclic.
\end{definition}
\begin{proposition}
    If \(K/F\) is abelian/cyclic, and \(L\) is an intermediate field,
    then \(L/F\) is abelian/cylic.
\end{proposition}
\begin{proof}
    Subgroups and quotient groups of abelian/cyclic groups are abelian/cyclic.
\end{proof}

\begin{theorem}
    Suppose \(K/F\) Galois, and \(L/F\) some other extension,
    under the assumption that \(K,L\) are both contained in some extension.
    Then \(KL/L\) and \(K/(K \cap L)\) are Galois.
\end{theorem}
\begin{proof}
    % TODO
\end{proof}
\begin{corollary}
    Suppose \(K/F\) finite Galois, and \(L/F\) some extension.
    Then \([KL:L] \mid [K:F]\).
\end{corollary}
\begin{proof}
    % TODO
\end{proof}
\begin{remark}
    This does not hold if \(K/F\) not Galois.
\end{remark}

\begin{theorem}[Artin's Theorem]
    Suppose \(K\) is a field, and \(G \subseteq \Aut K\) some group of automorphisms,
    with \(\abs{G} = n\).
    Let \(F = K^G\).
    Then \(K/F\) is a finite Galois extension, with \(\Gal(K/F) = G\).
\end{theorem}
\begin{proof}
    % TODO
\end{proof}

\begin{theorem}[Fundamental Theorem of Galois Theory]\label{thm:ftgt}
    Suppose \(K/F\) is (finite) Galois, and \(G = \Gal(K/F)\).
    \begin{enumerate}[label={(\alph*)}, itemsep=0mm]
        \item There is an inclusion reversing bijection
            between intermediate fields \(E\) of \(K/F\) and subgroups \(H \subseteq G\),
            given by associating \(H\) with \(K^H = E\).
        \item \([K:E] = \abs{H}\) and \([K:F] = \abs{G}\).
        \item \(K/E\) is Galois, \(H = \Gal(K/E)\).
        \item Embeddings \(E\) to \(\overline{F}\) are in bijection with
            left cosets of \(H\) in \(G\).
        \item \(E/F\) Galois if and only if \(H \lhd G\),
            and in that case, \(\Gal(E/F) \cong G/H\).
        \item Intersection of subgroups correspond to compositum of fields
            and joins of subgroups correspond to intersection of fields.
        \item Lattice of subgroups of \(G\) correspond under this bijection
            to inverted lattice of intermediate field extensions.
    \end{enumerate}
\end{theorem}
\begin{proof}
    % TODO
\end{proof}

\begin{theorem}
    Suppose \(K/F\) a field extension.
    Then the following are equivalent:
    \begin{enumerate}[label={(\alph*)}, itemsep=0mm]
        \item \(K/F\) finite Galois, i.e.\ \(\abs{\Aut_F(K)} = [K:F] < \infty\).
        \item \(K/F\) is the splitting field of some separable \(f(x) \in F[x]\).
        \item \(F\) is the fixed field of \(\Aut(K/F)\).
        \item \(K/F\) is normal, finite, and separable.
    \end{enumerate}
\end{theorem}
\begin{proof}
    % TODO
\end{proof}

\begin{definition}
    Suppose \(K/F\) Galois, \(G = \Gal(K/F)\),
    \(\sigma \in G\), \(\alpha \in K\).
    \(\{\sigma(\alpha) : \sigma \in G\}\) is the set of Galois conjugates of \(\alpha\).
    If \(F \subseteq E \subseteq K\) tower of extensions,
    then \(\sigma(E)\) is the conjugate of \(E\).
\end{definition}
\begin{proposition}
    Suppose \(F \subseteq E \subseteq K\) is a tower of field extensions,
    \(K/F\) finite Galois, and \(G = \Gal(K/F)\).
    Suppose that \(H \subseteq G\) corresponds to \(E\).
    For \(\sigma \in G\),
    the subgroup corresponding to \(\sigma(E)\)
    is the conjugate subgroup \(\sigma H \sigma^{-1} \subseteq G\).
    In particular \([E:F] = [G:H]\).
\end{proposition}
\begin{proof}
    % TODO
\end{proof}

\begin{theorem}
    Suppose \(F\) a field, and \(K = F(x_1,\hdots,x_n)\)
    the field of rational functions in \(n\) variables.
    Suppose \(S\) is the subfield of symmetric rational functions.
    Then:
    \begin{enumerate}[label={(\alph*)}, itemsep=0mm]
        \item \([K:S] = n!\)
        \item \(\Gal(K/S) = S_n\)
        \item Let \({\{s_i\}}_{i=1}^n\) be the elementary symmetric polynomials.
            Then \(S = F(s_1,\hdots,s_n)\).
        \item \(K\) is the splitting field over \(S\) of
            \(t^n - s_1 t^{n-1} + s_2 t^{n-2} - \cdots + {(-1)}^n s_n\).
    \end{enumerate}
\end{theorem}
\begin{proof}
    % TODO
\end{proof}

\begin{proposition}
    Suppose \(K/F\) Galois, and \(F'/F\) some extension
    such that \(K,F' \subseteq \overline{F}\).
    Then \(\vfunc{\psi}{\Gal(KF'/F)}{\Gal(K/F)}{\sigma}{\sigma\vert_K}\) is injective
    and it induces an isomorphism \(\Gal(KF'/F) \cong \Gal(K/(K \cap F'))\).
\end{proposition}
\begin{proof}
    % TODO: galois extension 3
\end{proof}
\begin{corollary}
    Suppose \(K/F\) Galois, and \(F'/F\) some extension
    such that \(K,F' \subseteq \overline{F}\).
    Then \([KF':F'] = [K:K \cap F']\).
    In particular, \([KF':F'] = [K:F][F':F]\) if and only if \(K \cap F' = F\).
\end{corollary}
\begin{proof}
    % TODO
\end{proof}

\begin{theorem}
    Suppose \(K/F\) and \(F'/F\) both finite Galois,
    such that \(K,F' \subseteq \overline{F}\).
    Then \(\vfunc{\psi}{\Gal(KF'/F)}{\Gal(K/F)\times\Gal(KF'/F)}{\sigma}{(\sigma\vert_K,\sigma\vert_{F'})}\)
    is injective,
    and isomorphism if \(K \cap F' = F\).
\end{theorem}
\begin{proof}
    % TODO
\end{proof}

\begin{corollary}
    Suppose \(E/F\) finite separable.
    Then there exists a Galois closure \(K \supseteq E\) such that
    \(K/F\) is Galois and minimal with respect to this property.
\end{corollary}
\begin{proof}
    % TODO
\end{proof}

% \subsubsection*{Finite Fields}

\begin{theorem}
    Let \(F = \bF_{p^n}\).
    Then \([F:\bF_p] = n\), \(F/\bF_p\) is cyclic,
    and the Galois group is generated by the Frobenius automorphism.
\end{theorem}
\begin{proof}
    % TODO: galois extension 2
\end{proof}
\begin{corollary}
    Suppose \(K/F\) is an extension of finite fields of characteristic \(p\).
    Then \(K/F\) is cyclic Galois.
    If \(\abs{F} = p^n\), then \(\Gal(K/F) = \langle\tau\rangle\)
    where \(\vfunc{\tau}{K}{K}{a}{a^{p^n}}\).
\end{corollary}
\begin{proof}
    % TODO
\end{proof}

\begin{theorem}
    Suppose \(\overline{\bF}_p\) an algebraic closure of \(\bF_p\).
    For any \(n \in \bN\),
    there exists a unique subfield of \(\overline{\bF}_p\) of order \(p^m\).
    If \(K,L\) are subfields of \(\overline{\bF}_p\) of orders \(p^m\) and \(p^n\),
    then \(K \subseteq L \iff m \mid n\).
    In that case, \(L/K\) Galois with \(\Gal(L/K)\) cyclic,
    generated by \(\vfunc{\tau}{L}{L}{a}{a^{p^m}}\).
\end{theorem}
\begin{proof}
    % TODO
\end{proof}

% \subsubsection*{Polynomials}

\begin{definition}
    Suppose \(f(x) \in F[x]\) irreducible and separable,
    and \(K\) splitting field of \(f(x)\).
    Then \(K/F\) is Galois,
    and the Galois group of the polynomial is \(\Gal(f) = \Gal(K/F)\).
\end{definition}
\begin{theorem}
    Suppose \(f(x) \in F[x]\) separable of degree \(n\).
    \begin{enumerate}[label={(\alph*)}, itemsep=0mm]
        \item If \(f(x)\) is irreducible over \(F\),
            then \(n \mid \abs{\Gal(f)}\).
        \item \(f(x)\) is irreducible over \(F[x]\)
            if and only if \(\Gal(f)\) is a transitive subgroup of \(S_n\).
    \end{enumerate}
\end{theorem}
\begin{proof}
    % TODO
\end{proof}


\subsection{Abelian Extensions}

\begin{definition}
    Suppose \(K/F\) is finite Galois.
    The norm is a multiplicative homomorphism
    \(\vfunc{N_{K/F}}{K}{F}{a}{\prod_{\sigma\in\Aut_F(K)} \sigma(a)}\);
    the trace is an additive homomorphism
    \(\vfunc{\tr_{K/F}}{K}{F}{a}{\sum_{\sigma\in\Aut_F(K)} \sigma(a)}\).
\end{definition}
\begin{proposition}
    Suppose \(K/F\) finite Galois,
    \(a \in K\) some element with minimal polynomial \(f(x) = \min_F(a) \in F[x]\).
    Let \(f(x) = \prod_{\sigma\in\Gal(K/F)} (x-\sigma(a)) = x^n + a_1 x^{n-1} + \cdots + a_n\).
    Then \(N_{K/F} = {(-1)}^n a_n\)  and \(\tr_{K/F} = -a_1\).
\end{proposition}
\begin{proof}
    % TODO
\end{proof}

\begin{theorem}[Hilbert Theorem 90, multiplicative]
    Suppose \(K/F\) is cyclic of degree \(n\), \(G = \Gal(K/F) = \langle\sigma\rangle\).
    For any \(b \in K^\ast\), if \(N_{K/F}(b) = 1\),
    then \(b = a/\sigma(a)\) for some \(a \in K\).
\end{theorem}
\begin{proof}
    % TODO
\end{proof}
\begin{theorem}[Hilbert Theorem 90, additive]
    Suppose \(K/F\) is cyclic of degree \(n\), \(G = \Gal(K/F) = \langle\sigma\rangle\).
    For any \(b \in K\), if \(\tr_{K/F}(b) = 0\),
    then \(b = a - \sigma(a)\) for some \(a \in K\).
\end{theorem}
\begin{proof}
    % TODO
\end{proof}

\begin{definition}
    Let \(K_1\) and \(K_2\) be extensions over \(F\).
    \(K_1\) and \(K_2\) are linearly disjoint over \(F\)
    if \(K_1 \cap K_2 = F\).
\end{definition}

\subsubsection*{Cyclotomic Extensions}

\begin{definition}
    For any \(n \in \bN\), \(\mu_n\) is the group of \(n\)th roots of unity,
    which are roots of \(x^n - 1\).
    A primitive \(n\)th root of unity is \(\zeta_n\),
    with \(\zeta_n^n = 1\) and \(\zeta_n^k \neq 1\) for all \(0 < k < n\).
\end{definition}
\begin{proposition}
    \(\mu_n \cong Z_n\);
    \(d \mid n\) if and only if \(\mu_d \subseteq \mu_n\);
    and \(\langle \zeta_n \rangle = \mu_n\).
\end{proposition}
\begin{proof}
    Obvious.
\end{proof}

\begin{proposition}
    Let \(F\) be a field of characteristic \(p > 0\).
    There does not exist primitive \(np\)th root of unity \(\zeta \in F\)
    for all \(n \in \bN\).
\end{proposition}
\begin{proof}
    \(\zeta\) is an \(np\)th root of unity,
    so \(\zeta^{np} - 1 = 0\).
    But in characteristic \(p\) we have
    \({(\zeta^n)}^p - 1^p = {(\zeta^n - 1)}^p = 0\),
    which allows us to conclude that \(\zeta\) is an \(n\)th root of unity.
    In that case, \(\zeta\) clearly cannot be a primitive \(np\)th root of unity.
\end{proof}

\begin{definition}
    The \(n\)th cyclotomic polynomial is
    \(\Phi_n(x) = \prod_{\text{primitive}\;\zeta\in\mu_n} (x-\zeta)\).
\end{definition}
\begin{proposition}
    Suppose \(\Phi_n(x) \in \bQ[x]\).
    Then \(\Phi_n(x)  = \prod_{1 \leq a < n,\; \gcd(a,n) = 1} (x-\zeta_n^a)\),
    and \(\deg \Phi_n(x) = [\bQ(\zeta_n):\bQ] = \varphi(n)\),
    where \(\varphi\) is the Euler totient function.
\end{proposition}
\begin{proof}
    % TODO
\end{proof}
\begin{proposition}
    Suppose \(\Phi_n(x) \in \bQ[x]\).
    Then \(\Phi_n(x) \in \bZ[x]\) is monic.
\end{proposition}
\begin{proof}
    % TODO
\end{proof}

\begin{lemma}
    For each \(\sigma \in \Gal(F(\mu_n)/F)\),
    there exists \(k_\sigma \in \bZ\) such that \(\gcd(k_\sigma,n) = 1\)
    and \(\sigma(\zeta) = \zeta^{k_\sigma}\) for all \(\zeta \in \mu_n\).
\end{lemma}
\begin{proof}
    % TODO: cyclotomic
\end{proof}
\begin{theorem}
    The mapping from \(\Gal(F(\mu_n)/F) \to {(\bZ/n\bZ)}^\ast\)
    via \(\zeta \mapsto k_\sigma \pmod{n}\)
    where \(\sigma(\zeta) = \zeta^{k_\sigma}\) for all \(\zeta \in \mu_n\),
    is an injective group homomorphism.
\end{theorem}
\begin{proof}
    % TODO
\end{proof}
\begin{theorem}
    The mapping from \(\Gal(\bQ(\mu_n)/\bQ) \to {(\bZ/n\bZ)}^\ast\)
    is an isomorphism.
\end{theorem}
\begin{proof}
    % TODO
\end{proof}

\begin{definition}
    The splitting field of \(x^n - 1\) over \(F\)
    is called a cyclotomic extension of order \(n\) over \(F\).
\end{definition}


\subsubsection*{Cyclic Extensions}

\begin{proposition}
    Suppose \(F\) is a field of characteristic \(p\),
    and consider the additive homomorphism \(\vfunc{\eta}{F}{F}{a}{a^p-a}\).
    We have\(\ker\eta = \bF_p\), and \(\eta^{-1}(a) = \{b+i : 0 \leq i \leq p-1\}\).
\end{proposition}
\begin{proof}
    % TODO
\end{proof}
\begin{theorem}[Artin-Schreier Extensions]
    Suppose \(K/F\) is a cyclic Galois extension of degree \(p\),
    and \(F\) is a field of characteristic \(p\).
    Then we have the following:
    \begin{enumerate}[label={(\alph*)}, itemsep=0mm]
        \item \(K = F(a)\) where \(a\) is a root of \(x^p-x-b\) for some \(b \in F\).
        \item If \(a \in F - \eta^{-1}(F)\), then \(f(x) = x^p-x-a\) is irreducible,
            and a splitting field of \(f(x)\) is cyclic of degree \(p\).
    \end{enumerate}
\end{theorem}
\begin{proof}
    % TODO
\end{proof}

\subsubsection*{Abelian Extensions}

\begin{lemma}
    Suppose a prime \(p \nmid n\) and \(m \mid n\), \(m \neq n\).
    Then \(\Phi_n(x)\) and \(x^m - 1\) cannot have a common root modulo \(p\).
\end{lemma}
\begin{proof}
    % TODO
\end{proof}
\begin{theorem}
    Suppose \(n \in \bN\).
    Then there are infinitely many primes congruent to \(1 \pmod{n}\).
\end{theorem}
\begin{proof}
    % TODO
\end{proof}

\begin{proposition}
    Suppose \(\gcd(a,b) = 1\).
    Let \(K = \bQ(\zeta_a)\) and \(L = \bQ(\zeta_b)\).
    We have:
    \begin{enumerate}[label={(\alph*)}, itemsep=0mm]
        \item \(KL = \bQ(\zeta_{ab})\);
        \item \(K \cap L = \bQ\); and
        \item \(\Gal(\bQ(\zeta_{ab})/\bQ) \cong \Gal(\bQ(\zeta_a)/\bQ) \times \Gal(\bQ(\zeta_b)/\bQ)\).
    \end{enumerate}
    In particular, if \(n = \prod_{i=1}^k p_i^{a_i}\) is the prime factorization,
    then \(\bQ(\zeta_n)\) is the compositum of \(\bQ(\zeta_{p_i^{a_i}})\),
    and the Galois group is the direct product of the prime-powered \(\zeta\) Galois groups.
    \begin{equation*}
        \Gal(\bQ(\zeta_n)/\bQ) \cong \prod_{i=1}^k \Gal(\bQ(\zeta_{p_i^{a_i}})/\bQ)
    \end{equation*}
\end{proposition}
\begin{proof}
    % TODO
\end{proof}

\begin{theorem}
    Every finite abelian group \(G\) is isomorphic to \(\Gal(K/\bQ)\)
    for some Galois extension \(K/\bQ\).
\end{theorem}
\begin{proof}
    % TODO
\end{proof}

\begin{theorem}[Kronecker-Weber Theorem]
    Any abelian extension of \(\bQ\)
    is a subextension of a cyclotomic extension.
\end{theorem}
\begin{proof}
    See footnote.\footnote{%
        \url{https://www.math.uchicago.edu/~may/VIGRE/VIGRE2007/REUPapers/FINALFULL/Culler.pdf}
    }
\end{proof}


\subsection{Solvable Extensions}

\subsubsection*{Radical Extensions}

\begin{definition}
    \(K/F\) is a simple radical extension if \(K = F(a)\)
    where \(a^n \in F\) for some \(n \in \bN\).
    \(K/F\) is a radical extension if there exists a chain of field extensions
    \begin{equation*}
        F = F_0 \subseteq F_1 \subseteq \cdots \subseteq F_n = K
    \end{equation*}
    of sequential radical extensions.
\end{definition}
\begin{definition}
    Polynomial \(f(x) \in F[x]\) is solvable by radicals
    if the splitting field of \(f(x)\) is contained in a radical extension.
\end{definition}
\begin{proposition}
    Suppose \(E/F\) is a separable radical extension.
    Suppose \(L \supseteq E\) is the smallest Galois extension of \(F\),
    so that \(L \subseteq \overline{F}\).
    Then \(L\) is a radical extension of \(F\).
\end{proposition}
\begin{proof}
    % TODO
\end{proof}
\begin{definition}
    The smallest Galois extension is called a Galois closure.
\end{definition}

\begin{lemma}
    Suppose \(F \subseteq L \subseteq M\) is a tower of extensions,
    with \(M/F\) Galois and \(L/F\) separable.
    Then the compositum of all conjugate fields of \(L\) in \(M\)
    is the Galois closure of \(L/F\).
\end{lemma}
\begin{proof}
    % TODO
\end{proof}

\begin{lemma}
    Suppose \(F \subseteq L \subseteq M\) is a tower of extensions.
    \begin{enumerate}[label={(\alph*)}, itemsep=0mm]
        \item If \(L/F\) and \(M/L\) radical, then \(M/F\) radical.
        \item If \(F \subseteq K_1,K_2 \subseteq L\) and \(K_1\) radical,
            then \(K_1K_2/K_2\) is radical.
        \item If \(F \subseteq K_1,K_2 \subseteq L\) and \(K_1,K_2\) radical,
            then \(K_1K_2/F\) is radical.
    \end{enumerate}
\end{lemma}
\begin{proof}
    % TODO
\end{proof}

\subsubsection*{Kummer Extensions}

\begin{proposition}
    Suppose \(K\) is a field
    that contains a primitive \(n\)th root of unity \(\zeta = \zeta_n\),
    with \(n \geq 2\).
    Suppose \(a \in K\), and adjoin a root of \(f(x) = x^n - a \in F[x]\)
    to form \(L/K\).
    Then \(L/K\) is Galois, and if \(G = \Gal(L/K)\),
    the map \(G \to \mu_n(K)\) via \(\sigma \mapsto \sigma(a^{1/n}/a^{1/n})\)
    is an injective group homomorphism.
    \(G\) is cyclic, and \(\abs{G} \mid n\).
\end{proposition}
\begin{proof}
    % TODO
\end{proof}

\begin{lemma}
    Suppose \(L/K\) is a Galois extension with Galois group \(Z_n\).
    If \(n\) is coprime to the characteristic of \(K\),
    and \(K\) contains \(\zeta = \zeta_n\),
    then \(L = K(z)\) with \(z^n \in K\).
\end{lemma}
\begin{proof}
    % TODO: kummer
\end{proof}
\begin{proposition}
    Suppose \(K\) is a field containing \(\zeta_n\),
    and \(L/K\) an extension such that \(L = K(\alpha)\) with \(\alpha^n = a \in K\).
    Then we have:
    \begin{enumerate}[label={(\alph*)}, itemsep=0mm]
        \item \(L/K\) is cyclic;
        \item the degree \(m = [L:K]\) is the order of image of \(a\) in \(K^\ast/{(K^\ast)}^n\); and
        \item there exists \(b \in K\) such that \(\min_K(\alpha) = x^m - b\).
    \end{enumerate}
\end{proposition}
\begin{proof}
    % TODO
\end{proof}

\subsubsection*{Solvable Extensions}

\begin{lemma}
    Suppose \(L/F\) Galois and \(\zeta = \zeta_m\) is a primitive \(m\)th root of unity.
    Then \(L(\zeta)/F\) and \(L(\zeta)/F(\zeta)\) are also Galois.
    Furthermore, the following are equivalent:
    \begin{enumerate}[label={(\alph*)}, itemsep=0mm]
        \item \(\Gal(L/F)\) solvable;
        \item \(\Gal(L(\zeta)/F)\) solvable; and
        \item \(\Gal(L(\zeta)/F(\zeta))\) solvable.
    \end{enumerate}
\end{lemma}
\begin{proof}
    % TODO
\end{proof}

\begin{theorem}
    Suppose \(F\) is a field of characteristic 0.
    If \(f(x) \in F[x]\) is solvable by radicals,
    then \(\Gal(f)\) is a solvable group.
\end{theorem}
\begin{proof}
    % TODO
\end{proof}
\begin{theorem}[Galois]
    Suppose \(F\) is a field of characteristic 0,
    and \(f(x) \in F[x]\).
    If \(\Gal(f)\) is solvable, then \(f(x)\) is solvable by radicals.
\end{theorem}
\begin{proof}
    % TODO
\end{proof}
\begin{corollary}
    Cubics and quartics over \(\bQ[x]\) are solvable by radicals.
\end{corollary}
\begin{proof}
    \(S_3\) and \(S_4\) are solvable.
\end{proof}
\begin{corollary}
    The general equation for degree \(n\) is insolvable for \(n \geq 5\).
\end{corollary}
\begin{proof}
    \(A_n\) simple for \(n \geq 5\).
\end{proof}


\subsection{Cubic and Quartic Polynomials}

\subsubsection*{Cubic Polynomials}

\begin{definition}
    The depressed cubic is a cubic polynomial without the square term,
    \(f(x) = x^3 + px + q\).
\end{definition}
\begin{proposition}
    The discriminant of the depressed cubic is \(\mca{D}(f) = -4p^3 - 27q^2\).
\end{proposition}
\begin{proof}
    % TODO
\end{proof}
\begin{theorem}[Galois Groups of Cubics]
    Suppose a cubic \(f(x) \in F[x]\) is irreducible and separable
    in a field \(F\) with characteristic not 2 or 3.
    Let \(K\) be the splitting field of \(f(x)\).
    The classification of Galois groups are:
    \begin{enumerate}[label={(\alph*)}, itemsep=0mm]
        \item \(\Gal(K/F) = S_3 \iff \mca{D} \notin {(F^\ast)}^2\); and
        \item \(\Gal(K/F) = A_3 \iff \mca{D} \in {(F^\ast)}^2\).
    \end{enumerate}
\end{theorem}
\begin{proof}
    % TODO
\end{proof}

\begin{lemma}
    A general cubic \(f(x) = x^3 + ax^2 + bx + c\)
    can be transformed into a depressed cubic
    via a translation \(x \mapsto x-\frac{a}{3}\).
    In particular,
    \begin{equation*}
        p = \frac{1}{3}(3b-a^2), \quad
        q = \frac{1}{27}(2a^3-9ab+27c)
    \end{equation*}
\end{lemma}
\begin{proof}
    % TODO
\end{proof}
\begin{proposition}
    The discriminant of the general cubic is
    \(\mca{D}(f) = a^2b^2 - 4b^3 - 4a^3c - 27c^2 + 18abc\).
\end{proposition}
\begin{proof}
    % TODO
\end{proof}
\begin{theorem}[Cardano's Formula]
    The roots to the depressed cubic \(f(x) = x^3 + px + q\)
    \begin{equation*}
        \frac{A+B}{3},\quad
        \frac{\zeta^2 A + \zeta B}{3},\quad
        \frac{\zeta A + \zeta^2 B}{3}
    \end{equation*}
    where
    \begin{equation*}
        \zeta = e^{2\pi i/3}, \quad
        A = \pqty{-\frac{27}{2}q + \frac{3}{2}\sqrt{-3\mca{D}}}^{1/3}, \quad
        B = \pqty{-\frac{27}{2}q - \frac{3}{2}\sqrt{-3\mca{D}}}^{1/3}
    \end{equation*}
\end{theorem}
\begin{proof}
    % TODO
\end{proof}

\subsubsection*{Quartic Polynomials}

\begin{definition}
    Suppose \(f(x) = x^4 + ax^3 + bx^2 + cx + d \in F[x]\) is a general quartic.
    Lagrange's resolvent is defined as
    \(g(x) = x^3 - bx^2 + (ac-4d)x + (4bd-a^2d-c^2)\).
\end{definition}
\begin{proposition}
    Suppose \({\{\alpha_i\}}_{i=1}^4\) are the roots of a general quartic \(f(x)\).
    Then the roots of the resolvent \(g(x)\) are
    \begin{equation*}
        \beta_1 = \alpha_1\alpha_2 + \alpha_3\alpha_4,\quad
        \beta_2 = \alpha_1\alpha_3 + \alpha_2\alpha_4,\quad
        \beta_3 = \alpha_1\alpha_4 + \alpha_2\alpha_3
    \end{equation*}
\end{proposition}

\begin{definition}
    The depressed quartic is a quartic polynomial without the cubic term,
    \(f(x) = x^4 + px^2 + qx + r\).
\end{definition}
\begin{corollary}
    The resolvent to the depressed quartic is
    \(g(x) = x^3 - px^2 - 4rx + (4pr-q^2)\).
\end{corollary}
\begin{proof}
    % TODO
\end{proof}
\begin{lemma}
    A general quartic can be transformed into a depressed quartic
    via a translation \(x \mapsto x-\frac{a}{4}\).
    In particular,
    \begin{equation*}
        p = \frac{1}{8}(8b-3a^2), \quad
        q = \frac{1}{8}(a^3-4ab+8c), \quad
        r = \frac{1}{256}(-3a^4+256d-64ac+16a^2b)
    \end{equation*}
\end{lemma}
\begin{proof}
    % TODO
\end{proof}
\begin{proposition}
    The discriminant of a quartic is the discriminant of its resolvent cubic.
\end{proposition}
\begin{proof}
    % TODO
\end{proof}
\begin{theorem}[Galois Groups of Quartics]
    Suppose a depressed quartic \(f(x) \in F[x]\) is irreducible
    in a field \(F\) with characteristic not 2 or 3.
    Let \(g(x)\) be its resolvent cubic.
    The classification of Galois groups are:
    \begin{enumerate}[label={(\alph*)}, itemsep=0mm]
        \item \(\Gal(f) = S_4 \iff g(x)\) irreducible in \(F\) and \(\mca{D} \notin {(F^\ast)}^2\);
        \item \(\Gal(f) = A_4 \iff g(x)\) irreducible in \(F\) and \(\mca{D} \in {(F^\ast)}^2\);
        \item \(\Gal(f) = V_4 \iff g(x)\) splits completely in \(F\);
        \item \(\Gal(f) = Z_4 \iff g(x)\) has one root \(s \in F\),
            and \(x^2 + s\) and \(x^2 + (s-p) + r\) both split in \(F(\sqrt{\mca{D}})\); and
        \item \(\Gal(f) = D_4 \iff g(x)\) has one root \(s \in F\),
            and at least one of \(x^2 + s\) and \(x^2 + (s-p) + r\) irreducible in \(F(\sqrt{\mca{D}})\).
    \end{enumerate}
\end{theorem}
\begin{proof}
    % TODO
\end{proof}
\begin{theorem}
    The roots to a depressed cubic \({\{\alpha_i\}}_{i=1}^4\)
    can be expressed in terms of the roots of the resolvent cubic \({\{\beta_j\}}_{j=1}^3\).
    \begin{align*}
        2\alpha_1 &= + \sqrt{-\beta_1} + \sqrt{-\beta_2} + \sqrt{-\beta_3} \\
        2\alpha_2 &= + \sqrt{-\beta_1} - \sqrt{-\beta_2} - \sqrt{-\beta_3} \\
        2\alpha_3 &= -\sqrt{-\beta_1} + \sqrt{-\beta_2} - \sqrt{-\beta_3} \\
        2\alpha_4 &= -\sqrt{-\beta_1} - \sqrt{-\beta_2} + \sqrt{-\beta_3}
    \end{align*}
\end{theorem}
\begin{proof}
    % TODO
\end{proof}

\subsubsection*{Other Polynomials}

\begin{theorem}
    Suppose \(f(x) \in \bQ[x]\) be irreducible of prime degree \(p\),
    with all but two roots in \(\bR\).
    Then \(\Gal(f) = S_p\).
\end{theorem}
\begin{proof}
    % TODO
\end{proof}

% TODO: quintics in the form x^5 + ax + b and x^5 + ax^2 + b

\chapter{Commutative Algebra}\label{ch:comm-alg}

\begin{remark}
    We shall assume that all rings,
    unless otherwise noted, are commutative;
    that is, the multiplication \(ab = ba\).
\end{remark}

\section{Ideals}

\begin{definition}
    Suppose \(I,J\) are two ideals.
    They are coprime if \(I+J = (1)\).
\end{definition}
\begin{theorem}
    Suppose \({\{J_i\}}_{i=1}^n\) are pairwise coprime ideals.
    Then \(\bigcap_{i=1}^n J_i = \prod_{i=1}^n J_i\).
\end{theorem}
\begin{proof}
    
\end{proof}

\begin{definition}
    Suppose \(\{R_i\}\) is a family of rings.
    The direct product is \(\prod_i R_i = \{(r_1,r_2,\hdots) : r_i \in R_i\}\).
\end{definition}
\begin{lemma}
    Suppose \(R\) is a ring, and \(\{J_i\}\) is a family of ideals.
    Let us define the homomorphism \(\vfunc{\phi}{R}{\prod_i R/J_i}{r}{(r+J_1,r+J_2,\hdots)}\).
    Then we have \(\ker\phi = \bigcap_i J_i\),
    and \(\img\phi = R/\prod_i J_i\).
\end{lemma}
\begin{proof}
    
\end{proof}
\begin{theorem}[Chinese Remainder Theorem]
    Suppose \(R\) is a ring, and \(\{J_i\}\) is a family of ideals.
    Consider the homomorphism \(\func{\phi}{R}{\prod_i R/J_i}\).
    Then \(\phi\) is surjective if and only if \(\{J_i\}\) are pairwise coprime.
    In that case, \(\prod_i R/J_i \cong R/\prod_i J_i\).
\end{theorem}
\begin{proof}
    
\end{proof}

\begin{definition}
    Suppose \(R\) is a ring, and \(I,J\) are ideals.
    The quotient ideal is \((I:J) = \{x \in R : xJ \subseteq I\}\).
\end{definition}
\begin{proposition}
    The following properties hold for quotient ideals:
    \begin{enumerate}[label={(\alph*)}, itemsep=0mm]
        \item \(I \subseteq (I:J)\);
        \item \((I:J)J \subseteq I\);
        \item \((I:I) = (1)\); and
        \item \(((I:J):K) = (I:JK) = ((I:K):J)\).
    \end{enumerate}
\end{proposition}
\begin{proof}
    
\end{proof}

\begin{definition}
    Suppose \(\func{\phi}{R}{S}\) is any ring homomorphism.
    Let \(I \subseteq R\) and \(J \subseteq S\) be ideals.
    \(\phi(I) \subseteq S\) might not be an ideal,
    so we define the ideal it generates \((\phi(I)) \subseteq \) the extension of \(I\).
\end{definition}
\begin{remark}
    \(\phi^{-1}(J) \subseteq R\) is always an ideal,
    and if \(J\) prime, \(\phi^{-1}(J)\) is also prime.
\end{remark}
\begin{definition}
    Suppose \(\func{\phi}{R}{S}\) is a ring monomorphism.
    If \(J \subseteq S\), we say \(\phi^{-1}(J) = J \cap R\) is the contraction of \(J\).
\end{definition}


\section{Modules}\label{sec:comm-modules}

\begin{proposition}
    Suppose \(R\) is a ring, and \(M,N\) are \(R\)-modules.
    Then \(\Hom_R(M,N)\) is also an \(R\)-module.
\end{proposition}
\begin{proof}
    
\end{proof}

\begin{proposition}
    Suppose \(R\) is a ring.
    The category \(\textbf{Mod}_R\) consisting of all \(R\)-modules
    and module homomorphisms form a category.
\end{proposition}
\begin{proof}
    
\end{proof}

\begin{proposition}
    Suppose \(M\) is an \(R\)-module.
    There exists a functor \(\func{F}{\textbf{Mod}_R}{\textbf{Mod}_R}\)
    that maps \(N \mapsto \Hom_R(M,N)\)
    and maps \(\func{\phi}{N}{L}\) to \(\func{\tilde{\phi}}{\Hom_R(M,N)}{\Hom_R(M,L)}\).
\end{proposition}
\begin{proof}
    
\end{proof}
\begin{corollary}
    \(\Hom_R(M,N) \cong N\).
\end{corollary}
\begin{proof}
    
\end{proof}

\begin{definition}
    Suppose \(L\) is an \(R\)-module,
    and \(M,N \subseteq L\) are submodules.
    The quotient ideal \((M:N) = \{x \in R : xN \subseteq M\}\).
\end{definition}
\begin{definition}
    Suppose \(N\) is an \(R\)-module.
    The annihilator of \(N\) is \(\Ann_R(N) = (0:N) = \{x \in R : xN = 0\}\).
\end{definition}

\begin{definition}
    Suppose \({\{M_i\}}_{i \in S}\) is a family of \(R\)-modules.
    The direct product is \(\prod_i M_i = \{{(m_i)}_{i \in S}\}\),
    and the direct sum is \(\bigoplus_i M_i = \{{(m_i)_{i \in S}}\}\) with finite support.
\end{definition}

\begin{definition}
    Suppose \(M\) is an \(R\)-module, and \({\{m_i\}}_{i \in S} \subseteq M\).
    The module it generates is \(N = \{\sum_{j=1}^n x_j m_j : x_j \in R, m_j \in {\{m_i\}}_{i \in S}\}\).
    If \(M = N\), we say \({\{m_i\}}_{i \in S}\) generates \(M\);
    in that case, if \(S\) is finite, we say \(M\) is finitely generated.
\end{definition}
\begin{proposition}
    Suppose \(M\) is an \(R\)-module, and \({\{m_i\}}_{i \in S} \subseteq M\).
    Consider the homomorphism \(\vfunc{\phi}{\bigoplus_{i \in S} R}{M}{{(x_i)}_i}{\sum_i x_i m_i}\).
    We have the following:
    \begin{enumerate}[label={(\alph*)}, itemsep=0mm]
        \item \(\img\phi\) is the module generated by \({\{m_i\}}_{i \in S}\);
        \item \(\phi\) is surjective if \({\{m_i\}}_{i \in S}\) generate \(M\); and
        \item \(M\) is finitely generated if there exists a surjective \(\func{\phi}{R^n}{M}\)./
    \end{enumerate}
\end{proposition}
\begin{proof}
    
\end{proof}

\begin{lemma}[Cayley-Hamilton Theorem]
    Suppose \(M\) is a finitely-generated \(R\)-module,
    and suppose we have an endomorphism \(\func{\phi}{M}{M}\).
    Then there exists a monic polynomial \(\phi^n + a_1\phi^{n-1} + \cdots + a_n = 0\).
    Moreover, if there exists \(I \subseteq R\) such that \(\img\phi \subseteq IM\),
    then the coefficients \(\{a_i\} \subseteq I\).
\end{lemma}
\begin{proof}
    
\end{proof}
\begin{corollary}[Nakayama's Lemma]
    Suppose \(M\) is a finitely-generated \(R\)-module,
    with \(I \subseteq R\) is an ideal, and \(IM = M\).
    Then there exists \(x \in R\), \(x \equiv 1\pmod{I}\), such that \(xM = 0\).
\end{corollary}
\begin{proof}
    
\end{proof}
\begin{corollary}
    Suppose \(M\) is a finitely-generated \(R\)-module,
    and a surjective homomorphism \(\func{\alpha}{M}{M}\).
    Then \(\alpha\) is an isomorphism.
\end{corollary}
\begin{proof}
    
\end{proof}

\begin{definition}
    Suppose \(R\) is a ring.
    The Jacobson radical of \(R\) is the intersection of all maximal ideals,
    \(\mca{R} = \bigcap m\).
\end{definition}
\begin{remark}
    The nilradical is inside the Jacobson radical.
\end{remark}
\begin{lemma}
    Suppose \(R\) is a ring, \(\mca{R}\) is its Jacobson radical.
    Any element \(x \in \mca{R}\) if and only if \(1 - xy\) is a unit for all \(y \in R\).
\end{lemma}
\begin{proof}
    
\end{proof}
\begin{theorem}[Nakayama's Lemma]
    Suppose \(M\) is a finitely generated \(R\)-module,
    and \(\mca{R}\) is the Jacobson radical of \(R\).
    Let \(I \subseteq \mca{R}\) be an ideal.
    If \(IM = M\), then \(M = 0\).
\end{theorem}
\begin{proof}
    
\end{proof}
\begin{corollary}
    Suppose \(M\) is a finitely generated \(R\)-module,
    and \(N \subseteq M\) is a submodule.
    Let \(I \subseteq \mca{R}\) be an ideal inside the Jacobson radical of \(R\).
    If \(IM + N = M\), then \(M = N\).
\end{corollary}
\begin{proof}
    
\end{proof}
\begin{corollary}
    Suppose \((R,m)\) is a local ring,
    and \(M\) a finitely generated \(R\)-module.
    Then \(M/mM\) is a finite-dimensional vector space.
\end{corollary}
\begin{proof}
    
\end{proof}

\begin{theorem}
    Suppose \((R,m)\) is a local ring,
    and \(M\) a finitely generated \(R\)-module.
    If \({\{x_i\}}_{i=1}^n \subset M\) span \(M/mM\) as a vector space,
    then \({\{x_i\}}_{i=1}^n\) generate \(M\).
\end{theorem}
\begin{proof}
    
\end{proof}


\section{Exact Sequences}

\begin{definition}
    Consider a sequence of \(R\)-modules with corresponding module homomorphisms.
    \begin{equation*}
        \begin{tikzcd}
            \cdots \arrow{r} & M_{i-1} \arrow{r}{\phi_i} &
            M_i \arrow{r}{\phi_{i+1}} & M_{i+1} \arrow{r} & \cdots
        \end{tikzcd}
    \end{equation*}
    This sequence is exact at \(M_i\) if \(\img\phi_i = \ker\phi_{i+1}\).
    The sequence is exact if it is exact at every \(M_i\).
\end{definition}
% \begin{proposition}
%     A sequence is exact at \(M_i\) if and only if the following two conditions hold:
%     \begin{enumerate}[label={(\alph*)}, itemsep=0mm]
%         \item \(\phi_{i+1}\circ\phi_i = 0\); and
%         \item \(\phi_{i+1}(m) = 0\) implies \(m = \)
%     \end{enumerate}
% \end{proposition}

\begin{proposition}
    We have the following two characterizations of homomorphisms:
    \begin{enumerate}[label={(\alph*)}, itemsep=0mm]
        \item \(0 \to M \to N\) exact if and only if \(M \to N\) is injective; and
        \item \(M \to N \to 0\) exact if and only if \(M \to N\) is surjective.
    \end{enumerate}
\end{proposition}
\begin{proof}
    
\end{proof}

\begin{definition}
    A short exact sequence is an exact sequence
    \begin{equation*}
        \begin{tikzcd}
            0 \arrow{r} & M_1 \arrow{r} & M_2 \arrow{r} & M_3 \arrow{r} & 0
        \end{tikzcd}
    \end{equation*}
\end{definition}
\begin{proposition}
    Such a sequence is a short exact sequence if and only if \(M_3 = M_2/M_1\).
\end{proposition}
\begin{proof}
    
\end{proof}
\begin{theorem}
    Every exact sequence can be cut into short exact sequences.
\end{theorem}
\begin{proof}
    
\end{proof}

\begin{proposition}[Euler Characteristic]
    Suppose we have an exact sequence of finite-dimensional \(K\)-vector spaces.
    \begin{equation*}
        \sigma: \qquad
        \begin{tikzcd}
            0 \arrow{r} & V_1 \arrow{r} & V_2 \arrow{r} & \cdots \arrow{r} &
            V_n \arrow{r} & 0
        \end{tikzcd}
    \end{equation*}
    Then the Euler characteristic obeys
    \begin{equation*}
        \chi(\sigma) = \sum_{i=1}^n {(-1)}^i \dim(V_i) = 0
    \end{equation*}
\end{proposition}
\begin{proof}
    
\end{proof}
\begin{lemma}
    Suppose \(C\) is a class of \(R\)-modules,
    closed under images and kernels.
    Let \(\func{\lambda}{C}{\bZ}\) be additive.
    If we have a short exact sequence in \(C\),
    \begin{equation*}
        \begin{tikzcd}
            0 \arrow{r} & M_1 \arrow{r} & M_2 \arrow{r} & M_3 \arrow{r} & 0
        \end{tikzcd}
    \end{equation*}
    then \(-\lambda(M_1) + \lambda(M_2) - \lambda(M_3) = 0\).
\end{lemma}
\begin{proof}
    
\end{proof}
\begin{theorem}
    An exact sequence in \(C\) yields \(\sum_{i=1}^n {(-1)}^i \lambda(M_i) = 0\).
\end{theorem}
\begin{proof}
    
\end{proof}

\begin{definition}
    Consider a functor \(\func{F}{\textbf{Mod}_R}{\textbf{Mod}_R}\).
    \(F\) is exact if it sends short exact sequences to short exact sequences.
    \(F\) is right-exact if it sends an exact \(M_1 \to M_2 \to M_3 \to 0\)
    to an exact \(F(M_1) \to F(M_2) \to F(M_3) \to 0\).
    \(F\) is left-exact if it sends an exact \(0 \to M_1 \to M_2 \to M_3\)
    to an exact \(0 \to F(M_1) \to F(M_2) \to F(M_3)\).
\end{definition}
\begin{proposition}
    Any exact functor sends exact sequences to exact sequences.
\end{proposition}
\begin{proof}
    
\end{proof}

\begin{theorem}
    Suppose \(M\) is an \(R\)-module.
    Let us consider two functors \(F,G\),
    with \(F: N \mapsto \Hom_R(M,N)\) and \(G: N \mapsto \Hom_R(N,M)\).
    Then \(F\) is a covariant left-exact functor,
    and \(G\) is a contravariant right-exact functor.
\end{theorem}
\begin{proof}
    
\end{proof}

\begin{lemma}[Snake Lemma]
    Suppose we have the two exact sequences,
    and the following diagram commutes.
    \begin{equation*}
        \begin{tikzcd}
            0 \arrow{r} & M_1 \arrow{d}{f} \arrow{r} &
            M_2 \arrow{d}{g} \arrow{r} & M_3 \arrow{d}{h} \arrow{r} & 0 \\
            0 \arrow{r} & N_1 \arrow{r} & N_2 \arrow{r} & N_3 \arrow{r} & 0
        \end{tikzcd}
    \end{equation*}
    Then the following sequence is exact.
    \begin{equation*}
        \begin{tikzcd}
            0 \arrow{r} & \ker f \arrow{r} & \ker g \arrow{r} & \ker h \arrow{r}{\delta} &
            \coker f \arrow{r} & \coker g \arrow{r} & \coker h \arrow{r} & 0
        \end{tikzcd}
    \end{equation*}

    This can be represented by the following commutative diagram:
    \begin{equation*}
        \begin{tikzcd}
            & 0 \arrow{d} & 0 \arrow{d} & 0 \arrow{d} \\
            0 \arrow{r} & \ker f \arrow{d} \arrow{r} &
            \ker g \arrow{d} \arrow{r} \arrow[ddd, phantom, ""{coordinate, name=Z}] & \ker h \arrow{d}
            \arrow[rounded corners, dashrightarrow, to path={
                -- ([xshift=12ex]\tikztostart.east)
                |- (Z) [near end]\tikztonodes
                -| ([xshift=-12ex]\tikztotarget.west)
                -- (\tikztotarget)
            }]{dddll}[at start]{\delta} \\
            0 \arrow{r} & M_1 \arrow{d}{f} \arrow{r} &
            M_2 \arrow{d}{g} \arrow{r} & M_3 \arrow{d}{h} \arrow{r} & 0 \\
            0 \arrow{r} & N_1 \arrow{d} \arrow{r} &
            N_2 \arrow{d} \arrow{r} & N_3 \arrow{d} \arrow{r} & 0 \\
            & \coker f \arrow{d} \arrow{r} &
            \coker g \arrow{d} \arrow{r} & \coker h \arrow{d} \arrow{r} & 0 \\
            & 0 & 0 & 0
        \end{tikzcd}
    \end{equation*}
\end{lemma}
\begin{proof}
    
\end{proof}
\begin{corollary}
    Suppose we have the two exact sequences,
    and the following diagram commutes.
    \begin{equation*}
        \begin{tikzcd}
            0 \arrow{r} & M_1 \arrow{d}{f} \arrow{r} &
            M_2 \arrow{d}{g} \arrow{r} & M_3 \arrow{d}{h} \arrow{r} & 0 \\
            0 \arrow{r} & N_1 \arrow{r} & N_2 \arrow{r} & N_3 \arrow{r} & 0
        \end{tikzcd}
    \end{equation*}
    If two of \(f,g,h\) are isomorphisms,
    then so is the third.
\end{corollary}
\begin{proof}
    
\end{proof}


\section{Tensor Products}

\begin{definition}[Universal Property of the Tensor Product]
    % Suppose \(M,N\) are \(R\)-modules.
    % Then the tensor product \(M \otimes_R N = M \otimes N\) is an \(R\)-module.
    Suppose \({\{M_i\}}_{i \in S}\) is a family of \(R\)-modules.
    The tensor product of \(R\)-modules is defined by the following universal property:
    \begin{enumerate}[label={(\roman*)}, itemsep=0mm]
        \item there exists a projection \(\func{\pi}{\prod_{i \in S} M_i}{\bigotimes_{i \in S} M_i}\)
            that is linear at every \(M_i\);
        \item for any module \(P\) and any \(i\)-linear map \(\func{\phi}{\prod_{i \in S} M_i}{P}\),
            there exists a unique homomorphism \(\func{\bar{\phi}}{\bigotimes_{i \in S} M_i}{P}\).
    \end{enumerate}

    This can be represented by the following commutative diagram:
    \begin{equation*}
        \begin{tikzcd}
            \prod_{i \in S} M_i \arrow{r}{\pi} \arrow{dr}{\phi} &
            \bigotimes_{i \in S} M_i \arrow[dashrightarrow]{d}{\exists!\bar{\phi}} \\
            & P
        \end{tikzcd}
    \end{equation*}
\end{definition}
\begin{theorem}
    The tensor product of modules exists and is unique up to isomorphism.
\end{theorem}
\begin{proof}
    
\end{proof}

\begin{definition}
    Suppose \({\{M_i, N_i\}}_{i \in S}\) are \(R\)-modules,
    and \(\func{f_i}{M_i}{N_i}\) are homomorphisms.
    Then \(\vfunc{\bigotimes f_i}{\bigotimes M_i}{\bigotimes N_i}{\bigotimes m_i}{\bigotimes f_i(m_i)}\)
    is a module homomorphism.
\end{definition}
\begin{theorem}
    There are canonical isomorphisms
    \begin{enumerate}[label={(\alph*)}, itemsep=0mm]
        \item \(M \otimes N \cong N \otimes M\) by exchanging positions;
        \item \(R \otimes N \cong N\) by multiplication;
        \item \((M \otimes N) \otimes P \cong M \otimes N \otimes P \cong M \otimes (N \otimes P)\); and
        \item the tensor distributes over the direct sum
            \((M \oplus N) \otimes P \cong (M \otimes P) \oplus (N \otimes P)\).
    \end{enumerate}
\end{theorem}
\begin{proof}
    
\end{proof}
\begin{corollary}
    \(\textbf{Mod}_R\) is a symmetric monoidal category;
    that is, \((\textbf{Mod}_R,\otimes)\) is a commutative monoid.
\end{corollary}
\begin{proof}
    
\end{proof}

\begin{lemma}
    \(\Hom(M \otimes N, P) \cong \Hom(M,\Hom(N,P))\),
    which is the set of bilinear maps from \(M \times N \to P\).
\end{lemma}
\begin{proof}
    
\end{proof}
\begin{theorem}
    Suppose \(N\) is an \(R\)-module.
    Consider the functor \(\vfunc{F}{\textbf{Mod}_R}{\textbf{Mod}_R}{M}{M \otimes N}\)
    that tensors by \(N\).
    Then \(F\) is right-exact.
\end{theorem}
\begin{proof}
    
\end{proof}

\begin{definition}
    Suppose \(N\) is an \(R\)-module.
    We say \(N\) is flat if tensoring by \(N\) is exact.
\end{definition}
\begin{lemma}
    Suppose \(N\) is an \(R\)-module,
    and \(F\) is the functor that tensors by \(N\).
    The following are equivalent:
    \begin{enumerate}[label={(\alph*)}, itemsep=0mm]
        \item \(N\) is flat;
        \item \(F\) sends exact sequences to exact sequences;
        \item \(F\) sends short exact sequences to short exact sequences;
        \item \(F\) sends an injective map to an injective map; and
        \item \(F\) sends an injective map of finitely generated modules
        to an injective map of finitely generated modules.
    \end{enumerate}
\end{lemma}
\begin{proof}
    
\end{proof}

\begin{definition}
    Suppose \(N\) is an \(R\)-module.
    Consider the functors \(F: M \mapsto \Hom(N,M)\) and \(G: M \mapsto \Hom(M,N)\).
    \(N\) is a projective module if \(F\) is exact,
    and an injective module of \(G\) is exact.
\end{definition}
\begin{proposition}
    Suppose \(P\) is a projective \(R\)-module,
    and \(M \to N\) is a surjective homomorphism.
    Then \(\Hom(P,M) \to \Hom(P,N)\) is surjective,
    and the following diagram holds:
    \begin{equation*}
        \begin{tikzcd}
            & P \arrow[dashrightarrow]{dl}[swap]{\exists!} \arrow{d} \\
            M \arrow{r} & N \arrow{r} & 0
        \end{tikzcd}
    \end{equation*}
\end{proposition}
\begin{proof}
    
\end{proof}
\begin{proposition}
    Suppose \(I\) is an injective \(R\)-module,
    and \(M \to N\) is a injective homomorphism.
    Then \(\Hom(M,I) \to \Hom(N,I)\) is injective,
    and the following diagram holds:
    \begin{equation*}
        \begin{tikzcd}
            & I \\
            0 \arrow{r} & M \arrow{u} \arrow{r} & N \arrow[dashrightarrow]{ul}[swap]{\exists!}
        \end{tikzcd}
    \end{equation*}
\end{proposition}
\begin{proof}
    
\end{proof}

\begin{proposition}
    Free modules are projective,
    projective modules are flat,
    and flat modules are torsion-free.
\end{proposition}
\begin{proof}
    
\end{proof}

\begin{definition}
    Suppose \(R \to S\) is a ring homomorphism.
    Then \(S\) is an \(R\)-algebra.
\end{definition}
\begin{definition}
    Suppose \(S,T\) are \(R\)-algebras.
    Then \(R\)-algebra homomorphisms are
    \begin{equation*}
        \begin{tikzcd}
            S \arrow{r} & T \\
            R \arrow{u} \arrow{r}{\id} & R \arrow{u}
        \end{tikzcd}
    \end{equation*}
\end{definition}
\begin{proposition}
    \(R\)-algebras are \(R\)-modules.
\end{proposition}
\begin{proof}
    
\end{proof}

\begin{proposition}
    Suppose \(S,T\) are \(R\)-algebras.
    Then \(S \otimes T\) is also an \(R\)-algebra.
\end{proposition}
\begin{proof}
    
\end{proof}

\begin{definition}
    Suppose \(S\) is an \(R\)-algebra.
    Then any \(S\)-module is also an \(R\)-module.
    This is the restriction of \(M\).
\end{definition}
\begin{definition}
    Suppose \(S\) is an \(R\)-algebra. and \(M\) is an \(R\)-module.
    Then \(M \otimes S\) is an \(S\)-module.
    This is the extension of \(M\).
\end{definition}


\section{Fractions}

\begin{definition}
    Suppose \(R\) is a ring, \(S \subseteq R\) is a multiplicative monoid.
    Then the ring of fractions is \(S^{-1}R = \{r/s : r \in R, s \in S\}/{\sim}\),
    where the equivalence relation is given by \(r_1/s_1 = r_2/s_2\)
    if there exists some \(s \in S\) such that \(s(r_1s_2 - r_2s_1) = 0\).
\end{definition}
\begin{proposition}
    There exists a ring homomorphism \(\vfunc{\iota}{R}{S^{-1}R}{n}{n/1}\).
\end{proposition}
\begin{proof}
    
\end{proof}
\begin{lemma}
    Suppose \(R\) is a ring, \(S \subseteq R\) is a multiplicative monoid.
    If \(s \in S\), then \(s\) is a unit in \(S^{-1}R\).
\end{lemma}
\begin{proof}
    
\end{proof}
\begin{lemma}
    Suppose \(R\) is a ring, \(S \subseteq R\) is a multiplicative monoid.
    \(r \in R\) is zero in \(S^{-1}R\) if and only if there exists some \(s\)
    such that \(sr = 0\) in \(R\).
\end{lemma}
\begin{proof}
    
\end{proof}

\begin{theorem}[Universal Property of Ring of Fractions]
    Suppose \(R\) is a ring, \(S \subseteq R\) is a multiplicative monoid.
    The ring of fractions is \(S^{-1}R\), if:
    \begin{enumerate}
        \item there exists a homomorphism \(\func{\iota}{R}{S^{-1}R}\); and
        \item for any ring \(T\), if \(\func{\phi}{R}{T}\) is a homomorphism that sends \(s \in S\) to a unit,
            then there exists a unique \(\func{\bar{\phi}}{S^{-1}R}{T}\)
            such that \(\phi = \bar{\phi}\circ\iota\).
    \end{enumerate}
    
    This is represented by the following commutative diagram:
    \begin{equation*}
        \begin{tikzcd}
            R \arrow{r}{\phi} \arrow{d}{\iota} & T \\
            S^{-1}R \arrow[dashrightarrow]{ur}[swap]{\exists!\bar{\phi}}
        \end{tikzcd}
    \end{equation*}
\end{theorem}
\begin{proof}
    
\end{proof}

\begin{proposition}
    Suppose \(R\) is a domain, and \(S\) is the set of units in \(R\).
    Then \(S^{-1}R\) is the field of fractions.
\end{proposition}
\begin{proof}
    
\end{proof}

\begin{definition}
    Suppose \(R\) is a ring, \(p \subseteq R\) is a prime ideal,
    and \(S = R - p\).
    Then we call \(S^{-1}R = R_p\) to be the localization of \(R\) at \(p\).
\end{definition}
\begin{proposition}
    Suppose \(R\) is a ring, and \(p \subseteq R\) is a prime ideal.
    \((R_p,pR_p)\) forms a local ring.
\end{proposition}
\begin{proof}
    
\end{proof}

\begin{definition}
    Suppose \(S^{-1}R\) is a ring of fractions, and \(M\) be an \(R\)-module.
    Then \(S^{-1}M = \{m/s : m \in M, s \in S\}/{\sim}\) is an \(S^{-1}R\)-module,
    with the equivalence relation is given by \(m_1/s_1 = m_2/s_2\)
    if there exists some \(s \in S\) such that \(s(m_1s_2 - m_2s_1) = 0\).
\end{definition}
\begin{proposition}
    There exists a module homomorphism \(\vfunc{\iota}{M}{S^{-1}M}{m}{m/1}\),
    where \(\ker\iota = \{m : \exists s \in S, ms = 0\}\).
\end{proposition}
\begin{proof}
    
\end{proof}

\begin{proposition}
    Suppose \(R\) is a ring, and \(S^{-1}R\) is a ring of fractions.
    Then \(\vfunc{S^{-1}}{\textbf{Mod}_R}{\textbf{Mod}_{S^{-1}R}}{M}{S^{-1}M}\)
    is a functor.
\end{proposition}
\begin{proof}
    
\end{proof}
\begin{theorem}
    \(S^{-1}\) is an exact functor.
\end{theorem}
\begin{proof}
    
\end{proof}

\begin{proposition}
    Suppose \(P\) is an \(R\)-module,
    and \(M,N \subseteq P\) are submodules.
    The following properties hold for modules of fractions:
    \begin{enumerate}[label={(\alph*)}, itemsep=0mm]
        \item \(S^{-1}(M+N) \cong S^{-1}M + S^{-1}N \subseteq S^{-1}P\);
        \item \(S^{-1}(M \cap N) \cong S^{-1}M \cap S^{-1}N \subseteq S^{-1}P\); and
        \item \(S^{-1}(M/N) \cong S^{-1}M/S^{-1}N \subseteq S^{-1}P\).
    \end{enumerate}
\end{proposition}
\begin{proof}
    
\end{proof}

\begin{theorem}
    There is an isomorphism \(S^{-1}R \otimes M \cong S^{-1}M\).
\end{theorem}
\begin{proof}
    
\end{proof}
\begin{corollary}
    \(S^{-1}R\) is a flat \(R\)-module.
\end{corollary}
\begin{proof}
    
\end{proof}

\begin{proposition}
    Suppose \(M,N\) are (appropriate) \(R\)-modules.
    The following properties hold for modules of fractions:
    \begin{enumerate}[label={(\alph*)}, itemsep=0mm]
        \item \(S^{-1}(M \oplus N) \cong S^{-1}M \oplus S^{-1}N\);
        \item \(S^{-1}(M/N) \cong S^{-1}M/S^{-1}N\); and
        \item \(S^{-1}(M \otimes N) \cong S^{-1}M \otimes S^{-1}N\).
    \end{enumerate}
\end{proposition}
\begin{proof}
    
\end{proof}
\begin{proposition}
    \(M\) is a flat \(R\)-module if and only if \(S^{-1}M\) is a flat \(S^{-1}R\)-module.
\end{proposition}
\begin{proof}
    
\end{proof}

\begin{definition}
    Suppose \(R,A\) are rings.
    \(N\) is an \(R,A\)-bimodule if it is both an \(R\)-module and an \(A\)-module.
\end{definition}
\begin{proposition}
    Suppose \(M\) is an \(R\)-module,
    \(N\) is an \(R,A\)-bimodule,
    and \(P\) is an \(A\)-module.
    Then \(M \otimes_R N \otimes_A P\) is an \(R,A\)-bimodule.
\end{proposition}
\begin{proof}
    
\end{proof}

\begin{proposition}
    Suppose \(R\) is a ring, \(M\) is an \(R\)-module,
    and \(M_p\) is a localization.
    If \(M\) is flat, then \(M_p\) is also flat.
    If \(M\) is finitely generated, then \(M_p\) is flat if and only if it is free.
\end{proposition}
\begin{proof}
    
\end{proof}
\begin{definition}
    Suppose \(M\) is an \(R\)-module, and \(M_p\) some localization.
    We say a module property is local when
    \(M\) has that property if and only if \(M_p\) for all prime \(p \subseteq R\) has that property.
    We say a module property is max-local when
    \(M\) has that property if and only if \(M_p\) for all maximal \(p \subseteq R\) has that property.
\end{definition}
\begin{lemma}
    Suppose \(M,N\) are \(R\)-modules, and \(\func{\phi}{M}{N}\) is a homomorphism.
    Then the following properties are local and max-local:
    \begin{enumerate}[label={(\alph*)}, itemsep=0mm]
        \item \(M = 0\);
        \item \(\phi\) injective; and
        \item \(\phi\) surjective.
    \end{enumerate}
\end{lemma}
\begin{proof}
    
\end{proof}
\begin{theorem}
    Flatness is local and max-local.
\end{theorem}
\begin{proof}
    
\end{proof}

\begin{proposition}
    Consider the ring homomorphism \(\func{\phi}{R}{S^{-1}R}\).
    If \(I \subseteq R\), its extension is \(I \otimes S^{-1}R = S^{-1}I\).
    If \(J \subseteq S^{-1}R\), its contraction is \(\phi^{-1}(J)\).
\end{proposition}
\begin{proof}
    
\end{proof}
\begin{theorem}
    There is a bijective correspondence between 
    the prime ideals of \(R\) that do not intersect with \(S\),
    and the prime ideals of \(S^{-1}R\).
    In that case, \(I \subseteq R\) corresponds to its extension,
    \(J \subseteq S^{-1}R\) corresponds to its contraction,
    and \(s \in S \subseteq R\) corresponds to \((1) \subseteq S^{-1}R\).
\end{theorem}
\begin{proof}
    
\end{proof}


\section{Primary Decomposition}

\begin{definition}
    Suppose \(R\) is a ring, and \(q \subsetneq R\) is a proper ideal.
    \(q\) is primary if \(xy \in q\) implies \(x \in q\) or \(y^n \in q\) for some \(n\).
\end{definition}
\begin{proposition}
    Suppose \(R\) is a ring, and \(q \subsetneq R\) is a proper ideal.
    \(q\) is primary if and only if every zero divisor in \(R/q\) is nilpotent.
\end{proposition}
\begin{proof}
    
\end{proof}
\begin{theorem}
    Suppose \(R\) is a ring, and \(q \subsetneq R\) is a proper ideal.
    If \(q\) is primary, then \(\sqrt{q}\) is prime.
\end{theorem}
\begin{proof}
    
\end{proof}
\begin{definition}
    Suppose \(q\) is primary.
    We say \(q\) is \(p\)-primary if \(\sqrt{q} = p\).
\end{definition}

\begin{lemma}
    Suppose \({\{q_i\}}_{i \in S}\) is a family of \(p\)-primary ideals.
    Then \(\bigcap_{i \in S} q_i\) is also \(p\)-primary.
\end{lemma}
\begin{proof}
    
\end{proof}
\begin{definition}
    Suppose \(q \subseteq R\) is an ideal.
    If \(q = \bigcap_i q_i\), we say \(q\) is decomposable.
    In this case, we may assume \(\sqrt{q_i} = p_i\) are distinct,
    and hence \(q_i\) should not be inside any \(q_j\).
    We call such a decomposition the minimal or irredundant decomposition.
\end{definition}
\begin{lemma}
    Suppose \(q_i\) is \(p_i\)-primary, and \(x \in R\).
    We have the following:
    \begin{enumerate}[label={(\alph*)}, itemsep=0mm]
        \item if \(x \in q_i\), then \((q_i:x) = (1)\); and
        \item if \(x \notin q_i\), then \((q_i:x)\) is \(p_i\)-primary.
    \end{enumerate}
\end{lemma}
\begin{proof}
    
\end{proof}
\begin{lemma}
    Suppose \(p = \bigcap_i J_i\) for a family of ideals.
    Then \(p = J_i\) for some \(i\).
\end{lemma}
\begin{proof}
    
\end{proof}
\begin{theorem}[First Uniqueness Theorem for Primary Decompositions]
    Suppose \(I\) is a decomposable ideal with an irredundant decomposition \(I = \bigcap_i q_i\).
    Then the associated primes \(p_i = \sqrt{q_i}\) are unique up to reordering.
\end{theorem}
\begin{proof}
    
\end{proof}

\begin{definition}
    Suppose \(I\) is a decomposable ideal,
    and \({\{p_i\}}_{i \in S}\) is a subset of the associated prime ideals.
    This set of ideals is isolated if any associated prime \(p \subseteq p_i\)
    is also in this subset \(p \in {\{p_i\}}_{i \in S}\).
\end{definition}
\begin{theorem}[Second Uniqueness Theorem for Primary Decompositions]
    Suppose \(I\) is a decomposable ideal with an irredundant decomposition \(I = \bigcap_i q_i\).
    Let \({\{q_{i_k}\}}_{k \in S}\) be an isolated set of associated primes.
    Then \(\bigcap_{i_k} q_{i_k}\) is uniquely determined independent of the decomposition.
\end{theorem}
\begin{proof}
    
\end{proof}


\section{Integral Dependence}

\begin{definition}
    Suppose \(A \subseteq R\), where \(R\) is an \(A\)-algebra.
    \(R\) is finite over \(A\) if \(R\) is a finitely generated \(A\)-module;
    that is, there is a surjection \(\bigoplus_{i=1}^n x_i A \to R\).
    \(R\) is finite-type over \(A\) if \(R\) is a finitely generated \(A\)-algebra;
    that is, there is a surjection \(A[x_1,\hdots,x_n] \to R\).
\end{definition}
\begin{definition}
    Suppose \(R\) is an \(A\)-algebra, and \(x \in R\).
    We say \(x\) is integral over \(A\) if it satisfies a monic polynomaial
    \(x^n + a_1 x^{n-1} + \cdots + a_n = 0\) for \(a_i \in A\).
    We say \(x\) is algebraic over \(A\) if it satisfies a polynomaial
    \(a_0 x^n + a_1 x^{n-1} + \cdots + a_n = 0\) for \(a_i \in A\).
\end{definition}
\begin{definition}
    \(A \subseteq R\) is integral if every \(x \in R\) is integral over \(A\).
\end{definition}
\begin{lemma}
    Suppose \(R\) is an \(A\)-algebra.
    \(x \in R\) is integral over \(A\) if and only if \(A[x]\) is a finite \(A\)-algebra.
\end{lemma}
\begin{proof}
    
\end{proof}
\begin{corollary}
    Suppose \(R\) is an \(A\)-algebra.
    If \(x \in R\) is integral over \(A\),
    then \(A[x]\) is integral over \(A\).
\end{corollary}
\begin{proof}
    
\end{proof}
\begin{corollary}
    Suppose \(R\) is an \(A\)-algebra,
    and \({\{x_i\}}_{i=1}^n \subset R\) are all integral over \(A\).
    Then \(A[x_1,\hdots,x_n]\) is finite over \(A\).
\end{corollary}
\begin{proof}
    
\end{proof}

\begin{definition}
    Suppose \(R\) is an \(A\)-algebra.
    The integral closure of \(A\) in \(R\) is an intermediate ring \(\overline{A}\)
    that consists of all elements in \(R\) that are integral over \(A\).
\end{definition}
\begin{proposition}
    Suppose \(R\) is an \(A\)-algebra.
    \(\overline{A}\) is indeed a ring.
\end{proposition}
\begin{proof}
    
\end{proof}
\begin{lemma}
    Suppose \(A \subseteq B \subseteq R\) are all rings.
    If \(A \subseteq B\) and \(B \subseteq R\) are both integral,
    then \(A \subseteq R\) is also integral.
\end{lemma}
\begin{proof}
    
\end{proof}
\begin{corollary}
    Suppose \(A \subseteq R\) is an extension of rings.
    Then \(\overline{\overline{A}} = \overline{A}\).
\end{corollary}
\begin{proof}
    
\end{proof}

\begin{lemma}
    Suppose \(A \subseteq R\) is an integral extension of rings.
    Then we have the following two integral extensions:
    \begin{enumerate}[label={(\alph*)}, itemsep=0mm]
        \item if \(I \subseteq R\) is an ideal, and \(J = I \cap A\),
        then \(A/J \subseteq R/I\) is integral; and
        \item if \(S \subseteq A\) is a multiplicative monoid,
        then \(S^{-1}A \subseteq S^{-1}R\) is integral.
    \end{enumerate}
\end{lemma}
\begin{proof}
    
\end{proof}
\begin{theorem}
    Suppose \(A \subseteq R\) is an integral extension of rings,
    and \(A,R\) are both integral domains.
    Then \(A\) is a field if and only if \(R\) is a field.
\end{theorem}
\begin{proof}
    
\end{proof}
\begin{corollary}
    Suppose \(A \subseteq R\) is an integral extension of rings.
    If \(q \subseteq R\) is prime, and \(p = q \cap A\) is also prime,
    then \(p\) is maximal if and only if \(q\) is maximal.
\end{corollary}
\begin{proof}
    
\end{proof}
\begin{theorem}
    Suppose \(A \subseteq R\) is an integral extension of rings,
    and \(p \subseteq A\) is prime.
    Then there exists some prime \(q \subseteq R\)
    such that \(p = q \cap A\).
\end{theorem}
\begin{proof}
    
\end{proof}

\begin{theorem}[Going Up Theorem]
    Suppose \(A \subseteq R\) is an integral extension of rings,
    \(p_1,p_2 \subseteq A\) and \(q_1 \subseteq R\) are prime ideals,
    with \(p_1 = q_1 \cap A\).
    Then there exists a prime ideal \(q_2 \subseteq R\)
    such that \(q_1 \subseteq q_2\) and \(p_2 = q_2 \cap A\).
\end{theorem}
\begin{proof}
    
\end{proof}
\begin{theorem}[Going Down Theorem]
    Suppose \(A \subseteq R\) is an integral extension of rings,
    \(p_1,p_2 \subseteq A\) and \(q_2 \subseteq R\) are prime ideals,
    with \(p_2 = q_2 \cap A\).
    Then there exists a prime ideal \(q_1 \subseteq R\)
    such that \(q_1 \subseteq q_2\) and \(p_1 = q_1 \cap A\).
\end{theorem}
\begin{proof}
    
\end{proof}

\begin{theorem}[Noetherian Normalization]
    Suppose \(K\) is an infinite field, and \(R\) is a finite-type \(K\)-algebra;
    that is, \(R = K[x_1,\hdots,x_n]/I\).
    Then there exists \({\{y_i\}}_{i=1}^m \subset R\) that are algebraically independent over \(K\),
    such that \(K[y_1,\hdots,y_m] \subseteq R\) is integral.
\end{theorem}
\begin{proof}
    
\end{proof}


\section{Noetherian and Artinian Rings}

\begin{proposition}
    Suppose \(A \subseteq B \subseteq R\) are rings.
    We have the following properties:
    \begin{enumerate}[label={(\alph*)}, itemsep=0mm]
        \item if \(A \subseteq B\) and \(B \subseteq C\) are finite, then \(A \subseteq C\) is also finite;
        \item if \(A \subseteq B\) and \(B \subseteq C\) are integral, then \(A \subseteq C\) is also integral;
        \item if \(A \subseteq R\) is finite, then \(B \subseteq R\) is finite;
        \item if \(A \subseteq R\) is integral, then \(B \subseteq R\) is integral; and
        \item if \(A \subseteq R\) is integral, then \(A \subseteq B\) is integral.
    \end{enumerate}
\end{proposition}
\begin{proof}
    
\end{proof}
\begin{remark}
    In general, if \(A \subseteq R\) is finite, \(A \subseteq B\) might not necessarily be finite.
\end{remark}

\begin{definition}
    Suppose \((\Sigma,\leq)\) is a partially ordered set.
    \(\Sigma\) satisfies the ascending chain condition
    if all ascending chains in \(\Sigma\) eventually stabilizes; that is,
    \begin{equation*}
        x_1 \leq x_2 \leq \cdots \implies x_n = x_{n+1} = \cdots
    \end{equation*}
    \(\Sigma\) satisfies the descending chain condition
    if all descending chains in \(\Sigma\) eventually stabilizes; that is,
    \begin{equation*}
        x_1 \geq x_2 \geq \cdots \implies x_n = x_{n+1} = \cdots
    \end{equation*}
\end{definition}
\begin{definition}
    Suppose \(M\) is an \(R\)-module.
    Let \((\Sigma, \subseteq)\) be the partially ordered set of all submodules of \(M\).
    We say \(M\) is Noetherian if \(\Sigma\) satisfies the ascending chain condition.
    We say \(M\) is Artinian if \(\Sigma\) satisfies the descending chain condition.
\end{definition}
\begin{definition}
    Suppose \(R\) is a ring.
    Let \((\Sigma, \subseteq)\) be the partially ordered set of all ideals of \(R\).
    We say \(R\) is Noetherian if \(\Sigma\) satisfies the ascending chain condition.
    We say \(R\) is Artinian if \(\Sigma\) satisfies the descending chain condition.
\end{definition}

\begin{theorem}
    \(M\) is a Noetherian module if and only if every submodule \(N \subseteq M\) is finitely generated.
\end{theorem}
\begin{proof}
    
\end{proof}
\begin{corollary}
    \(R\) is a Noetherian ring if and only if every ideal \(I \subseteq R\) is finitely generated.
\end{corollary}
\begin{proof}
    
\end{proof}

\begin{theorem}
    Suppose \(N \subset M\) is a submodule.
    Then \(M\) is Noetherian if and only if both \(N\) and \(M/N\) are Noetherian.
\end{theorem}
\begin{proof}
    
\end{proof}
\begin{corollary}
    Suppose \({\{M_i\}}_{i=1}^n\) is a family of Noetherian modules.
    Then \(\bigoplus_{i=1}^n M_i\) is Noetherian.
\end{corollary}
\begin{proof}
    
\end{proof}
\begin{corollary}
    Suppose \(R\) is a Noetherian ring,
    and \(M\) a finitely generated \(R\)-module.
    Then \(M\) is Noetherian.
\end{corollary}
\begin{proof}
    
\end{proof}
\begin{corollary}
    Suppose \(R\) is a Noetherian ring,
    and \(I \subseteq R\) an ideal.
    Then \(R/I\) is Noetherian.
\end{corollary}
\begin{proof}
    
\end{proof}

\begin{theorem}
    Suppose \(A\) is a Noetherian ring, and \(R\) is a finite \(A\)-algebra.
    Then \(R\) is a Noetherian ring.
\end{theorem}
\begin{proof}
    
\end{proof}
\begin{theorem}
    Suppose \(R\) is a Noetherian ring.
    Then any ring of fractions \(S^{-1}R\) is also Noetherian.
\end{theorem}
\begin{proof}
    
\end{proof}

\begin{lemma}
    Suppose \(R\) is a ring, and \(S^{-1}R\) is a ring of fractions.
    Then every ideal of \(S^{-1}R\) has the form \(S^{-1}I\),
    where \(I \subseteq R\) is an ideal.
\end{lemma}
\begin{proof}
    
\end{proof}
\begin{theorem}[Hilbert Basis Theorem]
    Suppose \(R\) is a Noetherian ring.
    Then \(R[x]\) is also a Noetherian ring.
\end{theorem}
\begin{proof}
    
\end{proof}
\begin{corollary}
    Suppose \(R\) is a Noetherian ring.
    Then \(R[x_1,\hdots,x_n]\) is Noetherian,
    and any finite-type \(R\)-algebra \(R[x_1,\hdots,x_n]/I\) is also Noetherian.
\end{corollary}
\begin{proof}
    
\end{proof}
\begin{remark}
    We essentially have the following results:
    \begin{enumerate}[label={(\roman*)}, itemsep=0mm]
        \item \(R\) is finite over \(A\) if and only if it is integral and finite-type; and
        \item if \(R\) is finite-type over \(A\), then \(R\) is Noetherian over \(A\).
    \end{enumerate}
\end{remark}

\begin{proposition}
    Suppose \(A\) is a Noetherian ring,
    and \(A \subseteq B \subseteq R\) is an extension of rings.
    If \(A \subseteq R\) is finite, then \(A \subseteq B\) is finite.
\end{proposition}
\begin{proof}
    
\end{proof}
\begin{proposition}
    Suppose \(A \subseteq B \subseteq R\) is an extension of rings.
    If \(R\) is Noetherian, then \(B\) is Noetherian.
\end{proposition}
\begin{proof}
    
\end{proof}
\begin{theorem}
    Suppose \(A \subseteq B \subseteq R\) is an extension of rings.
    If \(A\) is Noetherian, \(C\) is finite-type of \(A\),
    and \(C\) is finite over \(B\),
    then \(B\) is finite-type over \(A\).
\end{theorem}
\begin{proof}
    
\end{proof}
\begin{proposition}
    Suppose \(K\) is a field, and \(E\) a finitely generated \(K\)-algebra.
    If \(E\) is a field, then \(E/K\) is a finite algebraic extension.
\end{proposition}
\begin{proof}
    
\end{proof}
\begin{proposition}
    Suppose \(K(y_1,\hdots,y_n)/K\) is a purely transcendental extension.
    Then \(E\) cannot be finite-type over \(K\).
\end{proposition}
\begin{proof}
    
\end{proof}
\begin{corollary}[Weak Hilbert's Nullstellensatz]
    Suppose \(K\) is a field, and \(A\) is a finite-type \(K\)-algebra.
    Let \(m \subseteq A\) be a maximal ideal.
    Then \(A/m\) is a finite algebraic extension of \(K\).
    If \(K = \overline{K}\), then \(A/m = K\).
\end{corollary}
\begin{proof}
    
\end{proof}

\begin{definition}
    Suppose \(K\) is a field, and \(R = K[x_1,\hdots,x_n]\).
    Consider \({\{f_i\}}_{i \in S} \subset R\) a family of polynomial functions,
    \(\func{f_i}{K^n}{K}\).
    The vanishing locus is the set of points
    \(\Van(\{f_i\}) = \{p \in K^n: \forall i \in S,\,f_i(p) = 0\}\).
    Now consider \(X \subseteq K^n\).
    The ideal of all vanishing polynomials is
    \(\Ide(X) = \{f \in R: \forall x \in X,\,f(x) = 0\}\).
\end{definition}
\begin{corollary}
    Suppose \(K = \overline{K}\) is algebraically closed.
    Every maximal ideal of \(K[x_1,\hdots,x_n]\) is of the form
    \((x_1-a_1,\hdots,x_n-a_n)\) for some \(a_i \in K\).
    If \(I \subsetneq K[x_1,\hdots,x_n]\) is a proper ideal,
    then the vanishing locus \(\Van(I) \neq \emptyset\).
\end{corollary}
\begin{proof}
    
\end{proof}
\begin{theorem}[Hilbert's Nullstellensatz]\label{thm:nullstellensatz}
    Suppose \(K = \overline{K}\) is an algebraically closed field,
    and \(R = K[x_1,\hdots,x_n]\) is a polynomial ring.
    Let \(J \subseteq R\) be an ideal, and \(\Van(J)\) be its vanishing locus.
    Then \(\Ide(\Van(J)) = \sqrt{J}\).
\end{theorem}
\begin{proof}
    
\end{proof}

\begin{definition}
    Suppose \(R\) is a Noetherian ring, and \(I \subseteq R\) is an ideal.
    \(I\) is irreducible if \(I = I_1 \cap I_2\) implies \(I = I_1\) or \(I = I_2\).
\end{definition}
\begin{lemma}
    Suppose \(R\) is a Noetherian ring.
    Then every ideal \(I \subseteq R\) is a finite intersection of irreducible ideals
    \(I = \bigcap_{j=1}^n I_j\).
\end{lemma}
\begin{proof}
    
\end{proof}
\begin{lemma}
    Suppose \(R\) is a Noetherian ring.
    If \(I \subseteq R\) is an irreducible ideal,
    then it is also primary.
\end{lemma}
\begin{proof}
    
\end{proof}
\begin{proposition}
    Suppose \(R\) is a Noetherian ring, and \(I \subseteq R\) is an ideal.
    Then \({(\sqrt{I})^n} \subseteq I\) for some \(n\).
\end{proposition}
\begin{proof}
    
\end{proof}
\begin{corollary}
    Suppose \(R\) is a Noetherian ring, and \(\mca{N}(R)\) its nilradical.
    Then \({(\mca{N}(R))}^n = 0\) for some \(n\).
\end{corollary}
\begin{proof}
    
\end{proof}
\begin{corollary}
    Suppose \(R\) is a Noetherian ring, and \(m \subseteq R\) is a maximal ideal.
    Let \(q\) be a primary ideal.
    Then the following are equivalent:
    \begin{enumerate}[label={(\alph*)}, itemsep=0mm]
        \item \(q \subseteq R\) is \(m\)-primary;
        \item \(\sqrt{q} = m\); and
        \item \(m^n \subseteq q \subseteq m\) for some \(n\).
    \end{enumerate}
\end{corollary}
\begin{proof}
    
\end{proof}

\begin{theorem}
    Suppose \(R\) is a Noetherian ring.
    Then any ideal has a primary decomposition.
\end{theorem}
\begin{proof}
    
\end{proof}


\part{Theory of Structures}
\chapter{Homological Algebra}

\section{Basic Definitions}

\section{Derived Functors}


\backmatter

\end{document}