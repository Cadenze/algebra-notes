\chapter{Commutative Algebra}

\begin{remark}
    We shall assume that all rings,
    unless otherwise noted, are commutative;
    that is, the multiplication \(ab = ba\).
\end{remark}

\section{Ideals}

\begin{proposition}
    Suppose \(R\) is a ring.
    \(x \in R\) is a unit if and only if the ideals \((x) = (1)\).
\end{proposition}
\begin{proof}
    
\end{proof}

\begin{theorem}
    Suppose \(R\) is a nondegenerate ring.
    The following are equivalent:
    \begin{enumerate}[label={(\alph*)}, itemsep=0mm]
        \item \(R\) is a field;
        \item \(R\) has only two distinct ideals \((0)\) and \((1)\); and
        \item Every ring homomorphism \(\func{\phi}{R}{S}\) is injective.
    \end{enumerate}
\end{theorem}
\begin{proof}
    
\end{proof}

\begin{definition}
    Suppose \(R\) is a ring.
    An element \(x \in R\) is nilpotent if there exists \(n > 0\) such that \(x^n = 0\).
    Any nonzero nilpotent element is a zero divisor.
    A ring \(R\) is reduced if the only nilpotent is 0.
\end{definition}

\begin{definition}
    Suppose \(R\) is a ring, \(I \subseteq R\) an ideal.
    \(I\) is maximal if \(I \neq R\),
    and there does not exist any intermediate ideals \(J\)
    such that \(I \subseteq J \subseteq R\).
\end{definition}
\begin{theorem}\label{thm:maximal-quotient-field}
    Suppose \(R\) is a ring, \(I \subseteq R\) an ideal.
    \(I \subseteq R\) is maximal if and only if \(R/I\) is a field.
\end{theorem}
\begin{proof}
    Corollary~\ref{cor:maximal-quotient-field}.
\end{proof}

\begin{definition}
    Suppose \(R\) is a ring, \(I \subseteq R\) an ideal.
    \(I\) is prime if \(xy \in I\) implies \(x \in I\) or \(y \in I\);
    that is, the complement of \(I\) is closed under multiplication.
\end{definition}
\begin{theorem}\label{thm:prime-quotient-domain}
    Suppose \(R\) is a ring, \(I \subseteq R\) an ideal.
    \(I\) is prime if and only if \(R/I\) is a domain.
\end{theorem}
\begin{proof}
    Proposition~\ref{prop:prime-quotient-domain}.
\end{proof}

\begin{definition}
    Suppose \(R\) is a ring, \(I \subseteq R\) an ideal.
    \(I\) is radical if for all \(x \in R\) and \(x^n \in I\), then \(x \in I\).
\end{definition}
\begin{theorem}\label{thm:radical-quotient-reduced}
    Suppose \(R\) is a ring, \(I \subseteq R\) an ideal.
    \(I\) is radical if and only if \(R/I\) is reduced.
\end{theorem}
\begin{proof}
    Suppose \(I\) is not radical;
    then there exists some \(x \notin I\) such that \(x^n \in I\).
    Hence there exists some nonzero \([x] \in R/I\) such that \([x]^n = 0\),
    and \(R/I\) is not reduced.
    The reverse argument holds the same way since all those logical steps are bidirectional.
\end{proof}

\begin{theorem}
    Maximal ideals are prime, and prime ideals are radical.
\end{theorem}
\begin{proof}
    
\end{proof}

\begin{theorem}
    Every ring has a maximal ideal.
\end{theorem}
\begin{proof}
    Let \(\Sigma\) be the set of all proper ideals of some ring \(R\).
    Let this set be partially ordered by \(\subseteq\), the subset relation.
    We first see that \(\Sigma\neq\emptyset\) due to \((0) \in \Sigma\).
    We then claim that every chain \(I_1 \subseteq I_2 \subseteq I_3 \subseteq \cdots\)
    is contained within the union \(I = \bigcup_i I_i \in \Sigma\).
    By Proposition~\ref{prop:nested-ideals}, \(I\) is an ideal,
    and by definition, \(I_i \subseteq I\).
    We also see that \(I \neq R\) because if \(1 \in I\),
    then \(1 \in I_i\) for some \(i\), which gives us a contradiction.
    Hence by \hyperref[ax:zorn]{Zorn's lemma}, there is a maximal element, a maximal ideal.
\end{proof}
\begin{corollary}
    Every proper ideal lies in a maximal ideal.
\end{corollary}
\begin{proof}
    
\end{proof}
\begin{corollary}
    Suppose \(R\) is a ring, and \(x \in R\) is not a unit.
    Then \(x\) lies in a maximal ideal.
\end{corollary}
\begin{proof}
    
\end{proof}

\begin{definition}
    Suppose \(R\) is a ring.
    \(R\) is a local ring if it has a unique maximal ideal \(m \subseteq R\).
\end{definition}
\begin{theorem}
    Suppose \((R,m)\) forms a local ring.
    Then \(R-m\) is the group of units in \(R\).
\end{theorem}
\begin{proof}
    
\end{proof}
\begin{theorem}
    Suppose \(R\) is a ring, and \(I \subseteq R\) is a proper ideal.
    If every \(x \in R-I\) is a unit, then \((R,I)\) is a local ring.
\end{theorem}
\begin{proof}
    
\end{proof}
\begin{theorem}
    Suppose \(R\) is a ring, and \(I \subseteq R\) is proper ideal.
    If for every \(x \in I\), \(1+x\) is a unit, then \((R,I)\) is a local ring.
\end{theorem}
\begin{proof}
    
\end{proof}

\begin{definition}
    Suppose \(R\) is a ring.
    The set of nilpotent elements is called the nilradical of \(R\).
\end{definition}
\begin{theorem}
    Suppose \(R\) is a ring, and \(N\) its nilradical.
    Then \(N\) is an ideal, and \(R/N\) is reduced.
\end{theorem}
\begin{proof}
    
\end{proof}
\begin{theorem}
    The nilradical is the intersection of all prime ideals.
\end{theorem}
\begin{proof}
    
\end{proof}
\begin{proposition}
    Suppose \(I,J,K\) are three ideals.
    We have the two following distributive properties:
    \begin{enumerate}[label={(\alph*)}, itemsep=0mm]
        \item \(I(J+K) = IJ + IK\); and
        \item \(I \cap (J+K) \supseteq I \cap J + I \cap K\),
            with equality only if \(J \subseteq I\) or \(K \subseteq I\).
    \end{enumerate}
\end{proposition}
\begin{proof}
    
\end{proof}

\begin{definition}
    Suppose \(I,J\) are two ideals.
    They are coprime if \(I+J = (1)\).
\end{definition}
\begin{theorem}
    Suppose \({\{J_i\}}_{i=1}^n\) are pairwise coprime ideals.
    Then \(\bigcap_{i=1}^n J_i = \prod_{i=1}^n J_i\).
\end{theorem}
\begin{proof}
    
\end{proof}

\begin{definition}
    Suppose \(\{R_i\}\) is a family of rings.
    The direct product is \(\prod_i R_i = \{(r_1,r_2,\hdots) : r_i \in R_i\}\).
\end{definition}
\begin{lemma}
    Suppose \(R\) is a ring, and \(\{J_i\}\) is a family of ideals.
    Let us define the homomorphism \(\vfunc{\phi}{R}{\prod_i R/J_i}{r}{(r+J_1,r+J_2,\hdots)}\).
    Then we have \(\ker(\phi) = \bigcap_i J_i\),
    and \(\img(\phi) = R/\prod_i J_i\).
\end{lemma}
\begin{proof}
    
\end{proof}
\begin{theorem}[Chinese Remainder Theorem]
    Suppose \(R\) is a ring, and \(\{J_i\}\) is a family of ideals.
    Consider the homomorphism \(\func{\phi}{R}{\prod_i R/J_i}\).
    Then \(\phi\) is surjective if and only if \(\{J_i\}\) are pairwise coprime.
    In that case, \(\prod_i R/J_i \cong R/\prod_i J_i\).
\end{theorem}
\begin{proof}
    
\end{proof}

\begin{definition}
    Suppose \(R\) is a ring, and \(I,J\) are ideals.
    The quotient ideal is \((I:J) = \{x \in R : xJ \subseteq I\}\).
\end{definition}
\begin{proposition}
    The following properties hold for quotient ideals:
    \begin{enumerate}[label={(\alph*)}, itemsep=0mm]
        \item \(I \subseteq (I:J)\);
        \item \((I:J)J \subseteq I\);
        \item \((I:I) = (1)\); and
        \item \(((I:J):K) = (I:JK) = ((I:K):J)\).
    \end{enumerate}
\end{proposition}
\begin{proof}
    
\end{proof}

\begin{definition}
    Suppose \(\func{\phi}{R}{S}\) is any ring homomorphism.
    Let \(I \subseteq R\) and \(J \subseteq S\) be ideals.
    \(\phi(I) \subseteq S\) might not be an ideal,
    so we define the ideal it generates \((\phi(I)) \subseteq \) the extension of \(I\).
\end{definition}
\begin{remark}
    \(\phi^{-1}(J) \subseteq R\) is always an ideal,
    and if \(J\) prime, \(\phi^{-1}(J)\) is also prime.
\end{remark}
\begin{definition}
    Suppose \(\func{\phi}{R}{S}\) is a ring monomorphism.
    If \(J \subseteq S\), we say \(\phi^{-1}(J) = J \cap R\) is the contraction of \(J\).
\end{definition}


\section{Modules}

\begin{proposition}
    Suppose \(R\) is a ring, and \(M,N\) are \(R\)-modules.
    Then \(\Hom_R(M,N)\) is also an \(R\)-module.
\end{proposition}
\begin{proof}
    
\end{proof}

\begin{proposition}
    Suppose \(R\) is a ring.
    The category \(\textbf{Mod}_R\) consisting of all \(R\)-modules
    and module homomorphisms form a category.
\end{proposition}
\begin{proof}
    
\end{proof}

\begin{proposition}
    Suppose \(M\) is an \(R\)-module.
    There exists a functor \(\func{F}{\textbf{Mod}_R}{\textbf{Mod}_R}\)
    that maps \(N \mapsto \Hom_R(M,N)\)
    and maps \(\func{\phi}{N}{L}\) to \(\func{\tilde{\phi}}{\Hom_R(M,N)}{\Hom_R(M,L)}\).
\end{proposition}
\begin{proof}
    
\end{proof}
\begin{corollary}
    \(\Hom_R(M,N) \cong N\).
\end{corollary}
\begin{proof}
    
\end{proof}

\begin{definition}
    Suppose \(L\) is an \(R\)-module,
    and \(M,N \subseteq L\) are submodules.
    The quotient ideal \((M:N) = \{x \in R : xN \subseteq M\}\).
\end{definition}
\begin{definition}
    Suppose \(N\) is an \(R\)-module.
    The annihilator of \(N\) is \(\Ann_R(N) = (0:N) = \{x \in R : xN = 0\}\).
\end{definition}

\begin{definition}
    Suppose \({\{M_i\}}_{i \in S}\) is a family of \(R\)-modules.
    The direct product is \(\prod_i M_i = \{{(m_i)}_{i \in S}\}\),
    and the direct sum is \(\bigoplus_i M_i = \{{(m_i)_{i \in S}}\}\) with finite support.
\end{definition}

\begin{definition}
    Suppose \(M\) is an \(R\)-module, and \({\{m_i\}}_{i \in S} \subseteq M\).
    The module it generates is \(N = \{\sum_{j=1}^n x_j m_j : x_j \in R, m_j \in {\{m_i\}}_{i \in S}\}\).
    If \(M = N\), we say \({\{m_i\}}_{i \in S}\) generates \(M\);
    in that case, if \(S\) is finite, we say \(M\) is finitely generated.
\end{definition}
\begin{proposition}
    Suppose \(M\) is an \(R\)-module, and \({\{m_i\}}_{i \in S} \subseteq M\).
    Consider the homomorphism \(\vfunc{\phi}{\bigoplus_{i \in S} R}{M}{{(x_i)}_i}{\sum_i x_i m_i}\).
    We have the following:
    \begin{enumerate}[label={(\alph*)}, itemsep=0mm]
        \item \(\img(\phi)\) is the module generated by \({\{m_i\}}_{i \in S}\);
        \item \(\phi\) is surjective if \({\{m_i\}}_{i \in S}\) generate \(M\); and
        \item \(M\) is finitely generated if there exists a surjective \(\func{\phi}{R^n}{M}\)./
    \end{enumerate}
\end{proposition}
\begin{proof}
    
\end{proof}

\begin{lemma}[Cayley-Hamilton Theorem]
    Suppose \(M\) is a finitely-generated \(R\)-module,
    and suppose we have an endomorphism \(\func{\phi}{M}{M}\).
    Then there exists a monic polynomial \(\phi^n + a_1\phi^{n-1} + \cdots + a_n = 0\).
    Moreover, if there exists \(I \subseteq R\) such that \(\img(\phi) \subseteq IM\),
    then the coefficients \(\{a_i\} \subseteq I\).
\end{lemma}
\begin{proof}
    
\end{proof}
\begin{corollary}[Nakayama's Lemma]
    Suppose \(M\) is a finitely-generated \(R\)-module,
    with \(I \subseteq R\) is an ideal, and \(IM = M\).
    Then there exists \(x \in R\), \(x \equiv 1\pmod{I}\), such that \(xM = 0\).
\end{corollary}
\begin{proof}
    
\end{proof}
\begin{corollary}
    Suppose \(M\) is a finitely-generated \(R\)-module,
    and a surjective homomorphism \(\func{\alpha}{M}{M}\).
    Then \(\alpha\) is an isomorphism.
\end{corollary}
\begin{proof}
    
\end{proof}

\begin{definition}
    Suppose \(R\) is a ring.
    The Jacobson radical of \(R\) is the intersection of all maximal ideals,
    \(\mca{R} = \bigcap m\).
\end{definition}
\begin{remark}
    The nilradical is inside the Jacobson radical.
\end{remark}
\begin{lemma}
    Suppose \(R\) is a ring, \(\mca{R}\) is its Jacobson radical.
    Any element \(x \in \mca{R}\) if and only if \(1 - xy\) is a unit for all \(y \in R\).
\end{lemma}
\begin{proof}
    
\end{proof}
\begin{theorem}[Nakayama's Lemma]
    Suppose \(M\) is a finitely generated \(R\)-module,
    and \(\mca{R}\) is the Jacobson radical of \(R\).
    Let \(I \subseteq \mca{R}\) be an ideal.
    If \(IM = M\), then \(M = 0\).
\end{theorem}
\begin{proof}
    
\end{proof}
\begin{corollary}
    Suppose \(M\) is a finitely generated \(R\)-module,
    and \(N \subseteq M\) is a submodule.
    Let \(I \subseteq \mca{R}\) be an ideal inside the Jacobson radical of \(R\).
    If \(IM + N = M\), then \(M = N\).
\end{corollary}
\begin{proof}
    
\end{proof}
\begin{corollary}
    Suppose \((R,m)\) is a local ring,
    and \(M\) a finitely generated \(R\)-module.
    Then \(M/mM\) is a finite-dimensional vector space.
\end{corollary}
\begin{proof}
    
\end{proof}

\begin{theorem}
    Suppose \((R,m)\) is a local ring,
    and \(M\) a finitely generated \(R\)-module.
    If \({\{x_i\}}_{i=1}^n \subset M\) span \(M/mM\) as a vector space,
    then \({\{x_i\}}_{i=1}^n\) generate \(M\).
\end{theorem}
\begin{proof}
    
\end{proof}


\section{Exact Sequences}

\begin{definition}
    Consider a sequence of \(R\)-modules with corresponding module homomorphisms.
    \begin{equation*}
        \begin{tikzcd}
            \cdots \arrow{r} & M_{i-1} \arrow{r}{\phi_i} &
            M_i \arrow{r}{\phi_{i+1}} & M_{i+1} \arrow{r} & \cdots
        \end{tikzcd}
    \end{equation*}
    This sequence is exact at \(M_i\) if \(\img(\phi_i) = \ker(\phi_{i+1})\).
    The sequence is exact if it is exact at every \(M_i\).
\end{definition}
% \begin{proposition}
%     A sequence is exact at \(M_i\) if and only if the following two conditions hold:
%     \begin{enumerate}[label={(\alph*)}, itemsep=0mm]
%         \item \(\phi_{i+1}\circ\phi_i = 0\); and
%         \item \(\phi_{i+1}(m) = 0\) implies \(m = \)
%     \end{enumerate}
% \end{proposition}

\begin{proposition}
    We have the following two characterizations of homomorphisms:
    \begin{enumerate}[label={(\alph*)}, itemsep=0mm]
        \item \(0 \to M \to N\) exact if and only if \(M \to N\) is injective; and
        \item \(M \to N \to 0\) exact if and only if \(M \to N\) is surjective.
    \end{enumerate}
\end{proposition}
\begin{proof}
    
\end{proof}

\begin{definition}
    A short exact sequence is an exact sequence
    \begin{equation*}
        \begin{tikzcd}
            0 \arrow{r} & M_1 \arrow{r} & M_2 \arrow{r} & M_3 \arrow{r} & 0
        \end{tikzcd}
    \end{equation*}
\end{definition}
\begin{proposition}
    Such a sequence is a short exact sequence if and only if \(M_3 = M_2/M_1\).
\end{proposition}
\begin{proof}
    
\end{proof}
\begin{theorem}
    Every exact sequence can be cut into short exact sequences.
\end{theorem}
\begin{proof}
    
\end{proof}

\begin{proposition}[Euler Characteristic]
    Suppose we have an exact sequence of finite-dimensional \(K\)-vector spaces.
    \begin{equation*}
        \sigma: \qquad
        \begin{tikzcd}
            0 \arrow{r} & V_1 \arrow{r} & V_2 \arrow{r} & \cdots \arrow{r} &
            V_n \arrow{r} & 0
        \end{tikzcd}
    \end{equation*}
    Then the Euler characteristic obeys
    \begin{equation*}
        \chi(\sigma) = \sum_{i=1}^n {(-1)}^i \dim(V_i) = 0
    \end{equation*}
\end{proposition}
\begin{proof}
    
\end{proof}
\begin{lemma}
    Suppose \(C\) is a class of \(R\)-modules,
    closed under images and kernels.
    Let \(\func{\lambda}{C}{\bZ}\) be additive.
    If we have a short exact sequence in \(C\),
    \begin{equation*}
        \begin{tikzcd}
            0 \arrow{r} & M_1 \arrow{r} & M_2 \arrow{r} & M_3 \arrow{r} & 0
        \end{tikzcd}
    \end{equation*}
    then \(-\lambda(M_1) + \lambda(M_2) - \lambda(M_3) = 0\).
\end{lemma}
\begin{proof}
    
\end{proof}
\begin{theorem}
    An exact sequence in \(C\) yields \(\sum_{i=1}^n {(-1)}^i \lambda(M_i) = 0\).
\end{theorem}
\begin{proof}
    
\end{proof}

\begin{definition}
    Consider a functor \(\func{F}{\textbf{Mod}_R}{\textbf{Mod}_R}\).
    \(F\) is exact if it sends short exact sequences to short exact sequences.
    \(F\) is right-exact if it sends an exact \(M_1 \to M_2 \to M_3 \to 0\)
    to an exact \(F(M_1) \to F(M_2) \to F(M_3) \to 0\).
    \(F\) is left-exact if it sends an exact \(0 \to M_1 \to M_2 \to M_3\)
    to an exact \(0 \to F(M_1) \to F(M_2) \to F(M_3)\).
\end{definition}
\begin{proposition}
    Any exact functor sends exact sequences to exact sequences.
\end{proposition}
\begin{proof}
    
\end{proof}

\begin{theorem}
    Suppose \(M\) is an \(R\)-module.
    Let us consider two functors \(F,G\),
    with \(F: N \mapsto \Hom_R(M,N)\) and \(G: N \mapsto \Hom_R(N,M)\).
    Then \(F\) is a covariant left-exact functor,
    and \(G\) is a contravariant right-exact functor.
\end{theorem}
\begin{proof}
    
\end{proof}

\begin{lemma}[Snake Lemma]
    Suppose we have the two exact sequences,
    and the following diagram commutes.
    \begin{equation*}
        \begin{tikzcd}
            0 \arrow{r} & M_1 \arrow{d}{f} \arrow{r} &
            M_2 \arrow{d}{g} \arrow{r} & M_3 \arrow{d}{h} \arrow{r} & 0 \\
            0 \arrow{r} & N_1 \arrow{r} & N_2 \arrow{r} & N_3 \arrow{r} & 0
        \end{tikzcd}
    \end{equation*}
    Then the following sequence is exact.
    \begin{equation*}
        \begin{tikzcd}
            0 \arrow{r} & \ker f \arrow{r} & \ker g \arrow{r} & \ker h \arrow{r}{\delta} &
            \coker f \arrow{r} & \coker g \arrow{r} & \coker h \arrow{r} & 0
        \end{tikzcd}
    \end{equation*}

    This can be represented by the following commutative diagram:
    \begin{equation*}
        \begin{tikzcd}
            & 0 \arrow{d} & 0 \arrow{d} & 0 \arrow{d} \\
            0 \arrow{r} & \ker f \arrow{d} \arrow{r} &
            \ker g \arrow{d} \arrow{r} \arrow[ddd, phantom, ""{coordinate, name=Z}] & \ker h \arrow{d}
            \arrow[rounded corners, dashrightarrow, to path={
                -- ([xshift=12ex]\tikztostart.east)
                |- (Z) [near end]\tikztonodes
                -| ([xshift=-12ex]\tikztotarget.west)
                -- (\tikztotarget)
            }]{dddll}[at start]{\delta} \\
            0 \arrow{r} & M_1 \arrow{d}{f} \arrow{r} &
            M_2 \arrow{d}{g} \arrow{r} & M_3 \arrow{d}{h} \arrow{r} & 0 \\
            0 \arrow{r} & N_1 \arrow{d} \arrow{r} &
            N_2 \arrow{d} \arrow{r} & N_3 \arrow{d} \arrow{r} & 0 \\
            & \coker f \arrow{d} \arrow{r} &
            \coker g \arrow{d} \arrow{r} & \coker h \arrow{d} \arrow{r} & 0 \\
            & 0 & 0 & 0
        \end{tikzcd}
    \end{equation*}
\end{lemma}
\begin{proof}
    
\end{proof}
\begin{corollary}
    Suppose we have the two exact sequences,
    and the following diagram commutes.
    \begin{equation*}
        \begin{tikzcd}
            0 \arrow{r} & M_1 \arrow{d}{f} \arrow{r} &
            M_2 \arrow{d}{g} \arrow{r} & M_3 \arrow{d}{h} \arrow{r} & 0 \\
            0 \arrow{r} & N_1 \arrow{r} & N_2 \arrow{r} & N_3 \arrow{r} & 0
        \end{tikzcd}
    \end{equation*}
    If two of \(f,g,h\) are isomorphisms,
    then so is the third.
\end{corollary}
\begin{proof}
    
\end{proof}


\section{Tensor Products}

\begin{definition}[Universal Property of the Tensor Product]
    % Suppose \(M,N\) are \(R\)-modules.
    % Then the tensor product \(M \otimes_R N = M \otimes N\) is an \(R\)-module.
    Suppose \({\{M_i\}}_{i \in S}\) is a family of \(R\)-modules.
    The tensor product of \(R\)-modules is defined by the following universal property:
    \begin{enumerate}[label={(\roman*)}, itemsep=0mm]
        \item there exists a projection \(\func{\pi}{\prod_{i \in S} M_i}{\bigotimes_{i \in S} M_i}\)
            that is linear at every \(M_i\);
        \item for any module \(P\) and any \(i\)-linear map \(\func{\phi}{\prod_{i \in S} M_i}{P}\),
            there exists a unique homomorphism \(\func{\bar{\phi}}{\bigotimes_{i \in S} M_i}{P}\).
    \end{enumerate}

    This can be represented by the following commutative diagram:
    \begin{equation*}
        \begin{tikzcd}
            \prod_{i \in S} M_i \arrow{r}{\pi} \arrow{dr}{\phi} &
            \bigotimes_{i \in S} M_i \arrow[dashrightarrow]{d}{\exists!\bar{\phi}} \\
            & P
        \end{tikzcd}
    \end{equation*}
\end{definition}
\begin{theorem}
    The tensor product of modules exists and is unique up to isomorphism.
\end{theorem}
\begin{proof}
    
\end{proof}

\begin{definition}
    Suppose \({\{M_i, N_i\}}_{i \in S}\) are \(R\)-modules,
    and \(\func{f_i}{M_i}{N_i}\) are homomorphisms.
    Then \(\vfunc{\bigotimes f_i}{\bigotimes M_i}{\bigotimes N_i}{\bigotimes m_i}{\bigotimes f_i(m_i)}\)
    is a module homomorphism.
\end{definition}
\begin{theorem}
    There are canonical isomorphisms
    \begin{enumerate}[label={(\alph*)}, itemsep=0mm]
        \item \(M \otimes N \cong N \otimes M\) by exchanging positions;
        \item \(R \otimes N \cong N\) by multiplication;
        \item \((M \otimes N) \otimes P \cong M \otimes N \otimes P \cong M \otimes (N \otimes P)\); and
        \item the tensor distributes over the direct sum
            \((M \oplus N) \otimes P \cong (M \otimes P) \oplus (N \otimes P)\).
    \end{enumerate}
\end{theorem}
\begin{proof}
    
\end{proof}
\begin{corollary}
    \(\textbf{Mod}_R\) is a symmetric monoidal category;
    that is, \((\textbf{Mod}_R,\otimes)\) is a commutative monoid.
\end{corollary}
\begin{proof}
    
\end{proof}

\begin{lemma}
    \(\Hom(M \otimes N, P) \cong \Hom(M,\Hom(N,P))\),
    which is the set of bilinear maps from \(M \times N \to P\).
\end{lemma}
\begin{proof}
    
\end{proof}
\begin{theorem}
    Suppose \(N\) is an \(R\)-module.
    Consider the functor \(\vfunc{F}{\textbf{Mod}_R}{\textbf{Mod}_R}{M}{M \otimes N}\)
    that tensors by \(N\).
    Then \(F\) is right-exact.
\end{theorem}
\begin{proof}
    
\end{proof}

\begin{definition}
    Suppose \(N\) is an \(R\)-module.
    We say \(N\) is flat if tensoring by \(N\) is exact.
\end{definition}
\begin{lemma}
    Suppose \(N\) is an \(R\)-module,
    and \(F\) is the functor that tensors by \(N\).
    The following are equivalent:
    \begin{enumerate}[label={(\alph*)}, itemsep=0mm]
        \item \(N\) is flat;
        \item \(F\) sends exact sequences to exact sequences;
        \item \(F\) sends short exact sequences to short exact sequences;
        \item \(F\) sends an injective map to an injective map; and
        \item \(F\) sends an injective map of finitely generated modules
        to an injective map of finitely generated modules.
    \end{enumerate}
\end{lemma}
\begin{proof}
    
\end{proof}

\begin{definition}
    Suppose \(N\) is an \(R\)-module.
    Consider the functors \(F: M \mapsto \Hom(N,M)\) and \(G: M \mapsto \Hom(M,N)\).
    \(N\) is a projective module if \(F\) is exact,
    and an injective module of \(G\) is exact.
\end{definition}
\begin{proposition}
    Suppose \(P\) is a projective \(R\)-module,
    and \(M \to N\) is a surjective homomorphism.
    Then \(\Hom(P,M) \to \Hom(P,N)\) is surjective,
    and the following diagram holds:
    \begin{equation*}
        \begin{tikzcd}
            & P \arrow[dashrightarrow]{dl}[swap]{\exists!} \arrow{d} \\
            M \arrow{r} & N \arrow{r} & 0
        \end{tikzcd}
    \end{equation*}
\end{proposition}
\begin{proof}
    
\end{proof}
\begin{proposition}
    Suppose \(I\) is an injective \(R\)-module,
    and \(M \to N\) is a injective homomorphism.
    Then \(\Hom(M,I) \to \Hom(N,I)\) is injective,
    and the following diagram holds:
    \begin{equation*}
        \begin{tikzcd}
            & I \\
            0 \arrow{r} & M \arrow{u} \arrow{r} & N \arrow[dashrightarrow]{ul}[swap]{\exists!}
        \end{tikzcd}
    \end{equation*}
\end{proposition}
\begin{proof}
    
\end{proof}

\begin{proposition}
    Free modules are projective,
    projective modules are flat,
    and flat modules are torsion-free.
\end{proposition}
\begin{proof}
    
\end{proof}

\begin{definition}
    Suppose \(R \to S\) is a ring homomorphism.
    Then \(S\) is an \(R\)-algebra.
\end{definition}
\begin{definition}
    Suppose \(S,T\) are \(R\)-algebras.
    Then \(R\)-algebra homomorphisms are
    \begin{equation*}
        \begin{tikzcd}
            S \arrow{r} & T \\
            R \arrow{u} \arrow{r}{\id} & R \arrow{u}
        \end{tikzcd}
    \end{equation*}
\end{definition}
\begin{proposition}
    \(R\)-algebras are \(R\)-modules.
\end{proposition}
\begin{proof}
    
\end{proof}

\begin{proposition}
    Suppose \(S,T\) are \(R\)-algebras.
    Then \(S \otimes T\) is also an \(R\)-algebra.
\end{proposition}
\begin{proof}
    
\end{proof}

\begin{definition}
    Suppose \(S\) is an \(R\)-algebra.
    Then any \(S\)-module is also an \(R\)-module.
    This is the restriction of \(M\).
\end{definition}
\begin{definition}
    Suppose \(S\) is an \(R\)-algebra. and \(M\) is an \(R\)-module.
    Then \(M \otimes S\) is an \(S\)-module.
    This is the extension of \(M\).
\end{definition}


\section{Fractions}

\begin{definition}
    Suppose \(R\) is a ring, \(S \subseteq R\) is a multiplicative monoid.
    Then the ring of fractions is \(S^{-1}R = \{r/s : r \in R, s \in S\}/{\sim}\),
    where the equivalence relation is given by \(r_1/s_1 = r_2/s_2\)
    if there exists some \(s \in S\) such that \(s(r_1s_2 - r_2s_1) = 0\).
\end{definition}
\begin{proposition}
    There exists a ring homomorphism \(\vfunc{\iota}{R}{S^{-1}R}{n}{n/1}\).
\end{proposition}
\begin{proof}
    
\end{proof}
\begin{lemma}
    Suppose \(R\) is a ring, \(S \subseteq R\) is a multiplicative monoid.
    If \(s \in S\), then \(s\) is a unit in \(S^{-1}R\).
\end{lemma}
\begin{proof}
    
\end{proof}
\begin{lemma}
    Suppose \(R\) is a ring, \(S \subseteq R\) is a multiplicative monoid.
    \(r \in R\) is zero in \(S^{-1}R\) if and only if there exists some \(s\)
    such that \(sr = 0\) in \(R\).
\end{lemma}
\begin{proof}
    
\end{proof}

\begin{theorem}[Universal Property of Ring of Fractions]
    Suppose \(R\) is a ring, \(S \subseteq R\) is a multiplicative monoid.
    The ring of fractions is \(S^{-1}R\), if:
    \begin{enumerate}
        \item there exists a homomorphism \(\func{\iota}{R}{S^{-1}R}\); and
        \item for any ring \(T\), if \(\func{\phi}{R}{T}\) is a homomorphism that sends \(s \in S\) to a unit,
            then there exists a unique \(\func{\bar{\phi}}{S^{-1}R}{T}\)
            such that \(\phi = \bar{\phi}\circ\iota\).
    \end{enumerate}
    
    This is represented by the following commutative diagram:
    \begin{equation*}
        \begin{tikzcd}
            R \arrow{r}{\phi} \arrow{d}{\iota} & T \\
            S^{-1}R \arrow[dashrightarrow]{ur}[swap]{\exists!\bar{\phi}}
        \end{tikzcd}
    \end{equation*}
\end{theorem}
\begin{proof}
    
\end{proof}

\begin{proposition}
    Suppose \(R\) is a domain, and \(S\) is the set of units in \(R\).
    Then \(S^{-1}R\) is the field of fractions.
\end{proposition}
\begin{proof}
    
\end{proof}

\begin{definition}
    Suppose \(R\) is a ring, \(p \subseteq R\) is a prime ideal,
    and \(S = R - p\).
    Then we call \(S^{-1}R = R_p\) to be the localization of \(R\) at \(p\).
\end{definition}
\begin{proposition}
    Suppose \(R\) is a ring, and \(p \subseteq R\) is a prime ideal.
    \((R_p,pR_p)\) forms a local ring.
\end{proposition}
\begin{proof}
    
\end{proof}

\begin{definition}
    Suppose \(S^{-1}R\) is a ring of fractions, and \(M\) be an \(R\)-module.
    Then \(S^{-1}M = \{m/s : m \in M, s \in S\}/{\sim}\) is an \(S^{-1}R\)-module,
    with the equivalence relation is given by \(m_1/s_1 = m_2/s_2\)
    if there exists some \(s \in S\) such that \(s(m_1s_2 - m_2s_1) = 0\).
\end{definition}
\begin{proposition}
    There exists a module homomorphism \(\vfunc{\iota}{M}{S^{-1}M}{m}{m/1}\),
    where \(\ker\iota = \{m : \exists s \in S, ms = 0\}\).
\end{proposition}
\begin{proof}
    
\end{proof}

\begin{proposition}
    Suppose \(R\) is a ring, and \(S^{-1}R\) is a ring of fractions.
    Then \(\vfunc{S^{-1}}{\textbf{Mod}_R}{\textbf{Mod}_{S^{-1}R}}{M}{S^{-1}M}\)
    is a functor.
\end{proposition}
\begin{proof}
    
\end{proof}
\begin{theorem}
    \(S^{-1}\) is an exact functor.
\end{theorem}
\begin{proof}
    
\end{proof}

\begin{proposition}
    Suppose \(P\) is an \(R\)-module,
    and \(M,N \subseteq P\) are submodules.
    The following properties hold for modules of fractions:
    \begin{enumerate}[label={(\alph*)}, itemsep=0mm]
        \item \(S^{-1}(M+N) \cong S^{-1}M + S^{-1}N \subseteq S^{-1}P\);
        \item \(S^{-1}(M \cap N) \cong S^{-1}M \cap S^{-1}N \subseteq S^{-1}P\); and
        \item \(S^{-1}(M/N) \cong S^{-1}M/S^{-1}N \subseteq S^{-1}P\).
    \end{enumerate}
\end{proposition}
\begin{proof}
    
\end{proof}

\begin{theorem}
    There is an isomorphism \(S^{-1}R \otimes M \cong S^{-1}M\).
\end{theorem}
\begin{proof}
    
\end{proof}
\begin{corollary}
    \(S^{-1}R\) is a flat \(R\)-module.
\end{corollary}
\begin{proof}
    
\end{proof}

\begin{proposition}
    Suppose \(M,N\) are (appropriate) \(R\)-modules.
    The following properties hold for modules of fractions:
    \begin{enumerate}[label={(\alph*)}, itemsep=0mm]
        \item \(S^{-1}(M \oplus N) \cong S^{-1}M \oplus S^{-1}N\);
        \item \(S^{-1}(M/N) \cong S^{-1}M/S^{-1}N\); and
        \item \(S^{-1}(M \otimes N) \cong S^{-1}M \otimes S^{-1}N\).
    \end{enumerate}
\end{proposition}
\begin{proof}
    
\end{proof}
\begin{proposition}
    \(M\) is a flat \(R\)-module if and only if \(S^{-1}M\) is a flat \(S^{-1}R\)-module.
\end{proposition}
\begin{proof}
    
\end{proof}

\begin{definition}
    Suppose \(R,A\) are rings.
    \(N\) is an \(R,A\)-bimodule if it is both an \(R\)-module and an \(A\)-module.
\end{definition}
\begin{proposition}
    Suppose \(M\) is an \(R\)-module,
    \(N\) is an \(R,A\)-bimodule,
    and \(P\) is an \(A\)-module.
    Then \(M \otimes_R N \otimes_A P\) is an \(R,A\)-bimodule.
\end{proposition}
\begin{proof}
    
\end{proof}

\begin{proposition}
    Suppose \(R\) is a ring, \(M\) is an \(R\)-module,
    and \(M_p\) is a localization.
    If \(M\) is flat, then \(M_p\) is also flat.
    If \(M\) is finitely generated, then \(M_p\) is flat if and only if it is free.
\end{proposition}
\begin{proof}
    
\end{proof}
\begin{definition}
    Suppose \(M\) is an \(R\)-module, and \(M_p\) some localization.
    We say a module property is local when
    \(M\) has that property if and only if \(M_p\) for all prime \(p \subseteq R\) has that property.
    We say a module property is max-local when
    \(M\) has that property if and only if \(M_p\) for all maximal \(p \subseteq R\) has that property.
\end{definition}
\begin{lemma}
    Suppose \(M,N\) are \(R\)-modules, and \(\func{\phi}{M}{N}\) is a homomorphism.
    Then the following properties are local and max-local:
    \begin{enumerate}[label={(\alph*)}, itemsep=0mm]
        \item \(M = 0\);
        \item \(\phi\) injective; and
        \item \(\phi\) surjective.
    \end{enumerate}
\end{lemma}
\begin{proof}
    
\end{proof}
\begin{theorem}
    Flatness is local and max-local.
\end{theorem}
\begin{proof}
    
\end{proof}

\begin{proposition}
    Consider the ring homomorphism \(\func{\phi}{R}{S^{-1}R}\).
    If \(I \subseteq R\), its extension is \(I \otimes S^{-1}R = S^{-1}I\).
    If \(J \subseteq S^{-1}R\), its contraction is \(\phi^{-1}(J)\).
\end{proposition}
\begin{proof}
    
\end{proof}
\begin{theorem}
    There is a bijective correspondence between 
    the prime ideals of \(R\) that do not intersect with \(S\),
    and the prime ideals of \(S^{-1}R\).
    In that case, \(I \subseteq R\) corresponds to its extension,
    \(J \subseteq S^{-1}R\) corresponds to its contraction,
    and \(s \in S \subseteq R\) corresponds to \((1) \subseteq S^{-1}R\).
\end{theorem}
\begin{proof}
    
\end{proof}


\section{Primary Decomposition}

\begin{definition}
    Suppose \(R\) is a ring, and \(q \subsetneq R\) is a proper ideal.
    \(q\) is primary if \(xy \in q\) implies \(x \in q\) or \(y^n \in q\) for some \(n\).
\end{definition}
\begin{proposition}
    Suppose \(R\) is a ring, and \(q \subsetneq R\) is a proper ideal.
    \(q\) is primary if and only if every zero divisor in \(R/q\) is nilpotent.
\end{proposition}
\begin{proof}
    
\end{proof}
\begin{theorem}
    Suppose \(R\) is a ring, and \(q \subsetneq R\) is a proper ideal.
    If \(q\) is primary, then \(\sqrt{q}\) is prime.
\end{theorem}
\begin{proof}
    
\end{proof}
\begin{definition}
    Suppose \(q\) is primary.
    We say \(q\) is \(p\)-primary if \(\sqrt{q} = p\).
\end{definition}

\begin{lemma}
    Suppose \({\{q_i\}}_{i \in S}\) is a family of \(p\)-primary ideals.
    Then \(\bigcap_{i \in S} q_i\) is also \(p\)-primary.
\end{lemma}
\begin{proof}
    
\end{proof}
\begin{definition}
    Suppose \(q \subseteq R\) is an ideal.
    If \(q = \bigcap_i q_i\), we say \(q\) is decomposable.
    In this case, we may assume \(\sqrt{q_i} = p_i\) are distinct,
    and hence \(q_i\) should not be inside any \(q_j\).
    We call such a decomposition the minimal or irredundant decomposition.
\end{definition}
\begin{lemma}
    Suppose \(q_i\) is \(p_i\)-primary, and \(x \in R\).
    We have the following:
    \begin{enumerate}[label={(\alph*)}, itemsep=0mm]
        \item if \(x \in q_i\), then \((q_i:x) = (1)\); and
        \item if \(x \notin q_i\), then \((q_i:x)\) is \(p_i\)-primary.
    \end{enumerate}
\end{lemma}
\begin{proof}
    
\end{proof}
\begin{lemma}
    Suppose \(p = \bigcap_i J_i\) for a family of ideals.
    Then \(p = J_i\) for some \(i\).
\end{lemma}
\begin{proof}
    
\end{proof}
\begin{theorem}[First Uniqueness Theorem for Primary Decompositions]
    Suppose \(I\) is a decomposable ideal with an irredundant decomposition \(I = \bigcap_i q_i\).
    Then the associated primes \(p_i = \sqrt{q_i}\) are unique up to reordering.
\end{theorem}
\begin{proof}
    
\end{proof}

\begin{definition}
    Suppose \(I\) is a decomposable ideal,
    and \({\{p_i\}}_{i \in S}\) is a subset of the associated prime ideals.
    This set of ideals is isolated if any associated prime \(p \subseteq p_i\)
    is also in this subset \(p \in {\{p_i\}}_{i \in S}\).
\end{definition}
\begin{theorem}[Second Uniqueness Theorem for Primary Decompositions]
    Suppose \(I\) is a decomposable ideal with an irredundant decomposition \(I = \bigcap_i q_i\).
    Let \({\{q_{i_k}\}}_{k \in S}\) be an isolated set of associated primes.
    Then \(\bigcap_{i_k} q_{i_k}\) is uniquely determined independent of the decomposition.
\end{theorem}
\begin{proof}
    
\end{proof}


\section{Integral Dependence}

\begin{definition}
    Suppose \(A \subseteq R\), where \(R\) is an \(A\)-algebra.
    \(R\) is finite over \(A\) if \(R\) is a finitely generated \(A\)-module;
    that is, there is a surjection \(\bigoplus_{i=1}^n x_i A \to R\).
    \(R\) is finite-type over \(A\) if \(R\) is a finitely generated \(A\)-algebra;
    that is, there is a surjection \(A[x_1,\hdots,x_n] \to R\).
\end{definition}
\begin{definition}
    Suppose \(R\) is an \(A\)-algebra, and \(x \in R\).
    We say \(x\) is integral over \(A\) if it satisfies a monic polynomaial
    \(x^n + a_1 x^{n-1} + \cdots + a_n = 0\) for \(a_i \in A\).
    We say \(x\) is algebraic over \(A\) if it satisfies a polynomaial
    \(a_0 x^n + a_1 x^{n-1} + \cdots + a_n = 0\) for \(a_i \in A\).
\end{definition}
\begin{definition}
    \(A \subseteq R\) is integral if every \(x \in R\) is integral over \(A\).
\end{definition}
\begin{lemma}
    Suppose \(R\) is an \(A\)-algebra.
    \(x \in R\) is integral over \(A\) if and only if \(A[x]\) is a finite \(A\)-algebra.
\end{lemma}
\begin{proof}
    
\end{proof}
\begin{corollary}
    Suppose \(R\) is an \(A\)-algebra.
    If \(x \in R\) is integral over \(A\),
    then \(A[x]\) is integral over \(A\).
\end{corollary}
\begin{proof}
    
\end{proof}
\begin{corollary}
    Suppose \(R\) is an \(A\)-algebra,
    and \({\{x_i\}}_{i=1}^n \subset R\) are all integral over \(A\).
    Then \(A[x_1,\hdots,x_n]\) is finite over \(A\).
\end{corollary}
\begin{proof}
    
\end{proof}

\begin{definition}
    Suppose \(R\) is an \(A\)-algebra.
    The integral closure of \(A\) in \(R\) is an intermediate ring \(\overline{A}\)
    that consists of all elements in \(R\) that are integral over \(A\).
\end{definition}
\begin{proposition}
    Suppose \(R\) is an \(A\)-algebra.
    \(\overline{A}\) is indeed a ring.
\end{proposition}
\begin{proof}
    
\end{proof}
\begin{lemma}
    Suppose \(A \subseteq B \subseteq R\) are all rings.
    If \(A \subseteq B\) and \(B \subseteq R\) are both integral,
    then \(A \subseteq R\) is also integral.
\end{lemma}
\begin{proof}
    
\end{proof}
\begin{corollary}
    Suppose \(A \subseteq R\) is an extension of rings.
    Then \(\overline{\overline{A}} = \overline{A}\).
\end{corollary}
\begin{proof}
    
\end{proof}

\begin{lemma}
    Suppose \(A \subseteq R\) is an integral extension of rings.
    Then we have the following two integral extensions:
    \begin{enumerate}[label={(\alph*)}, itemsep=0mm]
        \item if \(I \subseteq R\) is an ideal, and \(J = I \cap A\),
        then \(A/J \subseteq R/I\) is integral; and
        \item if \(S \subseteq A\) is a multiplicative monoid,
        then \(S^{-1}A \subseteq S^{-1}R\) is integral.
    \end{enumerate}
\end{lemma}
\begin{proof}
    
\end{proof}
\begin{theorem}
    Suppose \(A \subseteq R\) is an integral extension of rings,
    and \(A,R\) are both integral domains.
    Then \(A\) is a field if and only if \(R\) is a field.
\end{theorem}
\begin{proof}
    
\end{proof}
\begin{corollary}
    Suppose \(A \subseteq R\) is an integral extension of rings.
    If \(q \subseteq R\) is prime, and \(p = q \cap A\) is also prime,
    then \(p\) is maximal if and only if \(q\) is maximal.
\end{corollary}
\begin{proof}
    
\end{proof}
\begin{theorem}
    Suppose \(A \subseteq R\) is an integral extension of rings,
    and \(p \subseteq A\) is prime.
    Then there exists some prime \(q \subseteq R\)
    such that \(p = q \cap A\).
\end{theorem}
\begin{proof}
    
\end{proof}

\begin{theorem}[Going Up Theorem]
    Suppose \(A \subseteq R\) is an integral extension of rings,
    \(p_1,p_2 \subseteq A\) and \(q_1 \subseteq R\) are prime ideals,
    with \(p_1 = q_1 \cap A\).
    Then there exists a prime ideal \(q_2 \subseteq R\)
    such that \(q_1 \subseteq q_2\) and \(p_2 = q_2 \cap A\).
\end{theorem}
\begin{proof}
    
\end{proof}
\begin{theorem}[Going Down Theorem]
    Suppose \(A \subseteq R\) is an integral extension of rings,
    \(p_1,p_2 \subseteq A\) and \(q_2 \subseteq R\) are prime ideals,
    with \(p_2 = q_2 \cap A\).
    Then there exists a prime ideal \(q_1 \subseteq R\)
    such that \(q_1 \subseteq q_2\) and \(p_1 = q_1 \cap A\).
\end{theorem}
\begin{proof}
    
\end{proof}

\begin{theorem}[Noetherian Normalization]
    Suppose \(K\) is an infinite field, and \(R\) is a finite-type \(K\)-algebra;
    that is, \(R = K[x_1,\hdots,x_n]/I\).
    Then there exists \({\{y_i\}}_{i=1}^m \subset R\) that are algebraically independent over \(K\),
    such that \(K[y_1,\hdots,y_m] \subseteq R\) is integral.
\end{theorem}
\begin{proof}
    
\end{proof}


\section{Noetherian and Artinian Rings}

\begin{proposition}
    Suppose \(A \subseteq B \subseteq R\) are rings.
    We have the following properties:
    \begin{enumerate}[label={(\alph*)}, itemsep=0mm]
        \item if \(A \subseteq B\) and \(B \subseteq C\) are finite, then \(A \subseteq C\) is also finite;
        \item if \(A \subseteq B\) and \(B \subseteq C\) are integral, then \(A \subseteq C\) is also integral;
        \item if \(A \subseteq R\) is finite, then \(B \subseteq R\) is finite;
        \item if \(A \subseteq R\) is integral, then \(B \subseteq R\) is integral; and
        \item if \(A \subseteq R\) is integral, then \(A \subseteq B\) is integral.
    \end{enumerate}
\end{proposition}
\begin{proof}
    
\end{proof}
\begin{remark}
    In general, if \(A \subseteq R\) is finite, \(A \subseteq B\) might not necessarily be finite.
\end{remark}

\begin{definition}
    Suppose \((\Sigma,\leq)\) is a partially ordered set.
    \(\Sigma\) satisfies the ascending chain condition
    if all ascending chains in \(\Sigma\) eventually stabilizes; that is,
    \begin{equation*}
        x_1 \leq x_2 \leq \cdots \implies x_n = x_{n+1} = \cdots
    \end{equation*}
    \(\Sigma\) satisfies the descending chain condition
    if all descending chains in \(\Sigma\) eventually stabilizes; that is,
    \begin{equation*}
        x_1 \geq x_2 \geq \cdots \implies x_n = x_{n+1} = \cdots
    \end{equation*}
\end{definition}
\begin{definition}
    Suppose \(M\) is an \(R\)-module.
    Let \((\Sigma, \subseteq)\) be the partially ordered set of all submodules of \(M\).
    We say \(M\) is Noetherian if \(\Sigma\) satisfies the ascending chain condition.
    We say \(M\) is Artinian if \(\Sigma\) satisfies the descending chain condition.
\end{definition}
\begin{definition}
    Suppose \(R\) is a ring.
    Let \((\Sigma, \subseteq)\) be the partially ordered set of all ideals of \(R\).
    We say \(R\) is Noetherian if \(\Sigma\) satisfies the ascending chain condition.
    We say \(R\) is Artinian if \(\Sigma\) satisfies the descending chain condition.
\end{definition}

\begin{theorem}
    \(M\) is a Noetherian module if and only if every submodule \(N \subseteq M\) is finitely generated.
\end{theorem}
\begin{proof}
    
\end{proof}
\begin{corollary}
    \(R\) is a Noetherian ring if and only if every ideal \(I \subseteq R\) is finitely generated.
\end{corollary}
\begin{proof}
    
\end{proof}

\begin{theorem}
    Suppose \(N \subset M\) is a submodule.
    Then \(M\) is Noetherian if and only if both \(N\) and \(M/N\) are Noetherian.
\end{theorem}
\begin{proof}
    
\end{proof}
\begin{corollary}
    Suppose \({\{M_i\}}_{i=1}^n\) is a family of Noetherian modules.
    Then \(\bigoplus_{i=1}^n M_i\) is Noetherian.
\end{corollary}
\begin{proof}
    
\end{proof}
\begin{corollary}
    Suppose \(R\) is a Noetherian ring,
    and \(M\) a finitely generated \(R\)-module.
    Then \(M\) is Noetherian.
\end{corollary}
\begin{proof}
    
\end{proof}
\begin{corollary}
    Suppose \(R\) is a Noetherian ring,
    and \(I \subseteq R\) an ideal.
    Then \(R/I\) is Noetherian.
\end{corollary}
\begin{proof}
    
\end{proof}

\begin{theorem}
    Suppose \(A\) is a Noetherian ring, and \(R\) is a finite \(A\)-algebra.
    Then \(R\) is a Noetherian ring.
\end{theorem}
\begin{proof}
    
\end{proof}
\begin{theorem}
    Suppose \(R\) is a Noetherian ring.
    Then any ring of fractions \(S^{-1}R\) is also Noetherian.
\end{theorem}
\begin{proof}
    
\end{proof}

\begin{lemma}
    Suppose \(R\) is a ring, and \(S^{-1}R\) is a ring of fractions.
    Then every ideal of \(S^{-1}R\) has the form \(S^{-1}I\),
    where \(I \subseteq R\) is an ideal.
\end{lemma}
\begin{proof}
    
\end{proof}
\begin{theorem}[Hilbert Basis Theorem]
    Suppose \(R\) is a Noetherian ring.
    Then \(R[x]\) is also a Noetherian ring.
\end{theorem}
\begin{proof}
    
\end{proof}
\begin{corollary}
    Suppose \(R\) is a Noetherian ring.
    Then \(R[x_1,\hdots,x_n]\) is Noetherian,
    and any finite-type \(R\)-algebra \(R[x_1,\hdots,x_n]/I\) is also Noetherian.
\end{corollary}
\begin{proof}
    
\end{proof}
\begin{remark}
    We essentially have the following results:
    \begin{enumerate}[label={(\roman*)}, itemsep=0mm]
        \item \(R\) is finite over \(A\) if and only if it is integral and finite-type; and
        \item if \(R\) is finite-type over \(A\), then \(R\) is Noetherian over \(A\).
    \end{enumerate}
\end{remark}

\begin{proposition}
    Suppose \(A\) is a Noetherian ring,
    and \(A \subseteq B \subseteq R\) is an extension of rings.
    If \(A \subseteq R\) is finite, then \(A \subseteq B\) is finite.
\end{proposition}
\begin{proof}
    
\end{proof}
\begin{proposition}
    Suppose \(A \subseteq B \subseteq R\) is an extension of rings.
    If \(R\) is Noetherian, then \(B\) is Noetherian.
\end{proposition}
\begin{proof}
    
\end{proof}
\begin{theorem}
    Suppose \(A \subseteq B \subseteq R\) is an extension of rings.
    If \(A\) is Noetherian, \(C\) is finite-type of \(A\),
    and \(C\) is finite over \(B\),
    then \(B\) is finite-type over \(A\).
\end{theorem}
\begin{proof}
    
\end{proof}
\begin{proposition}
    Suppose \(K\) is a field, and \(E\) a finitely generated \(K\)-algebra.
    If \(E\) is a field, then \(E/K\) is a finite algebraic extension.
\end{proposition}
\begin{proof}
    
\end{proof}
\begin{proposition}
    Suppose \(K(y_1,\hdots,y_n)/K\) is a purely transcendental extension.
    Then \(E\) cannot be finite-type over \(K\).
\end{proposition}
\begin{proof}
    
\end{proof}
\begin{corollary}[Weak Hilbert's Nullstellensatz]
    Suppose \(K\) is a field, and \(A\) is a finite-type \(K\)-algebra.
    Let \(m \subseteq A\) be a maximal ideal.
    Then \(A/m\) is a finite algebraic extension of \(K\).
    If \(K = \overline{K}\), then \(A/m = K\).
\end{corollary}
\begin{proof}
    
\end{proof}

\begin{definition}
    Suppose \(K\) is a field, and \(R = K[x_1,\hdots,x_n]\).
    Consider \({\{f_i\}}_{i \in S} \subset R\) a family of polynomial functions,
    \(\func{f_i}{K^n}{K}\).
    The vanishing locus is the set of points
    \(\Van(\{f_i\}) = \{p \in K^n: \forall i \in S,\,f_i(p) = 0\}\).
    Now consider \(X \subseteq K^n\).
    The ideal of all vanishing polynomials is
    \(\Ide(X) = \{f \in R: \forall x \in X,\,f(x) = 0\}\).
\end{definition}
\begin{corollary}
    Suppose \(K = \overline{K}\) is algebraically closed.
    Every maximal ideal of \(K[x_1,\hdots,x_n]\) is of the form
    \((x_1-a_1,\hdots,x_n-a_n)\) for some \(a_i \in K\).
    If \(I \subsetneq K[x_1,\hdots,x_n]\) is a proper ideal,
    then the vanishing locus \(\Van(I) \neq \emptyset\).
\end{corollary}
\begin{proof}
    
\end{proof}
\begin{theorem}[Hilbert's Nullstellensatz]\label{thm:nullstellensatz}
    Suppose \(K = \overline{K}\) is an algebraically closed field,
    and \(R = K[x_1,\hdots,x_n]\) is a polynomial ring.
    Let \(J \subseteq R\) be an ideal, and \(\Van(J)\) be its vanishing locus.
    Then \(\Ide(\Van(J)) = \sqrt{J}\).
\end{theorem}
\begin{proof}
    
\end{proof}

\begin{definition}
    Suppose \(R\) is a Noetherian ring, and \(I \subseteq R\) is an ideal.
    \(I\) is irreducible if \(I = I_1 \cap I_2\) implies \(I = I_1\) or \(I = I_2\).
\end{definition}
\begin{lemma}
    Suppose \(R\) is a Noetherian ring.
    Then every ideal \(I \subseteq R\) is a finite intersection of irreducible ideals
    \(I = \bigcap_{j=1}^n I_j\).
\end{lemma}
\begin{proof}
    
\end{proof}
\begin{lemma}
    Suppose \(R\) is a Noetherian ring.
    If \(I \subseteq R\) is an irreducible ideal,
    then it is also primary.
\end{lemma}
\begin{proof}
    
\end{proof}
\begin{proposition}
    Suppose \(R\) is a Noetherian ring, and \(I \subseteq R\) is an ideal.
    Then \({(\sqrt{I})^n} \subseteq I\) for some \(n\).
\end{proposition}
\begin{proof}
    
\end{proof}
\begin{corollary}
    Suppose \(R\) is a Noetherian ring, and \(\mca{N}(R)\) its nilradical.
    Then \({(\mca{N}(R))}^n = 0\) for some \(n\).
\end{corollary}
\begin{proof}
    
\end{proof}
\begin{corollary}
    Suppose \(R\) is a Noetherian ring, and \(m \subseteq R\) is a maximal ideal.
    Let \(q\) be a primary ideal.
    Then the following are equivalent:
    \begin{enumerate}[label={(\alph*)}, itemsep=0mm]
        \item \(q \subseteq R\) is \(m\)-primary;
        \item \(\sqrt{q} = m\); and
        \item \(m^n \subseteq q \subseteq m\) for some \(n\).
    \end{enumerate}
\end{corollary}
\begin{proof}
    
\end{proof}

\begin{theorem}
    Suppose \(R\) is a Noetherian ring.
    Then any ideal has a primary decomposition.
\end{theorem}
\begin{proof}
    
\end{proof}
