\chapter{Commutative Algebra}

\begin{remark}
    We shall assume that all rings,
    unless otherwise noted, are commutative;
    that is, the multiplication \(ab = ba\).
\end{remark}

\section{Ideals}

\begin{proposition}
    Suppose \(R\) is a ring.
    \(x \in R\) is a unit if and only if the ideals \((x) = (1)\).
\end{proposition}
\begin{proof}
    
\end{proof}

\begin{theorem}
    Suppose \(R\) is a nondegenerate ring.
    The following are equivalent:
    \begin{enumerate}[label={(\alph*)}, itemsep=0mm]
        \item \(R\) is a field;
        \item \(R\) has only two distinct ideals \((0)\) and \((1)\); and
        \item Every ring homomorphism \(\func{\phi}{R}{S}\) is injective.
    \end{enumerate}
\end{theorem}
\begin{proof}
    
\end{proof}

\begin{definition}
    Suppose \(R\) is a ring.
    An element \(x \in R\) is nilpotent if there exists \(n > 0\) such that \(x^n = 0\).
    Any nonzero nilpotent element is a zero divisor.
    A ring \(R\) is reduced if the only nilpotent is 0.
\end{definition}

\begin{definition}
    Suppose \(R\) is a ring, \(I \subseteq R\) an ideal.
    \(I\) is maximal if \(I \neq R\),
    and there does not exist any intermediate ideals \(J\)
    such that \(I \subseteq J \subseteq R\).
\end{definition}
\begin{theorem}\label{thm:maximal-quotient-field}
    Suppose \(R\) is a ring, \(I \subseteq R\) an ideal.
    \(I \subseteq R\) is maximal if and only if \(R/I\) is a field.
\end{theorem}
\begin{proof}
    Corollary~\ref{cor:maximal-quotient-field}.
\end{proof}

\begin{definition}
    Suppose \(R\) is a ring, \(I \subseteq R\) an ideal.
    \(I\) is prime if \(xy \in I\) implies \(x \in I\) or \(y \in I\);
    that is, the complement of \(I\) is closed under multiplication.
\end{definition}
\begin{theorem}\label{thm:prime-quotient-domain}
    Suppose \(R\) is a ring, \(I \subseteq R\) an ideal.
    \(I\) is prime if and only if \(R/I\) is a domain.
\end{theorem}
\begin{proof}
    Proposition~\ref{prop:prime-quotient-domain}.
\end{proof}

\begin{definition}
    Suppose \(R\) is a ring, \(I \subseteq R\) an ideal.
    \(I\) is radical if for all \(x \in R\) and \(x^n \in I\), then \(x \in I\).
\end{definition}
\begin{theorem}\label{thm:radical-quotient-reduced}
    Suppose \(R\) is a ring, \(I \subseteq R\) an ideal.
    \(I\) is radical if and only if \(R/I\) is reduced.
\end{theorem}
\begin{proof}
    Suppose \(I\) is not radical;
    then there exists some \(x \notin I\) such that \(x^n \in I\).
    Hence there exists some nonzero \([x] \in R/I\) such that \([x]^n = 0\),
    and \(R/I\) is not reduced.
    The reverse argument holds the same way since all those logical steps are bidirectional.
\end{proof}

\begin{theorem}
    Maximal ideals are prime, and prime ideals are radical.
\end{theorem}
\begin{proof}
    
\end{proof}

\begin{theorem}
    Every ring has a maximal ideal.
\end{theorem}
\begin{proof}
    Let \(\Sigma\) be the set of all proper ideals of some ring \(R\).
    Let this set be partially ordered by \(\subseteq\), the subset relation.
    We first see that \(\Sigma\neq\emptyset\) due to \((0) \in \Sigma\).
    We then claim that every chain \(I_1 \subseteq I_2 \subseteq I_3 \subseteq \cdots\)
    is contained within the union \(I = \bigcup_i I_i \in \Sigma\).
    By Proposition~\ref{prop:nested-ideals}, \(I\) is an ideal,
    and by definition, \(I_i \subseteq I\).
    We also see that \(I \neq R\) because if \(1 \in I\),
    then \(1 \in I_i\) for some \(i\), which gives us a contradiction.
    Hence by \hyperref[ax:zorn]{Zorn's lemma}, there is a maximal element, a maximal ideal.
\end{proof}
\begin{corollary}
    Every proper ideal lies in a maximal ideal.
\end{corollary}
\begin{proof}
    
\end{proof}
\begin{corollary}
    Suppose \(R\) is a ring, and \(x \in R\) is not a unit.
    Then \(x\) lies in a maximal ideal.
\end{corollary}
\begin{proof}
    
\end{proof}

\begin{definition}
    Suppose \(R\) is a ring.
    \(R\) is a local ring if it has a unique maximal ideal \(m \subseteq R\).
\end{definition}
\begin{theorem}
    Suppose \((R,m)\) forms a local ring.
    Then \(R-m\) is the group of units in \(R\).
\end{theorem}
\begin{proof}
    
\end{proof}
\begin{theorem}
    Suppose \(R\) is a ring, and \(I \subseteq R\) is a proper ideal.
    If every \(x \in R-I\) is a unit, then \((R,I)\) is a local ring.
\end{theorem}
\begin{proof}
    
\end{proof}
\begin{theorem}
    Suppose \(R\) is a ring, and \(I \subseteq R\) is proper ideal.
    If for every \(x \in I\), \(1+x\) is a unit, then \((R,I)\) is a local ring.
\end{theorem}
\begin{proof}
    
\end{proof}

\begin{definition}
    Suppose \(R\) is a ring.
    The set of nilpotent elements is called the nilradical of \(R\).
\end{definition}
\begin{theorem}
    Suppose \(R\) is a ring, and \(N\) its nilradical.
    Then \(N\) is an ideal, and \(R/N\) is reduced.
\end{theorem}
\begin{proof}
    
\end{proof}
\begin{theorem}
    The nilradical is the intersection of all prime ideals.
\end{theorem}
\begin{proof}
    
\end{proof}
\begin{proposition}
    Suppose \(I,J,K\) are three ideals.
    We have the two following distributive properties:
    \begin{enumerate}[label={(\alph*)}, itemsep=0mm]
        \item \(I(J+K) = IJ + IK\); and
        \item \(I \cap (J+K) \supseteq I \cap J + I \cap K\),
            with equality only if \(J \subseteq I\) or \(K \subseteq I\).
    \end{enumerate}
\end{proposition}
\begin{proof}
    
\end{proof}

\begin{definition}
    Suppose \(I,J\) are two ideals.
    They are coprime if \(I+J = (1)\).
\end{definition}
\begin{theorem}
    Suppose \({\{J_i\}}_{i=1}^n\) are pairwise coprime ideals.
    Then \(\bigcap_{i=1}^n J_i = \prod_{i=1}^n J_i\).
\end{theorem}
\begin{proof}
    
\end{proof}

\begin{definition}
    Suppose \(\{R_i\}\) is a family of rings.
    The direct product is \(\prod_i R_i = \{(r_1,r_2,\hdots) : r_i \in R_i\}\).
\end{definition}
\begin{lemma}
    Suppose \(R\) is a ring, and \(\{J_i\}\) is a family of ideals.
    Let us define the homomorphism \(\vfunc{\phi}{R}{\prod_i R/J_i}{r}{(r+J_1,r+J_2,\hdots)}\).
    Then we have \(\ker(\phi) = \bigcap_i J_i\),
    and \(\img(\phi) = R/\prod_i J_i\).
\end{lemma}
\begin{proof}
    
\end{proof}
\begin{theorem}[Chinese Remainder Theorem]
    Suppose \(R\) is a ring, and \(\{J_i\}\) is a family of ideals.
    Consider the homomorphism \(\func{\phi}{R}{\prod_i R/J_i}\).
    Then \(\phi\) is surjective if and only if \(\{J_i\}\) are pairwise coprime.
    In that case, \(\prod_i R/J_i \cong R/\prod_i J_i\).
\end{theorem}
\begin{proof}
    
\end{proof}

\begin{definition}
    Suppose \(R\) is a ring, and \(I,J\) are ideals.
    The quotient ideal is \((I:J) = \{x \in R : xJ \subseteq I\}\).
\end{definition}
\begin{proposition}
    The following properties hold for quotient ideals:
    \begin{enumerate}[label={(\alph*)}, itemsep=0mm]
        \item \(I \subseteq (I:J)\);
        \item \((I:J)J \subseteq I\);
        \item \((I:I) = (1)\); and
        \item \(((I:J):K) = (I:JK) = ((I:K):J)\).
    \end{enumerate}
\end{proposition}
\begin{proof}
    
\end{proof}

\begin{definition}
    Suppose \(\func{\phi}{R}{S}\) is any ring homomorphism.
    Let \(I \subseteq R\) and \(J \subseteq S\) be ideals.
    \(\phi(I) \subseteq S\) might not be an ideal,
    so we define the ideal it generates \((\phi(I)) \subseteq \) the extension of \(I\).
\end{definition}
\begin{remark}
    \(\phi^{-1}(J) \subseteq R\) is always an ideal,
    and if \(J\) prime, \(\phi^{-1}(J)\) is also prime.
\end{remark}
\begin{definition}
    Suppose \(\func{\phi}{R}{S}\) is a ring monomorphism.
    If \(J \subseteq S\), we say \(\phi^{-1}(J) = J \cap R\) is the contraction of \(J\).
\end{definition}

\section{Modules}

\begin{proposition}
    Suppose \(R\) is a ring, and \(M,N\) are \(R\)-modules.
    Then \(\Hom_R(M,N)\) is also an \(R\)-module.
\end{proposition}
\begin{proof}
    
\end{proof}

\begin{proposition}
    Suppose \(R\) is a ring.
    The category \(\textbf{Mod}_R\) consisting of all \(R\)-modules
    and module homomorphisms form a category.
\end{proposition}
\begin{proof}
    
\end{proof}

\begin{proposition}
    Suppose \(M\) is an \(R\)-module.
    There exists a functor \(\func{F}{\textbf{Mod}_R}{\textbf{Mod}_R}\)
    that maps \(N \mapsto \Hom_R(M,N)\)
    and maps \(\func{\phi}{N}{L}\) to \(\func{\tilde{\phi}}{\Hom_R(M,N)}{\Hom_R(M,L)}\).
\end{proposition}
\begin{proof}
    
\end{proof}
\begin{corollary}
    \(\Hom_R(M,N) \cong N\).
\end{corollary}
\begin{proof}
    
\end{proof}

\begin{definition}
    Suppose \(L\) is an \(R\)-module,
    and \(M,N \subseteq L\) are submodules.
    The quotient ideal \((M:N) = \{x \in R : xN \subseteq M\}\).
\end{definition}
\begin{definition}
    Suppose \(N\) is an \(R\)-module.
    The annihilator of \(N\) is \(\Ann_R(N) = (0:N) = \{x \in R : xN = 0\}\).
\end{definition}

\begin{definition}
    Suppose \({\{M_i\}}_{i \in S}\) is a family of \(R\)-modules.
    The direct product is \(\prod_i M_i = \{{(m_i)}_{i \in S}\}\),
    and the direct sum is \(\oplus_i M_i = \{{(m_i)_{i \in S}}\}\) with finite support.
\end{definition}

\begin{definition}
    Suppose \(M\) is an \(R\)-module, and \({\{m_i\}}_{i \in S} \subseteq M\).
    The module it generates is \(N = \{\sum_{j=1}^n x_j m_j : x_j \in R, m_j \in {\{m_i\}}_{i \in S}\}\).
    If \(M = N\), we say \({\{m_i\}}_{i \in S}\) generates \(M\);
    in that case, if \(S\) is finite, we say \(M\) is finitely generated.
\end{definition}

\section{Exact Sequences}

\section{Tensor Products}

\section{Fractions}

\section{Primary Decomposition}

\section{Integral Dependence}

\section{Noetherian \& Artinian Rings}
