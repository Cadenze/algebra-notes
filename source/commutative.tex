\chapter{Commutative Algebra}

\begin{remark}
    We shall assume that all rings,
    unless otherwise noted, are commutative;
    that is, the multiplication \(ab = ba\).
\end{remark}

\section{Ideals}

\begin{proposition}
    Suppose \(R\) is a ring.
    \(x \in R\) is a unit if and only if the ideals \((x) = (1)\).
\end{proposition}
\begin{proof}
    
\end{proof}

\begin{theorem}
    Suppose \(R\) is a nondegenerate ring.
    The following are equivalent:
    \begin{enumerate}[label={(\alph*)}, itemsep=0mm]
        \item \(R\) is a field;
        \item \(R\) has only two distinct ideals \((0)\) and \((1)\); and
        \item Every ring homomorphism \(\func{\phi}{R}{S}\) is injective.
    \end{enumerate}
\end{theorem}
\begin{proof}
    
\end{proof}

\begin{definition}
    Suppose \(R\) is a ring.
    An element \(x \in R\) is nilpotent if there exists \(n > 0\) such that \(x^n = 0\).
    Any nonzero nilpotent element is a zero divisor.
    A ring \(R\) is reduced if the only nilpotent is 0.
\end{definition}

\begin{definition}
    Suppose \(R\) is a ring, \(I \subseteq R\) an ideal.
    \(I\) is maximal if \(I \neq R\),
    and there does not exist any intermediate ideals \(J\)
    such that \(I \subseteq J \subseteq R\).
\end{definition}
\begin{theorem}
    Suppose \(R\) is a ring, \(I \subseteq R\) an ideal.
    \(I \subseteq R\) is maximal if and only if \(R/I\) is a field.
\end{theorem}
\begin{proof}
    Corollary~\ref{cor:maximal-quotient-field}.
\end{proof}

\begin{definition}
    Suppose \(R\) is a ring, \(I \subseteq R\) an ideal.
    \(I\) is prime if \(xy \in I\) implies \(x \in I\) or \(y \in I\);
    that is, the complement of \(I\) is closed under multiplication.
\end{definition}
\begin{theorem}
    Suppose \(R\) is a ring, \(I \subseteq R\) an ideal.
    \(I\) is prime if and only if \(R/I\) is a domain.
\end{theorem}
\begin{proof}
    Proposition~\ref{prop:prime-quotient-domain}.
\end{proof}

\begin{definition}
    Suppose \(R\) is a ring, \(I \subseteq R\) an ideal.
    \(I\) is radical if for all \(x \in R\) and \(x^n \in I\), then \(x \in I\).
\end{definition}
\begin{theorem}
    Suppose \(R\) is a ring, \(I \subseteq R\) an ideal.
    \(I\) is radical if and only if \(R/I\) is reduced.
\end{theorem}
\begin{proof}
    Suppose \(I\) is not radical;
    then there exists some \(x \notin I\) such that \(x^n \in I\).
    Hence there exists some nonzero \([x] \in R/I\) such that \([x]^n = 0\),
    and \(R/I\) is not reduced.
    The reverse argument holds the same way since all those logical steps are bidirectional.
\end{proof}

\section{Modules}

\section{Exact Sequences}

\section{Tensor Products}

\section{Fractions}

\section{Primary Decomposition}

\section{Integral Dependence}

\section{Noetherian \& Artinian Rings}
