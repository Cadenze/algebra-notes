\section{Galois Theory}\label{sec:galois}

\subsection{Galois Extensions}

\begin{definition}
    A field extension \(K/F\) is a Galois extension
    if it is algebraic, separable and normal.
\end{definition}
\begin{definition}
    The group of automorphisms that fix \(F\)
    is the the Galois group \(\Gal(K/F) = \Aut_F(K)\).
\end{definition}

\begin{lemma}
    Suppose \(K = F(S)\) is an extension generated by a set.
    If \(\sigma,\tau \in \Gal(K/F)\) and \(\sigma\vert_S = \tau\vert_S\),
    then \(\sigma  = \tau\).
\end{lemma}
\begin{proof}
    Write element of \(K\) as linear combination,
    apply linearity of automorphism.
\end{proof}
\begin{lemma}\label{lem:gal-permute-roots}
    Let \(\tau \in \Hom_F(K,L)\), and \(\alpha \in K\) an algebraic element.
    If \(f(x) \in F[x]\) and \(f(\alpha) = 0\),
    then \(f(\tau(\alpha)) = 0\).
    Therefore \(\tau\) permutes the roots of \(\min_F(\alpha)\),
    and \(\min_F(\alpha) = \min_F(\tau(\alpha))\).
\end{lemma}
\begin{proof}
    Simply by definition of an \(F\)-homomorphism,
    \(0 = f(\alpha) = \tau(f(\alpha)) = f(\tau(\alpha))\).
\end{proof}
\begin{corollary}
    \([K:F]\) finite implies \(\abs{\Gal(K/F)}\) finite.
\end{corollary}
\begin{proof}
    Suppose \(K = F(\alpha_1,\hdots,\alpha_n)\).
    Then \(\sigma \in \Gal(K/F)\) determined completely by \(\sigma(\alpha_i)\),
    and each one of these have finitely many possible images
    by Lemma~\ref{lem:gal-permute-roots}.
\end{proof}

\begin{definition}
    Suppose \(K\) a field, and \(G\) some group.
    A linear \(K\)-character on \(G\) is a homomorphism \(\func{\chi}{G}{L^\ast}\)
    where \(\chi(gh) = \chi(g)\chi(h)\).
\end{definition}
\begin{theorem}[Linear Independence of Characters]\label{thm:linear-independent-characters}
    Suppose \(K\) a field, and \(G\) some group.
    Let \({\{\chi_i\}}_{i=1}^n\) be distinct characters.
    Then \({\{\chi_i\}}_{i=1}^n\) is linearly independent.
\end{theorem}
\begin{proof}
    Suppose, by way of contradiction,
    that they are linearly dependent.
    Then there exists some minimal number \(m\) of nonzero coefficients \({\{a_i\}}_{i=1}^m\)
    such that
    \begin{equation*}
        a_1\chi_1 + a_2\chi_2 + \cdots + a_m\chi_m = 0
    \end{equation*}
    since without loss of generality we may assume by reordering the characters,
    the first \(m\) characters form our minimal dependent set.
    Then for any \(g \in G\), we have
    \begin{equation*}
        a_1\chi_1(g) + a_2\chi_2(g) + \cdots + a_m\chi_m(g) = 0
    \end{equation*}
    Let \(g_0 \in G\) be an element such that \(\chi_1(g_0) \neq \chi_m(g_0)\).
    Then we have
    \begin{align*}
        a_1\chi_1(g_0g) + a_2\chi_2(g_0g) + \cdots + a_m\chi_m(g_0g) &= 0 \\
        a_1\chi_1(g_0)\chi_1(g) + a_2\chi_2(g_0)\chi_2(g) + \cdots + a_m\chi_m(g_0)\chi_m(g) &= 0
    \end{align*}
    But we can also multiply by \(\chi_m(g_0)\) to get
    \begin{equation*}
        a_1\chi_m(g_0)\chi_1(g) + a_2\chi_m(g_0)\chi_2(g) + \cdots + a_m\chi_m(g_0)\chi_m(g) = 0
    \end{equation*}
    Subtracting the above two equations, we have
    \begin{equation*}
        (\chi_m(g_0)-\chi_1(g_0))a_1\chi_1(g) + (\chi_m(g_0)-\chi_2(g_0))a_2\chi_2(g)
        + \cdots + (\chi_m(g_0)-\chi_{m-1}(g_0))a_m\chi_m(g) = 0
    \end{equation*}
    We have now found a set of \(m-1\) characters that is linearly dependent,
    which contradicts minimality as assumed.
\end{proof}
\begin{corollary}[Dedekind's Lemma]\label{cor:dedekind}\label{lem:dedekind}
    Suppose \(K\) is a field, and \({\{\sigma_i\}}_{i=1}^n \subset \Aut K\).
    Then there does not exist \({\{a_i\}}_{i=1}^n \subset K\) not all zero
    such that \(a_1\sigma_1(u) + \cdots a_n\sigma_n(u) = 0\) for all \(u \in K\).
\end{corollary}
\begin{proof}
    Direct application of the \hyperref[thm:linear-independent-characters]{theorem above}.
\end{proof}

\begin{theorem}
    \([K:F]\) finite implies \(\abs{\Gal(K/F)} \leq [K:F]\).
\end{theorem}
\begin{proof}
    Let \(n = [K:F]\), and \({\{u_i\}}_{i=1}^n\) be an \(F\)-basis of \(K\).
    Suppose \({\{\sigma_i\}}_{i=1}^{n+1} \subset \Gal(K/F)\) are distinct.
    Then consider a system of \(n\) homogeneous linear equations in \(n+1\) unknowns.
    \begin{align*}
        \sigma_1(u_1)x_1 + \sigma_2(u_1)x_2  + \cdots + \sigma_{n+1}(u_1)x_{n+1} &= 0 \\
        \sigma_1(u_2)x_1 + \sigma_2(u_2)x_2  + \cdots + \sigma_{n+1}(u_2)x_{n+1} &= 0 \\
        &\;\;\vdots \\
        \sigma_1(u_n)x_1 + \sigma_2(u_n)x_2  + \cdots + \sigma_{n+1}(u_n)x_{n+1} &= 0
    \end{align*}
    This system has a nontrivial solution.
    But that exactly contradicts \hyperref[lem:dedekind]{Dedekind's lemma}.
\end{proof}

\begin{definition}
    Suppose \(K/F\) some extension, and \(G \subseteq \Aut_F(K)\) some subset.
    The fixed field of \(T\) is
    \(K^T =\{a \in K: \sigma(u) = u,\; \forall \sigma \in T\}\).
\end{definition}
\begin{proposition}
    Suppose \(F\) is a field of characteristic 0,
    \(K/F\) some finite extension, \(G = \Aut_F(K)\).
    Then \(K/F\) is normal if and only if \(K^G = F\).
\end{proposition}
\begin{proof}
    % TODO: galois extension
\end{proof}
\begin{proposition}
    Suppose \(L/K\) is a finite extension,
    \(\func{\phi}{K}{K'}\) a ring homomorphism,
    and \(L'/K'\) is normal.
    Then there is at least 1, and at most \([L:K]\) ring homomorphisms
    \(\func{\psi}{L}{L'}\) that extend \(\phi\),
    with equality if and only if \(L/K\) separable.
\end{proposition}
\begin{proof}
    % TODO
\end{proof}
\begin{proposition}
    Let \(E/F\) be a finite extension.
    Then \(\abs{\Aut_F(E)} \leq {[E:F]}_s\),
    with equality if and only if \(E/F\) normal.
\end{proposition}
\begin{proof}
    % TODO
\end{proof}

\begin{theorem}
    Suppose \(F\) is a field of characteristic 0,
    and \(K\) a (finite) normal extension of \(F\),
    \(H \subseteq \Gal(K/F)\) some subgroup.
    Let \(K^H\) be the fixed field of \(H\).
    Then \([K:K^H] = \abs{H}\), \(H = \Gal(K/K^H)\),
    and in particular, if \(H = \Gal(K/F)\) then \([K:F] = \Gal(K/F)\).
\end{theorem}
\begin{proof}
    % TODO
\end{proof}

\begin{theorem}
    Suppose \(K/F\) is a Galois extension, \(G = \Gal(K/F)\) is a Galois group.
    Then \(F = K^G\).
    If \(F \subseteq E \subseteq K\), then \(K/E\) is Galois.
    The map \(E \mapsto \Gal(K/E)\) from the set of intermediate field extensions
    to the set of subgroups of \(G\) is injective.
\end{theorem}
\begin{proof}
    % TODO
\end{proof}

\begin{lemma}
    Suppose \(K/F\) is Galois. Then the following hold:
    \begin{enumerate}[label={(\alph*)}, itemsep=0mm]
        \item If \(L_1 \subseteq L_2\) are subfields of \(K\),
            then \(\Gal(K/L_2) \subseteq \Gal(K/L_1)\).
        \item If \(L\) is a subfield of \(K\),
            then \(L \subseteq K^G\) where \(G = \Gal(K/L)\).
        \item If \(S_1 \subseteq S_2\) are subsets of \(Aut K\),
            then \(K^{S_2} \subseteq K^{S_1}\).
        \item If \(S\) is a subset of \(\Aut K\),
            then \(S \subseteq \Gal(K/K^S)\).
        \item If \(L = K^S\) for some \(S \subseteq \Aut K\),
            then \(L = K^G\) where \(G = \Gal(K/L)\).
        \item If \(H = \Gal(K/L)\) for some subfield \(L\),
            then \(H = \Gal(K/K^H)\).
    \end{enumerate}
\end{lemma}
\begin{proof}
    % TODO
\end{proof}
\begin{corollary}
    Suppose \(K/F\) is Galois, and \(G = \Gal(K/F)\).
    Then every subgroup \(H\) corresponds to some subfield \(L\)
    such that \(F \subseteq L \subseteq K\).
\end{corollary}
\begin{proof}
    % TODO
\end{proof}

\begin{theorem}
    Suppose \(K/F\) Galois and \(G = \Gal(K/F)\).
    Suppose \(L\) is a subfield with \(F \subseteq L \subseteq K\) and \(H = \Gal(K/L)\).
    Then \(L\) is normal over \(F\) if and only if \(H \lhd G\).
    If \(L\) is indeed normal over \(F\),
    then restricting \(\Gal(K/F) \to \Gal(L/F)\) via \(\sigma \mapsto \sigma\vert_L\)
    is a surjective homomorphism with kernel \(H\).
    Hence \(\Gal(L/F) \cong G/H\).
\end{theorem}
\begin{proof}
    % TODO
\end{proof}

\begin{definition}
    A Galois extension \(K/F\) is abelian/cyclic
    if \(\Gal(K/F)\) is abelian/cyclic.
\end{definition}
\begin{proposition}
    If \(K/F\) is abelian/cyclic, and \(L\) is an intermediate field,
    then \(L/F\) is abelian/cylic.
\end{proposition}
\begin{proof}
    Subgroups and quotient groups of abelian/cyclic groups are abelian/cyclic.
\end{proof}

\begin{theorem}
    Suppose \(K/F\) Galois, and \(L/F\) some other extension,
    under the assumption that \(K,L\) are both contained in some extension.
    Then \(KL/L\) and \(K/(K \cap L)\) are Galois.
\end{theorem}
\begin{proof}
    % TODO
\end{proof}
\begin{corollary}
    Suppose \(K/F\) finite Galois, and \(L/F\) some extension.
    Then \([KL:L] \mid [K:F]\).
\end{corollary}
\begin{proof}
    % TODO
\end{proof}
\begin{remark}
    This does not hold if \(K/F\) not Galois.
\end{remark}

\begin{theorem}[Artin's Theorem]
    Suppose \(K\) is a field, and \(G \subseteq \Aut K\) some group of automorphisms,
    with \(\abs{G} = n\).
    Let \(F = K^G\).
    Then \(K/F\) is a finite Galois extension, with \(\Gal(K/F) = G\).
\end{theorem}
\begin{proof}
    % TODO
\end{proof}

\begin{theorem}[Fundamental Theorem of Galois Theory]\label{thm:ftgt}
    Suppose \(K/F\) is (finite) Galois, and \(G = \Gal(K/F)\).
    \begin{enumerate}[label={(\alph*)}, itemsep=0mm]
        \item There is an inclusion reversing bijection
            between intermediate fields \(E\) of \(K/F\) and subgroups \(H \subseteq G\),
            given by associating \(H\) with \(K^H = E\).
        \item \([K:E] = \abs{H}\) and \([K:F] = \abs{G}\).
        \item \(K/E\) is Galois, \(H = \Gal(K/E)\).
        \item Embeddings \(E\) to \(\overline{F}\) are in bijection with
            left cosets of \(H\) in \(G\).
        \item \(E/F\) Galois if and only if \(H \lhd G\),
            and in that case, \(\Gal(E/F) \cong G/H\).
        \item Intersection of subgroups correspond to compositum of fields
            and joins of subgroups correspond to intersection of fields.
        \item Lattice of subgroups of \(G\) correspond under this bijection
            to inverted lattice of intermediate field extensions.
    \end{enumerate}
\end{theorem}
\begin{proof}
    % TODO
\end{proof}

\begin{theorem}
    Suppose \(K/F\) a field extension.
    Then the following are equivalent:
    \begin{enumerate}[label={(\alph*)}, itemsep=0mm]
        \item \(K/F\) finite Galois, i.e.\ \(\abs{\Aut_F(K)} = [K:F] < \infty\).
        \item \(K/F\) is the splitting field of some separable \(f(x) \in F[x]\).
        \item \(F\) is the fixed field of \(\Aut(K/F)\).
        \item \(K/F\) is normal, finite, and separable.
    \end{enumerate}
\end{theorem}
\begin{proof}
    % TODO
\end{proof}

\begin{definition}
    Suppose \(K/F\) Galois, \(G = \Gal(K/F)\),
    \(\sigma \in G\), \(\alpha \in K\).
    \(\{\sigma(\alpha) : \sigma \in G\}\) is the set of Galois conjugates of \(\alpha\).
    If \(F \subseteq E \subseteq K\) tower of extensions,
    then \(\sigma(E)\) is the conjugate of \(E\).
\end{definition}
\begin{proposition}
    Suppose \(F \subseteq E \subseteq K\) is a tower of field extensions,
    \(K/F\) finite Galois, and \(G = \Gal(K/F)\).
    Suppose that \(H \subseteq G\) corresponds to \(E\).
    For \(\sigma \in G\),
    the subgroup corresponding to \(\sigma(E)\)
    is the conjugate subgroup \(\sigma H \sigma^{-1} \subseteq G\).
    In particular \([E:F] = [G:H]\).
\end{proposition}
\begin{proof}
    % TODO
\end{proof}

\begin{theorem}
    Suppose \(F\) a field, and \(K = F(x_1,\hdots,x_n)\)
    the field of rational functions in \(n\) variables.
    Suppose \(S\) is the subfield of symmetric rational functions.
    Then:
    \begin{enumerate}[label={(\alph*)}, itemsep=0mm]
        \item \([K:S] = n!\)
        \item \(\Gal(K/S) = S_n\)
        \item Let \({\{s_i\}}_{i=1}^n\) be the elementary symmetric polynomials.
            Then \(S = F(s_1,\hdots,s_n)\).
        \item \(K\) is the splitting field over \(S\) of
            \(t^n - s_1 t^{n-1} + s_2 t^{n-2} - \cdots + {(-1)}^n s_n\).
    \end{enumerate}
\end{theorem}
\begin{proof}
    % TODO
\end{proof}

\begin{proposition}
    Suppose \(K/F\) Galois, and \(F'/F\) some extension
    such that \(K,F' \subseteq \overline{F}\).
    Then \(\vfunc{\psi}{\Gal(KF'/F)}{\Gal(K/F)}{\sigma}{\sigma\vert_K}\) is injective
    and it induces an isomorphism \(\Gal(KF'/F) \cong \Gal(K/(K \cap F'))\).
\end{proposition}
\begin{proof}
    % TODO: galois extension 3
\end{proof}
\begin{corollary}
    Suppose \(K/F\) Galois, and \(F'/F\) some extension
    such that \(K,F' \subseteq \overline{F}\).
    Then \([KF':F'] = [K:K \cap F']\).
    In particular, \([KF':F'] = [K:F][F':F]\) if and only if \(K \cap F' = F\).
\end{corollary}
\begin{proof}
    % TODO
\end{proof}

\begin{theorem}
    Suppose \(K/F\) and \(F'/F\) both finite Galois,
    such that \(K,F' \subseteq \overline{F}\).
    Then \(\vfunc{\psi}{\Gal(KF'/F)}{\Gal(K/F)\times\Gal(KF'/F)}{\sigma}{(\sigma\vert_K,\sigma\vert_{F'})}\)
    is injective,
    and isomorphism if \(K \cap F' = F\).
\end{theorem}
\begin{proof}
    % TODO
\end{proof}

\begin{corollary}
    Suppose \(E/F\) finite separable.
    Then there exists a Galois closure \(K \supseteq E\) such that
    \(K/F\) is Galois and minimal with respect to this property.
\end{corollary}
\begin{proof}
    % TODO
\end{proof}

% \subsubsection*{Finite Fields}

\begin{theorem}
    Let \(F = \bF_{p^n}\).
    Then \([F:\bF_p] = n\), \(F/\bF_p\) is cyclic,
    and the Galois group is generated by the Frobenius automorphism.
\end{theorem}
\begin{proof}
    % TODO: galois extension 2
\end{proof}
\begin{corollary}
    Suppose \(K/F\) is an extension of finite fields of characteristic \(p\).
    Then \(K/F\) is cyclic Galois.
    If \(\abs{F} = p^n\), then \(\Gal(K/F) = \langle\tau\rangle\)
    where \(\vfunc{\tau}{K}{K}{a}{a^{p^n}}\).
\end{corollary}
\begin{proof}
    % TODO
\end{proof}

\begin{theorem}
    Suppose \(\overline{\bF}_p\) an algebraic closure of \(\bF_p\).
    For any \(n \in \bN\),
    there exists a unique subfield of \(\overline{\bF}_p\) of order \(p^m\).
    If \(K,L\) are subfields of \(\overline{\bF}_p\) of orders \(p^m\) and \(p^n\),
    then \(K \subseteq L \iff m \mid n\).
    In that case, \(L/K\) Galois with \(\Gal(L/K)\) cyclic,
    generated by \(\vfunc{\tau}{L}{L}{a}{a^{p^m}}\).
\end{theorem}
\begin{proof}
    % TODO
\end{proof}

% \subsubsection*{Polynomials}

\begin{definition}
    Suppose \(f(x) \in F[x]\) irreducible and separable,
    and \(K\) splitting field of \(f(x)\).
    Then \(K/F\) is Galois,
    and the Galois group of the polynomial is \(\Gal(f) = \Gal(K/F)\).
\end{definition}
\begin{theorem}
    Suppose \(f(x) \in F[x]\) separable of degree \(n\).
    \begin{enumerate}[label={(\alph*)}, itemsep=0mm]
        \item If \(f(x)\) is irreducible over \(F\),
            then \(n \mid \abs{\Gal(f)}\).
        \item \(f(x)\) is irreducible over \(F[x]\)
            if and only if \(\Gal(f)\) is a transitive subgroup of \(S_n\).
    \end{enumerate}
\end{theorem}
\begin{proof}
    % TODO
\end{proof}


\subsection{Cyclic Extensions}

\begin{definition}
    Suppose \(K/F\) is finite Galois.
    The norm is a multiplicative homomorphism
    \(\vfunc{N_{K/F}}{K}{F}{a}{\prod_{\sigma\in\Aut_F(K)} \sigma(a)}\);
    the trace is an additive homomorphism
    \(\vfunc{\tr_{K/F}}{K}{F}{a}{\sum_{\sigma\in\Aut_F(K)} \sigma(a)}\).
\end{definition}
\begin{proposition}
    Suppose \(K/F\) finite Galois,
    \(a \in K\) some element with minimal polynomial \(f(x) = \min_F(a) \in F[x]\).
    Let \(f(x) = \prod_{\sigma\in\Gal(K/F)} (x-\sigma(a)) = x^n + a_1 x^{n-1} + \cdots + a_n\).
    Then \(N_{K/F} = {(-1)}^n a_n\)  and \(\tr_{K/F} = -a_1\).
\end{proposition}
\begin{proof}
    % TODO
\end{proof}

\begin{theorem}[Hilbert Theorem 90, multiplicative]
    Suppose \(K/F\) is cyclic of degree \(n\), \(G = \Gal(K/F) = \langle\sigma\rangle\).
    For any \(b \in K^\ast\), if \(N_{K/F}(b) = 1\),
    then \(b = a/\sigma(a)\) for some \(a \in K\).
\end{theorem}
\begin{proof}
    % TODO
\end{proof}
\begin{theorem}[Hilbert Theorem 90, additive]
    Suppose \(K/F\) is cyclic of degree \(n\), \(G = \Gal(K/F) = \langle\sigma\rangle\).
    For any \(b \in K\), if \(\tr_{K/F}(b) = 0\),
    then \(b = a - \sigma(a)\) for some \(a \in K\).
\end{theorem}
\begin{proof}
    % TODO
\end{proof}

\begin{definition}
    Let \(K_1\) and \(K_2\) be extensions over \(F\).
    \(K_1\) and \(K_2\) are linearly disjoint over \(F\)
    if \(K_1 \cap K_2 = F\).
\end{definition}

\subsubsection*{Cyclotomic Extensions}

\begin{definition}
    For any \(n \in \bN\), \(\mu_n\) is the group of \(n\)th roots of unity,
    which are roots of \(x^n - 1\).
    A primitive \(n\)th root of unity is \(\zeta_n\),
    with \(\zeta_n^n = 1\) and \(\zeta_n^k \neq 1\) for all \(0 < k < n\).
\end{definition}
\begin{proposition}
    \(\mu_n \cong Z_n\);
    \(d \mid n\) if and only if \(\mu_d \subseteq \mu_n\);
    and \(\langle \zeta_n \rangle = \mu_n\).
\end{proposition}
\begin{proof}
    Obvious.
\end{proof}

\begin{definition}
    The \(n\)th cyclotomic polynomial is
    \(\Phi_n(x) = \prod_{\text{primitive}\;\zeta\in\mu_n} (x-\zeta)\).
\end{definition}
\begin{proposition}
    Suppose \(\Phi_n(x) \in \bQ[x]\).
    Then \(\Phi_n(x)  = \prod_{1 \leq a < n,\; \gcd(a,n) = 1} (x-\zeta_n^a)\),
    and \(\deg \Phi_n(x) = [\bQ(\zeta_n):\bQ] = \varphi(n)\),
    where \(\varphi\) is the Euler totient function.
\end{proposition}
\begin{proof}
    % TODO
\end{proof}
\begin{proposition}
    Suppose \(\Phi_n(x) \in \bQ[x]\).
    Then \(\Phi_n(x) \in \bZ[x]\) is monic.
\end{proposition}
\begin{proof}
    % TODO
\end{proof}

\begin{lemma}
    For each \(\sigma \in \Gal(F(\mu_n)/F)\),
    there exists \(k_\sigma \in \bZ\) such that \(\gcd(k_\sigma,n) = 1\)
    and \(\sigma(\zeta) = \zeta^{k_\sigma}\) for all \(\zeta \in \mu_n\).
\end{lemma}
\begin{proof}
    % TODO: cyclotomic
\end{proof}
\begin{theorem}
    The mapping from \(\Gal(F(\mu_n)/F) \to {(\bZ/n\bZ)}^\ast\)
    via \(\zeta \mapsto k_\sigma \pmod{n}\)
    where \(\sigma(\zeta) = \zeta^{k_\sigma}\) for all \(\zeta \in \mu_n\),
    is an injective group homomorphism.
\end{theorem}
\begin{proof}
    % TODO
\end{proof}
\begin{theorem}
    The mapping from \(\Gal(\bQ(\mu_n)/\bQ) \to {(\bZ/n\bZ)}^\ast\)
    is an isomorphism.
\end{theorem}
\begin{proof}
    % TODO
\end{proof}

\begin{definition}
    The splitting field of \(x^n - 1\) over \(F\)
    is called a cyclotomic extension of order \(n\) over \(F\).
\end{definition}
\begin{theorem}[Kronecker-Weber Theorem]
    Any abelian extension of \(\bQ\)
    is a subextension of a cyclotomic extension.
\end{theorem}
\begin{proof}
    See footnote.\footnote{%
        \url{https://www.math.uchicago.edu/~may/VIGRE/VIGRE2007/REUPapers/FINALFULL/Culler.pdf}
    }
\end{proof}

\subsubsection*{Cyclic Extensions}




\subsection{Radical and Kummer Extensions}


\subsection{Solvable Extensions}


\subsection{Cubic and Quartic Polynomials}
