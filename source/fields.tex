\section{Fields}

\begin{remark}
    We begin by recalling the definition of a field.
    Fields are quintuples \((F,+,\cdot,0,1)\) where
    \((F,+,0)\) and \((F^\star,\cdot,1)\) are both abelian groups,
    and multiplication distributes over addition.
\end{remark}
\begin{remark}
    Fields are commutative rings,
    so theorems in Section~\ref{sec:rings} will entirely hold,
    and we will be using those theorems throughout.
\end{remark}
\begin{remark}
    Much of our foundation will come from extending any integral domain \(R\)
    (in particular \(\bZ\))
    to its field of fractions \(F\) (in particular \(\bQ\)).
\end{remark}


\subsection{Field Extensions}

\begin{definition}
    Suppose \(F\) is a field, and \(F[x]\) its polynomial ring.
    Corollary~\ref{cor:field-polynomial-pid} tells us that
    \(F[x]\) is a domain,
    and hence we can construct its field of fractions,
    which we denote \(F(X) = \{\frac{f(x)}{g(x)} : f,g \in F[x], g \neq 0\}\).
    This is often called the transcendental extension of \(F\),
    or the univariate function field over \(F\).
\end{definition}

\begin{remark}
    Recall that if \(R\) is a commutative ring,
    \(P\) a prime ideal, and \(M\) a maximal ideal,
    then \(R/P\) \hyperref[prop:prime-quotient-domain]{is an integral domain}
    and \(R/M\) \hyperref[cor:maximal-quotient-field]{is a field}.
    But when \(F\) a field,
    \hyperref[prop:pid-maximal-is-prime]{prime ideals and maximal ideals are the same thing}.
\end{remark}

\begin{definition}
    Suppose \(F\) is a field.
    A field extension \(K/F\) is a field \(K \supseteq F\).
    We say \(F\) is a subfield of \(K\),
    where \(F\) is a base field, and \(K\) is an over field of \(F\).

    In commutative diagrams, we often denote this as
    \begin{center}
        \begin{tikzcd}
            K \arrow[dash]{d} \\ F
        \end{tikzcd}
    \end{center}
\end{definition}

\begin{proposition}
    Suppose \(K/F\) a field extension.
    \(K\) can be characterized as a vector space over \(F\).
\end{proposition}
\begin{proof}
    Let \(k,\ell \in K\), and \(\lambda \in F\).
    Trivially by field addition \(k + \ell \in K\),
    and \(\lambda \in F \subseteq K\),
    so by field multiplication we have \(\lambda k \in K\).
    Moreover, distributivity holds because of
    \(\lambda(k+\ell) = \lambda k + \lambda \ell\)
    due to distributivity in \(K\).
\end{proof}

\begin{definition}
    Suppose \(K/F\) field extension.
    We call \(K/F\) finite or \(K\) is a finite extension of \(F\)
    if \(K\) is a finite-dimensional vector space over \(F\).
    If \(K/F\) finite, the degree of \(K/F\),
    denoted \([K:F] = \dim_F K\)
    is the dimension of \(K\) as an \(F\)-space.
\end{definition}
\begin{theorem}\label{thm:finite-extension-stack}
    Suppose \(K/F\) and \(L/K\) are finite field extensions.
    Then \(L/F\) is a finite extension,
    and in particular, \([L:F] = [L:K][K:F]\).
\end{theorem}
\begin{proof}
    Since \(K/F\) finite, let \({\{a_i\}}_{i=1}^m\) be a basis for \(K/F\),
    so \([K:F] = m\).
    For every \(a \in K\), we can write \(a = \sum_{i=1}^m \lambda_i a_i\)
    for some \(\lambda_i \in F\).
    Similarly since \(L/K\) finite,
    let \({\{b_j\}}_{j=1}^n\) be a basis for \(L/K\), so \([L:K] = n\).
    For every \(b \in L\), we can write \(b = \sum_{j=1}^n \mu_j b_j\)
    for some \(\mu_j \in K\).
    But then we have a linear combination of \(mn\) elements
    \begin{equation*}
        b = \sum_{j=1}^n \mu_j b_j
        = \sum_{j=1}^n \qty(\sum_{i=1}^m \lambda_{ij} a_i) b_j
        = \sum_{i,j=1}^{m,n} \lambda_{ij} (a_i b_j)
    \end{equation*}
    It suffices to check that \({{\{a_i b_j\}}_{i=1}^m}_{j=1}^n\) forms a basis.
    Suppose that
    \begin{equation*}
        0 = \sum_{i,j=1}^{m,n} \lambda_{ij} (a_i b_j)
        = \sum_{j=1}^n \qty(\sum_{i=1}^m \lambda_{ij} a_i) b_j
    \end{equation*}
    But by linear independence of \(b_j\),
    for each individual \(j\), \(\sum_{i=1}^m \lambda_{ij} a_i = 0\),
    and by linear independence of \(a_i\),
    for all \(i\), \(\lambda_{ij} = 0\).
    This proves linear independence.
\end{proof}
\begin{corollary}
    Finiteness is a property that is stackable,
    and degree of extension is multiplicative.
\end{corollary}
\begin{proof}
    Merely proceed by induction on the tower of field extensions,
    applying the \hyperref[thm:finite-extension-stack]{theorem above}.
\end{proof}

\begin{theorem}\label{thm:field-hom-injective}
    Every field homomorphism is injective.
\end{theorem}
\begin{proof}
    Suppose \(\func{\phi}{F}{K}\) is a field homomorphism.
    By way of contradiction, suppose \(\phi\) is not injective.
    Then there exists \(a,b \in F\) such that \(\phi(a) = \phi(b)\) and \(a \neq b\).
    But that tells us \(f(a-b) = 0 = f(0)\), and in particular,
    since \(a-b \neq 0\), \(f(a-b)f({(a-b)}^{-1}) = f(1) = 1\).
    But that implies \(0f({(a-b)}^{-1}) = 1\) which is a contradiction.
\end{proof}


\subsection{Algebraic Extensions}

\begin{definition}
    Suppose \(K/F\) is a field extension, and \(a \in K\) some element.
    We call \(a\) algebraic over \(F\)
    if there exists a (monic) polynomial \(f(x) \in F[x]\) such that \(f(a) = 0\).
    If \(a \in K\) is not algebraic over \(F\),
    we call \(a\) transcendental over \(F\).
\end{definition}

\begin{definition}
    Suppose \(a \in K\) is algebraic over \(F\).
    The minimal polynomial of \(a\) over \(F\),
    either denoted \(\min_F(a)\) or \(\min(a;F)\)
    is the irreducible polynomial \(f(x) \in F[x]\)
    with the lowest degree such that \(f(a) = 0\).
\end{definition}
\begin{proposition}
    A monic minimal polynomial exists and is unique for every algebraic \(a \in K\).
\end{proposition}
\begin{proof}
    By definition there must be some polynomial that it satisfies.
    If it is reducible, reduce it into irreducible components,
    and \(a\) must satisfy one of those irreducible components.
    It is unique since if another monic polynomial \(g(x)\) satisfies our conditions,
    \((g-f)(x)\) also satisfies our conditions,
    and since they are the same degree, \(\deg(g-f) < \deg f\) violates minimality.
\end{proof}
\begin{definition}
    Suppose \(a \in K\) is algebraic over \(F\).
    We define the degree of \(a\) to be the degree of its minimal polynomial,
    \(\deg a = \deg \min_F(a)\).
\end{definition}

\begin{definition}
    Suppose \(K/F\) is a field extension.
    If all elements \(a \in K\) are algebraic over \(F\),
    we call \(K/F\) an algebraic extension.
\end{definition}
\begin{definition}
    Suppose \(K/F\) is an extension, and \(a \in K\) some element.
    \(F(a) \subseteq K\) is the smallest subfield that contains \(F\)
    and all elements \(\lambda = \sum_i \ell_i a^i\) for some \(\ell_i \in F\).
\end{definition}
\begin{proposition}
    \(F(a)\) is the field of fractions of \(F[a]\).
\end{proposition}
\begin{proof}
    By definition of a field of fractions,
    \(F(a)\) is the smallest field that contains \(F[a]\).
\end{proof}
\begin{theorem}\label{thm:algebraic-finite-extension}
    Suppose \(K/F\) is a field extension, and \(a \in K\) some element.
    The following are equivalent:
    \begin{enumerate}[label={(\alph*)}, itemsep=0mm]
        \item \(a\) is algebraic over \(F\);
        \item \(F(a)\) is a finite extension of \(F\); and
        \item \(F[a] = F(a)\).
    \end{enumerate}
\end{theorem}
\begin{proof}
    We first prove that (a) implies (c).
    Since \(a\) is algebraic, there is some minimal polynomial
    \(a^n + \lambda_1 a^{n-1} + \cdots + \lambda_n = 0\) for \(\lambda_i \in F\).
    But we can rewrite the above as
    \begin{equation*}
        -\lambda_n = a^n + \lambda_1 a^{n-1} + \cdots + \lambda_{n-1}a
        = a(a^{n-1} + \lambda_1 a^{n-2} + \cdots + \lambda_{n-1})
    \end{equation*}
    and clearly \(\lambda_n \neq 0\),
    since otherwise \(a^{n-1} + \lambda_1 a^{n-2} + \cdots + \lambda_{n-1}\)
    would be a minimal polynomial of degree less than \(n\),
    violating minimality of the original polynomial.
    Hence we have a polynomial expression for the inverse
    \begin{equation*}
        a^{-1} = -\frac{1}{\lambda_n}
        (a^{n-1} + \lambda_1 a^{n-2} + \cdots + \lambda_{n-1})
    \end{equation*}
    so \(F(a) \subseteq F[a]\).
    \(F[a] \subseteq F(a)\) by definition
    and we have equality.
    
    We now prove that (c) implies (b).
    For any \(i \geq 0\), \(a^{n+i}\) is in the \(F\)-span of \({\{a^j\}}_{j=0}^{n-1}\).
    and since we write elements in \(F(a) = F[a]\)
    as linear combinations of powers of \(a\),
    the degree of those elements must be at most \(n\),
    hence \([F(a):F]\) is finite.

    Lastly we prove that (b) implies (a).
    Suppose \(F(a)\) is a finite extension.
    Then \({\{a^i\}}_{i=0}^\infty \subseteq F(a)\) is a linearly dependent set.
    We shall find the smallest \(ell\) such that
    \({\{a^i\}}_{i=0}^\ell\) is a linearly dependent set.
    By linear dependence we will have
    \(\lambda_0 + \lambda_1 a + \cdots + \lambda_\ell a^\ell = 0\)
    for some \(\lambda_i \in F\), \(\lambda_\ell \neq 0\).
    Hence we have found a degree \(\ell\) polynomial that \(a\) satisfies,
    and \(a\) must be algebraic.
\end{proof}
\begin{corollary}\label{cor:algebraic-finite-extension}
    Suppose \(K/F\) is a finite extension.
    Then \(K\) is algebraic over \(F\).
\end{corollary}
\begin{proof}
    Suppose \(a \in K\).
    Since we have \(F \subseteq F(a) \subseteq K\),
    \(F(a)/F\) must be finite,
    which by the \hyperref[thm:algebraic-finite-extension]{theorem above}
    must be algebraic.
\end{proof}
\begin{remark}
    We have shown that finite extensions are algebraic,
    but algebraic extensions are not necessarily finite.
\end{remark}

\begin{theorem}
    Suppose \(K/F\) is a field extension.
    The set of all \(F\)-algebraic elements in \(K\)
    form a subfield of \(K\) containing \(F\).
\end{theorem}
\begin{proof}
    Let \(a,b \in K\) be algebraic over \(F\).
    It is sufficient to prove that \(a \pm b\), \(ab\),
    and \(a/b\) when \(b \neq 0\) are all algebraic.
    Observe that \([F(a):F],[F(b):F]\) are finite by the
    \hyperref[thm:algebraic-finite-extension]{previous theorem}.
    Let \(F_1 = F(a) \subseteq K\).
    Then again, \([F_1(b):F]\) is also finite by Theorem~\ref{thm:finite-extension-stack}.
    This process is repeatable for finitely many elements.
    From this we can see that \(F(a)(b) = F(a,b)\)
    by minimality of the extension by an element.
    But then this tells us that \(a,b \in F(a,b)\) is a field,
    so \(a \pm b\), \(ab\), and \(a/b\) are all in \(F(a,b)\).
    and finite implies algebraic
    by the \hyperref[thm:algebraic-finite-extension]{theorem above}.
    % It is now sufficient to prove that \(\)
    % Now suppose \(S \subseteq K\) is a subset,
    % and let \(F(S)\) be the smallest subfield of \(K\) containing \(F\) and \(S\).
\end{proof}
\begin{definition}
    Suppose \(K/F\) is an extension.
    Then the subfield of all \(F\)-algebraic elements in \(K\)
    is usually denoted \(F^\text{alg} \subseteq K\).
    \(K/F^\text{alg}\) has no algebraic elements over \(F\),
    while \(F^\text{alg}/F\) is an algebraic extension.
\end{definition}

\begin{definition}
    Suppose \(K/F\) is an extension.
    \(K/F\) is finitely generated if there exists a finite set \(S \subseteq K\)
    such that \(K = F(S)\).
\end{definition}
\begin{theorem}\label{thm:finitely-gen-algebraic}
    Suppose \(K/F\) is finitely generated by \(S\).
    If all elements of \(S\) are algebraic, then \(K/F\) is algebraic.
\end{theorem}
\begin{proof}
    If \(S = {\{s_i\}}_{i=1}^n\),
    then merely apply Theorem~\ref{thm:algebraic-finite-extension}
    to see that \(F(S)/F\) is finite,
    so Corollary~\ref{cor:algebraic-finite-extension}
    tells us that it is algebraic.
\end{proof}
\begin{corollary}\label{cor:algebraic-extension-stack}
    Suppose \(K/F\) and \(L/K\) are algebraic extensions.
    Then \(L/F\) is algebraic.
\end{corollary}
\begin{proof}
    Suppose \(\lambda \in L\).
    Then there exists \(f(x) \in K[x]\) such that \(f(\lambda) = 0\).
    \(f(x) = x^n + b_1 x^{n-1} + \cdots + b_0\) where \(b_i \in K\).
    But notice that \(b_i \in K\) are algebraic over \(F\),
    so \(F(b_1,\hdots,b_n)/F \subseteq K\) is a finite extension
    by the \hyperref[thm:finitely-gen-algebraic]{theorem above}.
    We can see that \(f(x)\) is also a polynomial in \(F(b_1,\hdots,b_n)[x]\),
    which tells us that \(F(b_1,\hdots,b_n)(\lambda)/F(b_1,\hdots,b_n)\) is finite.
    But these two finite extensions combined gives us that
    \(F(b_1,\hdots,b_n)(\lambda)/F\) is finite by Theorem~\ref{thm:finite-extension-stack}.
    Since \(F(\lambda) \subseteq F(b_1,\hdots,b_n)\),
    \(F(\lambda)/F\) must also be finite,
    which implies \(\lambda\) is algebraic over \(F\)
    by Theorem~\ref{thm:algebraic-finite-extension}.
\end{proof}


\subsection{Splitting Fields and Algebraic Closures}

\begin{theorem}\label{thm:field-extension-gain-root}
    Suppose \(f(x) \in F[x]\) is an irreducible polynomial of degree at least 2,
    There exists an extension \(K/F\) such that \(f(x)\) has roots in \(K\).
\end{theorem}
\begin{proof}
    Let \(I = (f(x)) \subseteq F[x]\) be a maximal ideal,
    since \(f(x)\) is irreducible by Theorem~\ref{thm:ideal-divisibility}.
    Then we have a field \(K = F[x]/I \supseteq F\)
    by Corollary~\ref{cor:maximal-quotient-field},
    with degree \([K:F] = \deg f\) which we will say is \(n\).
    This induces a homomorphism \(\func{\pi\circ\iota}{F}{K}\)
    where \(\func{\iota}{F}{F[x]}\) is the inclusion of \(F\) into its polynomial ring,
    and \(\func{\pi}{F[x]}{K}\) is the quotient by \(I\),
    both of them ring homomorphisms.
    We can see that \(\pi\circ\iota\) is a field homomorphism,
    since elements in \(F\) within \(F[x]\) will not get collapsed by \(I\)
    as their degrees are always 1.
    Now see that \(\pi(f(x)) = \overline{f(x)} = 0\) in \(K\),
    so \(f(\bar{x}) = 0\), and \(\pi\) maps \(x \mapsto \bar{x}\),
    so \(\bar{x}\) is a root in \(K\),
    and \(K = \langle 1, x, \hdots, x^{n-1} \rangle\).
\end{proof}
\begin{corollary}
    If \(f(x) \in F[x]\) has degree \(n\),
    then there exists a field extension \(K/F\)
    with \([K:F] \leq n\) such that \(f(x)\) acquires a root.
\end{corollary}
\begin{proof}
    Split \(f(x)\) into irreducible components
    and then apply the \hyperref[thm:field-extension-gain-root]{theorem above}.
\end{proof}

\begin{definition}
    Suppose \(L_1/F\) and \(L_2/F\) are field extensions,
    where \(L_1,L_2 \subseteq L\).
    The composite extension \(L_1 \cdot L_2\)
    is the smallest field extension of \(F\) in \(L\)
    that contains both \(L_1\) and \(L_2\).
\end{definition}

\begin{definition}
    A polynomial \(f(x) \in F[x]\) splits over an extension \(K/F\)
    if \(f(x) \in K[x]\) can be written as a product of linear polynomials,
    i.e.\ all roots exist in \(K\).
\end{definition}

\begin{definition}
    Suppose \(K/F\) is an extension.
    An automorphism of \(K\) is a field automorphism \(\sigma \in \Aut(L)\)
    that is bijective;
    an \(F\)-automorphism is a field automorphism \(\sigma \in \Aut_F(L)\),
    with the additional condition that it is the identity when restricted to \(F\),
    \(\sigma\vert_F = \id_F\).
\end{definition}
\begin{lemma}
    Suppose \(K/F\) is an extension,
    and \(\sigma \in \Aut_F(K)\) an \(F\)-automorphism.
    If \(a \in K\) is an element with a minimal polynomial \(\min_F(a) = f(x)\),
    then \(\sigma(a)\) is also a root of \(f(x)\).
\end{lemma}
\begin{proof}
    By definition \(a\) is a root of \(f(x)\), so \(f(a) = 0\).
    But \(\sigma(f(a)) = \sigma(0) = 0\),
    and notice that \(\sigma(f(x)) = f(x)\),
    since the coefficients are in \(F\) so they remain unchanged under \(\sigma\),
    so \(f(\sigma(a)) = 0\).
\end{proof}

\begin{definition}
    The splitting field of a polynomial \(f(x) \in F[X]\) over \(F\)
    is a field \(K\) such that
    \begin{enumerate}[label={(\roman*)}, itemsep=0mm]
        \item \(f(x)\) splits completely over \(K\);
        \item \(K \supseteq F\); and
        \item if \(K'\) satisfies the previous two conditions,
            then \(K' \supseteq K\).
    \end{enumerate}
\end{definition}
\begin{proposition}[Existence of Splitting Fields]\label{prop:splitting-field-existence}
    For any \(f(x) \in F[x]\), its splitting field exists.
\end{proposition}
\begin{proof}
    Without loss of generality suppose \(f(x)\) irreducible and monic.
    By Theorem~\ref{thm:field-extension-gain-root},
    we can find a field extension \(F_1/F\)
    such that \(f(x) = (x-a_1)f_1(x) \in F_1[x]\).
    Iterate on \(f_i(x)\) until \(f(x)\) is comprised of linear factors.
\end{proof}
\begin{corollary}
    Let \(f(x) \in F[x]\) be a polynomial of degree \(n\).
    Then there exists an extension \(K/F\) with degree \([K:F] \leq n!\)
    such that \(F\) splits completely over \(K\).
\end{corollary}
\begin{proof}
    Same proof as \hyperref[prop:splitting-field-existence]{proposition above},
    but notice that \([F_1:F] \leq n\),
    and then induct on \(n\), since \(\deg f_1(x) \leq n-1\).
\end{proof}

% \begin{definition}
%     Suppose \(F\) is a field.
%     An algebraic closure \(\overline{F}\) is a field that
%     every polynomial \(f(x) \in F[x]\) splits completely over \(\overline{F}\),
%     and every element is algebraic over \(F\).
% \end{definition}
\begin{lemma}
    Suppose \(F\) is a field.
    Then the following are equivalent:
    \begin{enumerate}[label={(\alph*)}, itemsep=0mm]
        \item The only algebraic extension of \(F\) is itself;
        \item The only finite extension of \(F\) is itself;
        \item If \(K/F\) is an extension,
            then \(F\) is the set of \(F\)-algebraic elements in \(K\);
        \item Every \(f(x) \in F[x]\) splits over \(F\); and
        \item Every \(f(x) \in F[x]\) has a root in \(F\).
    \end{enumerate}
\end{lemma}
\begin{proof}
    (a) implies (b) is the contrapositive of Corollary~\ref{cor:algebraic-finite-extension}.

    (b) implies (c) because if there exists another \(F\)-algebraic element \(a \in K-F\),
    then there is a finite extension \(F(a)/F\).

    (c) implies (d) since if \(f(x) \in F[x]\) and \(a\) is a root,
    then \(a\) is algebraic over \(F\), and must be in \(F\).
    Induct on order of \(f\).

    (d) implies (e) trivially true.

    (e) implies (d) simply by induction on degree of \(f\).

    (d) implies (a) since if there are other algebraic extensions,
    then that means there is some polynomial that does not have a root in \(F\).
\end{proof}
\begin{definition}
    \(F\) is algebraically closed if it satisfies any of the conditions above.
\end{definition}
% \begin{proposition}
%     Algebraic closures are algebraically closed.
% \end{proposition}
% \begin{proof}
%     It is sufficient to prove that \(\overline{\overline{F}} = \overline{F}\).
%     Suppose \(p(x) \in \overline{F}[x]\) is a polynomial
%     and let \(a\) be a root of \(p(x)\).
%     Let \(f(x) = \min_{\overline{F}}(a)\) be the minimal polynomial.
% \end{proof}

\begin{definition}
    If \(K\) is an algebraic extension of \(F\) and is algebraically closed,
    then \(K\) is an algebraic closure of \(F\).
\end{definition}
\begin{theorem}[Existence of Algebraic Closures]
    Suppose \(F\) is a field.
    Then there exists an algebraically closed field \(K \supseteq F\).
\end{theorem}
\begin{proof}
    We first attempt to construct an extension \(F_1/F\)
    such that every polynomial \(f(x) \in F[x]\) of degree at least 1
    has at least one root.
    For every \(f(x) \in F[x]\) let us assign an arbitrary letter
    (some unique indeterminate) \(X_f\),
    and let \(S\) be the set of all \(X_f\).
    We can get a bijection between \(S\) and \(F[x] - F\).
    We now get a multivariate polynomial ring \(F[S]\).

    We now wish to show that the ideal \(I\) generated by all \(f(X_f)\)
    is not the unit ideal \((1) = F[S]\).
    Suppose, by way of contradiction, that \(I = (1) = F[S]\).
    % TODO: see alg closed in 422
\end{proof}
\begin{corollary}
    Suppose \(F\) is a field,
    then there exists an algebraic extension \(F^\text{alg}/F\)
    that is algebraically closed.
\end{corollary}
\begin{proof}
    % TODO: see alg closed in 422
\end{proof}

\begin{proposition}
    Suppose \(\func{\sigma}{F}{L}\) is a field homomorphism,
    where \(L = \overline{L}\) iss algebraically closed.
    Let \(\alpha\) be algebraic over \(F\),
    \(p(x) = \min_F(a) = \sum_{i=0}^n a_i x^i\) a monic minimal polynomial,
    and \(\sigma(p)(x) = \sum_{i=0}^n \sigma(a_i)x^i\) its image.
    Then given a homomorphism \(\func{\tau}{F(\alpha)}{L}\)
    that is an extension of \(\sigma\) (\(\tau\vert_F = \sigma\)),
    the evaluation map \(\tau \mapsto \tau(\alpha)\)
    is a bijection between homomorphisms extending \(\sigma\)
    (\(\{\tau\in\Hom(F(\alpha),L): \tau\vert_F = \sigma\}\))
    and the roots of the image polynomial in \(L\)
    (\(\{\beta \in L : \sigma(p)(\beta) = 0\}\)).
\end{proposition}
\begin{proof}
    Let \(\func{\tau}{F(\alpha)}{L}\) extending \(\sigma\).
    Then \(\tau(p(\alpha)) = \sigma(p)(\tau(\alpha)) = 0\)
    since \(p(\alpha) = 0\).
    % TODO
\end{proof}

\begin{theorem}[Uniqueness of Splitting Fields]
    Suppose \(p(x) \in E[x]\) an irreducible polynomial,
    and \(K\) is a splitting field of \(p(x) \in E[x]\).
    Suppose \(\func{\phi}{E}{F}\) is a field isomorphism,
    and \(L\) is a splitting field of \(\phi(p)(x) \in F[x]\).
    Then \(\phi\) extends to an isomorphism \(\func{\sigma}{K}{L}\)
    where \(\sigma\vert_E = \phi\).
    In particular, splitting fields of \(p(x) \in F[x]\) over \(F\)
    are \(F\)-isomorphic.
\end{theorem}


\subsection{Separability}

\begin{definition}
    An (irreducible) polynomial \(f(x) \in F[x]\) of degree \(n\)
    is separable if \(f(x)\) has \(n\) distinct roots in its splitting field,
    or equivalently, if its irreducible factors are distinct and separable.
\end{definition}
\begin{proposition}\label{prop:derivative-inseparability}
    \(f(x) \in F[x]\) is inseparable
    if and only if \(f\) and \(f'\) share a root.
\end{proposition}
\begin{proof}
    Suppose \(f(x)\) is separable.
    Then we can write, in its splitting field,
    that \(f(x) = \prod_i (x-r_i)\) for all distinct \(r_i\).
    By the product rule,
    \(f'(x)\) has one term that is missing \(x-r_i\),
    and all other terms contain \(x-r_i\),
    Let us inspect \(f'(r_i)\).
    All the terms that contain \(r_i\) will evaluate to 0,
    while the term that has \(r_i\) missing will evaluate to nonzero.
    Hence \(0 = f(r_i) \neq f'(r_i)\).

    Now suppose \(f(x)\) is inseparable.
    Then there is some root \(r\) that has multiplicity \(m > 1\),
    so we write in a splitting field \(f(x) = {(x-r)}^m g(x)\).
    The formal derivative is then \(f'(x) = m{(x-r)}^{m-1} g(x) + {(x-r)}^m g'(x)\).
    Hence \(f(r) = f'(r) = 0\).
\end{proof}

\begin{definition}
    Suppose \(f(x) \in F[x]\) is a (monic) polynomial,
    and over a splitting field is written as
    \(f(x) = \prod_{i=1}^k {(x-r_i)}^{m_i}\)
    where \(r_i\) are all distinct and \(m_i \geq 1\).
    The multiplicity of a root \(r_i\) is \(m_i\).
    We call \(r_i\) a simple root if \(m_i = 1\),
    and we say \(f(x)\) has repeated roots if there exists \(i\)
    such that \(m_i > 1\).
\end{definition}

\subsubsection*{Resultant and Determinant}

\begin{remark}
    Historically the resultant is very important in determining separability,
    but in modern times the discriminant is more often used.
    Nevertheless, we shall include it here for sake of completeness.
\end{remark}
\begin{definition}
    Suppose \(f(x),g(x) \in F[x]\) are polynomials;
    \(f(x) = \sum_{i=0}^n a_i x^{n-i}\) and \(g(x) = \sum_{j=0}^m b_j x^{m-j}\),
    and the leading terms are nonzero.
    The resultant of \(f\) and \(g\) is
    \begin{equation*}
        \mathcal{R}(f,g) = a_0^m b_0^n \prod_{i,j} (\alpha_i - \beta_j)
    \end{equation*}
    where \(\alpha_i\) are the roots of \(f\),
    and \(\beta_j\) are the roots of \(g\).
\end{definition}
\begin{lemma}\label{lem:resultant-zero}
    \(\mathcal{R}(f,g) = 0\) if and only if \(f\) and \(g\) share a root.
\end{lemma}
\begin{proof}
    If they share a root then one of the terms in the product is 0.
\end{proof}
\begin{corollary}
    \(f\) is inseparable if and only if \(\mathcal{R}(f,f') = 0\).
\end{corollary}
\begin{proof}
    \(f\) and \(f'\) sharing a root is equivalent to
    the resultant being zero by the \hyperref[lem:resultant-zero]{lemma above}.
    Then apply Proposition~\ref{prop:derivative-inseparability}.
\end{proof}
\begin{remark}
    If \(f(x) = \prod_i (x-\alpha_i)\) and \(g(x) = \prod_j (x-\beta_j)\) are monic,
    then \(\prod_i g(\alpha_i) = {(-1)}^{mn} f(\beta) = \mathcal{R}(f,g)\).
    Also notice that the resultant is closely related to the Vandermonde determinant.
    \begin{equation*}
        \mathcal{R}(f,g) = \det
        \begin{bmatrix}
            a_0 & a_1 & \cdots & a_n \\
            & a_0 & a_1 & \cdots & a_n \\
            & & \ddots & & & \ddots \\
            & & & a_0 & a_1 & \cdots & a_n \\
            b_0 & b_1 & \cdots & b_m \\
            & b_0 & b_1 & \cdots & b_m \\
            & & \ddots & & & \ddots \\
            & & & b_0 & b_1 & \cdots & b_m
        \end{bmatrix}
    \end{equation*}
    where you have \(m\) rows of \(a_i\) and \(n\) rows of \(b_i\),
    resulting in a \((m+n)\times(m+n)\) matrix.
\end{remark}

\begin{definition}
    Suppose \(f(x) \in F[x]\) is a polynomial.
    The discriminant is
    \begin{equation*}
        \mathcal{D}(f) = a_0^{2n-2} {(-1)}^{n(n-1)/2}
        \prod_{i \neq j} (\alpha_i - \alpha_j)
        = a_0^{2n-2} \prod_{i<j} {(\alpha_i - \alpha_j)}^2
    \end{equation*}
    where \(\alpha_i\) are roots.
\end{definition}
\begin{proposition}
    \(\mathcal{R}(f,f') = {(-1)}^{n(n-1)/2} a_0 \mathcal{D}(f)\).
\end{proposition}
\begin{proof}
    % TODO
\end{proof}

\subsubsection*{Separable Extensions}

\begin{definition}
    Suppose \(F\) is a field, and \(a\) algebraic over \(F\).
    \(a\) is a separable element if \(\min_F(a)\) is separable.
\end{definition}
\begin{definition}
    Suppose \(E/F\) is an algebraic extension.
    \(E/F\) is a separable extension if every \(a \in E\) is separable.
\end{definition}

\begin{lemma}
    Let \(f(x) = \min_F(a) \in F[x]\).
    \(f(x)\) is separable if and only if \((f,f') = 1\) are coprime.
\end{lemma}
\begin{proof}
    Let \(f(x)\) is separable.
    By way of contradiction,
    suppose \(g(x) \in F[x]\) is a monic irreducible polynomial of degree \(d\),
    where \(g \mid f\) and \(g \mid f'\).
    Let us write \(f(x) = g(x)h(x)\)
    Without loss of generality we may assume \((g,h) = 1\) are also coprime,
    since if they share factors, then send all those factors to \(g\),
    and that does not change irreducibility.
    But then \(f'(x) = g'(x)h(x) + h'(x)g(x)\),
    which implies that \(g(x) \mid g'(x)\)
    since \(g \mid h'g\), \(g \mid g'h\), and \(g \nmid h\).
    But that is not possible, since \(g'\) must have a lower degree that \(g\).
    Hence \((f,f') = 1\) are coprime.

    Now suppose \((f,f') = 1\).
    Notice they cannot share a root in the splitting field in this case,
    which by Proposition~\ref{prop:derivative-inseparability}
    means \(f(x)\) is separable.
\end{proof}

\begin{definition}
    Suppose \(E/F\) is an extension,
    and \(\func{\sigma}{F}{L}\) an embedding of fields.
    We denote the set of all homomorphisms \(\func{\sigma^\ast}{E}{L}\)
    extending \(\sigma\), that is, \(\sigma^\ast\vert_F = \sigma\) as
    \(S_\sigma = \{\sigma^\ast\in\Hom(E,L) : \sigma^\ast\vert_F = \sigma\}\).
\end{definition}
\begin{theorem}
    Suppose \(E = F(\alpha)\) is an algebraic extension of \(F\),
    \(L = \overline{L}\) is an algebraically closed field extension of \(F\),
    % \(E = F(\alpha)\) is an algebraic extension of \(F\),
    and \(p(x) = \min_F(\alpha) \in F[x]\) is the minimal polynomial.
    If \(\func{\sigma}{F}{L}\) is an embedding,
    % that is, \(\sigma\vert_F = \id\).
    then \(\abs{S_\sigma}\),
    the number of embeddings of \(E\) into \(L\) extending \(\sigma\)
    is the number of distinct roots of \(\sigma(p)(x)\).
    In particular, \(\abs{S_\sigma} = [E:F]\)
    if and only if \(E/F\) is separable.
\end{theorem}
\begin{proof}
    % TODO: separability 1
\end{proof}

\begin{proposition}
    Suppose \(E/F\) is an algebraic extension,
    \(L = \overline{L}\) algebraically closed,
    and \(\func{\sigma}{F}{L}\) an embedding of fields,
    Then there exists an embedding \(\func{\tau}{E}{L}\) that extends \(\sigma\),
    i.e.\ \(\tau\vert_F = \sigma\).
    If \(E\) is an algebraic closure of \(F\),
    and \(L\) is the algebraic closure of \(\sigma(F)\),
    then \(\tau\) is an isomorphism.
\end{proposition}
\begin{proof}
    % TODO: separability 1
\end{proof}
\begin{proposition}
    Suppose \(E/F\) is a finite extension,
    \(L = \overline{L}\) is algebraically closed,
    and \(\func{\sigma}{F}{L}\) an embedding of fields.
    % If \(E/F\) is a finite extension,
    There are at most \([E:F]\) different embeddings extending \(\sigma\),
    % extending \(\func{\sigma}{F}{L}\),
    % for \(L = \overline{L}\) algebraically closed.
    % That is, if \(S_\sigma = \{\sigma^\ast\in\Hom(E,L) : \sigma^\ast\vert_F = \sigma\}\),
    that is, \(\abs{S_\sigma} \leq [E:F]\).
\end{proposition}
\begin{proof}
    % TODO: separability 2
\end{proof}
\begin{proposition}
    Suppose \(F \subseteq E \subseteq L\) is a tower of field extensions.
    If \(L/F\) is separable, then \(L/E\) and \(E/F\) are separable.
\end{proposition}
\begin{proof}
    % TODO: separability 2
\end{proof}
\begin{theorem}
    Suppose \(E/F\) is a finite extension,
    \(L = \overline{L}\) algebraically closed,
    and \(\func{\sigma}{F}{L}\) an embedding of fields.
    Then \(\abs{S_\sigma}\),
    the number of distinct embeddings of \(E\) into \(L\) extending \(\sigma\)
    is exactly \([E:F]\) if and only if \(E/F\) is separable.
\end{theorem}
\begin{proof}
    % TODO: separability 2
\end{proof}

\begin{theorem}[Transitivity of Separable Extensions]
    Suppose \(F \subseteq E \subseteq L\) is a tower of field extensions.
    If \(E/F\) and \(L/E\) are both separable,
    then \(L/F\) is separable.
\end{theorem}
\begin{proof}
    % TODO: separability 2
\end{proof}

\subsubsection*{Simple Extensions}

\begin{definition}
    Suppose \(E/F\) is an extension.
    \(E/F\) is simple (algebraic)
    if \(E = F(\alpha)\) (for \(\alpha\) algebraic over \(F\)).
\end{definition}
\begin{theorem}[Primitive Element Theorem]
    Suppose \(E/F\) is a finite separable extension.
    Then \(E/F\) is simple.
\end{theorem}
\begin{proof}
    % TODO: wikipedia
\end{proof}


\subsection{Finite Fields and Inseparability}
