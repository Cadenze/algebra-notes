\section{Fields}

\begin{remark}
    We begin by recalling the definition of a field.
    Fields are quintuples \((F,+,\cdot,0,1)\) where
    \((F,+,0)\) and \((F^\star,\cdot,1)\) are both abelian groups,
    and multiplication distributes over addition.
\end{remark}
\begin{remark}
    Fields are commutative rings,
    so theorems in Section~\ref{sec:rings} will entirely hold,
    and we will be using those theorems throughout.
\end{remark}
\begin{remark}
    Much of our foundation will come from extending any integral domain \(R\)
    (in particular \(\bZ\))
    to its field of fractions \(F\) (in particular \(\bQ\)).
\end{remark}


\subsection{Field Extensions}

\begin{definition}
    Suppose \(F\) is a field, and \(F[x]\) its polynomial ring.
    Corollary~\ref{cor:field-polynomial-pid} tells us that
    \(F[x]\) is a domain,
    and hence we can construct its field of fractions,
    which we denote \(F(X) = \{\frac{f(x)}{g(x)} : f,g \in F[x], g \neq 0\}\).
    This is often called the transcendental extension of \(F\),
    or the univariate function field over \(F\).
\end{definition}

\begin{remark}
    Recall that if \(R\) is a commutative ring,
    \(P\) a prime ideal, and \(M\) a maximal ideal,
    then \(R/P\) \hyperref[prop:prime-quotient-domain]{is an integral domain}
    and \(R/M\) \hyperref[cor:maximal-quotient-field]{is a field}.
    But when \(F\) a field,
    \hyperref[prop:pid-maximal-is-prime]{prime ideals and maximal ideals are the same thing}.
\end{remark}

\begin{definition}
    Suppose \(F\) is a field.
    A field extension \(K/F\) is a field \(K \supseteq F\).
    We say \(F\) is a subfield of \(K\),
    where \(F\) is a base field, and \(K\) is an over field of \(F\).

    In commutative diagrams, we often denote this as
    \begin{center}
        \begin{tikzcd}
            K \arrow[dash]{d} \\ F
        \end{tikzcd}
    \end{center}
\end{definition}

\begin{proposition}
    Suppose \(K/F\) a field extension.
    \(K\) can be characterized as a vector space over \(F\).
\end{proposition}
\begin{proof}
    Let \(k,\ell \in K\), and \(\lambda \in F\).
    Trivially by field addition \(k + \ell \in K\),
    and \(\lambda \in F \subseteq K\),
    so by field multiplication we have \(\lambda k \in K\).
    Moreover, distributivity holds because of
    \(\lambda(k+\ell) = \lambda k + \lambda \ell\)
    due to distributivity in \(K\).
\end{proof}

\begin{definition}
    Suppose \(K/F\) field extension.
    We call \(K/F\) finite or \(K\) is a finite extension of \(F\)
    if \(K\) is a finite-dimensional vector space over \(F\).
    If \(K/F\) finite, the degree of \(K/F\),
    denoted \([K:F] = \dim_F K\)
    is the dimension of \(K\) as an \(F\)-space.
\end{definition}
\begin{theorem}\label{thm:finite-extension-stack}
    Suppose \(K/F\) and \(L/K\) are finite field extensions.
    Then \(L/F\) is a finite extension,
    and in particular, \([L:F] = [L:K][K:F]\).
\end{theorem}
\begin{proof}
    Since \(K/F\) finite, let \({\{a_i\}}_{i=1}^m\) be a basis for \(K/F\),
    so \([K:F] = m\).
    For every \(a \in K\), we can write \(a = \sum_{i=1}^m \lambda_i a_i\)
    for some \(\lambda_i \in F\).
    Similarly since \(L/K\) finite,
    let \({\{b_j\}}_{j=1}^n\) be a basis for \(L/K\), so \([L:K] = n\).
    For every \(b \in L\), we can write \(b = \sum_{j=1}^n \mu_j b_j\)
    for some \(\mu_j \in K\).
    But then we have a linear combination of \(mn\) elements
    \begin{equation*}
        b = \sum_{j=1}^n \mu_j b_j
        = \sum_{j=1}^n \qty(\sum_{i=1}^m \lambda_{ij} a_i) b_j
        = \sum_{i,j=1}^{m,n} \lambda_{ij} (a_i b_j)
    \end{equation*}
    It suffices to check that \({{\{a_i b_j\}}_{i=1}^m}_{j=1}^n\) forms a basis.
    Suppose that
    \begin{equation*}
        0 = \sum_{i,j=1}^{m,n} \lambda_{ij} (a_i b_j)
        = \sum_{j=1}^n \qty(\sum_{i=1}^m \lambda_{ij} a_i) b_j
    \end{equation*}
    But by linear independence of \(b_j\),
    for each individual \(j\), \(\sum_{i=1}^m \lambda_{ij} a_i = 0\),
    and by linear independence of \(a_i\),
    for all \(i\), \(\lambda_{ij} = 0\).
    This proves linear independence.
\end{proof}
\begin{corollary}
    Finiteness is a property that is stackable,
    and degree of extension is multiplicative.
\end{corollary}
\begin{proof}
    Merely proceed by induction on the tower of field extensions,
    applying the \hyperref[thm:finite-extension-stack]{theorem above}.
\end{proof}


\subsection{Algebraic Extensions}

\begin{definition}
    Suppose \(K/F\) is a field extension, and \(a \in K\) some element.
    We call \(a\) algebraic over \(F\)
    if there exists a (monic) polynomial \(f(x) \in F[x]\) such that \(f(a) = 0\).
    If \(a \in K\) is not algebraic over \(F\),
    we call \(a\) transcendental over \(F\).
\end{definition}

\begin{definition}
    Suppose \(a \in K\) is algebraic over \(F\).
    The minimal polynomial of \(a\) over \(F\),
    either denoted \(\min_F(a)\) or \(\min(a;F)\)
    is the irreducible polynomial \(f(x) \in F[x]\)
    with the lowest degree such that \(f(a) = 0\).
\end{definition}
\begin{proposition}
    A monic minimal polynomial exists and is unique for every algebraic \(a \in K\).
\end{proposition}
\begin{proof}
    By definition there must be some polynomial that it satisfies.
    If it is reducible, reduce it into irreducible components,
    and \(a\) must satisfy one of those irreducible components.
    It is unique since if another monic polynomial \(g(x)\) satisfies our conditions,
    \((g-f)(x)\) also satisfies our conditions,
    and since they are the same degree, \(\deg(g-f) < \deg f\) violates minimality.
\end{proof}
\begin{definition}
    Suppose \(a \in K\) is algebraic over \(F\).
    We define the degree of \(a\) to be the degree of its minimal polynomial,
    \(\deg a = \deg \min_F(a)\).
\end{definition}

\begin{definition}
    Suppose \(K/F\) is a field extension.
    If all elements \(a \in K\) are algebraic over \(F\),
    we call \(K/F\) an algebraic extension.
\end{definition}
\begin{definition}
    Suppose \(K/F\) is an extension, and \(a \in K\) some element.
    \(F(a) \subseteq K\) is the smallest subfield that contains \(F\)
    and all elements \(\lambda = \sum_i \ell_i a^i\) for some \(\ell_i \in F\).
\end{definition}
\begin{proposition}
    \(F(a)\) is the field of fractions of \(F[a]\).
\end{proposition}
\begin{proof}
    By definition of a field of fractions,
    \(F(a)\) is the smallest field that contains \(F[a]\).
\end{proof}
\begin{theorem}\label{thm:algebraic-finite-extension}
    Suppose \(K/F\) is a field extension, and \(a \in K\) some element.
    The following are equivalent:
    \begin{enumerate}[label={(\alph*)}, itemsep=0mm]
        \item \(a\) is algebraic over \(F\);
        \item \(F(a)\) is a finite extension of \(F\); and
        \item \(F[a] = F(a)\).
    \end{enumerate}
\end{theorem}
\begin{proof}
    We first prove that (a) implies (c).
    Since \(a\) is algebraic, there is some minimal polynomial
    \(a^n + \lambda_1 a^{n-1} + \cdots + \lambda_n = 0\) for \(\lambda_i \in F\).
    But we can rewrite the above as
    \begin{equation*}
        -\lambda_n = a^n + \lambda_1 a^{n-1} + \cdots + \lambda_{n-1}a
        = a(a^{n-1} + \lambda_1 a^{n-2} + \cdots + \lambda_{n-1})
    \end{equation*}
    and clearly \(\lambda_n \neq 0\),
    since otherwise \(a^{n-1} + \lambda_1 a^{n-2} + \cdots + \lambda_{n-1}\)
    would be a minimal polynomial of degree less than \(n\),
    violating minimality of the original polynomial.
    Hence we have a polynomial expression for the inverse
    \begin{equation*}
        a^{-1} = -\frac{1}{\lambda_n}
        (a^{n-1} + \lambda_1 a^{n-2} + \cdots + \lambda_{n-1})
    \end{equation*}
    so \(F(a) \subseteq F[a]\).
    \(F[a] \subseteq F(a)\) by definition
    and we have equality.
    
    We now prove that (c) implies (b).
    For any \(i \geq 0\), \(a^{n+i}\) is in the \(F\)-span of \({\{a^j\}}_{j=0}^{n-1}\).
    and since we write elements in \(F(a) = F[a]\)
    as linear combinations of powers of \(a\),
    the degree of those elements must be at most \(n\),
    hence \([F(a):F]\) is finite.

    Lastly we prove that (b) implies (a).
    Suppose \(F(a)\) is a finite extension.
    Then \({\{a^i\}}_{i=0}^\infty \subseteq F(a)\) is a linearly dependent set.
    We shall find the smallest \(ell\) such that
    \({\{a^i\}}_{i=0}^\ell\) is a linearly dependent set.
    By linear dependence we will have
    \(\lambda_0 + \lambda_1 a + \cdots + \lambda_\ell a^\ell = 0\)
    for some \(\lambda_i \in F\), \(\lambda_\ell \neq 0\).
    Hence we have found a degree \(\ell\) polynomial that \(a\) satisfies,
    and \(a\) must be algebraic.
\end{proof}



% \begin{theorem}
%     Suppose \(K/F\) is a field extension.
%     The set of all \(F\)-algebraic elements in \(K\)
%     form a subfield of \(K\) containing \(F\).
% \end{theorem}
% \begin{proof}
%     Let \(a,b \in K\) be algebraic over \(F\).
%     Then \([F(a):F],[F(b):F]\) are finite by the
%     \hyperref[thm:algebraic-finite-extension]{previous theorem}.
%     Let \(F_1 = F(a) \subseteq K\).
%     Then again, \([F_1(b):F]\) is also finite by Theorem~\ref{thm:finite-extension-stack}.
%     This process is repeatable for finitely many elements.
%     From this we can see that \(F(a)(b) = F(a,b)\).

%     % It is now sufficient to prove that \(\)
%     Now suppose \(S \subseteq K\) is a subset,
%     and let \(F(S)\) be the smallest subfield of \(K\) containing \(F\) and \(S\).
% \end{proof}

\subsection{Splitting Fields}

\begin{theorem}\label{thm:field-extension-gain-root}
    Suppose \(f(x) \in F[x]\) is an irreducible polynomial of degree at least 2,
    There exists an extension \(K/F\) such that \(f(x)\) has roots in \(K\).
\end{theorem}
\begin{proof}
    Let \(I = (f(x)) \subseteq F[x]\) be a maximal ideal,
    since \(f(x)\) is irreducible by Theorem~\ref{thm:ideal-divisibility}.
    Then we have a field \(K = F[x]/I \supseteq F\)
    by Corollary~\ref{cor:maximal-quotient-field},
    with degree \([K:F] = \deg f\) which we will say is \(n\).
    This induces a morphism \(\func{\pi\circ\iota}{F}{K}\)
    where \(\func{\iota}{F}{F[x]}\) is the inclusion of \(F\) into its polynomial ring,
    and \(\func{\pi}{F[x]}{K}\) is the quotient by \(I\),
    both of them ring homomorphisms.
    We can see that \(\pi\circ\iota\) is a field homomorphism,
    since elements in \(F\) within \(F[x]\) will not get collapsed by \(I\)
    as their degrees are always 1.
    Now see that \(\pi(f(x)) = \overline{f(x)} = 0\) in \(K\),
    so \(f(\bar{x}) = 0\), and \(\pi\) maps \(x \mapsto \bar{x}\),
    so \(\bar{x}\) is a root in \(K\),
    and \(K = \langle 1, x, \hdots, x^{n-1} \rangle\).
\end{proof}
\begin{corollary}
    If \(f(x) \in F[x]\) has degree \(n\),
    then there exists a field extension \(K/F\)
    with \([K:F] \leq n\) such that \(f(x)\) acquires a root.
\end{corollary}
\begin{proof}
    Split \(f(x)\) into irreducible components
    and then apply the \hyperref[thm:field-extension-gain-root]{theorem above}.
\end{proof}
