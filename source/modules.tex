\chapter{Modules}\label{sec:modules}

\section{Basic Definitions}

\begin{definition}
    We recall from the previous chapter that
    a left module over a ring \(R\) is a quadruple \((M,+,\cdot,0)\)
    where \(0 \in M\) is a set equipped with addition \(+\)
    with an identity \(0\)
    and scalar multiplication,
    with the following four properties:
    \begin{enumerate}[label={(\roman*)}, itemsep=0mm]
        \item \((M,+,0)\) forms an abelian group;
        \item scalar multiplication
            \(\vfunc{\cdot}{R \times M}{M}{(\alpha,m)}{\alpha m}\),
            and in particular \(1m\);
        \item \(\forall\{\alpha,\beta\} \in R,\, \forall\{m,n\} \subset M\),
            the distributive laws \((\alpha+\beta)(m+n)
            = \alpha m+\alpha n+\beta m+\beta n\) hold; and
        \item associativity
            \(\forall\{\alpha,\beta\} \subset R,\,\forall m \in M\),
            \((\alpha\beta)m = \alpha(\beta m)\).
    \end{enumerate}
    A right module is similarly defined with the multiplication \(MR \subseteq M\) instead.
\end{definition}

\begin{definition}
    Suppose \(M\) is an \(R\)-module.
    \(M\) is generated by \(X \subseteq M\)
    if for any arbitrary \(m \in M\),
    there exists scalars \(a_i \in R\) and \(x_i \in X\) such that
    we have the finite sum \(m = \sum_i a_i x_i\).
\end{definition}

\begin{theorem}
    Suppose \(M\) is an abelian group.
    Then \(M\) is also:
    \begin{enumerate}[label={(\alph*)}, itemsep=0mm]
        \item a \(\bZ\)-module; and
        \item an \(R\)-module, where \(R = \End_\bZ(M)\).
    \end{enumerate}
\end{theorem}
\begin{proof}
    By definition, \(M\) is an abelian group.
    We can define scalar multiplication to be \(am = \underbrace{m+\cdots+m}_{a\,\text{times}}\),
    Then of course
    \begin{equation*}
        (a+b)(m+n) = \underbrace{(m+n)+\cdots+(m+n)}_{a+b\,\text{times}}
        = \underbrace{m+\cdots+m}_{a+b\,\text{times}} + \underbrace{n+\cdots+n}_{a+b\,\text{times}}
        = am + bm + an + bn
    \end{equation*}
    which is also obviously associative because
    \begin{equation*}
        (ab)m = \underbrace{m+\cdots+m}_{ab\,\text{times}}
        = \underbrace{bm+\cdots+bm}_{a\,\text{times}}
        = a(bm)
    \end{equation*}
    This proves part (a).

    For part (b), we have to first show that \(R = \End(M)\) is indeed a ring.
    Let \(\phi,\psi \in \End(M)\),
    and define addition as \((\phi+\psi)(m) = \phi(m)+\psi(m)\),
    and multiplication as function composition \((\phi\psi)(m) = (\phi\circ\psi)(m)\).
    Addition forms an abelian group,
    since the element-wise definition inherits closure, associativity, and commutativity,
    and the identity is defined trivially as the zero map \(m \mapsto 0\),
    so the inverse can be defined as \(m \mapsto -m\).
    Multiplication is associative by function composition,
    closure is inherited from \(M\),
    and identity is defined by \(m \mapsto m\).
    Distributivity is given by
    \begin{gather*}
        \phi(\psi+\chi)(m) = \phi(\psi(m)+\chi(m))
        = (\phi\circ\psi)(m) + (\phi\circ\chi)(m)
        = (\phi\psi+\phi\chi)(m)
    \end{gather*}

    Then, scalar multiplication can be defined as evaluation \(\phi m = \phi(m)\);
    and associativity holds by function composition.
    By the fact that we have homomorphisms,
    distributivity holds.
    \begin{equation*}
        (\phi+\psi)(m+n) = \phi(m+n) + \psi(m+n)
        = \phi(m) + \phi(n) + \psi(m) + \psi(n)
    \end{equation*}
    This proves part (b).
\end{proof}

% \begin{theorem}[Cayley's Theorem]
%     Suppose \(R\) is a ring,
%     and let \(M = (R,+,0)\) be its additive group, treated as a \(\bZ\)-module.
%     Then \(R \cong \End(M)\).
% \end{theorem}
% \begin{proof}
%     Consider \(\func{\ell_r}{m}{rm}\), and let us show that \(\ell_r \in \End_\bZ(M)\).
%     \begin{equation*}
%         \ell_r(am+n) = r(am+n) = a(rm)+rn = a\ell_r(m) + \ell_r(n)
%     \end{equation*}
%     Then we shall show that \(\vfunc{\phi}{R}{\End(M)}{r}{\ell_r}\) is an isomorphism.
%     Suppose \(\ell_r = \ell_s\).
%     Then for all \(m \in R\), \(\ell_r(m) = \ell_r(s)\),
%     and in particular, \(\ell_r(1) = r = s = \ell_r(1)\).
%     We have shown injectivity.
%     Suppose \(\phi \in \End(M)\), and that \(\phi(1) = r\).
%     We claim that \(\phi = \ell_r\),
%     and suppose, by way of contradiction, that it is not.
%     Then \(\phi - \ell_r \in \End(M)\) is a nonzero homomorphism,
%     so there exists some \(m \in M\) such that
%     \(\phi(m) - rm \neq 0\).
% \end{proof}
% \begin{remark}
%     This is the ring-theoretic analogue of
%     the \hyperref[thm:cayley]{group-theoretic Cayley's theorem}.
% \end{remark}

\begin{definition}
    Suppose \(M\) is an \(R\)-module.
    Then \(N \subseteq M\) is a subring if:
    \begin{enumerate}[label={(\roman*)}, itemsep=0mm]
        \item \((N,+,0) \subseteq (M,+,0)\) is a subgroup; and
        \item for all \(r \in R\), \(rn \in N\).
    \end{enumerate}
\end{definition}
\begin{lemma}
    Suppose \(M\) is an \(R\)-module,
    and \(N \subseteq M\) is a submodule.
    Then \(M/N\) forms an \(R\)-module.
\end{lemma}
\begin{proof}
    Clearly \(M/N\) readily forms an additive quotient group,
    as Proposition~\ref{prop:abelian-subgroup-normal} informs us that abelian subgroups are normal.
    Then scalar multiplication is \(r(x+N) = rx + rN = rx + N\).
    Distributivity and associativity is then inherited from \(M\).
\end{proof}
\begin{corollary}
    \(\vfunc{\pi}{M}{M/N}{x}{x+N}\) is the quotient homomorphism.
\end{corollary}
\begin{proof}
    All properties are inherited.
\end{proof}

\begin{definition}
    Suppose \(M\) is an \(R\)-module. Then \(M\) is 
    \begin{enumerate}[label={(\roman*)}, itemsep=0mm]
        \item finitely generated if there exists \({\{x_i\}}_{i=1}^n \subset M\) that generates \(M\); and
        \item a cyclic module if it can be generated by one element.
    \end{enumerate}
\end{definition}
\begin{definition}
    Suppose \(R\) is a commutative integral domain,
    and \(M\) is an \(R\)-module.
    \begin{enumerate}[label={(\roman*)}, itemsep=0mm]
        \item \(M\) is a torsion module if for all \(m \in M\),
            there exists a nonzero \(r \in R\) such that \(rm = 0\).
        \item \(M\) is torsion-free if \(rm = 0\) implies either \(r = 0\) or \(m = 0\).
    \end{enumerate}
\end{definition}
\begin{definition}
    Suppose \(M\) is an \(R\)-module.
    The annihilator of \(M\) is \(\Ann M = \{r \in R: rM = 0\}\).
\end{definition}


\section{Free Modules}

\begin{definition}
    Suppose \(M\) is an \(R\)-module.
    \(M\) is free if there is a basis \(X \subset M\);
    that is, \(X\) generates \(M\) (analogous to spanning)
    and \(X\) is linearly independent.
\end{definition}
\begin{remark}
    Throughout this chapter,
    we assume that any free module must be finitely generated;
    that is, they are free modules of finite rank.
\end{remark}
\begin{lemma}
    For any free \(R\)-module \(M\), \(M \cong R^n\).
\end{lemma}
\begin{proof}
    Consider the basis \({\{x_i\}}_{i=1}^n \subset M\).
    Let us construct the homomorphism \(\vfunc{\phi}{R^n}{M}{(a_1,\hdots,a_n)}{\sum a_i x_i}\).
    Linearity is obvious.
    \begin{equation*}
        \phi(c(a_1,\hdots,a_n)+(b_1,\hdots,b_n))
        = \sum (ca_i + b_i)x_i
        = c \sum a_i x_i + \sum b_i x_i
        = c\phi(a_1,\hdots,a_n) + \phi(b_1,\hdots,b_n)
    \end{equation*}
    Injectivity is given by linear independence.
    Surjectivity is given by spanning.
\end{proof}

\begin{lemma}\label{lem:free-module-homomorphism}
    Suppose \(M\) is a free \(R\)-module, \(N\) is any \(R\)-module,
    and \(X \subset M\) is a basis.
    Then \(\func{\phi}{M}{N}\) is wholly determined by the values \(\phi(x)\) for all \(x \in X\).
\end{lemma}
\begin{proof}
    Consider the basis \({\{x_i\}}_{i=1}^n \subset M\).
    If \(\phi(x_i) = \psi(x_i)\) for all \(x_i\),
    then any \(m = \sum a_i x_i\) must have an image
    \(\phi(m) = \sum a_i \phi(x_i) = \psi(m)\).
\end{proof}
\begin{theorem}
    Suppose \(M\) is a free \(R\)-module, and \(N\) is any \(R\)-module,
    and \(X \subset M\) is a basis.
    Given any \(x \in X\) and correspondingly \(n_x \in N\),
    there exists a unique \(\func{\phi}{M}{N}\) with \(\phi(x) = n_x\).
\end{theorem}
\begin{proof}
    The \hyperref[lem:free-module-homomorphism]{lemma above} proves existence.
    If there exists \(\phi \neq \psi\) that both fit this property,
    the \hyperref[lem:free-module-homomorphism]{lemma above}
    tells us that their images are the same.
\end{proof}
\begin{definition}
    Suppose \(\func{\phi}{R^n}{R^m}\) is a homomorphism between free \(R\)-modules.
    If \({\{x_i\}}_{i=1}^n \subset R^n\) is a basis,
    then we can write \(\phi(x_i) = (a_{1i},\hdots,a_{mi})\),
    and the entries \((a_{ij})\) is the matrix representation of \(\phi\).
\end{definition}

\begin{theorem}
    Suppose \(R\) is a (nontrivial) commutative ring.
    Then \(R^m \cong R^n\) implies \(m = n\).
\end{theorem}
\begin{proof}
    Let us have an isomorphism \(\func{\phi}{R^m}{R^n}\)
    Suppose, by way of contradiction, that \(m \neq n\).
    Without loss of generality let us assume that \(m < n\).
    There exists some inverse \(\func{\psi}{R^n}{R^m}\),
    and in particular, \(\phi\psi = \id\) on \(R^n\).
    But we know that we can extend \(\phi\) to some \(\func{\bar{\phi}}{R^m \oplus R^{n-m}}{R^n}\),
    with \(\eval{\bar{\phi}}_{R^{n-m}} = 0\),
    and similarly we can write \(\func{\bar{\psi}}{R^n}{R^m \oplus R^{n-m}}\)
    by padding it with zeroes.
    Let \(\bar{\phi}\) and \(\bar{\psi}\) have matrices \(\vb{A}\) and \(\vb{B}\) respectively.
    Then as \(\vb{A}\vb{B} = \vb{I}\), we have \((\det\vb{A})(\det\vb{B}) = 1\).
    But by construction, the padding of zeroes on \(\bar{\psi}\)
    is going to leave us with rows of zeroes on \(\vb{B}\),
    which implies \(\det\vb{B} = 0\).
    Hence \(0 = 1\), and we have reached a contradiction.
\end{proof}
\begin{remark}
    This proof only works for commutative rings,
    as the notion of a determinant does not exist without commutativity.
\end{remark}
\begin{corollary}
    Suppose \(F\) is a field.
    Then \(F^m \cong F^n\) implies \(m = n\).
\end{corollary}
\begin{proof}
    Fields are commutative rings.
\end{proof}
\begin{definition}
    The dimension of a vector space is the cardinality of its basis.
\end{definition}
\begin{remark}
    Any such ring with this property is said to have an invariant basis number,
    which means its rank is well-defined;
    this is analogous to the dimension vector space being well-defined.
\end{remark}
