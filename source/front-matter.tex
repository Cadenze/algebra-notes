\begin{titlepage}
\begin{center}
    \vspace*{1cm}
    {\Huge\textbf{Notes on Algebra}}

    \vspace{5mm}
    {\LARGE at the Undergraduate Level}

    \vspace{15mm}
    {\Large\textbf{Boris Li}}

    \vspace*{4cm}
    \adjustbox{scale=3,center}{%
        \begin{tikzcd}[row sep=6ex, column sep=5ex]
        X \arrow{d}[swap]{\pi} \arrow{r}{\phi} & Y \\
        X/{\sim} \arrow[dashrightarrow]{ru}[swap]{\exists!\bar{\phi}} &{}
        \end{tikzcd}
    }

    \vfill

    {\large Last compiled: \today}
\end{center}
\end{titlepage}

\vspace*{30mm}

\begin{center}
    \textit{For an undergraduate algebra student:}

    \vspace{5mm}

    \textit{You're gonna get better at this.}
\end{center}

\chapter*{Foreword}\markboth{FOREWORD}{}\addcontentsline{toc}{chapter}{Foreword}

\section*{Version 1.0}

This document was written in the summer of 2023
for the purpose of helping myself review the material
in various classes of algebra.
It is based on Prof.\ Vatsal's MATH 322 (groups) \& 323 (rings) in-class notes.
% and Prof.\ Silberman's MATH 412 in-class notes.

While this does not replace any course notes or the textbook,
I have personally found Jacobson's \textit{Basic Algebra}~\cite{jacobson}
to be overly concise,
with too many details hidden within singular paragraphs;
and Herstein's \textit{Topics in Algebra}~\cite{herstein} sometimes overly verbose.
I aim to lift out definitions from large proofs
as to aid my own digestion of the material,
and to ease the process of revisiting ideas from group theory.

Morgan suggests that Rotman's \textit{An Introduction to the Theory of Groups}~\cite{rotman}
to be a useful book to self-learn from,
while I have occasionally seen Justin
pick up a copy of Lang's \textit{Algebra}~\cite{lang} as reference.
Both of these are good textbooks,
but my current pick still goes to Dummit and Foote's \textit{Abstract Algebra}~\cite{dummitfoote},
as I find it as a great reference book,
with neatly numbered theorems, and detailed proofs;
but it is currently out of print,
so I presume the reader knows where to find such copies.
Although I have never been one to absorb knowledge directly from textbooks,
the reader must be way more diligent than I am,
so go ahead and experiment,
go find out the correct textbook for you.

\medskip

To Madeline:
I truly hope that you can find math as beautiful as I thought it is,
and find understanding algebra for the first time
to be less painful than when I understood it for the first time.
I want you to be better at this than I ever was.

To Arsam:
Maybe this one course will help you find your true place in math,
and can help you decide the direction that you wish to pursue.
While we may want rigour for the sake of understanding,
it may sometimes get in the way of clarity;
choose wisely when it comes to writing proofs.

To Morgan and Aryan:
I know the pain of
Jacobson not having a search-able PDF copy online,
so hopefully this document comes in handy.
Or you guys are just too smart
and can memorize all those definitions first try.

% \vspace{5mm}
\medskip

Boris

August 24, 2023

% \section*{Version 2.0}

% This document was continuously worked on
% over the 2023--2024 academic year
% in an attempt to fulfill my grand ambition
% to consolidate all of my knowledge in algebra
% learnt at the undergraduate level.
% The new sections are based on course material taught in
% Prof.\ Vatsal's MATH 323 (modules),
% Prof.\ Silberman's MATH 412 (linear algebra),
% Prof.\ Ramdorai's MATH 422 (fields and Galois theory),
% and Prof.\ Karu's MATH 423 (commutative algebra),
% with bits and pieces coming from Prof.\ Bryan's MATH 426 (topology).
% Newly included reference texts include
% Halmos' \textit{Finite-Dimensional Vector Spaces}~\cite{halmos},
% Roman's \textit{Advanced Linear Algebra}~\cite{roman} (both for linear algebra),
% Atiyah and MacDonald's \textit{Introduction to Commutative Algebra}~\cite{atiyahmacdonald} (for commutative algebra),
% and by Morgan's recommendation,
% Mac Lane's \textit{Categories for the Working Mathematician}~\cite{maclane} (for category theory).
% Numerous proofs are directly referenced from the typeset notes of
% Profs.\ Silberman, Ramdorai, and Williams.


% A small token of thanks to Justin
% suggesting that the section on categories can be shifted earlier
% as to avoid reproving tedious amounts of theorems later on.
% It didn't really reduce the number of proofs,
% but I guess it looks nicer now?
% I would also like to apologize,
% because Lang is indeed superior to Dummit \& Foote in a variety of ways.

% \medskip

% Boris

% Date?

\vspace{5mm}

Please do not redistribute without prior permission.
