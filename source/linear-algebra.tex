\chapter{Vector Spaces}\label{sec:linear-algebra}\label{sec:vsp}

\begin{remark}
    We shall resume the following convention as in Section~\ref{sec:matrix-rings}:
    \begin{enumerate}[label={(\roman*)}, itemsep=0mm]
        \item \(a\) a lowercase symbol denotes a generic element in a set;
        \item \(\vec{a}\) an underlined symbol denotes a vector; and
        \item \(\vb{A}\) a bold symbol (usually uppercase)
            denotes a matrix or tensorial quantity.
    \end{enumerate}
\end{remark}

\section{Basic Definitions}

\begin{definition}
    A left module over a ring \(R\) is a quadruple \((M,+,\cdot,\vec{0})\)
    where \(\vec{0} \in M\) is a set equipped with addition \(+\)
    with an identity \(\vec{0}\)
    and scalar multiplication,
    with the following four properties:
    \begin{enumerate}[label={(\roman*)}, itemsep=0mm]
        \item \((M,+,\vec{0})\) forms an abelian group;
        \item scalar multiplication
            \(\vfunc{\cdot}{R \times M}{M}{(\alpha,\vec{m})}{\alpha\vec{m}}\),
            and in particular \(1\vec{m} = \vec{m}\);
        \item \(\forall\{\alpha,\beta\} \in R,\, \forall\{\vec{m},\vec{n}\} \subset M\),
            the distributive laws \((\alpha+\beta)(\vec{m}+\vec{n})
            = \alpha\vec{m}+\alpha\vec{n}+\beta\vec{m}+\beta\vec{n}\) hold; and
        \item associativity
            \(\forall\{\alpha,\beta\} \subset R,\,\forall\vec{m} \in M\),
            \((\alpha\beta)\vec{m} = \alpha(\beta\vec{m})\).
    \end{enumerate}
\end{definition}
\begin{definition}
    A vector space over a field \(F\) is similarly defined with \(R = F\).
    It is a quadruple \((V,+,\cdot,\vec{0})\)
    where \(\vec{0} \in V\) is a set equipped with addition \(+\)
    with an identity \(\vec{0}\)
    and scalar multiplication,
    with the following four properties:
    \begin{enumerate}[label={(\roman*)}, itemsep=0mm]
        \item \((V,+,\vec{0})\) forms an abelian group;
        \item scalar multiplication
            \(\vfunc{\cdot}{F \times V}{M}{(\alpha,\vec{v})}{\alpha\vec{v}}\),
            and in particular \(1\vec{v} = \vec{v}\);
        \item \(\forall\{\alpha,\beta\} \in R,\, \forall\{\vec{u},\vec{v}\} \subset M\),
            the distributive laws \((\alpha+\beta)(\vec{u}+\vec{v})
            = \alpha\vec{u}+\alpha\vec{v}+\beta\vec{u}+\beta\vec{v}\) hold; and
        \item associativity
            \(\forall\{\alpha,\beta\} \subset R,\,\forall\vec{v} \in M\),
            \((\alpha\beta)\vec{v} = \alpha(\beta\vec{v})\).
    \end{enumerate}
\end{definition}
\begin{proposition}
    Suppose \(F\) is a field,
    and \(F^X = \{\func{f}{X}{F}\}\) the set of all functions from \(X\) to \(F\).
    When equipped with pointwise addition and multiplication,
    this forms a vector space over \(F\).
\end{proposition}
\begin{proof}
    Since addition and multiplication is pointwise,
    all proofs essentially start with `for all \(x \in X\) \(f(x) \in F\)',
    so distributive laws and associativity laws are inherited from \(F\).
\end{proof}
\begin{corollary}
    \(F^n\) forms a vector space.
\end{corollary}
\begin{proof}
    Simply consider \(X = \{0,1,\hdots,n-1\}\) an \(n\)-element set.
\end{proof}

\begin{definition}
    Suppose \((V,+,\cdot,0,1)\) is a vector space.
    Then \(W\) is a subspace if \(W \subseteq V\)
    and its additive and multiplicative groups form subgroups.
\end{definition}

\begin{definition}
    For any sequence of vectors \({\{\vec{v}_i\}}_{i=1}^r\),
    finite summation is recursively defined as
    \begin{equation*}
        \sum_{i=1}^0 \vec{v}_i = \vec{0} \qquad
        \sum_{i=1}^{r+1} \vec{v}_i = \vec{v}_{r+i} + \sum_{i=1}^r \vec{v}_i \qquad
        \sum_{i=m}^n \vec{v}_i = \sum_{i=1}^n \vec{v}_i - \sum_{i=1}^{m-1} \vec{v}_i
    \end{equation*}
\end{definition}

\begin{definition}
    Suppose \(V\) is a vector space over \(F\)
    (not necessarily finite-dimensional),
    and \(S \subseteq V\) some subset.
    Then
    \begin{enumerate}[label={(\roman*)}, itemsep=0mm]
        \item a vector \(\vec{v} \in V\) (linearly) depends on \(S\)
            if there are finite sets \(\exists {\{\vec{w}_i\}}_{i=1}^r \subset V\),
            \(\exists {\{a_i\}}_{i=1}^r \subset F\),
            such that it can written as a linear combination
            \(\sum_{i=1}^r a_i \vec{w}_i\);
        \item the span (or the \(F\)-span), denoted \(\Span(S)\) or \(\Span_F(S)\),
            is the set of all vectors in \(V\) that depend on \(S\);
        \item \(S\) is (linearly) dependent if \(\exists \vec{v} \in S\)
            such that \(\vec{v}\) depends on \(S \setminus \{\vec{v}\}\),
            and it is (linearly) independent otherwise; and
        \item \(S\) forms a basis of \(V\)
            if \(S\) is (linearly) independent and \(V = \Span(S)\).
    \end{enumerate}
\end{definition}
\begin{lemma}\label{lem:intersection-subspace}
    Suppose \(V\) a vector space, and \(W_i \subseteq V\) subspaces.
    Then \(\bigcap_i W_i\) is also a subspace of \(V\).
\end{lemma}
\begin{proof}
    Apply Lemma~\ref{lem:intersection-subgroup} twice
    exactly like Lemma~\ref{lem:intersection-subring}.
\end{proof}
\begin{proposition}\label{prop:subset-generated-subspace}
    \(\Span(S) = \bigcap_{S \subset W \subset V} W\)
    where \(W\) is a subspace of \(V\).
\end{proposition}
\begin{proof}
    Apply Proposition~\ref{prop:subset-generated-subgroup} twice
    exactly like Proposition~\ref{prop:subset-generated-subring}.
\end{proof}

\begin{lemma}
    \(S\) is dependent if and only if 
    there are distinct \({\{\vec{v}_i\}}_{i=1}^r \subset S\)
    and coefficients \({\{a_i\}}_{i=1}^r \subset F\) that are not all 0
    such that \(\sum_{i=1}^r a_i \vec{v}_i = \vec{0}\).
\end{lemma}
\begin{proof}
    \begin{equation*}
        \vec{v} = \sum_{i=1}^r a_i \vec{v}_i
        \iff \vec{0} = -\vec{v} + \sum_{i=1}^r a_i \vec{v}_i
    \end{equation*}
\end{proof}

\begin{definition}
    Suppose \(U,V\) are vector spaces, and \(\func{f}{U}{V}\) a function.
    \(f\) is linear if \(f(a\vec{u}+\vec{v}) = af(\vec{u}) + f(\vec{v})\),
    i.e.\ it preserves both operations.
\end{definition}
\begin{definition}
    Vector space homomorphisms are linear maps.
\end{definition}
\begin{proposition}
    \(\End(V) = \Hom(V,V)\) endomorphisms of a vector space
    has both a ring structure and a vector space structure.
\end{proposition}
\begin{proof}
    Let \(\func{0}{V}{V}\) map every vector to \(\vec{0}\),
    and \(\func{1}{V}{V}\) be the identity map.
    Clearly if \(\func{f,g}{V}{V}\),
    \(af(\vec{x}) = f(a\vec{x})\) by the fact that it is a linear map,
    and addition is defined the obvious way.
    If we define ring multiplication as function composition,
    then by linearity we have distributivity.
\end{proof}


\section{General Constructions}

\subsection*{Direct Sum}

\begin{definition}[Universal Property of Direct Sum]
    Suppose \({\{V_i\}}_{i \in I}\) is a family of vector spaces.
    The direct sum of vector spaces \(\bigoplus_{i \in I} V_i\)
    is the categorical coproduct of vector spaces, that is,
    \begin{enumerate}[label={(\roman*)}, itemsep=0mm]
        \item there exist embeddings \(\func{\iota_i}{V_i}{\bigoplus_{i \in I} V_i}\); and
        \item for any vector space \(X\),
            if \(\func{\phi_i}{V_i}{X}\) are homomorphisms,
            then there exists a unique \(\func{\bar{\phi}}{\bigoplus_{i \in I} V_i}{X}\)
            such that \(\phi_i = \bar{\phi}\circ\iota_i\).
    \end{enumerate}

    This can be represented by the logical statement
    \begin{equation*}
        \forall X\in\mathbf{Vect},\;
        \forall i \in I,\; \forall \phi_i\in\Hom(V_i,X),\;
        \exists! \phi\in\Hom\qty(\bigoplus_{i \in I} V_i,X),\;
        \phi_i = \bar{\phi}\circ\iota_i
    \end{equation*}
    and the following commutative diagram:
    \begin{center}
        \begin{tikzcd}
            V_i \arrow{r}{\phi_i} \arrow{d}{\iota_i} & X \\
            \bigoplus_{i \in I} V_i \arrow[dashrightarrow]{ru}[swap]{\exists! \bar{\phi}}
        \end{tikzcd}
    \end{center}
\end{definition}

\begin{definition}
    Suppose \(U,V\) are \(F\)-spaces.
    The external direct sum \(U \oplus V\)
    is a vector space \(U \times V\)
    equipped with coordinate-wise operations.
\end{definition}
\begin{lemma}
    \(U \oplus V\) indeed forms a vector space.
\end{lemma}
\begin{proof}
    The additive groups is a direct sum of (abelian) groups,
    which is (abelian) group.
    Check scalar multiplication by
    \begin{equation*}
        (ab)(\vec{u},\vec{v}) = ((ab)\vec{u},(ab)\vec{v})
        = (a(b\vec{u}),a(b\vec{v})) = a(b\vec{u},b\vec{v})
        = a(b(\vec{u},\vec{v}))
    \end{equation*}
\end{proof}
\begin{lemma}
    \(\dim_F(U \oplus V) = \dim_F U + \dim_F V\).
\end{lemma}
\begin{proof}
    Suppose \(B_U \subset U\) and \(B_V \subset V\) be bases.
    Let \(B = {\{(\vec{u},\vec{0})\}}_{\vec{u} \in B_U} \sqcup
    {\{(\vec{0},\vec{v})\}}_{\vec{v} \in B_V}\) be a disjoint union.
    We wish to prove that this forms a basis.
    \begin{equation*}
        \sum_i a_i(\vec{u}_i,\vec{0}) + \sum_j b_j(\vec{0},\vec{v}_j)
        = (\vec{0},\vec{0})
    \end{equation*}
    invokes linear independence by each of \(U\) and \(V\),
    so all \(a_i\) and \(b_j\) are zero.
    Then clearly \(B\) spans \((\vec{u},\vec{v})\)
    simply by span of \(B_U\) and \(B_V\).
\end{proof}

\begin{theorem}[Uniqueness of Direct Sum]
    The direct sum of vector spaces is associative and commutative up to isomorphism.
\end{theorem}
\begin{proof}
    \begin{center}
        \begin{tikzcd}
            V_i \arrow{r}{\iota_i'} \arrow{d}{\iota_i} &
            \bigoplus_{i \in I} V_i' \arrow[rightharpoondown, shift right=0.25ex]{ld}%
                [xshift=.5ex, yshift=-.5ex, swap]{\bar{\phi}'} \\
            \bigoplus_{i \in I} V_i \arrow[rightharpoondown, shift right=0.25ex]{ru}%
                [xshift=-.5ex, yshift=.5ex, swap]{\bar{\phi}}
            % F' \arrow[rightharpoondown, shift right=0.25ex]{ld}%
            %     [xshift=.5ex, yshift=-.5ex, swap]{\bar{\phi}'} \\
            % F \arrow[rightharpoondown, shift right=0.25ex]{ru}%
            %     [xshift=-.5ex, yshift=.5ex, swap]{\bar{\phi}}
        \end{tikzcd}
    \end{center}
    It is obvious from the universal property that
    the unique inclusions do not get affected by order or bracketing.
    Hence \(\bar{\phi}^{-1} = \bar{\phi}'\),
    and we have isomorphism.
\end{proof}


\subsection*{Direct Product}

\begin{definition}[Universal Property of Direct Product]
    Suppose \({\{V_i\}}_{i \in I}\) is a family of vector spaces.
    The direct product of vector spaces \(\prod_{i \in I} V_i\)
    is the categorical product of vector spaces, that is,
    \begin{enumerate}[label={(\roman*)}, itemsep=0mm]
        \item there exists projections \(\func{\pi_i}{\prod_{i \in I} V_i}{V_i}\); and
        \item for any vector space \(X\), if \(\func{\phi_i}{X}{V_i}\) are homomorphisms,
            then there exists a unique \(\func{\bar{\phi}}{X}{\prod_{i \in I} V_i}\)
            such that \(\phi_i = \pi_i\circ\bar{\phi}\).
    \end{enumerate}

    This can be represented by the logical statement
    \begin{equation*}
        \forall X \in \mathbf{Vect},\;
        \forall i \in I,\;
        \forall \phi_i \in \Hom(X,V_i),\;
        \exists! \phi \in \Hom\qty(X,\prod_{i \in I} V_i),\;
        \phi_i = \pi_i \circ \bar{\phi}
    \end{equation*}
    and the following commutative diagram:
    \begin{center}
        \begin{tikzcd}
            V_i & X \arrow{l}[swap]{\phi_i}%
                \arrow[dashrightarrow]{ld}{\exists! \bar{\phi}} \\
            \prod_{i \in I} V_i \arrow{u}[swap]{\pi_i}
                % \arrow[dashrightarrow]{ru}[swap]{\exists! \bar{\phi}}
        \end{tikzcd}
    \end{center}
\end{definition}

\begin{theorem}[Uniqueness of Direct Product]
    The direct product of vector spaces is associative and commutative up to isomorphism.
\end{theorem}
\begin{proof}
    \begin{center}
        \begin{tikzcd}
            V_i & %
                % \arrow[dashrightarrow]{ld}{\exists! \bar{\phi}} \\
            % \prod_{i \in I} V_i \
            \prod_{i \in I} V_i' \arrow[rightharpoondown, shift right=0.25ex]{ld}%
                [xshift=.5ex, yshift=-.5ex, swap]{\bar{\phi}'} \arrow{l}[swap]{\pi_i'} \\
            \prod_{i \in I} V_i \arrow[rightharpoondown, shift right=0.25ex]{ru}%
                [xshift=-.5ex, yshift=.5ex, swap]{\bar{\phi}} \arrow{u}[swap]{\pi_i}
        \end{tikzcd}
    \end{center}
    Again, from the universal property,
    the unique projections do not get affected by order or bracketing.
    Hence we have an isomorphism between direct products.
\end{proof}

\begin{theorem}
    Suppose \({\{V_i\}}_{i=1}^n\) is a finite family of vector spaces.
    Then the direct sum and direct product of finitely many vector spaces are isomorphic.
    \begin{equation*}
        \bigoplus_{i=1}^n V_i \cong \prod_{i=1}^n V_i
    \end{equation*}
\end{theorem}
\begin{proof}
    The following projections and inclusions clearly give a surjection
    from the direct product to the direct sum.
    \begin{center}
        \begin{tikzcd}
            \prod_{i=1}^n V_i \arrow{r}{\pi_i} &
            V_i \arrow{r}{\iota_i} &
            \bigoplus_{i=1}^n V_i
        \end{tikzcd}
    \end{center}
    It suffices to prove that this is also injective.
    But the \hyperref[thm:pigeonhole]{pigeonhole principle}
    gives us that a surjection of finitely many vector spaces
    must also be an injection.
\end{proof}

\begin{remark}
    In infinite indices,
    the direct sum is a smaller space than the direct product.
    Choosing canonical coordinates,
    the direct sum can only have finite support with respect to its basis,
    but the direct product can have infinite support.
\end{remark}


\subsection*{Quotient}

\begin{proposition}[Universal Property of Quotients]
    Suppose \(V\) is a vector space, and \(U \subseteq V\) is a subspace,
    and \(\func{\pi}{V}{U}\) is the quotient map as groups.
    \begin{enumerate}[label={(\alph*)}, itemsep=0mm]
        \item \(\pi\) is a linear map, and there exists a unique vector space structure on \(V/U\);
        \item For any vector space \(Z\), if we have \(f \in \Hom(V,Z)\) and \(\ker(f) \supseteq U\),
            then there exists a unique linear map \(\bar{f} \in \Hom(V/U,Z)\)
            such that \(\bar{f} \circ g = f\).
    \end{enumerate}

    This can be represented by the following commutative diagram:
    \begin{center}
        \begin{tikzcd}
            V \arrow{d}{\pi} \arrow{r}{f} & Z \\
            V/U \arrow[dashrightarrow]{ru}[swap]{\exists! \bar{f}}
        \end{tikzcd}
    \end{center}
\end{proposition}
\begin{proof}
    Suppose \(\vec{x} \in V/U\).
    If \(\vec{x} = \vec{v} + U + \vec{v}' + U\),
    then we can see that \(a\vec{v}-a\vec{v}' = a(\vec{v}-\vec{v}') \in U\),
    which allows us to conclude that \(\pi(a\vec{v}) = \pi(a\vec{v}') = a\pi(\vec{v})\)
    is well-defined.
    The additive axioms are clearly satisfied,
    so it suffices to check distributivity.
    \begin{equation*}
        \pi(a\vec{v}+a\vec{w}) = a\pi(\vec{v}) + a\pi(\vec{w})
        = a\pi(\vec{v} +\vec{w})
    \end{equation*}
\end{proof}

\begin{theorem}[First Isomorphism Theorem for Vector Spaces]\label{thm:iso-1-vsp}
    Suppose \(\func{\phi}{V}{W}\) is a vector space homomorphism,
    and \(U = \ker(\phi)\).
    We have:
    \begin{enumerate}[label={(\alph*)}, itemsep=0mm]
        \item \(U \subseteq V\), the kernel is a subspace;
        \item \(\phi(V) \subseteq W\), the image is a subspace; and
        \item \(\phi(V) \cong V/W\), the image is uniquely isomorphic to the quotient subspace.
    \end{enumerate}
\end{theorem}
\begin{theorem}[Third and Fourth Isomorphism Theorems for Vector Spaces]\label{thm:iso-3-vsp}\label{thm:iso-4-vsp}
    Suppose \(V\) is a vector space, \(U \subseteq V\) some subspace,
    and \(\func{\pi}{V}{V/U}\) is the quotient homomorphism.
    Then \(\pi\) is a bijection between the subspaces of \(V/U\)
    and the subspaces of \(V\) containing \(U\).
    In particular, if \(W\) is one such intermediate subspace,
    then \((V/U)/(W/U) \cong V/W\).
\end{theorem}
\begin{remark}
    Observe that vector spaces are modules over fields,
    so these isomorphism theorems are direct consequences of
    \hyperref[thm:iso-1-module]{isomorphism theorems on modules},
    which we will give a treatment in Chapter~\ref{sec:modules}.
    However, as the reader might be observed
    from the proofs of the isomorphism theorems
    for \hyperref[thm:iso-1-group]{groups} and \hyperref[thm:iso-1-ring]{rings},
    the direct proof of these theorems follow the exact same line of reasoning.
\end{remark}
% EDIT WHEN UNIVERSAL ALGEBRA WRITTEN


\section{Duality}

\begin{definition}
    Suppose \(V\) is an \(F\)-vector space.
    The dual vector space is \(V' = \Hom_F(V,F)\)
    the set of all homomorphisms into the base field.
    In some other fields of math the dual is sometimes denoted \(V^\ast\).
\end{definition}
\begin{lemma}
    Suppose \(W\) and \(V\) are \(F\)-vector spaces. Then
    \begin{enumerate}[label={(\alph*)}, itemsep=0mm]
        \item \(W^V\) is a vector space under coordinate-wise operations; and
        \item \(\Hom_F(W,V) \subseteq W^V\) is a subspace.
    \end{enumerate}
\end{lemma}
\begin{proof}
    \(W^V = \prod_{\vec{v} \in V} W\),
    which by the universal property of the direct product is a vector space.

    The homomorphisms are clearly a subset,
    so it suffices to prove that it is a vector space.
    The zero morphism is trivially linear,
    so we show that if \(f,g\) are linear,
    \begin{align*}
        (af+g)(b\vec{u}+\vec{v}) &= af(b\vec{u}+\vec{v}) + g(b\vec{u}+\vec{v})
        = baf(\vec{u}) + af(\vec{v}) + bg(\vec{u}) + g(\vec{v}) \\
        &= b(af(\vec{u}) + g(\vec{u})) + af(\vec{v}) + g(\vec{v})
        = b(af+g)(\vec{u}) + (af+g)(\vec{v})
    \end{align*}
\end{proof}
\begin{proposition}
    \({(V/U)}' = \{\phi \in V' : \phi\vert_U = \vec{0}\}\).
\end{proposition}
\begin{proof}
    Suppose \(\phi \in {(V/U)}'\).
    Then we can extend this to \(\phi\circ\pi\),
    which since it factors through \(V/U\),
    the kernel contains \(U\),
    so \({(V/U)}' \subseteq \{\phi \in V' : \phi\vert_U = \vec{0}\}\).
    On the other hand, if \(\phi\vert_U = \vec{0}\),
    by the universal property of quotients,
    we can find a corresponding morphism \(V/U \to F\),
    i.e.\ in \({(V/U)}'\).
\end{proof}

\begin{definition}
    Suppose \(B \subset V\) is a basis.
    The dual basis \(B' \subset V'\) is a set where for each \(\vec{u} \in B\),
    there is a corresponding \(\phi_{\vec{u}} \in B'\)
    such that \(\phi_{\vec{u}}(\vec{v}) = \delta_{\vec{u},\vec{v}}\)
    for all \(\vec{v} \in B\),
    where \(\delta\) is the Kronecker delta.
\end{definition}
\begin{lemma}
    Suppose \(B = {\{\vec{v}_i\}}_{i \in I}\),
    and \(B' = {\{\phi_i\}}_{i \in I}\),
    where \(\phi_i(\vec{v}_j) = \delta_{ij}\).
    \begin{enumerate}[label={(\alph*)}, itemsep=0mm]
        \item \(\phi_i(\sum_{j \in I} a_j \vec{v}_j) = a_i\); and
        \item Each \(\phi_i \in B'\) is uniquely defined by choice of \(B\).
    \end{enumerate}
\end{lemma}
\begin{proof}
    \(\phi_i\) are linear functionals, so we have
    \begin{equation*}
        \phi_i\pqty{\sum_{j \in I} a_j \vec{v}_j}
        = \sum_{j \in I} a_j \phi_i(\vec{v}_j)
        = \sum_{j \in I} a_j \delta_{ij}
        = a_i
    \end{equation*}

    Suppose \(\phi_i(\vec{v}_j) = \phi_{i'}(\vec{v}_j) = \delta_{ij}\).
    Since they agree on the basis \(B \subset V\),
    they must also agree on \(V\),
    so \(\phi_i = \phi_{i'}\).
\end{proof}

\begin{lemma}
    The dual basis \(B' \subset V'\) is linearly independent.
\end{lemma}
\begin{proof}
    Suppose \(B'\) not linearly independent.
    Then there exists some \(a_i\) such that \(\sum_{i \in I} a_i \phi_i = 0\).
    But we can evaluate at \(\vec{v}_j \in B\) to conclude that
    \(\sum_{i \in I} a_i \phi_i(\vec{v}_j) = a_j = 0\) for all \(j \in I\).
\end{proof}
\begin{proposition}
    If \(V\) is finite dimensional,
    then \(B' \subset V'\) is a basis.
\end{proposition}
\begin{proof}
    It suffices to prove that it is spanning.
    Suppose \(B = {\{\vec{v}_i\}}_{i=1}^n\),
    and \(\phi \in V'\) some functional.
    For any \(\vec{x} \in V\),
    \(\vec{x} = \sum_{i=1}^n a_i \vec{v}_i\),
    so
    \begin{equation*}
        \phi(\vec{x}) = \sum_{i=1}^n a_i \phi(\vec{v}_i)
        = \sum_{i=1}^n \phi(\vec{v}_i) \phi_i(\vec{x})
    \end{equation*}
    Hence \(\phi = \sum_{i=1}^n \phi(\vec{v}_i) \phi_i\).
\end{proof}
\begin{corollary}
    If \(V\) is finite dimensional,
    then \(\dim V = \dim V'\).
\end{corollary}
\begin{proof}
    The bases are the same size.
\end{proof}
\begin{remark}
    Note that `dual basis' is kind of a misnomer,
    since it does not form a basis in the case of infinite dimensions.
    Rather, the dual basis is the basis for the subspace of \(V'\)
    with finite support.
\end{remark}

\begin{proposition}
    \(\pqty{\bigoplus_i V_i}' \cong \prod_i V_i'\).
\end{proposition}
\begin{proof}
    Suppose \(\phi \in \pqty{\bigoplus_i V_i}'\).
    If we consider the usual restriction of \(\phi\) to each individual \(V_i\),
    and denote each of them \(\phi_i\),
    then the direct product of each of the \(\phi_i\)
    gives us a natural homomorphism
    from \(\pqty{\bigoplus_i V_i}' \to \prod_i V_i'\),
    \(\phi \mapsto (\phi_i)\).

    We shall explicitly construct the inverse as to show that it is an isomorphism.
    Consider \((\psi_i) \in \prod_i V_i'\).
    We can map this to some element \(\sum_i \psi_i\).
    Observe that although there might be infinitely many nonzero \(\psi_i\),
    when applied to \(\vec{x}_i \in V_i\), there are only finitely many nonzero \(\psi_i(\vec{x}_i)\) terms.

    We first check that these two homomorphisms are indeed homomorphisms,
    by checking linearity.
    The forward direction is almost by definition:
    \(a\phi + \chi\) restricts to \(a\phi_i + \chi_i\),
    so \(a\phi + \chi \mapsto (a\phi_i + \chi_i) = a(\phi_i) + (\chi_i)\).
    The reverse direction is similar:
    \(a(\psi_i) + (\omega_i) = (a\psi_i + \omega_i) \mapsto
    \sum_i a\psi_i + \omega_i = a\sum_i \psi_i + \sum_i \omega_i\).

    We then check that it is indeed an inverse.
    \begin{equation*}
        \phi\pqty{\sum_i \vec{x}_i} \overset{f}{\mapsto} (\phi_i(\vec{x}_i))
        = \sum_i \phi_i(\vec{x}_i) = \phi\pqty{\sum_i \vec{x}_i}
    \end{equation*}
\end{proof}

\begin{theorem}
    Suppose \(\vfunc{f}{V}{V''}{\vec{v}}{(\phi\mapsto\phi(\vec{v})) = \delta_{\vec{v}}}\),
    where \(\delta_{\vec{v}}\) is the evaluation map.
    This homomorphism is injective;
    furthermore, it is an isomorphism if and only if \(\dim_F(V) < \infty\).
\end{theorem}
\begin{proof}
    Suppose, by way of contradiction, that there is some \(\vec{v} \neq \vec{0}\)
    such that \(f(\vec{v}) = \delta_{\vec{v}} = 0\);
    this tells us there is some vector \(\vec{v} \in V\)
    such that any \(\phi \in V'\) evaluates it to \(\phi(\vec{v}) = 0\).
    But then, if we choose a basis for \(V\), such that it includes \(\vec{v}\),
    the corresponding dual basis would include some \(\phi_{\vec{v}}\)
    such that \(\phi_{\vec{v}}(\vec{v}) = 1\), which is a contradiction.

    Suppose \(V\) is a finite-dimensional vector space.
    Then it suffices to prove that \(f\) is a surjection.
    Let \({\{\vec{v}_i\}}_{i=1}^n \subset V\) be a basis,
    and \({\{\phi_i\}}_{i=1}^n \subset V'\) be its dual basis.
    Consider any \(\delta \in V''\).
    If we let \(a_i = \delta(\phi_i)\),
    and \(\vec{v} = \sum_{i=1}^n a_i\vec{v}_i\),
    then we see that
    \begin{equation*}
        f(\vec{v}) = \sum_{i=1}^n a_i f(\vec{v}_i)
        = \sum_{i=1}^n \delta(\phi_i) \delta_{\vec{v}_i}
        = \delta
    \end{equation*}
    Hence we have surjection.

    Now suppose \(V\) is an infinite-dimensional vector space.
    Then if \({\{\vec{v}_i\}}_{i \in I} \subset V\) is a basis,
    then \(V = \bigoplus_{i \in I} F\),
    and \(V' = \prod_{i \in I} F\).
    But we can immediately see that in infinite dimensions the dual basis does not form a basis,
    and hence there is no surjection from \(V \to V'\).
    Analogously there is no surjection from \(V' \to V''\).
    Hence there exists no surjection \(V \to V''\).
\end{proof}


\section{Tensor Product}

\begin{definition}
    Suppose \(U,V,W\) are vector spaces,
    and \(\func{f}{U \times V}{W}\) is a function.
    We say \(f\) is bilinear if it is linear in each variable.
    Similarly, if \(\func{f}{\prod_{i=1}^n V_i}{W}\) is a function,
    \(f\) is \(n\)-linear if it is linear in each variable.
\end{definition}


\section{Symmetry and Antisymmetry}


\section{Minimal Polynomial}


\section{Jordan Canonical Form}
