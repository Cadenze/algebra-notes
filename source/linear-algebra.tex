\section{Linear Algebra}\label{sec:linear-algebra}

\begin{remark}
    We shall resume the following convention as in Section~\ref{sec:matrix-rings}:
    \begin{enumerate}[label={(\roman*)}, itemsep=0mm]
        \item \(a\) a lowercase symbol denotes a generic element in a set;
        \item \(\vec{a}\) an underlined symbol denotes a vector; and
        \item \(\vb{A}\) a bold symbol (usually uppercase)
            denotes a matrix or tensorial quantity.
    \end{enumerate}
\end{remark}

\subsection{Basic Definitions}

\begin{definition}
    A left module over a ring \(R\) is a quadruple \((M,+,\cdot,\vec{0})\)
    where \(\vec{0} \in M\) is a set equipped with addition \(+\)
    with an identity \(\vec{0}\)
    and scalar multiplication,
    with the following four properties:
    \begin{enumerate}[label={(\roman*)}, itemsep=0mm]
        \item \((M,+,\vec{0})\) forms an abelian group;
        \item scalar multiplication
            \(\vfunc{\cdot}{R \times M}{M}{(\alpha,\vec{m})}{\alpha\vec{m}}\),
            and in particular \(1\vec{m} = \vec{m}\);
        \item \(\forall\{\alpha,\beta\} \in R,\, \forall\{\vec{m},\vec{n}\} \subset M\),
            the distributive laws \((\alpha+\beta)(\vec{m}+\vec{n})
            = \alpha\vec{m}+\alpha\vec{n}+\beta\vec{m}+\beta\vec{n}\) hold; and
        \item associativity
            \(\forall\{\alpha,\beta\} \subset R,\,\forall\vec{m} \in M\),
            \((\alpha\beta)\vec{m} = \alpha(\beta\vec{m})\).
    \end{enumerate}
\end{definition}
\begin{definition}
    A vector space over a field \(F\) is similarly defined with \(R = F\).
    It is a quadruple \((V,+,\cdot,\vec{0})\)
    where \(\vec{0} \in V\) is a set equipped with addition \(+\)
    with an identity \(\vec{0}\)
    and scalar multiplication,
    with the following four properties:
    \begin{enumerate}[label={(\roman*)}, itemsep=0mm]
        \item \((V,+,\vec{0})\) forms an abelian group;
        \item scalar multiplication
            \(\vfunc{\cdot}{F \times V}{M}{(\alpha,\vec{v})}{\alpha\vec{v}}\),
            and in particular \(1\vec{v} = \vec{v}\);
        \item \(\forall\{\alpha,\beta\} \in R,\, \forall\{\vec{u},\vec{v}\} \subset M\),
            the distributive laws \((\alpha+\beta)(\vec{u}+\vec{v})
            = \alpha\vec{u}+\alpha\vec{v}+\beta\vec{u}+\beta\vec{v}\) hold; and
        \item associativity
            \(\forall\{\alpha,\beta\} \subset R,\,\forall\vec{v} \in M\),
            \((\alpha\beta)\vec{v} = \alpha(\beta\vec{v})\).
    \end{enumerate}
\end{definition}
\begin{proposition}
    Suppose \(F\) is a field,
    and \(F^X = \{\func{f}{X}{F}\}\) the set of all functions from \(X\) to \(F\).
    When equipped with pointwise addition and multiplication,
    this forms a vector space over \(F\).
\end{proposition}
\begin{proof}
    Since addition and multiplication is pointwise,
    all proofs essentially start with `for all \(x \in X\) \(f(x) \in F\)',
    so distributive laws and associativity laws are inherited from \(F\).
\end{proof}
\begin{corollary}
    \(F^n\) forms a vector space.
\end{corollary}
\begin{proof}
    Simply consider \(X = \{0,1,\hdots,n-1\}\) an \(n\)-element set.
\end{proof}
