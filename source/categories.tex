\section{Categories}\label{sec:categories}

\subsection{Basic Definitions}

\begin{definition}
    A category \(\mca{C}\) is a collection of objects \(\Ob(\mca{C})\)
    alongisde a collection of morphisms \(\Mor(\mca{C})\) or \(\Hom(\mca{C})\),
    where
    \begin{enumerate}[label={(\roman*)}, itemsep=0mm]
        \item \(\forall f \in \Mor(\mca{C}),\; \exists X,Y \in \Mor(\mca{C}),\; \func{f}{X}{Y}\),
            a morphism is a mapping from a source object to a target object;
            we define the collection of morphisms from \(X\) to \(Y\)
            as \(\Hom(X,Y) = \{f\in\Mor(\mca{C}) \mid \func{f}{X}{Y}\}\);
        \item \(\forall f \in \Hom(X,Y),\; \forall g \in \Hom(Y,Z),\; g \circ f \in \Hom(X,Z)\),
            the composition is a morphism;
        \item \(\forall f,g,h \in \Mor(\mca{C}),\; (f \circ g) \circ h = f \circ (g \circ h)\),
            morphisms are associative; and
        \item \(\forall X \in \Ob(\mca{C}),\; \exists \id_X \in \Hom(X,X),\; \id_X \circ g = g,\; f \circ \id_X = f\)
            the identity morphism exists.
    \end{enumerate}
\end{definition}
\begin{remark}
    \(\Mor\) and \(\Hom\) are often interchangeable in this context,
    since in algebra, the morphisms that we deal with are usually homomorphisms.
    We shall stick to the convention that the morphisms of a category is \(\Mor\),
    while the homomorphisms between objects are \(\Hom\).
\end{remark}

\begin{definition}
    A morphism \(f \in \Hom(X,Y)\) is an isomorphism
    if it has an inverse \(g \in \Hom(Y,X)\)
    such that \(f \circ g = \id_Y\) and \(g \circ f = \id_X\).
\end{definition}
\begin{definition}
    A morphism \(f \in \Hom(X,X)\) that maps to itself is called an endomorphism.
    The collection of endomorphisms is \(\End(X) = \Hom(X,X)\).
\end{definition}
\begin{definition}
    A morphism \(f \in \Aut(X)\) if it is both an isomorphism and an endomorphism.
\end{definition}

\begin{definition}
    A category \(\mca{C}\) is small
    if \(\Ob(\mca{C})\) and \(\Mor(\mca{C})\) both form sets.
    A category is large otherwise.
\end{definition}

\begin{proposition}
    The category \(\mathbf{Set}\) consisting of all sets and set maps form a category.
\end{proposition}
\begin{proposition}
    The category \(\mathbf{Grp}\) consisting of all groups and group homomorphisms
    form a category.
\end{proposition}
\begin{proposition}
    The category \(\mathbf{Ab}\) consisting of all abelian groups
    and group homomorphisms between them form a category.
\end{proposition}
\begin{proposition}
    The category \(\mathbf{Ring}\) consisting of all rings and ring homomorphisms
    form a category.
\end{proposition}
\begin{proof}
    Obvious for the above four.
\end{proof}

\begin{theorem}
    Suppose \(\mca{C}\) is a category, and \(X \in \Ob(\mca{C})\).
    Then \(\Aut(X)\) forms a group.
\end{theorem}
\begin{proof}
    First off, if \(f,g \in \Aut(X)\),
    then \(f \circ g \in \Aut(X)\),
    since \(f \circ g \in \End(X)\) and composition of bijections are bijections.
    Associativity and identity is given for free.
    By definition, the inverse exists since they are isomorphisms.
\end{proof}

\begin{definition}
    A universal property is a property
    that characterizes an object up to isomorphism.
    This is often stated as given some objects and morphisms,
    there exists a (unique) morphism to some desired object that we want to define.
\end{definition}


\subsection{Duality}

\begin{remark}
    Since morphisms are directed,
    the notion of duality is to reverse such morphisms
    and see whether they make sense.
\end{remark}

\begin{definition}
    A functor \(\func{F}{\mca{C}}{\mca{D}}\) is a mapping between categories,
    and \(F\) maps both objects \(X \in \Ob(\mca{C}) \mapsto F(X) \in \Ob(\mca{D})\)
    and morphisms \(f \in \Mor(\mca{C}) \mapsto F(f) \in \Mor(\mca{D})\).
    Functors respect the identity \(F(\id_\mca{C}) = \id_\mca{D}\).
\end{definition}

\begin{definition}
    We call a functor covariant if it preserves the direction of morphisms
    \(f \in \Hom(X,Y)\) maps to \(F(f) \in \Hom(F(X),F(Y))\),
    and composition \(F(g \circ f) = F(g) \circ F(f)\).
    We call a functor contravariant if it reverses the direction of morphisms
    \(f \in \Hom(X,Y)\) maps to \(F(f) \in \Hom(F(Y),F(X))\),
    and composition \(F(g \circ f) = F(f) \circ F(g)\).
\end{definition}

\subsubsection*{Subobject}

\begin{remark}
    We first generalize the notion of a subset to any categorical object.
\end{remark}
\begin{definition}
    Suppose \(X,Y\) are objects.
    We say \(X \subseteq Y\) is (isomorphic to) a subobject
    if there exists an inclusion (injective homomorphism)
    \(\func{\iota}{X}{Y}\).
\end{definition}

% \begin{theorem}[Universal Property of Subobjects]
%     Let \(X \subseteq Y\) be a subobject-superobject pair,
%     and \(Z\) some arbitrary object.
%     Suppose \(\func{\iota}{X}{Y}\) is the subobject inclusion,
%     and \(\func{\phi}{Z}{Y}\) is any injective homomorphism
%     with the image \(\phi(Z) = \iota(X)\).
%     Then there exists a unique homomorphism \(\func{\bar{\phi}}{Z}{X}\)
%     such that \(\phi = \iota\circ\bar{\phi}\).

%     This is represented by the following commutative diagram:
%     \begin{center}
%         \begin{tikzcd}
%             Y & Z \arrow{l}[swap]{\phi}
%             \arrow[dashrightarrow]{ld}{\exists!\bar{\phi}} \\
%             X \arrow{u}[swap]{\iota}
%         \end{tikzcd}
%     \end{center}
% \end{theorem}
% \begin{proof}
%     % We wish to prove uniqueness by assuming existence.
%     % Suppose \(\phi = \iota\circ\bar{\phi} = \iota\circ\bar{\phi}'\).
%     Since \(\phi(Z) = \iota(X)\),
%     there exists a bijection \(Z \to \phi(Z) \to X\).
%     That bijection is our desired map \(\bar{\phi}\).
% \end{proof}

\subsubsection*{Quotient Object}

\begin{remark}
    The dual to a subobject is a quotient object.
\end{remark}
\begin{definition}
    Suppose \(X\) is an object, and \(\sim\) an equivalence relation on \(X\).
    We say \(X/{\sim}\) is a quotient object of \(X\)
    if there exists a projection \(\func{\pi}{X}{X/{\sim}}\),
    where we map any element to its equivalence class.
\end{definition}

% \begin{theorem}[Universal Property of Quotients]
%     Let \(X\) be an object with a relation \(\sim\).
%     Suppose \(\func{\pi}{X}{X/{\sim}}\) is the quotient projection morphism,
%     and \(\func{\phi}{X}{Z}\) is any surjective homomorphism
% \end{theorem}


\subsection{Products}

\subsubsection*{Product}

\begin{definition}
    Suppose \(X,Y\) are two objects.
    We call \(X \times Y\) the product of objects if
    \begin{enumerate}[label={(\roman*)}, itemsep=0mm]
        \item there exists projections \(\func{\pi_1}{X \times Y}{X}\)
            and \(\func{\pi_2}{X \times Y}{Y}\); and
        \item for any object \(Z\),
            if \(\func{\phi_1}{Z}{X}\) and \(\func{\phi_2}{Z}{Y}\) are homomorphisms,
            then there exists a unique \(\func{\bar{\phi}}{Z}{X \times Y}\)
            such that \(\phi_i = \pi_i\circ\bar{\phi}\) for \(i = 1,2\).
    \end{enumerate}

    This can be represented by the following commutative diagram:
    \begin{center}
        \begin{tikzcd}
            & Z \arrow{ld}[swap]{\phi_1} \arrow{rd}{\phi_2}
            \arrow[dashrightarrow]{d}{\exists!\bar{\phi}} \\
            X & X \times Y \arrow{l}{\pi_1} \arrow{r}[swap]{\pi_2} & Y
        \end{tikzcd}
    \end{center}
\end{definition}

\begin{definition}
    Suppose \({\{X_i\}}_{i \in I}\) is a family of objects.
    We call \(\prod_{i \in I} X_i\) the (infinite) product of objects if
    \begin{enumerate}[label={(\roman*)}, itemsep=0mm]
        \item there exists projections \(\func{\pi_i}{\prod_{i \in I} X_i}{X_i}\); and
        \item for any object \(Z\),
            if \(\func{\phi_i}{Z}{X_i}\) are homomorphisms,
            then there exists a unique \(\func{\bar{\phi}}{Z}{\prod_{i \in I} X_i}\)
            such that \(\phi_i = \pi_i\circ\bar{\phi}\).
    \end{enumerate}

    This can be represented by the logical statement
    \begin{equation*}
        \forall Z \in \mca{C},\;
        \forall i \in I,\;
        \forall \phi_i \in \Hom(X,V_i),\;
        \exists! \phi \in \Hom\pqty{Z,\prod_{i \in I} X_i},\;
        \phi_i = \pi_i\circ\bar{\phi}
    \end{equation*}
    and the following commutative diagram:
    \begin{center}
        \begin{tikzcd}
            & Z \arrow{ld}[swap]{\phi_i}
            \arrow[dashrightarrow]{d}{\exists!\bar{\phi}} \\
            X_i & \prod_{i \in I} X_i \arrow{l}{\pi_i}
        \end{tikzcd}
    \end{center}
\end{definition}

\subsubsection*{Coproduct}

\begin{remark}
    The coproduct is the dual to the product.
\end{remark}

\begin{definition}
    Suppose \(X,Y\) are two objects.
    We call \(X \oplus Y\) the coproduct of objects if
    \begin{enumerate}[label={(\roman*)}, itemsep=0mm]
        \item there exists inclusions \(\func{\iota_1}{X}{X \oplus Y}\)
            and \(\func{\iota_2}{Y}{X \oplus Y}\); and
        \item for any object \(Z\),
            if \(\func{\phi_1}{X}{Z}\) and \(\func{\phi_2}{Y}{Z}\) are homomorphisms,
            then there exists a unique \(\func{\bar{\phi}}{X \oplus Y}{Z}\)
            such that \(\phi_i = \bar{\phi}\circ\iota_i\) for \(i = 1,2\).
    \end{enumerate}

    This can be represented by the following commutative diagram:
    \begin{center}
        \begin{tikzcd}
            & Z \\
            X \arrow{r}[swap]{\iota_1} \arrow{ru}{\phi_1} &
            X \oplus Y \arrow[dashrightarrow]{u}[swap]{\exists!\bar{\phi}} &
            Y \arrow{l}{\iota_2} \arrow{lu}[swap]{\phi_2}
        \end{tikzcd}
    \end{center}
\end{definition}

\begin{definition}
    Suppose \({\{X_i\}}_{i \in I}\) is a family of objects.
    We call \(\bigoplus_{i \in I} X_i\) the (infinite) coproduct of objects if
    \begin{enumerate}[label={(\roman*)}, itemsep=0mm]
        \item there exists inclusions \(\func{\iota_i}{X}{\bigoplus_{i \in I} X_i}\); and
        \item for any object \(Z\),
            if \(\func{\phi_i}{X_i}{Z}\) are homomorphisms,
            then there exists a unique \(\func{\bar{\phi}}{\bigoplus_{i \in I} X_i}{Z}\)
            such that \(\phi_i = \bar{\phi}\circ\iota_i\).
    \end{enumerate}

    This can be represented by the logical statement
    \begin{equation*}
        \forall Z \in \mca{C},\;
        \forall i \in I,\;
        \forall \phi_i \in \Hom(X_i,Z),\;
        \exists! \phi \in \Hom\pqty{\bigoplus_{i \in I} X_i, Z},\;
        \phi_i = \bar{\phi}\circ\iota_i
    \end{equation*}
    and the following commutative diagram:
    \begin{center}
        \begin{tikzcd}
            & Z \\
            X_i \arrow{r}[swap]{\iota_i} \arrow{ru}{\phi_i} &
            \bigoplus_{i \in I} X_i \arrow[dashrightarrow]{u}[swap]{\exists!\bar{\phi}}
        \end{tikzcd}
    \end{center}
\end{definition}

\subsubsection*{Pullback}

\begin{definition}
    Suppose \(W,X,Y\) are three objects.
    We call \(X \times_W Y\) the pullback or fibre product with respect to \(W\) if
    \begin{enumerate}[label={(\roman*)}, itemsep=0mm]
        \item there exists projections \(\func{\pi_X}{X}{W}\) and \(\func{\pi_Y}{Y}{W}\);
        \item there exists projections \(\func{\pi_1}{X \times_W Y}{X}\)
            and \(\func{\pi_2}{X \times_W Y}{Y}\); and
        \item for any object \(Z\),
            if \(\func{\phi_1}{Z}{X}\) and \(\func{\phi_2}{Z}{Y}\) are homomorphisms,
            then there exists a unique \(\func{\bar{\phi}}{Z}{X \times_W Y}\)
            such that the following diagram commutes.
    \end{enumerate}
    \begin{center}
        \begin{tikzcd}
            & Z \arrow{ld}[swap]{\phi_1} \arrow{rd}{\phi_2}
            \arrow[dashrightarrow]{d}{\exists!\bar{\phi}} \\
            X \arrow{rd}[swap]{\pi_X} &
            X \times Y \arrow{l}{\pi_1} \arrow{r}[swap]{\pi_2} &
            Y \arrow{ld}{\pi_Y} \\
            & W
        \end{tikzcd}
    \end{center}
\end{definition}

\subsubsection*{Pushout}

\begin{remark}
    The pushout is the dual to the pullback.
\end{remark}

\begin{definition}
    Suppose \(W,X,Y\) are three objects.
    We call \(X +_W Y\) the pushout or fibre coproduct with respect to \(W\) if
    \begin{enumerate}[label={(\roman*)}, itemsep=0mm]
        \item there exists inclusions \(\func{\iota_X}{W}{X}\) and \(\func{\iota_Y}{W}{Y}\);
        \item there exists inclusions \(\func{\iota_1}{X}{X +_W Y}\)
            and \(\func{\iota_2}{Y}{X +_W Y}\); and
        \item for any object \(Z\),
            if \(\func{\phi_1}{X}{Z}\) and \(\func{\phi_2}{Z}{Y}\) are homomorphisms,
            then there exists a unique \(\func{\bar{\phi}}{X +_W Y}{Z}\)
            such that the following diagram commutes.
    \end{enumerate}
    \begin{center}
        \begin{tikzcd}
            & Z \\
            X \arrow{r}[swap]{\iota_1} \arrow{ru}{\phi_1} &
            X +_W Y \arrow[dashrightarrow]{u}[swap]{\exists!\bar{\phi}} &
            Y \arrow{l}{\iota_2} \arrow{lu}[swap]{\phi_2} \\
            & W \arrow{lu}{\iota_X} \arrow{ru}[swap]{\iota_Y}
        \end{tikzcd}
    \end{center}
\end{definition}


\subsection{Limits}

\subsubsection*{Direct Limit}

\begin{definition}
    Suppose \({\{X_i\}}_{i \in I}\) is a family of objects,
    and the index set \(I\) is ordered with the relation \(\leq\).
    A direct system over \(I\) is when \(i \leq j\),
    let \(\func{f_{ij}}{X_i}{X_j}\) be a homomorphism such that
    \begin{enumerate}[label={(\roman*)}, itemsep=0mm]
        \item \(f_{ii} = \id_{X_i}\); and
        \item \(f_{ik} = f_{jk} \circ f_{ij}\) for all \(i \leq j \leq k\).
    \end{enumerate}

    This can be represented by the following commutative diagram:
    \begin{center}
        \begin{tikzcd}
            X_i \arrow{r}[swap]{f_{ij}} \arrow[bend left]{rr}{f_{ik}} &
            X_j \arrow{r}[swap]{f_{jk}} & X_k
        \end{tikzcd}
    \end{center}
\end{definition}

\begin{definition}
    Suppose \({(X_i,f_{ij})}_{i,j \in I}\) is a direct system.
    The direct limit of this system is an object \(\dirlim X_i\)
    that satisfies the following universal property:
    \begin{enumerate}[label={(\roman*)}, itemsep=0mm]
        \item there exists inclusions \(\func{\iota_i}{X_i}{\dirlim X_i}\)
            such that for all \(i \leq j\), \(\iota_i = \iota_j \circ f_{ij}\); and
        \item for any object \(Z\), if \(\func{\phi_i}{X_i}{Z}\) are homomorphisms
            such that for all \(i \leq j\), \(\phi_i = \phi_j \circ f_{ij}\),
            then there exists a unique \(\func{\bar{\phi}}{\dirlim X_i}{Z}\)
            such that \(\phi_i = \bar{\phi}\circ\iota_i\).
    \end{enumerate}

    This can be represented by the following commutative diagram:
    \begin{center}
        \begin{tikzcd}
            & Z \\
            & \dirlim X_i \arrow[dashrightarrow]{u}[swap]{\exists!\bar{\phi}} \\
            X_i \arrow{rr}[swap]{f_{ij}} \arrow{ur}[swap]{\iota_i} \arrow[bend left]{uur}{\phi_i} &&
            X_j \arrow{ul}{\iota_j} \arrow[bend right]{uul}[swap]{\phi_j}
        \end{tikzcd}
    \end{center}
\end{definition}

\subsubsection*{Inverse Limit}

\begin{remark}
    The inverse system and inverse limit
    is dual to the direct system and direct limit.
\end{remark}

\begin{definition}
    Suppose \({\{X_i\}}_{i \in I}\) is a family of objects,
    and the index set \(I\) is ordered with the relation \(\leq\).
    An inverse system over \(I\) is when \(i \leq j\),
    let \(\func{f_{ij}}{X_j}{X_i}\) be a homomorphism such that
    \begin{enumerate}[label={(\roman*)}, itemsep=0mm]
        \item \(f_{ii} = \id_{X_i}\); and
        \item \(f_{ik} = f_{ij} \circ f_{jk}\) for all \(i \leq j \leq k\).
    \end{enumerate}

    This can be represented by the following commutative diagram:
    \begin{center}
        \begin{tikzcd}
            X_i & X_j \arrow{l}{f_{ij}} &
            X_k \arrow{l}{f_{jk}} \arrow[bend right]{ll}[swap]{f_{ik}}
        \end{tikzcd}
    \end{center}
\end{definition}

\begin{definition}
    Suppose \({(X_i,f_{ij})}_{i,j \in I}\) is an inverse system.
    The inverse limit of this system is an object \(\invlim X_i\)
    that satisfies the following universal property:
    \begin{enumerate}[label={(\roman*)}, itemsep=0mm]
        \item there exists projections \(\func{\pi_i}{\invlim X_i}{X_i}\)
            such that for all \(i \leq j\), \(\pi_i = f_{ij}\circ\pi_j\); and
        \item for any object \(Z\), if \(\func{\phi_i}{Z}{X_i}\) are homomorphisms
            such that for all \(i \leq j\), \(\phi_i = f_{ij}\circ\phi_j\),
            then there exists a unique \(\func{\bar{\phi}}{Z}{\invlim X_i}\)
            such that \(\phi_i = \pi_i\circ\bar{\phi}\).
    \end{enumerate}

    This can be represented by the following commutative diagram:
    \begin{center}
        \begin{tikzcd}
            & Z \arrow[bend right]{ddl}[swap]{\phi_i} \arrow[bend left]{ddr}{\phi_j}
            \arrow[dashrightarrow]{d}{\exists!\bar{\phi}} \\
            & \invlim X_i \arrow{dl}{\pi_i} \arrow{dr}[swap]{\pi_j} \\
            X_i && X_j \arrow{ll}{f_{ij}}
        \end{tikzcd}
    \end{center}
\end{definition}


% \subsection{Universal Algebra}


% \subsection{Categorical Constructions}
