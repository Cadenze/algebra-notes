\section{Categories}

\subsection{Basic Definitions}

\begin{definition}
    A category \(\mca{C}\) is a collection of objects \(\Ob(\mca{C})\)
    alongisde a collection of morphisms \(\Mor(\mca{C})\) or \(\Hom(\mca{C})\),
    where
    \begin{enumerate}[label={(\roman*)}, itemsep=0mm]
        \item \(\forall f \in \Mor(\mca{C}),\; \exists X,Y \in \Mor(\mca{C}),\; \func{f}{X}{Y}\),
            a morphism is a mapping from a source object to a target object;
            we define the collection of morphisms from \(X\) to \(Y\)
            as \(\Hom(X,Y) = \{f\in\Mor(\mca{C}) \mid \func{f}{X}{Y}\}\);
        \item \(\forall f \in \Hom(X,Y),\; \forall g \in \Hom(Y,Z),\; g \circ f \in \Hom(X,Z)\),
            the composition is a morphism;
        \item \(\forall f,g,h \in \Mor(\mca{C}),\; (f \circ g) \circ h = f \circ (g \circ h)\),
            morphisms are associative; and
        \item \(\forall X \in \Ob(\mca{C}),\; \exists \id_X \in \Hom(X,X),\; \id_X \circ g = g,\; f \circ \id_X = f\)
            the identity morphism exists.
    \end{enumerate}
\end{definition}

\begin{definition}
    A morphism \(f \in \Hom(X,Y)\) is an isomorphism
    if it has an inverse \(g \in \Hom(Y,X)\)
    such that \(f \circ g = \id_Y\) and \(g \circ f = \id_X\).
\end{definition}
\begin{definition}
    A morphism \(f \in \Hom(X,X)\) that maps to itself is called an endomorphism.
    The collection of endomorphisms is \(\End(X) = \Hom(X,X)\).
\end{definition}
\begin{definition}
    A morphism \(f \in \Aut(X)\) if it is both an isomorphism and an endomorphism.
\end{definition}

\begin{definition}
    A category \(\mca{C}\) is small
    if \(\Ob(\mca{C})\) and \(\Mor(\mca{C})\) both form sets.
    A category is large otherwise.
\end{definition}

\begin{proposition}
    The category \(\mathbf{Set}\) consisting of all sets and set maps form a category.
\end{proposition}
\begin{proposition}
    The category \(\mathbf{Grp}\) consisting of all groups and group homomorphisms
    form a category.
\end{proposition}
\begin{proposition}
    The category \(\mathbf{Ab}\) consisting of all abelian groups
    and group homomorphisms between them form a category.
\end{proposition}
\begin{proposition}
    The category \(\mathbf{Ring}\) consisting of all rings and ring homomorphisms
    form a category.
\end{proposition}

\begin{theorem}
    Suppose \(\mca{C}\) is a category, and \(X \in \Ob(\mca{C})\).
    Then \(\Aut(X)\) forms a group.
\end{theorem}
\begin{proof}
    First off, if \(f,g \in \Aut(X)\),
    then \(f \circ g \in \Aut(X)\),
    since \(f \circ g \in \End(X)\) and composition of bijections are bijections.
    Associativity and identity is given for free.
    By definition, the inverse exists since they are isomorphisms.
\end{proof}

\begin{definition}
    A universal property is a property
    that characterizes an object up to isomorphism.
    This is often stated as given some objects and morphisms,
    there exists a (unique) morphism to some desired object that we want to define.
\end{definition}


\subsection{Duality}

\begin{remark}
    Since morphisms are directed,
    the notion of duality is to reverse such morphisms
    and see whether they make sense.
\end{remark}


\subsection{Products}

\subsubsection*{Product}

\subsubsection*{Coproduct}


\subsection{Limits}

\subsubsection*{Direct Limit}

\subsubsection*{Inverse Limit}


\subsection{Universal Algebra}


\subsection{Categorical Constructions}
