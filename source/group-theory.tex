\section{Groups}

\subsection{Basic Definitions}

\begin{definition}
    A group is a triple \((G,\cdot,1)\),
    where \(G \neq \emptyset\) is a set
    equipped with a multiplicative operation \(\cdot\)
    with an identity \(1\),
    with these four properties:
    \begin{enumerate}[label={(\roman*)}, itemsep=0mm]
        \item closure \(\func{\cdot}{G \cross G}{G}\);
        \item associativity \(\forall \{a,b,c\} \subset G, (ab)c = a(bc)\);
        \item identity \(\exists 1 \in G, \forall g \in G, 1g = g1 = g\); and
        \item inverse \(\forall g \in G, \exists h \in G, gh = hg = 1\),
            which we usually denote \(g^{-1} = h\).
    \end{enumerate}
\end{definition}
\begin{remark}
    If we think of groups as objects that we use
    to describe actions that preserve symmetry,
    these are exactly the four criteria that we require:
    any composition of two actions must also be an action (closure),
    you must be able to compose actions in any order (associativity),
    the act of doing nothing preserves symmetry (identity),
    and reversing an action also preserves symmetry (inverse).
\end{remark}

\begin{definition}
    A monoid is a triple \((X,\cdot,1)\),
    where \(X \neq \emptyset\) is a set
    equipped with a multiplicative operation \(\cdot\)
    with an identity \(1\),
    with the first three properties of a group,
    that is, closure, associativity, and identity.
\end{definition}
% \begin{definition}
%     A monad is a triple \((X,\cdot,1)\)
%     with the same hypotheses as a monoid,
%     but with only closure and identity.
% \end{definition}
% \begin{definition}
%     A semigroup is a double \((X,\cdot)\)
%     with the same hypotheses,
%     but only closure is guaranteed.
% \end{definition}

\begin{definition}
    Suppose \(G\) is a group.
    The order of an element \(g \in G\)
    is the smallest positive integer \(n\)
    such that \(g^n = 1\).
    If no such \(n\) exists,
    we say the order of \(g\) is infinite.
    This is often denoted \(\abs{g} = \ord{g} = \#g = n\).
\end{definition}
\begin{definition}
    The order of a group \(G\),
    also the cardinality of a group,
    is the number of elements in a group \(G\).
    If \(G\) has infinitely many elements,
    we say the order of \(G\) is infinite.
    This is often denoted \(\abs{G} = \ord{G} = \#G\).
\end{definition}

\begin{definition}
    A commutative group or an abelian group
    is a group \(G\) with a commutative operation \(\cdot\),
    that is, \(gh = hg\) for all \(\{g,h\} \subset G\).
\end{definition}
\begin{remark}
    We sometimes denote abelian groups with additive notation,
    where \((G,+,0)\) is a group.
\end{remark}


\subsection{Transformation Groups}

\begin{definition}
    Suppose we have a set \(X\).
    \(G\) is a group of transformations on \(X\)
    if all elements of \(G\) are bijective functions that map \(X \to X\),
    with multiplication being function composition.
\end{definition}
\begin{remark}
    It is worth remembering that \(X\) can be any set.
    Sometimes \(X\) indeed carries more structure,
    namely being a group \(X\),
    but remember that a group is but a set of elements
    with some additional rules.
\end{remark}

\begin{definition}
    The rotations and reflections on an \(n\)-gon
    trivially form a transformation group.
    These are transformations on \(n\) vertices that preserve adjacency.
    This is often denoted as \(D_n\),
    the dihedral group of an \(n\)-gon.
\end{definition}
\begin{proposition}
    \(D_n\) is a group of order \(\abs{D_n} = 2n\),
    with elements
    \begin{equation*}
        D_n = \{1,\tau,\tau^2,\hdots,\tau^{n-1},
        \sigma,\sigma\tau,\sigma\tau^2,\hdots,\sigma\tau^{n-1}\}
    \end{equation*}
    in which \(\tau^n = \sigma^2 = 1\)
    and \(\sigma\tau = \tau^{-1}\sigma\).
\end{proposition}
\begin{proof}
    We think of \(\tau\) as a rotation by \(2\pi/n\),
    and \(\sigma\) as a reflection.

    We prove that all such elements are distinct.
    Clearly by definition,
    all \(\tau^j\) are distinct when \(j \in [0,n-1]\).
    Moreover, all \(\sigma\tau^j\) are distinct when \(j \in [0,n-1]\)
    because if \(\sigma\tau^i = \sigma\tau^j\),
    then \(\sigma^2\tau^i = \sigma^2\tau^j\),
    which implies \(\tau^i = \tau^j\).
    Lastly, if \(\sigma\tau^i = \tau^j\),
    then \(\sigma\tau^i\tau^{-j} = \tau^j\tau^{-j} = 1\),
    which would imply there is \(\sigma = \tau^{j-i}\),
    a contradiction,
    as reflections are not rotations.

    With this, we also prove that all \(\sigma\tau^j\)
    are reflections.
\end{proof}

\begin{definition}
    A (group) homomorphism or a homomorphic function
    is \(\func{\phi}{G_1}{G_2}\),
    where \(G_1,G_2\) are groups,
    with the properties that \(\phi(gh) = \phi(g)\phi(h)\),
    and \(\phi(1)\) is the identity on \(G_2\).
    \(\phi\) provides a correspondence
    between the multiplication in \(G_1\) and \(G_2\).
    We call the set of all homomorphisms between \(G_1\) and \(G_2\)
    the hom-set \(\Hom(G_1,G_2)\).
\end{definition}
\begin{definition}
    Injective homomorphisms are called monomorphisms.
    Surjective homomorphisms are called epimorphisms.
    And most importantly,
    bijective homomorphisms are called isomorphisms.
    Two groups \(G,H\) that are isomorphic to each other
    are denoted \(G \cong H\).
\end{definition}
\begin{definition}
    Isomorphisms that map \(G \to G\) are called automorphisms;
    the set of all automorphisms of \(G\) is the automorphism group \(\Aut(G)\).
    The homomorphism that map a group \(G \to \{1\}\)
    by mapping all elements to \(1\)
    is called the trivial homomorphism.
\end{definition}

\begin{theorem}[Cayley's Theorem]\label{thm:cayley}
    Suppose \(G\) is a finite group.
    Then we can find a set \(X\)
    such that \(G\) can be represented
    as a transformation group on \(X\).
    More specifically,
    we can find a transformation group \(H\) on \(X\)
    such that \(G \cong H\).
\end{theorem}
\begin{proof}
    We consider \(X = G\),
    i.e.\ \(G\) potentially being isomorphic
    to a group that transforms all elements of itself.

    We first define a function \(\vfunc{\ell_a}{G}{G}{x}{ax}\),
    which we choose some \(a\)
    that multiply some element \(x \in G\) on the left.
    This is clearly a permutation on \(G\),
    i.e.\ it is a bijection \(G \to G\),
    because if \(\ell_a(x) = \ell_a(y)\) for some \(\{x,y\} \subset G\)
    hence \(ax = ay\), then \(a^{-1}ax = a^{-1}ay\), and \(x = y\),
    which proves that this is an injection;
    the pigeonhole principle proves that this is a surjection.

    For every element \(g \in G\),
    we can create a function \(\ell_g\).
    We let \(H = \{\ell_g : g \in G\}\)
    be the set of all such left permutations.
    We claim that \(H\) is a group,
    with function composition as multiplication,
    and \(\ell_1\) being the identity.
    We have closure because \(\ell_a\ell_b\)
    will map \(x \mapsto bx \mapsto abx\),
    which is equivalent to \(\ell_{ab}\).
    Function composition is naturally associative,
    so we get that for free.
    The inverse mapping from \(ax \to x\)
    is achieved by left multiplying by \(a^{-1}\),
    which itself is equivalent to \(\ell_{a^{-1}}\).
    Lastly, the identity works,
    because \(\ell_a\ell_1\) maps \(x \mapsto 1x = x \mapsto ax\)
    which is equivalent to \(\ell_a\),
    and similarly with \(\ell_1\ell_a = \ell_a\).

    We then want to prove that
    the mapping from \(a \to \ell_a\) is an isomorphism.
    Suppose \(\vfunc{\phi}{G}{H}{a}{\ell_a}\).
    This is a homomorphism because
    \begin{equation*}
        \phi(ab) = \ell_{ab} = \ell_a\ell_b = \phi(a)\phi(b)
    \end{equation*}
    And by construction, this is a bijection,
    because if \(\phi(a) = \phi(b)\),
    then \(\ell_a = \ell_b\),
    which implies \(ax = bx\) for all \(x \in G\),
    and hence \(axx^{-1} = bxx^{-1}\),
    so \(a = b\),
    which gives us injectivity;
    the pigeonhole principle again gives us surjectivity.

    We have now found a transformation group \(H\)
    that acts on a set \(G\),
    and that \(G\) is isomorphic to the transformation group \(H\).
\end{proof}
\begin{remark}
    One can repeat the entire proof with right multiplication instead,
    it is equally valid.
\end{remark}


\subsection{Cyclic Groups}

\begin{definition}
    Suppose \(G\) is a group.
    A subgroup \(H\) of \(G\) is a subset \(H \subseteq G\),
    with \(1 \in H\), and properties of a group hold.
    We call the group \(\{1\}\) the trivial subgroup.
\end{definition}

\begin{definition}
    Suppose we have a group \(G\),
    and some non-empty subset \(S \subseteq G\).
    The subgroup generated by \(S\)
    is the group of all elements that can be formed
    by the products of elements in \(S\) and their inverses.
    We denote it as
    \(\langle S \rangle = \{\prod_i s_i : s_i \in S \lor s_i^{-1} \in S\}\).
\end{definition}
\begin{remark}
    One can think of this as almost like how a vector space
    spanned by a set of basis vectors.
\end{remark}

\begin{lemma}\label{lem:intersection-subgroup}
    Suppose \(G\) is a group,
    and \(H_i \subseteq G\) subgroups.
    Then \(\bigcap_i H_i\) is also a subgroup of \(G\).
\end{lemma}
\begin{proof}
    If \(\{a,b\} \subset \bigcap_i H_i\),
    then \(\{a,b\} \subset H_i\) for all \(i\),
    and hence \(ab \in H_i\),
    which gives us closure \(ab \in \bigcap_i H_i\).

    Associativity is inherited from \(G\).

    The identity must be in each of \(H_i\),
    so it must also be in \(\bigcap_i H_i\).

    Lastly, if \(a \in \bigcap_i H_i\),
    then \(a \in H_i\) for all \(i\),
    so \(a^{-1} \in H_i\) for all \(i\),
    so we have inverses as \(a^{-1} \in \bigcap_i H_i\).
\end{proof}

\begin{proposition}\label{prop:subset-generated-subgroup}
    \(\langle S \rangle = \bigcap_i H_i\),
    where \(H_i\) is any subgroup of \(G\) containing \(S\).
\end{proposition}
\begin{proof}
    We first need to prove that
    \(\bigcap_i H_i \subseteq \langle S \rangle\),
    which it is sufficient to prove that \(\langle S \rangle\)
    is any subgroup of \(G\) containing \(S\),
    i.e.\ \(\langle S \rangle\) corresponds to some \(H_i\),
    since the intersection must be contained in any \(H_i\).
    It is trivial to see that \(1 \in \langle S \rangle\)
    by choosing \(s_2 = s_1^{-1}\), which gives us \(1\).
    Closure and associativity is inherited from \(G\),
    so we need not prove anything.
    Inverses exist since it is easy to check that
    \({(s_1 s_2 \hdots s_n)}^{-1} = s_n^{-1} \hdots s_2^{-1} s_1^{-1}\).
    Hence \(\langle S \rangle \subseteq G\) is a subgroup,
    which must contain \(S\) by construction.

    We then see that \(H_i \subseteq G\),
    so every element of \(\langle S \rangle\) is in \(H_i\),
    as it is closed under multiplication and inversion.
    Hence we have \(\langle S \rangle \subseteq \bigcap_i H_i\),
    and subsequently \(\langle S \rangle = \bigcap_i H_i\).
\end{proof}

\begin{definition}
    Suppose we have a group \(G\),
    and some element \(g \in G\).
    The subgroup generated by one element,
    otherwise known as the cyclic group generated by \(g\),
    is denoted \(\langle g \rangle = \{g^n : n \in \bZ\}\).
\end{definition}
\begin{proposition}
    Cyclic groups are abelian.
\end{proposition}
\begin{proof}
    \(g\) will always commute with itself.
\end{proof}

\begin{theorem}
    Cyclic groups of the same order (finite or infinite) are isomorphic.
\end{theorem}
\begin{proof}
    Suppose we have a cyclic group \(\langle g \rangle\).
    We first attempt to construct a homomorphism
    \(\vfunc{\phi}{(\bZ,+,0)}{\langle g \rangle}{n}{g^n}\).
    A quick proof shows us that it is indeed a homomorphism,
    with \(\phi(0) = g^0 = 1\), and
    \begin{equation*}
        \phi(m+n) = g^{m+n} = g^m g^n = \phi(m)\phi(n)
    \end{equation*}
    We also see that this is surjective by construction,
    but not necessarily injective,
    if the order of \(g\) is finite.

    We first consider the case of infinite order of \(g\).
    Then it is clear that no two integers will map to the same element,
    and hence \(\phi\) is injective, and therefore and isomorphism.
    This proves the part where we claim that
    all infinite cyclic groups are isomorphic,
    and specifically with \((\bZ,+,0)\).

    We now consider some \(g\) with order \(\abs{g} = d\).
    Then clearly \(\phi(d) = \phi(0) = g^d = 1\),
    which shows us that \(\phi\) is not injective.
    We can however show that \(\{1,g,g^2,\hdots,g^{d-1}\}\)
    are all distinct,
    which proves that \((\bZ/d\bZ,+,0) \cong \langle g \rangle\).
    We proceed by contradiction,
    assuming that there exists some \(0 \leq i < j < d\)
    such that \(g^i = g^j\).
    But then we know \(g^{j-i} = g^j g^{-i} = g^i g^{-i} = 1\),
    which tells us there exists some number \(j-i < d\)
    such that \(g^{j-i} = 1\),
    contradicting that \(\abs{g} = d\).
    Hence all cyclic groups with order \(d\)
    are isomorphic to \(\bZ/d\bZ,+,0\),
    which in turn are isomorphic to each other.
\end{proof}

\begin{definition}
    \(C_n\) is the cyclic group with order \(\abs{C_n} = n\).
    It is also sometimes denoted \(C_n = Z_n = \bZ/n\bZ = \bZ/(n) = \bZ/n\).
\end{definition}

\begin{theorem}\label{thm:cyclic-subgroup}
    Any subgroup of a cyclic group is cyclic.
    In particular, if \(H \subseteq G\) a finite cyclic group,
    \(\abs{H} \mid \abs{G}\).
\end{theorem}
\begin{proof}
    Suppose \(G = \langle g \rangle\) is generated by this element.
    Then all elements of \(H\) can be written as some power of \(g\).
    We pick the smallest such integer \(s\) such that \(g^s \in H\).
    We claim that \(H = \langle g^s \rangle\).
    No matter what element we pick in \(H\),
    we can write it as \(g^{s'}\).
    Via long division, we write \(s' = qs + r\), with \(0 \leq r < s\),
    so we know our element is
    \(g^{s'} = g^{qs+r} = {(g^s)}^q g^r\).
    Since \(g^s \in H\), by closure we have \({(g^s)}^q \in H\);
    which gives us \(g^r = {\qty({(g^s)}^q)}^{-1} g^{s'} \in H\).
    But if \(r \neq 0\),
    we will have found a smaller integer than \(s\)
    such that \(g^r \in H\),
    which itself is a contradiction.
    We are then forced to conclude that \(r = 0\),
    so all elements in \(H\) must be written as
    \(g^{s'} = g^{qs} = {(g^s)}^q \in \langle g^s \rangle\),
    and hence \(H \subseteq \langle g^s \rangle\).
    But obviously, \(g^s \in H\), so \({(g^s)}^q \in H\),
    and we have \(\langle g^s \rangle \subseteq H\).
    We then have our claim of equality \(H = \langle g^s \rangle\),
    which is generated by a single element.

    For the second part of the theorem,
    it is sufficient to prove that \(\abs{g^s} \mid \abs{g}\).
    Suppose \(\abs{g^s} = d\) and \(\abs{g} = n\).
    Then we know that \(g^{sd} = 1\), which we apply long division on.
    \(sd = qn + r\), with \(0 \leq r < n\),
    so \(g^{sd} = g^{qn}g^r = {(g^n)}^q g^r = g^r = 1\),
    which gives us \(r = 0\),
    because \(r\) is smaller than the order of \(g\).
    This tells us that \(sd \mid n\).
    Now, we know that \(d\) is the smallest number
    that we need to multiply \(s\) by to obtain a multiple of \(n\),
    which tells us \(d\) is the product of
    all the factors that \(n\) has,
    but \(s\) has not (counted with multiplicity).
    A simple decomposition of \(n = \gcd(s,n) \cdot n/\gcd(s,n)\)
    tells us that the gcd must be the factors that \(n\) and \(s\) share,
    so the other part must be the ones \(n\) has, but \(s\) has not,
    meaning \(d = n/\gcd(s,n)\).
    But by definition, these are factors of \(n\),
    so \(d \mid n\).
\end{proof}
\begin{corollary}
    \(\abs{g^s} = \abs{g}/\gcd(s,\abs{g})\).
\end{corollary}
\begin{remark}
    The above theorem gives us the ability to
    find the order of a given subgroup of a cyclic group.
    But to do the reverse, given an order finding a subgroup,
    each order might correspond to multiple generators,
    which means we need to prove the next theorem.
\end{remark}

\begin{theorem}\label{thm:cyclic-subgroup-uniqueness}
    Suppose \(G = \langle g \rangle\), with \(\abs{g} = n\).
    Given that \(d \mid n\), the subgroup of order \(d\) is unique.
\end{theorem}
\begin{proof}
    Let \(s = n/d\).
    We first see that \(H = \langle g^s \rangle\)
    is one such subgroup of order \(d\),
    since the corollary gives us
    \(\abs{H} = n/\gcd(n/d,n) = n/\gcd(n/d) = d\).

    Now suppose we have another subgroup \(H' \subseteq G\)
    with order \(d\).
    Then from the \hyperref[thm:cyclic-subgroup]{above theorem},
    we know that \(H' = \langle g^{s'} \rangle\) is cyclic.
    By the corollary, we get
    \begin{equation*}
        d = \frac{n}{\gcd(n,s')} \implies s = \frac{n}{d} = \gcd(n,s')
    \end{equation*}
    with \(s \mid s'\) by definition of gcd.
    Hence \(s'\) is a multiple of \(s\),
    so \(g^{s'} \in H\), and therefore \(H' \subseteq H\).
    But as the orders are the same,
    the pigeonhole principle gives us \(H' = H\).
\end{proof}

\begin{theorem}
    Suppose \(g\) and \(h\) commute (\(gh = hg\)),
    with \(m = \abs{g}\), \(n = \abs{h}\),
    \(m,n\) coprime (\(\gcd(m,n) = 1\)).
    Then \(\abs{gh} = mn = \abs{g}\abs{h}\).
\end{theorem}
\begin{proof}
    Let \(d = \abs{gh}\).
    We first notice that
    \({(gh)}^{mn} = g^{mn}h^{mn} = {(g^m)}^n{(h^n)}^m = 1\),
    so \(d \mid mn\).

    Now we claim that \(mn\) is minimal.
    Suppose we have a \(0 < d < mn\) such that \({(gh)}^d = g^d h^d = 1\),
    so we know that \(g^d = h^{-d}\).
    Since their orders are coprime,
    unless one of the groups is \(\{1\}\),
    they are not subgroups of each other,
    so we have \(1 = g^d = h^{-d}\),
    and \(m \mid d\) and \(n \mid d\).
    But then \(d\) must be some multiple of \(\lcm(m,n) = mn\),
    which is impossible within the range \(0 < d < mn\).
    Hence \(d = mn\).
\end{proof}
\begin{remark}
    In general, \(g\) and \(h\) will not commute,
    so it is impossble to relate \(\abs{gh}\) to \(\abs{g}\abs{h}\),
    since you would have to understand the groups
    \(\langle g \rangle\) and \(\langle h \rangle\) together.
\end{remark}


\subsection{Permutations}

\begin{definition}
    Let \(X = \bN_n = \{1,2,3,\hdots,n\}\)
    a set (any set) of \(n\) elements.
    We call a bijective function \(\func{\sigma}{X}{X}\) a permutation.
\end{definition}
\begin{proposition}\label{prop:symmetric-group}
    The set of all possible permutations,
    with function composition as multiplication
    and the identity mapping as the identity,
    forms a group of order \(n!\).
\end{proposition}
\begin{proof}
    It is easy to see that the composition of two bijections
    is also a bijection,
    which gives us closure.
    We inherit associativity from function composition,
    and the identity mapping works as intended
    on both the left and the right side.
    Lastly, the inverse of a bijection
    must also be a bijection.

    To count all the ways we can construct such permutations,
    we see that for any given \(\sigma\),
    there are \(n\) possible elements that we can map \(1\) to,
    and after that choice is taken,
    \(n-1\) possible elements that we can map \(2\) to,
    and so forth until there is only one choice
    for what we can map \(n\) to.
    This gives us a total of \(n(n-1)\hdots(2)(1) = n!\) choices.
\end{proof}
\begin{definition}
    It is clear that the groups of permutations
    of any two sets of \(n\) items
    are isomorphic to each other.
    We call this group \(S_n\),
    the symmetric group of \(n\) elements.
\end{definition}

\begin{definition}
    We can write elements of \(S_n\) as disjoint cycles
    \((a_1 a_2 \hdots a_p) \hdots (b_1 b_2 \hdots b_q)\)
    where this permutation will map \(a_1 \mapsto a_2\),
    \(a_2 \mapsto a_3\), and so on until \(a_p \to a_1\),
    repeating this process for every cycle.
    We will often omit cycles of length 1.
\end{definition}
\begin{remark}
    It is worth noting that disjoint cycles commute.
\end{remark}

\begin{definition}
    We call \(\tau = (ab)\) a cycle of two elements
    a flip or a transposition.
\end{definition}
\begin{lemma}[Breaking the cycle]\label{lem:breaking-cycles-sn}
    Suppose \(\rho = (a c_1 \hdots c_m b d_1 \hdots d_n)\)
    is a cycle of length \(m+n+2\).
    Then \(\rho = (ab)(a c_1 \hdots c_m)(b d_1 \hdots d_n)\).
\end{lemma}
\begin{proof}
    This is equivalent to proving that
    \((ab)\rho = (a c_1 \hdots c_m)(b d_1 \hdots d_n)\).
    We can see that \(\rho\) maps the following way,
    so \((ab)\rho\) must simply swap all occurrences of \(a\) with \(b\)
    in the results, and vice versa.
    \begin{equation*}
        \rho = \begin{cases}
            a \mapsto c_1 \\
            c_i \mapsto c_{i+1} & i < m \\
            c_m \mapsto b \\
            b \mapsto d_1 \\
            d_j \mapsto d_{j+1} & j < n \\
            d_n \mapsto a
        \end{cases} \qquad
        (ab)\rho = \begin{cases}
            a \mapsto c_1 \\
            c_i \mapsto c_{i+1} & i < m \\
            c_m \mapsto a \\
            b \mapsto d_1 \\
            d_j \mapsto d_{j+1} & j < n \\
            d_n \mapsto b
        \end{cases}
    \end{equation*}
\end{proof}
\begin{theorem}\label{thm:sn-product-transpositions}
    Any \(\sigma \in S_n\) is a non-unique product of transpositions.
\end{theorem}
\begin{proof}
    We write \(\sigma = \gamma_1\hdots\gamma_n\) as disjoint cycles.
    We now decompose each of \(\gamma_i\).
    First suppose \(\gamma_i\) is a transposition already.
    Then there is no need to do anything.

    Now suppose \(\gamma_i = (abc)\) is of length 3.
    Then by our \hyperref[lem:breaking-cycles-sn]{lemma above},
    \(\gamma_i = (ac)(ab)(c) = (ac)(ab)\)
    and we have a product of transpositions.

    Then suppose \(\gamma_i = (abcd)\) is of length 4.
    By our \hyperref[lem:breaking-cycles-sn]{lemma above},
    \(\gamma_i = (ac)(ab)(cd)\)
    and we have a product of transpositions.

    Applying the induction step,
    assume that \(\gamma_i = (a b c_1 \hdots c_n)\) is a cycle of \(n+2\),
    and the case of cycles of length \(n\) is already proven.
    Then we know by our lemma \(\gamma_i = (ac_1)(ab)(c_1 \hdots c_n)\),
    so we have two transpositions and a cycle of length \(n\),
    which by the inductive hypothesis can be decomposed into transpositions.

    This representation is not unique,
    since if \(\sigma = \tau_1\hdots\tau_n\),
    and \(\tau'\) is another transposition,
    we can write \(\sigma = \tau'{\tau'}^{-1}\tau_1\hdots\tau_n
    = \tau'\tau'\tau_1\hdots\tau_n\)
    a different representation.
\end{proof}

\begin{proposition}\label{prop:sn-conjugation}
    Suppose \(\sigma \in S_n\).
    Then \(\sigma(n_1 n_2 \hdots n_d)\sigma^{-1}
    = (\sigma(n_1) \sigma(n_2) \hdots \sigma(n_d))\).
\end{proposition}
\begin{proof}
    Since all \(\sigma\) can be written a a product of transpositions
    (Theorem~\ref{thm:sn-product-transpositions}),
    it is sufficient to consider \(\tau(n_1 n_2 \hdots n_d)\tau^{-1}\).
    Suppose we have \((ab)\rho(ab)\),
    and \(\rho = (ac_1 \hdots c_m bd_1 \hdots d_n)\)
    Then we have
    \begin{equation*}
        (ab) = \begin{cases}
            a \mapsto b \\
            c_i \mapsto c_i \\
            b \mapsto a \\
            d_j \mapsto d_j
        \end{cases} \qquad
        \rho(ab) = \begin{cases}
            a \mapsto d_1 \\
            c_i \mapsto c_{i+1} & i < m \\
            c_m \mapsto b
            b \mapsto c_1 \\
            d_j \mapsto d_{j+1} & j < n\\
            d_n \mapsto a
        \end{cases}
    \end{equation*}
    which in cycle notation is
    \(\rho(ab) = (ad_1 \hdots d_n)(bc_1 \hdots c_m)\),
    and if we \hyperref[lem:breaking-cycles-sn]{break the cycle}
    we get \((ab)\rho(ab) = (ad_1 \hdots d_n bc_1 \hdots c_m)\)
    which is exactly as desired, swapping the positions of \(a\) and \(b\).
\end{proof}

\begin{definition}
    Suppose \(\sigma = \gamma_1\gamma_2\hdots\gamma_n\)
    is a permutation written in disjoint cycle notation,
    with \(\gamma_i\) having length \(d_i\).
    Then we define the sign of the permutation as
    \begin{equation*}
        \sgn(\sigma) = {(-1)}^{\sum_{i=1}^n (d_i-1)}
        = \prod_{i=1}^n {(-1)}^{(d_i-1)}
    \end{equation*}
    If \(\sgn(\sigma) = 1\), \(\sigma\) is called even;
    conversely, if \(\sgn(\sigma) = -1\), \(\sigma\) is called odd.
\end{definition}
\begin{remark}
    It is worth noting that cycles of odd length are even,
    and cycles of even length are odd.
    Then cycles of length 1 (identity) are even,
    so it does not change sign.
\end{remark}

\pagebreak

\begin{proposition}
    Suppose \(\tau = (ab)\) is a transposition, and \(\sigma \in S_n\).
    Then \(\sgn(\tau\sigma) = -\sgn(\sigma)\).
\end{proposition}
\begin{proof}
    We write \(\sigma = \gamma_1\hdots\gamma_r\) as disjoint cycles,
    inclding 1-cycles.
    We first consider the case that \(a,b\) are in the same cycle,
    so without loss of generality let \(a,b\) be in \(\gamma_1\),
    which has length \(m+n+2\).
    Then we have, by the lemma
    \begin{gather*}
        \tau\sigma
        = (ab)(ac_1 \hdots c_m bd_1 \hdots d_n)\gamma_2\hdots\gamma_r
        = (ac_1 \hdots c_m)(bd_1 \hdots d_n)\gamma_2\hdots\gamma_r \\
        \sgn(\tau\sigma)
        = {(-1)}^m{(-1)}^n \prod_{i=2}^r {(-1)}^{(d_i-1)}
        = {(-1)}^{m+n} \prod_{i=2}^r {(-1)}^{(d_i-1)}
        = -{(-1)}^{m+n+1} \prod_{i=2}^r {(-1)}^{(d_i-1)}
        = -\sgn(\sigma)
    \end{gather*}

    Now consider the case that \(a,b\) are in different cycles,
    so without loss of generality
    let \(a\) be in \(\gamma_1\) of length \(m+1\),
    and \(b\) be in \(\gamma_2\) of length \(n+1\).
    By the lemma again
    \begin{gather*}
        \tau\sigma
        = (ab)(ac_1 \hdots c_m)(bd_1 \hdots d_n)\gamma_3\hdots\gamma_r
        = (ac_1 \hdots c_m bd_1 \hdots d_n)\gamma_3\hdots\gamma_r \\
        \sgn(\tau\sigma)
        = {(-1)}^{m+n+1} \prod_{i=3}^r {(-1)}^{(d_i-1)}
        = -{(-1)}^{m+n} \prod_{i=3}^r {(-1)}^{(d_i-1)}
        = -{(-1)}^m{(-1)}^n \prod_{i=2}^r {(-1)}^{(d_i-1)}
        = -\sgn(\sigma)
    \end{gather*}
\end{proof}
\begin{corollary}\label{cor:sgn-transposition}
    Suppose \(\sigma = \tau_1\tau_2\cdots\tau_k\)
    written as a product of transpositions.
    Then \(\sgn(\sigma) = {(-1)}^k\).
\end{corollary}
\begin{proof}
    \(\sgn(\sigma) = \sgn(\tau_1\tau_2\cdots\tau_k)
    = (-1)\sgn(\tau_2\cdots\tau_k)
    = \cdots = {(-1)}^{k-1}\sgn(\tau_k)
    = {(-1)}^k\)
\end{proof}
\begin{remark}
    Since the sign of a permutation is either odd or even,
    this tells us that the number of transpositions might not be unique,
    but whether there are an odd or even number of them
    is inherent to each permutation.
\end{remark}

\begin{theorem}\label{thm:sgn-mult}
    \(\sgn(\sigma_1\sigma_2) = \sgn(\sigma_1)\sgn(\sigma_2)\).
\end{theorem}
\begin{proof}
    Suppose \(\sigma_1 = \tau_1\hdots\tau_m\),
    and \(\sigma_2 = \tau'_1\hdots\tau'_n\).
    Then by the \hyperref[cor:sgn-transposition]{corollary above},
    \begin{align*}
        \sgn(\sigma_1\sigma_2)
        &= \sgn(\tau_1\hdots\tau_m\tau'_1\hdots\tau'_n)
        = {(-1)}^{m+n}
        = {(-1)}^m{(-1)}^n \\
        &= \sgn(\tau_1\hdots\tau_m)\sgn(\tau'_1\hdots\tau'_m)
        = \sgn(\sigma_1)\sgn(\sigma_2)
    \end{align*}
\end{proof}
\begin{corollary}
    The even permutations form a subgroup of \(S_n\).
\end{corollary}
\begin{proof}
    Suppose \(\sigma_1,\sigma_2\) are both even,
    Then since \(\sgn(\sigma_1) = \sgn(\sigma_2) = 1\),
    we have \(\sgn(\sigma_1\sigma_2) = \sgn(\sigma_1)\sgn(\sigma_2) = 1\).
    We have closure.

    The identity is also even,
    because \(\sgn(\sigma) = \sgn(1\sigma) = \sgn(1)\sgn(\sigma)\),
    which gives us \(\sgn(1) = 1\), an even permutation.

    Lastly, if \(\sigma\) is even,
    then \(\sigma^{-1}\) is too,
    because if \(\sigma = 1\),
    then \(1 = \sgn(1) = \sgn(\sigma\sigma^{-1})
    = \sgn(\sigma)\sgn(\sigma^{-1}) = \sgn(\sigma^{-1})\).
\end{proof}
\begin{definition}
    We call the subgroup of even permutations of \(S_n\)
    the alternating group of \(n\) elements,
    usually denoted as \(A_n\).
\end{definition}


\subsection{Cosets}\label{sec:cosets}

\begin{proposition}
    Suppose \(G\) is a transformation group of some set \(X\).
    For any \(\{x,y\} \subset X\),
    if we define a relation \(x \sim y\) to be
    when there exists some \(g \in G\)
    such that \(g(x) = y\),
    such a relation is an equivalence relation.
\end{proposition}
\begin{proof}
    This is reflexive because \(1 \in G\),
    and \(1(x) = x\), so \(x \sim x\).
    This is symmetric because suppose \(x \sim y\),
    then there exists some \(g \in G\) such that \(g(x) = y\);
    it is easy to see that then \(g^{-1}(y) = x\),
    so we have \(y \sim x\).
    This is transitive because if \(x \sim y\) and \(y \sim z\),
    then there exists \(g(x) = y\) and \(h(y) = z\),
    and since \(hg \in G\), \((hg)(x) = h(y) = z\),
    so \(x \sim z\).
\end{proof}
\begin{definition}
    We call the equivalence class of \(x \in X\) the orbit of \(x\),
    which is often denoted as \(\orb(x) = Gx = \{g(x) : g \in G\}\).
    If this is the entire set,
    i.e.\ there exists some \(x \in X\)
    (and therefore all \(x \in X\) by Theorem~\ref{thm:equiv-class-partition})
    such that \(Gx = X\),
    we call \(G\) a transitive transformation group.
\end{definition}

\begin{proposition}
    Suppose \(G\) is a group, and \(H \subseteq G\) some subgroup.
    For any \(\{g_1,g_2\} \subset G\),
    if we define a relation \(g_1 \sim g_2\) to be
    when there exists some \(h \in H\) such that \(g_2 = g_1h\),
    such a relation is an equivalence relation.
\end{proposition}
\begin{proof}
    This is reflexive because \(1 \in H\),
    and \(g_1 = g_1 1\), so \(g_1 \sim g_1\).
    This is symmetric because suppose \(g_1 \sim g_2\),
    then there exists \(h \in H\) such that \(g_2 = g_1 h\);
    it is easy to see that \(g_1 = g_2 h^{-1}\), so we have \(g_2 \sim g_1\).
    This is transitive because if \(g_1 \sim g_2\) and \(g_2 \sim g_3\),
    then there exists \(h_1\) and \(h_2\) such that
    \(g_2 = g_1 h_1\) and \(g_3 = g_2 h_2\),
    so \(g_3 = g_1 h_1 h_2\),
    and hence \(g_1 \sim g_3\).
\end{proof}
\begin{definition}
    We call the equivalence class of \(g \in G\) a left coset of \(H\) in \(G\),
    which is denoted \(gH = \{gh : h \in H\}\).
    We can similarly define a right coset of \(H\) in \(G\),
    which is denoted \(Hg = \{hg : h \in H\}\).
\end{definition}
\begin{definition}
    The set of all left cosets is denoted \(G/H\),
    while the set of all right cosets is denoted \(H \backslash G\).
\end{definition}
\begin{definition}
    The number of left cosets is called the index of \(H\) in \(G\)
    and is denoted \(\abs{G/H} = [G:H]\).
\end{definition}

\begin{lemma}\label{lem:order-coset}
    Suppose \(G\) a group, and \(H \subseteq G\) some subgroup.
    Then for any \(g \in G\),
    the cosets are the same size,
    and in particular \(\abs{gH} = \abs{H}\).
\end{lemma}
\begin{proof}
    We write the left multiplication function
    \(\vfunc{\ell_g}{H}{gH}{h}{gh}\).
    We can show that it is injective,
    because if \(gh_1 = gh_2\),
    then \(g^{-1}gh_1 = g^{-1}gh_2\),
    so \(h_1 = h_2\).
    We can also show that it is surjective,
    because for every element \(gh \in gH\),
    clearly \(h \mapsto gh\) and \(h \in H\) by definition.
    Hence \(\ell_g\) is a bijection,
    which shows that \(\abs{H} = \abs{gH}\).
\end{proof}
\begin{theorem}[Lagrange's Theorem]\label{thm:lagrange}
    For some finite group \(G\), and \(H \subseteq G\) any subgroup,
    \(\abs{G} = \abs{H}[G:H]\).
\end{theorem}
\begin{proof}
    By definition, there are a total of \([G:H]\) cosets,
    and the subgroup itself is a coset \(H = 1H\).
    Since cosets are equivalence classes,
    they partition \(G\),
    so the order of \(G\) must be the sum of the orders of the cosets.
    But we also know that all the cosets are the of size \(\abs{H}\)
    from the \hyperref[lem:order-coset]{lemma above},
    so we have \(\abs{G} = \abs{H}[G:H]\).
\end{proof}
\begin{corollary}\label{cor:prime-order-subgroup}
    Suppose \(\abs{G} = p\) some prime order.
    Then if \(H \subseteq G\) is a subgroup,
    either \(H = \{1\}\) or \(H = G\),
    and in the second case,
    \(H\) is generated by any element \(g \in G\) when \(g \neq 1\).
\end{corollary}
\begin{proof}
    Since from the \hyperref[thm:lagrange]{theorem above}
    we have \(\abs{H}\mid\abs{G}\),
    \(\abs{H}\) is either 1 or \(p\).
    If \(\abs{H} = 1\), since all groups must have the identity,
    \(H = \{1\}\).

    On the other hand, if \(\abs{H} = p\),
    then it must be the whole group, so \(H = G\).
    Now pick any \(g \in G\).
    We know that \(\langle g \rangle \subseteq G\) is a subgroup,
    and if \(g \neq 1\),
    we cannot have \(\langle g \rangle = \{1\}\),
    so we are forced to conclude otherwise,
    and we have \(\langle g \rangle = G\).
\end{proof}
\begin{corollary}\label{cor:order-element-group}
    Suppose \(G\) is a finite group, with \(g \in G\).
    If \(n = \abs{G}\), then \(g^n = 1\).
\end{corollary}
\begin{proof}
    Suppose \(\abs{g} = m\), so \(g^m = 1\).
    By the \hyperref[thm:lagrange]{theorem above},
    we know that \(m \mid n\),
    so there exists some \(r \in \bN\) such that \(n = mr\).
    Hence \(g^n = g^{mr} = {(g^m)}^r = 1^r = 1\).
\end{proof}
\begin{corollary}
    \(\abs{A_n} = n!/2\).
\end{corollary}
\begin{proof}
    The even and odd permutations form two cosets in \(S_n\),
    since multiplying by even permutations
    (and hence by elements of \(A_n\)) do not change sign
    by Theorem~\ref{thm:sgn-mult}.
    Then we have \(\abs{S_n} = \abs{A_n}[S_n:A_n]\),
    so \(\abs{A_n} = \abs{S_n}/2 = n!/2\).
\end{proof}


\subsection{Normal Subgroups}

\begin{definition}
    Suppose \(G\) is a group, and \(H \subseteq G\) some subgroup.
    \(H\) is a normal subgroup of \(G\)
    if for all \(g \in G\) and \(h \in H\),
    \(ghg^{-1} \in H\).
    We often denote this as \(H \lhd G\).
\end{definition}
\begin{theorem}
    If \(H \lhd G\),
    then the set of left cosets \(G/H\) forms a group,
    and in particular, there exists a epimorphism
    from \(G\) to \(G/H\).
\end{theorem}
\begin{proof}
    We attempt to write a function \(\vfunc{\phi}{G}{G/H}{g}{gH}\)
    that maps every element into its coset.
    By definition, this is a surjective mapping,
    since every coset must have some element,
    and those elements must be in \(G\).

    We also see that every coset must be represented by some element,
    so in the following, let \(g'_i \in G\) represent the coset \(g_i H\);
    by definition of a coset we have \(g_i = g'_i h_i\) for some \(h_i \in H\).
    A simple proof of the homomorphism shows that
    \(\phi(g_1g_2) = (g_1 g_2)H\) which is the coset of \(g_1 g_2\),
    and \(\phi(g_1)\phi(g_2) = (g_1 H)(g_2 H)\)
    which is the coset of the representatives \(g'_1 g'_2\).
    For them to be in the same coset,
    we see \(g_1 g_2 = g'_1 h_1 g'_2 h_2
    = g'_1 g'_2 ({(g'_2)}^{-1} h_1 g'_2) h_2\)
    needs to be written in the form \(gH\),
    so we want \({(g'_2)}^{-1} h_1 g'_2 \in H\)
    for every possible \(g'_2 \in G\) and \(h_1 \in H\).
    But this is exactly the condition that normal subgroups provide,
    and hence there exists an epimorphism from \(G\) to \(G/H\),
    which shows that \(G/H\) forms a group,
    inheriting the identity as the coset of \(1 \in G\),
    and the multiplication itself from \(G\).
\end{proof}
\begin{definition}
    Suppose \(G\) some group, and \(H \lhd G\).
    We call \(G/H\) the quotient group of \(G\) by \(H\).
\end{definition}

\begin{proposition}\label{prop:abelian-subgroup-normal}
    Subgroups of abelian groups must be normal.
\end{proposition}
\begin{proof}
    Suppose \(G\) is our abelian group,
    and \(H \subseteq G\) our subgroup.
    Then for all \(g \in G\) and \(h \in H\),
    we realize that both \(g,h\) are elements of \(G\),
    and hence will commute,
    so we have \(ghg^{-1} = gg^{-1}h = h \in H\),
    and hence \(H \lhd G\).
\end{proof}

\begin{theorem}\label{thm:equal-coset-normal}
    \(H \lhd G\) is a normal subgroup if and only if
    the left and right cosets are identical, that is,
    \(gH = Hg\) for all \(g \in G\).
\end{theorem}
\begin{proof}
    In the forward direction,
    assuming a normal subgroup,
    by definition we have for all \(g \in G\) and \(h \in H\),,
    \(ghg^{-1} \in H\),
    which tells us that there exists some \(h' \in H\)
    such that \(ghg^{-1} = h'\).
    When we right-multiply by \(g\) on both sides,
    we get \(gh = h'g\).
    This tells us that for all elements \(g \in G\) and \(h \in H\)
    \(gh \in Hg\), so we get \(gH \subseteq Hg\).
    Similarly, we can left-multiply by \(g^{-1}\) on both sides,
    and get \(hg^{-1} = g^{-1}h'\),
    which tells us for all \(g \in G\) and \(h \in H\),
    \(hg \in gH\), so we get \(Hg \subseteq gH\).
    Combining these two statements, we get \(gH = Hg\).

    In the reverse direction,
    assuming that left and right cosets are equal,
    we reverse the argument to see that
    for all \(g \in G\) and \(h \in H\),
    \(gh \in Hg\), so there exists some \(h' \in H\)
    such that \(gh = h'g\), which implies \(ghg^{-1} = h' \in H\).
\end{proof}

\begin{remark}
    It is good to remind ourselves that
    \(gH \subseteq G\), cosets are subsets of the whole group;
    but \(gH \in G/H\), these are now elements of the quotient group.
\end{remark}


\subsection{Homomorphisms and Isomorphisms}

% \begin{proposition}
%     Suppose \(\func{\phi}{G_1}{G_2}\) is a homomorphism between groups,
%     not necessarily injective nor surjective.
%     The image of \(G_1\) under \(\phi\) is a subgroup of \(G_2\),
%     that is, \(\phi(G_1) \subseteq G_2\) forms a subgroup.
% \end{proposition}

% \begin{lemma}
%     Suppose we have a homomorphism \(\func{\phi}{G_1}{G_2}\).
%     The preimage of the identity is a normal subgroup,
%     that is, \(\phi^{-1}(1_{G_2}) \lhd G_1\).
% \end{lemma}

\begin{definition}
    Suppose we have a group homomorphism \(\func{\phi}{G}{H}\).
    We call the preimage of the identity \(\phi^{-1}(1)\)
    the kernel of \(\phi\),
    sometimes denoted \(\ker(\phi) = \{g \in G : \phi(g) = 1\}\).
\end{definition}

\begin{theorem}[Universal Property of Quotient Groups]\label{thm:univ-prop-quotient-group}
    Let \(G,H\) be groups,
    and \(N \lhd G\) be a normal subgroup.
    Suppose \(\func{\pi}{G}{G/N}\) is the quotient homomorphism
    and \(\func{\phi}{G}{H}\) is any group homomorphism
    with \(N \subseteq \ker(\phi)\).
    Then there exists a unique group homomorphism
    \(\func{\bar{\phi}}{G/N}{H}\) such that \(\phi = \bar{\phi}\circ\pi\).

    This is represented by the following commutative diagram:
    \begin{center}
        \begin{tikzcd}
            G \arrow{r}{\phi} \arrow{d}{\pi} & H \\
            G/N \arrow{ru}[swap]{\exists! \bar{\phi}}
        \end{tikzcd}
    \end{center}
\end{theorem}
\begin{proof}
    This universal property asserts that
    both the existence and and the uniqueness of \(\bar{\phi}\),
    which needs to be proven separately.
    We will write our proof in reverse order,
    first proving the uniqueness assuming existence,
    and then proving existence without any assumptions.

    Suppose \(\phi = \bar{\phi}\circ\pi = \bar{\phi}'\circ\pi\).
    Since \(N \subseteq \ker(\phi)\),
    all elements \(n \in N\) obey \(\phi(n) = 1\).
    Knowing that all elements in \(G\) belong to some coset \(gN\),
    all \(g' \in gN\) maps to
    \(\phi(g') = \phi(gn) = \phi(g)\phi(n) = \phi(g)\),
    which tells us the image of each coset
    is a single element \(\phi(gN) = \{\phi(g)\}\).
    Now, seeing that \(\pi\) maps \(g' \mapsto gN\),
    its coset, by definition of a quotient mapping
    if \(\bar{\phi} \neq \bar{\phi}'\),
    there must be one such coset \(gN\)
    that \(\bar{\phi}(gN) \neq \bar{\phi}'(gN)\) disagrees on.
    However, this is a contradiction,
    because for all \(g' \in gN \subseteq G\),
    \(\bar{\phi}'(gN) = \bar{\phi}'(\pi(g')) = \phi(g')
    = \bar{\phi}(\pi(g')) = \bar{\phi}(gN)\),
    contradicting with our assumed inequality above.
    Hence we have established uniqueness of \(\bar{\phi}\).

    We will now prove existence by constructing such a homomorphism.
    Let \(\vfunc{\bar{\phi}}{G/N}{H}{gN}{\phi(g)}\),
    mapping all cosets \(gN\) to the function output
    of its coset representative.
    Suppose some arbitrary element \(g' \in gN \subseteq G\)
    in an arbitrary coset.
    Then we know that there exists \(n \in N\) such that \(g' = gn\),
    which allows us to conclude that
    \begin{equation*}
        \phi(g') = \phi(gn) = \phi(g)\phi(n) = \phi(g)
        = \bar{\phi}(gN) = \bar{\phi}(\pi(g'))
    \end{equation*}
\end{proof}
\begin{remark}
    Logically speaking,
    the existence part is proven before the uniqueness part
    in the sense that one requires existence to prove uniqueness.
    However, it is convenient to write universal property proofs
    in the reverse order
    because during the proof of uniqueness,
    we often demonstrate criteria that must be followed
    by our unique function \(\bar{\phi}\),
    which greatly helps us in deciding how to construct \(\bar{\phi}\)
    during the proof of existence.
\end{remark}

\begin{remark}
    Jacobson here groups all the statements below into something
    known as the Fundamental Theorem of Homomorphisms of Groups
    and the Isomorphism Theorems.
    % which concerns only with the image of \(G\),
    % which is a subset of \(H\).
    We will group the theorems differently,
    following the convention as given in Rotman and Dummit \& Foote,
    and with the correspondence theorem denoted the Fourth Isomorphism Theorem.
    Nevertheless, armed with the universal property,
    these are much easier to prove now.
\end{remark}
% \begin{corollary}[Fundamental Homomorphism Theorem for Groups]
%     Suppose \(\func{\phi}{G}{H}\) is a group homomorphism,
%     and \(N = \ker(\phi)\).
%     Then there exists a unique isomorphism
%     that makes the image \(\phi(G) \cong G/N\).
% \end{corollary}
% \begin{proof}
%     We want to prove that for all \(g \in \phi^{-1}(x)\)
%     we have \(\phi^{-1}(x) = gH\) being the coset.
%     Suppose we have \(\phi(g_1) = \phi(g_2) = x\)
%     mapping to the same element.
%     Then we see that \(\phi(g_1^{-1} g_2) = \phi(g_1)^{-1}\phi(g_2)
%     = x^{-1}x = 1\).
%     This implies that \(g_1^{-1} g_2 \in H\) inside the kernel,
%     so that there exists \(h \in H\) such that \(h = g_1^{-1}g_2\),
%     and therefore \(g_1 h = g_2\),
%     telling us that they are in the same coset,
%     \(g_2 \in g_1 H\), implying that \(\phi^{-1}(x) \subseteq gH\).
% \end{proof}

\begin{theorem}[First Isomorphism Theorem for Groups]\label{thm:iso-1-group}
    Suppose \(\func{\phi}{G}{H}\) is a group homomorphism,
    and \(N = \ker(\phi)\).
    We have:
    \begin{enumerate}[label={(\alph*)}, itemsep=0mm]
        \item \(N \lhd G\), the kernel is a normal subgroup;
        \item \(\phi(G) \subseteq H\), the image is a subgroup; and
        \item \(\phi(G) \cong G/N\),
            the image is uniquely isomorphic to the quotient group.
    \end{enumerate}
\end{theorem}
\begin{proof}
    We first prove that the preimage of the identity
    forms a subgroup of \(H\).
    Suppose \(\{g_1,g_2\} \subset \phi^{-1}(1)\).
    Then we know \(\phi(g_1) = \phi(g_2) = 1\),
    which gives us closure with \(\phi(g_1 g_2) = \phi(g_1)\phi(g_2) = 1\).
    Associativity is inherited from the multiplication of \(G_1\),
    while the identity by definition must obey \(\phi(1) = 1\).
    Lastly we have \(1 = \phi(1) = \phi(g_1 g_1^{-1})
    = \phi(g_1)\phi(g_1^{-1})\),
    which forces us to conclude that \(g_1^{-1} \in \phi^{-1}(1)\),
    giving us an inverse.
    Hence \(\phi^{-1}(1)\) is a group.

    Now to prove that this is a normal subgroup,
    suppose we have some arbitrary element \(g \in G\),
    and \(h \in \phi^{-1}(1)\).
    We see that \(\phi(ghg^{-1}) = \phi(g)\phi(h)\phi(g^{-1})
    = \phi(g)\phi(g^{-1}) = \phi(gg^{-1}) = \phi(1) = 1\),
    which implies \(ghg^{-1} \in \phi^{-1}(1)\),
    exactly the definition of a normal subgroup,
    which proves statement (a).

    \medskip

    If \(\{\phi(x),\phi(y)\} \subset \phi(G)\),
    then \(\phi(x)\phi(y) = \phi(xy) \in \phi(G)\),
    and thus we have closure.
    Then \(\phi(1_G) \in \phi(G)\)
    which must be the identity by definition of a homomorphism.
    Associativity is obtained for free
    when we equate the multiplicative operation
    between \(G\) and \(\phi(G)\).
    Lastly the inverse must be given by
    \(1_H = \phi(1_G) = \phi(xx^{-1}) = \phi(x)\phi(x^{-1})\).
    Hence \(\phi(G)\) forms a subgroup of \(H\),
    which proves statement (b).

    \medskip

    By the \hyperref[thm:univ-prop-quotient-group]{universal property},
    there exists a unique homomorphism \(\func{\bar{\phi}}{G/N}{\phi(G)}\)
    such that \(\phi = \bar{\phi}\circ\pi\),
    where \(\pi\) is the quotient homomorphism.
    Since by definition, \(\phi\) is a surjective mapping
    from \(G\) to its image \(\phi(G)\),
    \(\bar{\phi}\) must also be a surjection.

    Again taking the same definition for \(\bar{\phi}\)
    as in the universal property,
    \(\vfunc{\bar{\phi}}{G/N}{\phi(G)}{gN}{\phi(g)}\),
    suppose we have two cosets \(g_1 N\) and \(g_2 N\).
    such that \(\bar{\phi}(g_1 N) = \bar{\phi}(g_2 N)\),
    giving us \(\phi(g_1) = \bar{\phi}(\pi(g_1)) = \bar{\phi}(g_1 N)
    = \bar{\phi}(g_2 N) = \bar{\phi}(\pi(g_2)) = \phi(g_2)\)
    But then we have \(1 = \phi(g_1){\phi(g_2)}^{-1} = \phi(g_1 g_2^{-1})\),
    implying that \(g_1 g_2^{-1} \in N\), the kernel.
    Hence we have \(g_1 \in g_2 N\),
    which gives us \(g_1 N \subseteq g_2 N\),
    so without loss of generality, \(g_2 N \subseteq g_1 N\),
    giving us \(g_1 N = g_2 N\), the cosets must be equal.
    This proves injectivity.

    Combining surjection and injection
    gives us our required bijective homomorphism,
    which is an isomorphism,
    proving statement (c).
\end{proof}

\begin{definition}
    Suppose \(G\) is a group,
    and \(S,T \subseteq G\) some subgroups.
    We define the product to be \(ST = \{st : s \in S, t \in T\}\).
\end{definition}

\begin{theorem}[Second Isomorphism Theorem for Groups]\label{thm:iso-2-group}
    Suppose \(G\) is a group,
    \(S \subseteq G\) some subgroup,
    and \(N \lhd G\) a normal subgroup.
    We have:
    \begin{enumerate}[label={(\alph*)}, itemsep=0mm]
        \item \(SN \subseteq G\), the product is a subgroup;
        \item \(N \lhd SN\),
            the normal subgroup is also normal to the product;
        \item \(S \cap N \lhd S\),
            the intersection is normal to the subset; and
        \item \(SN/N \cong S/(S \cap N)\),
            these two quotients are isomorphic.
    \end{enumerate}
    %
    % This is represented by the following commutative diagram:
    % \begin{center}
    %     \begin{tikzcd}
    %
    %         % G \arrow{r}{\phi} \arrow{d}{\pi} & H \\
    %         % G/N \arrow{ru}[swap]{\exists! \bar{\phi}}
    %     \end{tikzcd}
    % \end{center}
\end{theorem}
\begin{proof}
    We can see that for all \(s_i n_i \in SN\),
    \(s_1 n_1 s_2 n_2 = s_1 s_2 s_2^{-1} n_1 s_2 n_2 \in SN\),
    since \(s_2^{-1} n_1 s_2 \in N\) by definition of normal,
    and we have closure.
    Associativity is inherited from \(G\),
    and the identity \(1\) is in both \(S\) and \(N\).
    Lastly, \({(sn)}^{-1} = n^{-1}s^{-1} = s^{-1}sn^{-1}s^{-1} \in SN\),
    since \(sn^{-1}s^{-1} \in SN\) by definiton of normal,
    which gives us inverse.
    Therefore \(SN \subseteq G\) forms a group,
    giving us statement (a).

    \medskip

    Now, clearly \(N \subseteq SN\),
    because every element \(n \in N\) multiplied by \(1 \in S\)
    gives us an element of \(SN\).
    To prove normality,
    we need to prove that for all \(s \in S\) and \(n_i \in N\),
    \(sn_1 n_2 {(sn_1)}^{-1} \in N\).
    This is easy because \(sn_1 n_2 {(sn_1)}^{-1}
    = s(n_1 n_2 n_1^{-1})s^{-1} \in N\)
    by definition of \(N \lhd G\).
    Hence \(N \lhd SN\), proving statement (b).

    \medskip

    Also, clearly \(S \cap N \subseteq S\),
    because by definition \(g \in S \cap N\)
    implies \(g \in S\) (and \(g \in N\)).
    Now it is sufficient to prove that
    for all \(g \in S \cap N\) and \(s \in S\),
    \(sgs^{-1} \in S \cap N\).
    It is clear that since both \(g,s\) are elements of \(S\),
    \(sgs^{-1} \in S\).
    Then since \(g \in N\), and \(s \in S \subseteq G\),
    we have \(sgs^{-1} \in N\) since \(N \lhd G\).
    Hence \(sgs^{-1} \in S \cap N\),
    so we have \(S \cap N \lhd S\),
    proving statement (c).

    \medskip

    We now attempt to construct a homomorphism
    \(\vfunc{\phi}{S}{SN/N}{s}{sN}\).
    % \(\vfunc{\phi}{SN/N}{S/(S \cap N)}{sN}{s(S \cap N)}\).
    % Obviously the kernel is \(\ker(\phi) = N\),
    % because to map to 1, \(s = 1\),
    % which implies \(sn = n \in N\);
    % conversely, when \(s \neq 1\), then \(sn \notin N\),
    % and \(\phi(sn) = \phi(s) = s \neq 1\).
    We demonstrate that this is a valid homomorphism,
    by showing that
    \begin{equation*}
        \phi(s_1 s_2) = (s_1 s_2)N = (s_1 N)(s_2 N) = \phi(s_1)\phi(s_2)
    \end{equation*}
    since \(s_1,s_2\) are elements of \(G\),
    so same logic as the quotient subgroup epimorphism applies.

    We now want to show that \(\phi\) is surjective.
    The elements \(sn \in SN\) must belong in some coset \(snN\),
    which we can see is equivalent to \(sN\).
    By definition, \(\phi\) maps \(s \mapsto sN\),
    so every coset is covered by \(\phi\),
    and therefore it is an epimorphism.

    We can then demonstrate that \(\ker(\phi) = S \cap N\).
    We can see that if \(g \in S \cap N\),
    then \(g \in N\), so \(\phi(g) = gN = N\),
    which gives us \(S \cap N \subseteq \ker(\phi)\).
    On the other hand, if \(g \in \ker(\phi)\),
    then \(\phi(g) = gN \subseteq N\),
    which requires \(g \in N\),
    giving us \(\ker(\phi) \subseteq S \cap N\).
    Hence \(\ker(\phi) = S \cap N\).

    Lastly, we can apply the
    \hyperref[thm:iso-1-group]{first isomorphism theorem},
    and prove that there exists a unique isomorphism
    between \(S/(S \cap N) \cong SN/N\).
\end{proof}

\begin{theorem}[Third Isomorphism Theorem for Groups]\label{thm:iso-3-group}
    Suppose \(G\) is a group, \(N \lhd G\) a normal subgroup.
    Then:
    % Then we have:
    \begin{enumerate}[label={(\alph*)}, itemsep=0mm]
        \item if \(K\) is a subgroup such that \(N \subseteq K \subseteq G\),
            then \(K/N \subseteq G/N\) is a subgroup;
        \item a subgroup of \(G/N\) must be of the form \(K/N\)
            such that \(K\) is a subgroup with \(N \subseteq K \subseteq G\);
        \item if \(K\) is a normal subgroup such that \(N \subseteq K \subseteq G\),
            then \(K/N \lhd G/N\) is a normal subgroup;
        \item a normal subgroup of \(G/N\) must be of the form \(K/N\)
            where \(K \lhd G\) is a normal subgroup with \(N \subseteq K \subseteq G\); and
        \item if \(K \lhd G\) is a normal subgroup such that \(N \subseteq K \subseteq G\),
            then \((G/N)/(K/N) \cong G/K\).
    \end{enumerate}
\end{theorem}
\begin{proof}
    We first have to prove that \(N \lhd K\),
    which is obvious because by definition of normality,
    for all \(n \in N\) and \(g \in G\),
    \(gng^{-1} \in N\), which can be restricted to \(g \in K \subseteq G\).
    % We will first demonstrate that \(N \lhd K\).
    % For some \(k \in K \subseteq G\) and \(n \in N\),
    % \(knk^{-1} \in N\) by normality of \(N\) in \(G\),
    % since we can treat \(k\) as elements of \(G\).
    % Hence all three \(G/N, K/N, G/K\) are valid quotient groups.

    It is now easy to see that
    with the quotient homomorphism \(\func{\pi}{G}{G/N}\),
    the image of the subgroup \(\pi(K) = K/N\),
    so by the \hyperref[thm:iso-1-group]{first isomorphism theorem}
    \(K/N\) forms a subgroup.
    This proves statement (a).

    % Secondly, we can see that \(K/N \lhd G/N\),
    % because for some arbitrary cosets \(kN \in K/N\) and \(gN \in G/N\),
    % there are \(n_i \in N\)
    % such that the coset representatives \(k' = kn_1\) and \(g' = gn_2\),
    % \((gN)(kN){(gN)}^{-1} = (NgkN)\)

    \medskip

    Suppose \(K' \subseteq G/N\) is a subgroup.
    We can look at the preimage \(\pi^{-1}(K')\),
    which since \(1 \in K'\), we have \(K' \supseteq \ker(\pi) = N\).
    Notice that the preimage of a group is still a group:
    suppose \(\func{\phi}{G'}{H'}\) is a homomorphism,
    since \(\{x,y\}\subseteq\phi^{-1}(H')\)
    implies \(\phi(xy) = \phi(x)\phi(y) \in H'\) (closure),
    \(1 \in \ker(\phi) \subseteq \phi^{-1}(H')\) (identity),
    associativity inherited from \(G'\),
    and \(x \in \pi^{-1}(H')\)
    implies \({\pi(x)}^{-1} = \pi(x^{-1}) \in H'\) (inverse).
    This proves statement (b).

    \medskip

    It is now sufficient to prove the normality condition.
    % If we have \(k \in K\) and \(g \in G\) such that \(gkg^{-1} \in K\),
    % then under the quotient homomorphism \(\func{\pi}{G}{G/N}\)
    % \(\pi(gkg^{-1}) = \pi(g)\pi(k)\pi(g)^{-1}\)
    \(K \lhd G\) tells us that for all \(g \in G\), \(gKg^{-1} \subseteq K\),
    which under mapping is \(\pi(gKg^{-1}) = \pi(g)\pi(K){\pi(g)}^{-1}\).
    Notice that by Theorem~\ref{thm:equal-coset-normal},
    \(\pi(g)\pi(K) = \pi(K)\pi(g)\),
    which allows us to cancel out the \(g\) and its inverse,
    letting us conclude that \(\pi(gKg^{-1}) \subseteq \pi(K)\),
    giving us \(\pi(K) = K/N \lhd G/N\).
    This proves statement (c).

    \medskip

    Similarly, assume \(K' \lhd G/N\);
    then for all \(x \in G/N\), \(xK'x^{-1} \subseteq K'\).
    No matter what element \(g \in G\), \(k \in \pi^{-1}(K')\) we choose,
    we have \(\pi(gkg^{-1}) = \pi(g)\pi(k){\pi(g)}^{-1} \in K'\)
    by definition of normality in \(G/N\),
    so \(ghg^{-1} \in \pi^{-1}(K')\),
    and the preimage is normal.
    This proves statement (d).

    \medskip
    
    We can now attempt to construct a homomorphism
    \(\vfunc{\phi}{G/N}{G/K}{gN}{gK}\).
    % We demonstrate that this is a valid homomorphism, as
    % first by showing that all elements in the same \(N\)-coset
    % will be mapped to the same \(K\)-coset;
    % observe that if \(g' = gn\),
    % then \(\phi(g'N) = g'K = gnK = gK = \phi(gN)\)
    % since \(n \in N \subseteq K\).
    % Moreover, we have
    % \begin{equation*}
    %     \phi(g_1 g_2 N) = g_1 g_2 K = (g_1 g_2 K)K = (K g_1 g_2)K
    %     = (g_1 K)(g_2 K) = \phi(g_1 N)\phi(g_2 N)
    % \end{equation*}
    This is valid because
    by the \hyperref[thm:univ-prop-quotient-group]{universal property},
    we have \(\func{\pi}{G}{G/N}\) and \(\func{\eta}{G}{G/K}\),
    so there is a unique homomorphism that makes \(\eta = \phi\circ\pi\).
    We can also show that this is surjective,
    since the \(K\)-cosets partition \(G\),
    so each \(K\)-coset must have some element \(gK\)
    that represents it,
    and clearly this element \(gN\) in the \(N\)-cosets
    must get sent to it,
    alongside all other elements in that \(N\)-coset.

    We claim the kernel is \(\ker(\phi) = K/N\).
    % To see this, first observe \(kN \in K/N\)
    % must get mapped to \(\phi(kN) = kK = K\),
    % which is the coset of the identity,
    % giving us \(K/N \in \ker(\phi)\).
    % For the reverse argument,
    % for the image to be the identity coset \(K\),
    % the element \(g\) must belong in \(K\),
    % which means the kernel must be the cosets of the elements in \(K\),
    % otherwise written as \(\ker(\phi) \subseteq K/N\).
    % This proves our claim.
    Observe that \(\ker(\eta) = K\) and \(\ker(\pi) = N\),
    so \(\phi\) must map all the \(N\)-cosets
    that are represented by elements of \(K\) into \(1\).

    Lastly, we invoke the \hyperref[thm:iso-1-group]{first isomorphism theorem},
    which gives us \(\ker(\phi) = K/N \lhd G/N\),
    making our quotient \((G/N)/(K/N)\) valid;
    and also that \((G/N)/\ker(\phi) = (G/N)/(K/N) \cong G/K\),
    proving statement (e).
\end{proof}

\begin{theorem}[Fourth Isomorphism Theorem for Groups]\label{thm:iso-4-group}
    Suppose \(G\) is a group,
    \(N \lhd G\) some normal subgroup,
    and \(\vfunc{\pi}{G}{G/N}{g}{gN}\) the quotient homomorphism.
    Then \(\pi\) is a bijection
    between the subgroups of \(G/N\)
    and the subgroups of \(G\) containing \(N\);
    and is also a bijection between normal subgroups of \(G/N\)
    and normal subgroups of \(G\) containing \(N\).
\end{theorem}
\begin{proof}
    % We can split this theorem into 4 statments:
    % first that subgroups of \(G\)
    % First suppose \(KN \subseteq H \subseteq G\) for some subgroup \(H\);
    % then clearly \(\pi(H) \subseteq G/N\) is a subgroup.
    % Now suppose \(H' \subseteq G/N\) is a subgroup;
    % since \(1 \in H'\), the preimage \(\pi^{-1}(H')\) must contain \(N\),
    % and notice that the preimage of a group under homomorphism is still a group,
    % elements of \(H'\) must be cosets \(hN\),
    % where \(h \in H \subseteq G\) forms a group
    This is merely a corollary of
    the \hyperref[thm:iso-3-group]{third isomorphism theorem}.
    Statements (a) and (b) prove the correspondence between subgroups,
    while statements (c) and (d)
    prove the correspondence between normal subgroups.

    % Then suppose \(N \lhd H \lhd G\) for some normal subgroup \(H\);
    % this tells us that for all \(g \in G\), \(gHg^{-1} \subseteq H\),
    % which under mapping is \(\pi(gHg^{-1}) = \pi(g)\pi(H){\pi(g)}^{-1}\).
    % Notice that \(\pi(g)\pi(H) = \pi(H)\pi(g)\),
    % which allows us to cancel out the \(g\) and its inverse,
    % letting us conclude that \(\pi(gHg^{-1}) \subseteq \pi(H)\),
    % giving us \(\pi(H) \lhd G/N\).
    % Now in the other direction, 

    % Lastly, we wish to prove that \(G/H \cong (G/N)/\pi(H)\).
    % But this is exactly the conclusion of
    % the \hyperref[thm:iso-3-group]{third isomorphism theorem}.
\end{proof}


\subsection{Action on Sets}

\begin{definition}
    Suppose \(X\) is a set of \(n\) elements.
    The group of all possible permutations
    (guaranteed by Proposition~\ref{prop:symmetric-group})
    is denoted \(S_X\),
    which is of course isomorphic to \(S_n\).
\end{definition}

\begin{definition}
    Suppose \(G\) a group, and \(X\) some set.
    The action of \(G\) on \(X\) is a homomorphism \(\func{\phi}{G}{S_X}\).
\end{definition}
\begin{remark}
    Notice that by \hyperref[thm:cayley]{Cayley's theorem},
    every finite group is a transformation group on itself.
    With this knowledge,
    we can formally say that we can let \(G\) act on itself
    by left multiplication,
    where the homomorphism is \(\vfunc{\phi}{G}{S_G}{g}{\ell_g}\),
    and \(\vfunc{\ell_g}{G}{G}{\alpha}{g\alpha}\).
    Beware that when considering actions,
    the homomorphism \(\phi(g)\) outputs a function
    on the set \(X\) that we are acting on,
    so in general \(\phi(g)(x) = \phi_g(x)\) are both valid notation,
    the former focusing on that we have a homomorphism \(\phi(g)\),
    and the latter focusing on that we have a function \(\phi_g\).
\end{remark}

\begin{proposition}[Action by Conjugation]
    Suppose \(G\) is a group.
    \(G\) can act on itself by conjugation,
    that is, \(\vfunc{\phi}{G}{S_G}{g}{\gamma_g}\)
    where \(\vfunc{\gamma_g}{G}{G}{\alpha}{g \alpha g^{-1}}\).
\end{proposition}
\begin{proof}
    We need to first prove that \(\gamma_g\) is a bijection.
    Suppose \(\{\alpha,\beta\} \subset G\),
    \(g \alpha g^{-1} = g \beta g^{-1}\);
    then \(\alpha = g^{-1}g \alpha g^{-1}g
    = g^{-1}g \beta g^{-1}g = \beta\),
    which implies \(\gamma_g\) is an injection.
    Now, clearly, for every element \(x \in G\),
    we can map \(\gamma_g(g^{-1}xg) = gg^{-1}xgg^{-1} = x\),
    which implies \(\gamma_g\) is a surjection.
    Hence \(\gamma_g\) is a bijection.

    We now then prove that \(\phi\) is a homomorphism.
    Clearly \(\gamma_1\) is the identity mapping
    \(\alpha \mapsto 1\alpha 1^{-1} = \alpha\).
    \begin{equation*}
        \gamma_{gh}(\alpha) = (gh)\alpha{(gh)}^{-1}
        = g h\alpha h^{-1}g^{-1} = \gamma_g(h \alpha h^{-1})
        = \gamma_g(\gamma_h(\alpha))
    \end{equation*}
    We have showed that
    \(\phi(gh) = \gamma_{gh} = \gamma_g \circ \gamma_h = \phi(g)\phi(h)\),
    which proves homomorphism.
\end{proof}

\begin{definition}
    Suppose \(G\) is a group that acts on some set \(X\)
    via the homomorphism \(\func{\phi}{G}{S_X}\).
    The stabilizer of some element \(x \in X\)
    is the set of all actions that fix \(x\),
    \(\stab(x) = \{g \in G : \phi_g(x) = x\}\).
\end{definition}
\begin{proposition}\label{prop:stabilizer-subgroup}
    The stabilizer \(\stab(x)\) is a subgroup of \(G\).
\end{proposition}
\begin{proof}
    Suppose our action is \(\vfunc{\phi}{G}{S_G}{g}{\phi_g}\),
    where \(\func{\phi_g}{G}{G}\) is a permutation.
    Then the stabilizer is a set of all \(g\)
    such that \(\phi_g(x) = x\).
    Suppose \(\phi_g(x) = \phi_h(x) = x\);
    then \(\phi_{gh}(x) = \phi_g(x)\phi_h(x) = x\),
    so the stabilizer is closed under multiplication.
    Associativity is given to us for free with function composition,
    and \(\phi_1(x) = x\) is trivially the identity mapping.
    Lastly if \(\phi_g(x) = x\), we know that
    \(x = \phi_1(x) = \phi_g(x)\phi_{g^{-1}}(x) = \phi_{g^{-1}}(x)\)
    which gives us an inverse.
\end{proof}

\begin{definition}
    Suppose \(G\) is a group
    and \(x \in G\) some element in the group.
    The centralizer of the element
    is the set of all elements in \(G\) that commute with it,
    \(\Centralizer(x) = \{g \in G : gx = xg\}\).
    Similarly, the centralizer of a subset \(S \subseteq G\)
    is the set of all elements in \(G\)
    that commute with every element in \(S\),
    \(\Centralizer(S) = \Centralizer_G(S)
    = \{g \in G : \forall x \in S, gx = xg\}\).
\end{definition}
\begin{proposition}\label{prop:centralizer-stabilizer}
    Suppose \(G\) acts on itself by conjugation.
    Then for any \(x \in G\), \(\Centralizer(x) = \stab(x)\).
\end{proposition}
\begin{proof}
    \(\stab(x) = \{g \in G : \gamma_g(x) = x\}
    = \{g \in G : gxg^{-1} = x\} = \{g \in G : gx = xg\}
    = \Centralizer(x)\).
\end{proof}
\begin{corollary}\label{cor:centralizer-subgroup}
    The centralizer \(\Centralizer(S) \subseteq G\) is a subgroup.
\end{corollary}
\begin{proof}
    We first observe that
    \(\Centralizer(S) = \bigcap_{s \in S} \Centralizer(s)\),
    because it should include all elements
    that commute with every element of \(S\).
    Then we know from the
    \hyperref[prop:centralizer-stabilizer]{proposition above}
    that \(\Centralizer(s) = \stab(s)\),
    which are subgroups by Proposition~\ref{prop:stabilizer-subgroup}.
    Lastly, from Lemma~\ref{lem:intersection-subgroup} we know
    the intersection of subgroups is a subgroup.
\end{proof}

\begin{definition}
    We sometimes call the centralizer of the entire group \(\Centralizer(G)\)
    the center of the group \(\Centre(G)\).
    It consists of the elements that commute with everything in the group.
\end{definition}
\begin{proposition}z
    Suppose \(G\) acts on itself by conjugation.
    Then the kernel of the homomorphism is the centralizer of the group,
    \(\ker(\phi) = \Centre(G)\).
\end{proposition}
\begin{proof}
    \(\ker(\phi) = \{g \in G : \phi(g) = 1\}
    = \{g \in G : \gamma_g = \gamma_1\}
    = \{g \in G : \forall x \in G, gxg^{-1} = x\}
    = \Centre(G)\).
\end{proof}
\begin{remark}
    The centralizer tells us what and how many elements of \(G\)
    commute with everything,
    so in a sense, self-action by conjugation
    allows us to `measure' the noncommutativity of \(G\).
\end{remark}

\begin{definition}
    Suppose \(G\) is a group
    and \(S \subseteq G\) some subset of the group.
    The normalizer of the subset
    is the set of elements that fix \(S\) under conjugation,
    i.e.\ conjugation by the normalizer sends elements of \(S\)
    to (potentially other) elements of \(S\).
    This is denoted \(\Normalizer(S) = \Normalizer_G(S)
    = \{g \in G : gSg^{-1} = S\}\).
\end{definition}
\begin{proposition}\label{prop:normalizer-subgroup}
    The normalizer \(\Normalizer(S) \subseteq G\) is a subgroup.
\end{proposition}
\begin{proof}
    We clearly have closure with
    \((gh)S{(gh)}^{-1} = ghSh^{-1}g^{-1} = gSg^{-1} = S\).
    Associativity and identity is given to us,
    as per usual, for free.
    Since \(gSg^{-1} = S\),
    then \(S = g^{-1}gSg^{-1}g = g^{-1}Sg\),
    and we get inverse.
\end{proof}

\begin{remark}
    Recall from the definition of cosets and orbits
    in Section~\ref{sec:cosets}
    that the actions of \(G\) form orbits,
    which are equivalence classes demonstrating that
    there exists an action \(g \in G\)
    such that \(\phi_g(x) = y\)
    for all \(\{x,y\} \subset X\) in the same orbit.
    If we allows \(G\) to act on itself,
    these now define equivalence classes on \(G\) itself,
    which allow us to deduce facts about subgroups of \(G\).
    In particular, under self-action,
    the set of orbits \(G/{\sim}\) partition \(G\).
\end{remark}
\begin{definition}
    Suppose \(G\) acts on a set \(X\) by conjugation.
    The orbits of \(X\) are equivalence classes,
    which we call conjugacy classes,
    sometimes denoted \([x] = \{\gamma_g(x) = gxg^{-1} : g \in G\}\).
\end{definition}

\pagebreak

\begin{definition}
    Suppose \(G\) acts on two sets \(X\) and \(Y\)
    via \(\vfunc{\phi}{G}{S_X}{g}{\phi_g}\)
    and \(\vfunc{\psi}{G}{S_Y}{g}{\psi_g}\) respectively.
    We say \(X\) is equivalent to \(Y\) under \(G\)-action
    if there exists a bijection \(\func{f}{X}{Y}\)
    such that \(f(\phi_g(x)) = \psi_g(f(x))\)
    for all \(x \in X\) and \(g \in G\).

    This can be summarized by the following commutative diagram:
    \begin{center}
        \begin{tikzcd}
            X \arrow{r}{f} \arrow{d}{\phi_g} & Y \arrow{d}{\psi_g} \\
            X \arrow{r}{f} & Y
        \end{tikzcd}
    \end{center}
\end{definition}
\begin{lemma}\label{lem:transitive-orbit-subgroup}
    Suppose \(G\) acts on \(X\) transitively (there is only one orbit).
    Then there exists a subgroup \(H \subseteq G\)
    such that \(X\) is equivalent to \(G/H\).
\end{lemma}
\begin{proof}
    Suppose the action of \(G\) on \(X\)
    is \(\vfunc{\phi}{G}{S_X}{g}{\phi_g}\).
    To prove such an equivalence,
    first we have to construct \(H\).
    Choosing some arbitrary \(x \in X\),
    let us find its stabilizer \(H = \stab(x) = \{g \in G : \phi_g(x) = x\}\),
    which by Proposition~\ref{prop:stabilizer-subgroup} we know is a subgroup.
    % We claim that \(H \lhd G\).
    % If \(\phi_h(x) = x\),
    % we see that \(\phi_{ghg^{-1}} = \phi_g\phi_h\phi_{g^{-1}}\),
    % so we have \(\phi_{ghg^{-1}}(\phi_g(x)) = \phi_g(x)\),
    % which stabilizes some arbitrary element \(\phi_g(x)\),
    % and hence fixes all elements \(x \in X\) as \(\phi_g\) is a bijection,
    % giving us normality.

    Now, suppose \(G\) acts on \(G/H\) by left multiplication,
    so that we have \(\vfunc{\psi}{G}{S_{G/H}}{g}{\ell_g}\),
    where \(\vfunc{\ell_g}{G/H}{G/H}{rH}{grH}\).
    Note that \(G/H\) here is not a quotient group,
    but rather the set of all left cosets.
    We claim that the function \(\vfunc{f}{X}{G/H}{\phi_g(x)}{gH}\)
    forms a bijection.
    We first see that this is a completely valid definition,
    because the action is transitive, so \(\phi_g(x)\) covers all of \(X\).
    We then have to check that if \(\phi_{g_1}(x) = \phi_{g_2}(x)\),
    \(g_1H = g_2H\);
    this is obvious because \(\phi_{g_2}^{-1}\circ\phi_{g_1}(x) = x\),
    and \(g_2^{-1} g_1 \in H\), so \(g_2 \in g_1 H\),
    and without loss of generality we also have \(g_1 \in g_2 H\).
    \(f\) is injective because if \(g_1 H = g_2 H\),
    then \((g_2^{-1}g_1)H = H\),
    so \(\phi_{g_2^{-1}g_1}(x) = \phi_{g_2}^{-1}\circ\phi_{g_1}(x) = x\),
    and hence \(\phi_{g_1}(x) = \phi_{g_2}(x)\).
    \(f\) is surjective also because for every \(gH\),
    at least \(gx\) will map to it.
    Therefore, \(f\) is a bijection.

    We will now check the equivalence under multiplication.
    For some arbitrary \(r \in G\),
    and hence some arbitrary \(\phi_r(x) \in X\),
    \begin{equation*}
        \ell_g(f(\phi_r(x))) = \ell_g(rH) = grH
        = f(\phi_{gr}(x)) = f(\phi_g(\phi_r(x)))
    \end{equation*}
    We can conclude that there is equivalency between \(X\) and \(G/H\).
\end{proof}
\begin{corollary}[Orbit-Stabilizer Theorem]\label{cor:orbit-stabilizer}
    Suppose \(G\) is a group that acts on a finite set \(X\).
    Let \(x \in X\) denote an arbitrary element.
    Then we have an equivalence between the size of the orbit of \(x\)
    and the index of the stabilizer in \(G\),
    that is, \(\abs{\orb(x)} = [G:\stab(x)]\).
\end{corollary}
\begin{proof}
    We shall restrict the action on \(G\) on \(X\)
    to only acting onto the subset \(\orb(x) \subseteq X\).
    This new action is now by definition transitive.
    The \hyperref[lem:transitive-orbit-subgroup]{lemma above}
    asserts an equivalence between \(\orb(x)\) and \(G/\stab(x)\),
    that is, a bijection between elements of these two sets.
    Hence we can equate the cardinality of these two sets.
\end{proof}
\begin{corollary}\label{cor:transitive-cardinality}
    If \(G\) acts on \(X\) transitively,
    then \(\abs{X} \mid \abs{G}\).
\end{corollary}
\begin{proof}
    From the \hyperref[lem:transitive-orbit-subgroup]{lemma above},
    since there is a bijection between \(X\) and \(G/H\)
    we know that \(\abs{X} = \abs{G/H}\).
    But \hyperref[thm:lagrange]{Lagrange's theorem}
    guarantees that the number of cosets is \(\abs{G/H} = [G:H] \mid \abs{G}\).
\end{proof}

\begin{theorem}[Class Equation]\label{thm:class-equation}
    Suppose \(G\) acts on \(X\).
    Then \(\abs{X} = \sum_i \abs{X_i} = \sum_i \abs{G/H_i}\),
    where \(H_i \subseteq G\) subgroups,
    and hence \(\abs{G/H_i} \mid \abs{G}\).
\end{theorem}
\begin{proof}
    Suppose the orbits of \(X\) are disjoint sets \(X_i\).
    Then if we restrict to any subset \(X_i \subseteq X\),
    \(G\) acts on \(X_i\) transitively.
    By the \hyperref[lem:transitive-orbit-subgroup]{lemma above},
    \(X_i\) is equivalent to \(G/H_i\),
    so \(\abs{X_i} = \abs{G/H_i}\).
    Now, since orbits partition \(X\),
    \(\abs{X} = \sum \abs{X_i}\).
\end{proof}
\begin{corollary}
    \(\abs{G} = \abs{\Centre(G)} + \sum_i n_i\)
    where \(n_i \mid \abs{G}\).
\end{corollary}
\begin{proof}
    Let \(G\) act on itself by conjugation.
    For the elements \(x \in G\) that get fixed by conjugation,
    i.e.\ they commute with all \(g \in G\),
    their orbits (conjugacy classes) are a single element
    \([x] = \{\gamma_g(x) = gxg^{-1} : g \in G\} = [x] = \{x\}\).
    Together, all these elements form the center \(\Centre(G)\).

    Then all other conjugacy classes (with more than one element)
    are equivalent to \(G/H_i\)
    by the \hyperref[thm:class-equation]{class equation},
    which are divisors of \(\abs{G}\).
\end{proof}

\begin{theorem}\label{thm:pr-nontrivial-center}
    Suppose \(G\) is a finite group,
    with order \(\abs{G} = p^r\) where \(p\) is prime.
    Then \(G\) has a nontrivial center,
    that is, \(\Centre(G) \neq \{1\}\).
\end{theorem}
\begin{proof}
    Suppose we have a trivial center.
    Then the \hyperref[thm:class-equation]{class equation} tells us that
    \(p^r = 1 + \sum_i p^{r_i}\),
    since the order is \(\abs{G} = p^r\), \(\abs{\Centre(G)} = 1\),
    and all divisors of \(p^r\) must be some power of \(p\),
    with \(r_i \neq 0\), as those orbits cannot be trivial.
    But clearly the left side is a multiple of \(p\),
    and the right side has a multiple of \(p\) plus 1,
    which is a contradiction.
\end{proof}
\begin{corollary}\label{cor:p2-abelian}
    Suppose \(G\) is a group with order \(\abs{G} = p^2\),
    where \(p\) is prime.
    Then \(G\) is abelian.
\end{corollary}
\begin{proof}
    We know from the \hyperref[thm:class-equation]{theorem above}
    that it has a nontrivial center (\(\abs{\Centre(G)} \neq 1\)),
    so if the order of some element is \(p^2\),
    then it generates the entire group, and it must be cyclic (abelian).
    % But since the center is a subgroup
    % (Corollary~\ref{cor:centralizer-subgroup}),
    % its order must divide \(\abs{G} = p^2\) (Theorem~\ref{thm:lagrange}),
    % i.e.\ \(\abs{\Centre(G)} = p\) or \(p^2\).

    % Suppose, by way of contradiction, that \(\abs{\Centre(G)} = p\).
    % Then we can clearly see that all elements in their orbits
    Now suppose there are no elements of order \(p^2\);
    then all elements other than the identity must have order \(p\),
    which implies the center must also be a subgroup
    (Corollary~\ref{cor:centralizer-subgroup})
    that is cyclic with order \(p\) (Theorem~\ref{thm:lagrange}).
    % Let \(H = \Centre(G)\).
    % We write down the number of cosets that we get,
    % that is, \([G:H] = \abs{G}/\abs{H} = p^2/p = p\).
    Let \(\Centre(G) = \langle g \rangle\), where \(\abs{g} = p\).
    Clearly there is another element \(k \in G\), but \(k \neq g^r\)
    such that \(\abs{k} = p\).
    If \(gk = kg\), then a simple counting argument gives us
    \(\langle g,k \rangle\) a group of order \(p^2\),
    since every element can be written in the form \(g^i k^j\),
    \(i,j\) each having \(p\) choices each.
    If on the other hand, \(gk \neq kg\),
    then \(\langle g,k \rangle\) must include
    all previously mentioned \(p^2\) elements, and also \(kg\),
    which implies \(\abs{\langle g,k \rangle} > p^2\),
    and contradicts the fact that \(\langle g,k \rangle \subseteq G\).
    Hence \(gk = kg\),
    and any power of \(g\) commutes with any power of \(k\),
    so all elements of \(G\) commute with each other.
\end{proof}

\begin{theorem}
    Conjugacy classes in \(S_n\) are the cycle types;
    same cycle types are in the same conjugacy class,
    and different cycle types are in different conjugacy classes.
\end{theorem}
\begin{proof}
    Suppose we have a \(d\)-cycle \((n_1 n_2 \hdots n_d)\).
    We know that for all \(g \in S_n\), \(n_i \in \{1,\hdots,n\}\),
    any \(g(n_1 n_2 \hdots n_d)g^{-1} = (g(n_1) g(n_2) \hdots g(n_d))\)
    by Proposition~\ref{prop:sn-conjugation}.
    This is also a \(d\)-cycle.
    As all elements of \(S_n\) are disjoint cycles,
    all such disjoint cycle types must only ever map to their own cycle type
    under conjugation.
\end{proof}
\begin{remark}
    In general it is difficult to determine
    exactly the conjugacy classes of a group.
\end{remark}


\subsection{Sylow's Theorems}

\begin{theorem}[Cauchy's Theorem]\label{thm:cauchy}
    Suppose \(G\) is a finite abelian group,
    and there is some prime \(p \mid \abs{G}\)
    that divides the order of the group.
    Then there exists an element \(g \in G\) of order \(\abs{g} = p\).
\end{theorem}
\begin{proof}
    We first consider the base case where \(\abs{G} = p\).
    By Corollary~\ref{cor:prime-order-subgroup},
    we know that there exists a subgroup \(\langle g \rangle \subseteq G\)
    where \(g \in G\) is any arbitrary element \(g \neq 1\).

    We now proceed by strong induction on the order of the group.
    Suppose \(\abs{G} = mp\), where \(m \in \bN\),
    and that for all \(k < m\) Cauchy's Theorem for order \(kp\) is proven.
    Let us arbitrarily pick some \(x \neq 1\), \(x \in G\).
    Suppose the order of this element is some \(\abs{x} = n\).
    If \(p \mid n\), then Theorem~\ref{thm:cyclic-subgroup-uniqueness}
    guarantees us that there exists a subgroup
    \(\langle x^m \rangle \subseteq \langle x \rangle\)
    with order \(\abs{x^m} = p\).
    If \(p \nmid n\), then we can construct a subgroup \(G/\langle x \rangle\)
    (\(\langle x \rangle\) guaranteed to be normal
    by Proposition~\ref{prop:abelian-subgroup-normal}),
    which by \hyperref[thm:lagrange]{Lagrange's theorem}
    has order \(\abs{G/\langle x \rangle} = \abs{G}/\abs{x} = mp/n\),
    and \hyperref[lem:euclid]{Euclid's lemma} tells us \(p \mid mp/n\).
    Then clearly \(m/n \in \bN\), specifically \(m/n < m\),
    so the induction hypothesis applies.
\end{proof}

\begin{theorem}[Sylow's First Theorem]\label{thm:sylow-1}
    Suppose \(G\) a finite group, \(p\) a prime number,
    and that the order is \(\abs{G} = Np^s\),
    with \(\gcd(N,p) = 1\) coprime
    (that is, \(p^s\) is the largest power of \(p\) that divides \(\abs{G}\)).
    Then for all \(0 \leq t \leq s\),
    \(G\) contains a subgroup of order \(p^t\).
\end{theorem}
\begin{proof}
    We first consider the case where \(G\) is abelian.
    % The most simple case is when \(\abs{G} = Np\);
    % there trivially exists a subgroup \(\{1\}\) of order 1,
    % and by \hyperref[thm:cauchy]{Cauchy's theorem}
    % there exists a subgroup \(\langle g \rangle\) of order \(p\).
    Trivially, \(G\) itself is a subgroup of order \(\abs{G} = Np^s\),
    which we will use as our base case.

    We now suppose, by way of weak induction,
    that we have a subgroup \(H_k\) of order \(\abs{H_k} = Np^k\)
    (for \(k \geq 1\)).
    We wish to prove that a subgroup \(H_{k-1}\)
    of order \(\abs{H_{k-1}} = Np^{k-1}\) exists.
    By \hyperref[thm:cauchy]{Cauchy's theorem},
    we can find an \(x_k \in H_k\) of order \(\abs{x_k} = p\).
    Proposition~\ref{prop:abelian-subgroup-normal}
    tells us that \(\langle x_k \rangle \lhd H_k\),
    so \(H_k/\langle x_k \rangle\) is a subgroup,
    with \hyperref[thm:lagrange]{Lagrange's theorem} telling us
    its order is \(\abs{H_k/\langle x_k \rangle}
    = \abs{H_k}/\abs{x_k} = Np^k/p = Np^{k-1}\).
    This allows us to conclude that there are subgroups of \(G\)
    with orders \(Np^r\) where \(0 \leq r \leq s\).

    Again invoking Proposition~\ref{prop:abelian-subgroup-normal},
    \(H_{s-t} \lhd G\) for all \(0 \leq t \leq s\).
    Hence \(G/H_{s-t}\) is a subgroup,
    with its order determined by \hyperref[thm:lagrange]{Lagrange's theorem}
    to be \(\abs{G/H_{s-t}} = \abs{G}/\abs{H_{s-t}}
    = (Np^s)/(Np^{s-t}) = p^t\).

    \medskip

    Now consider the case where \(G\) is not abelian.
    Then clearly there are elements that do not commute,
    so \(\Centre(G) \subsetneq G\).
    We can write down the \hyperref[thm:class-equation]{class equation}
    \(\abs{G} = \abs{\Centre(G)} + \sum_i [G:H_i]\),
    where there is at least one \(i\) in the summation.

    We can first prove the simplest case when \(\abs{G} = p^s\).
    It suffices to consider \(G\) not abelian,
    since the abelian case is already proven above.
    The center of the group \(\Centre(G)\)
    is a subgroup (Corollary~\ref{cor:centralizer-subgroup}),
    so its order must divide the order of the whole group
    (\(\abs{\Centre(G)} \mid \abs{G} = p^s\),
    \hyperref[thm:lagrange]{Lagrange's theorem}),
    which implies \(p \mid \abs{\Centre(G)}\).
    Then \hyperref[thm:cauchy]{Cauchy's theorem} tells us that
    we can find an element \(x \in \Centre(G) \subsetneq G\)
    such that \(\abs{x} = p\).
    Proceeding similarly to the induction in the abelian case,
    we see that \(\langle x \rangle \lhd \Centre(G)\)
    (Proposition~\ref{prop:abelian-subgroup-normal}),
    and in fact \(\langle x \rangle \lhd G\),
    simply because \(g \in \Centre(G)\),
    and will commute with every element of \(G\)
    (\(gx^r g^{-1} = x^r gg^{-1} = x^r \in \langle x \rangle\)).
    Hence \(G/\langle x \rangle\) is a subgroup
    with order \(\abs{G}/\abs{x} = p^{s-1}\)
    (\hyperref[thm:lagrange]{Lagrange}).
    Recursively applying this process on the quotient group
    generates subgroups of order \(1,p,p^2,\hdots,p^s\).

    Let us then go back to the general case of \(\abs{G} = Np^s\).
    We shall additionally assume, by way of strong induction,
    that any group of order \(mp^s\), where \(m < N\)
    has subgroups of order \(1,p,p^2,\hdots,p^s\).
    Suppose there exists a subgroup \(H_i\) in the class equation
    such that \(p^s \mid \abs{H_i}\).
    Then clearly \(\abs{H_i} = mp^s\), \(m < N\),
    so the induction hypothesis immediately gives our desired result.

    Suppose otherwise, when \(p^s \nmid \abs{H_i}\);
    we can then see that \(p \mid [G:H_i]\) for all \(i\),
    since \hyperref[thm:lagrange]{Lagrange's theorem}
    guarantees \(Np^s = \abs{G} = \abs{H_i}[G:H_i]\),
    and our supposition tells us that the prime factorization of \(\abs{H_i}\)
    includes at most \(p^{s-1}\),
    and therefore \([G:H_i]\) prime factorizes to include at least one \(p\).
    Then every term in the summation inside the class equation
    is divisible by \(p\),
    and the left side is \(Np^s\), which also is divisible by \(p\),
    so the remaining term,
    the centralizer must also be \(p \mid \abs{\Centre(G)}\).
    But this condition means we can follow the proof
    as in the previous paragraph;
    given some group \(H_k\) of order \(Np^k\),
    we can find an element \(x_k\) of order \(p\)
    such that \(H_{k-1} = H_k/\langle x_k \rangle\)
    is a group of order \(\abs{H_{k-1}} = Np^{k-1}\).
    The natural quotient homomorphism \(\func{\pi_k}{H_k}{H_{k-1}}\)
    has a kernel of order \(p\).
    It is then possible to construct a chain of homomorphisms
    \(\pi_{s-r}\circ\pi_{s-r+1}\circ\cdots\circ\pi_s\),
    which has a kernel of order \(p^r\);
    and as we know
    from the \hyperref[thm:iso-1-group]{first isomorphism theorem}
    the kernel is a group,
    which gets us our desired result.
\end{proof}
\begin{remark}
    This theorem does not hold for order \(d^s\) when \(d\) is not prime.
\end{remark}
\begin{corollary}[Cauchy's Theorem]
    Suppose \(G\) is a finite group, abelian or not,
    and there is a some prime \(p \mid \abs{G}\)
    that divides the order of the group.
    Then there exists an element \(g \in G\) of order \(\abs{g} = p\).
\end{corollary}
\begin{proof}
    Special case of the \hyperref[thm:sylow-1]{theorem above},
    with \(t = 1\).
\end{proof}

\begin{definition}
    A group with prime power order,
    that is, \(\abs{G} = p^r\) for some \(r \in \bN\),
    is called a \(p\)-group.
\end{definition}
\begin{corollary}
    There exists subgroups of all possible orders
    that divide the order of a \(p\)-group.
\end{corollary}
\begin{proof}
    The order of a \(p\)-group is \(p^s\),
    so the \hyperref[thm:sylow-1]{theorem above} applies.
\end{proof}

\begin{definition}
    Suppose \(G\) is a finite group with order \(Np^s\), \(\gcd(N,p) = 1\).
    A subgroup of order \(p^s\) (maximal prime power order)
    is often called a Sylow \(p\)-subgroup of \(G\),
    and less often called a \(p\)-Sylow subgroup of \(G\).
\end{definition}
\begin{lemma}\label{lem:prime-power-sylow-subgroup}
    Suppose \(G\) is a finite group with \(\abs{G} = Np^s\), \(\gcd(N,p) = 1\),
    with \(P \subseteq G\) a Sylow \(p\)-subgroup.
    If \(K \subseteq \Normalizer(P)\) a subgroup, and \(\abs{K} = p^t\),
    then \(K \subseteq P \subseteq \Normalizer(P)\).
\end{lemma}
\begin{proof}
    By definition of a normalizer, if \(g \in \Normalizer(P)\),
    we have \(gPg^{-1} \subseteq P\), so \(P \lhd \Normalizer(P)\).

    Since \(P \subseteq \Normalizer(P) \subseteq G\),
    \hyperref[thm:lagrange]{Lagrange's theorem} gives us
    \(\Normalizer(P) = mp^s\), \(m \mid N\).
    We form a quotient group \(\Normalizer(P)/P\),
    which has an order \(mp^s/p^s = m\), which is coprime to \(p\)
    since \(N\) is already coprime to \(p\).
    The natural quotient homomorphism is
    \(\func{\pi}{\Normalizer(P)}{\Normalizer(P)/P}\).
    We now inspect the image \(\pi(K)\),
    which by the \hyperref[thm:iso-1-group]{first isomorphism theorem}
    is a subgroup of \(\Normalizer(P)/P\).

    Now, as elements of \(k \in K\) have \(k^{p^t} = 1\)
    by Corollary~\ref{cor:order-element-group},
    under the homomorphism we should have \({\pi(k)}^{p^t} = 1\),
    so the order \(\abs{\pi(k)} \mid p^t\).
    But the order also \(\abs{\pi(k)} \mid m\),
    because it should divide the order of the codomain group;
    since \(m,p\) coprime, \(m,p^t\) also coprime,
    so the order must be \(\abs{\pi(k)} = 1\),
    and therefore the image must be \(\pi(K) = \{1\}\).
    Hence we can conclude that \(K \subseteq \ker(\pi) = P\).
\end{proof}
\begin{theorem}[Sylow's Second Theorem]\label{thm:sylow-2}
    Suppose \(G\) is a finite group, \(p\) a prime number,
    and that the order is \(\abs{G} = Np^s\), with \(\gcd(N,p) = 1\).
    Let us denote \(P\) and \(P'\) as arbitrary Sylow \(p\)-subgroups.
    Then the following will hold:
    \begin{enumerate}[label={(\alph*)}, itemsep=0mm]
        \item Any two Sylow \(p\)-subgroups are conjugate,
            that is, if \(\abs{P} = \abs{P'} = p^s\),
            then there exists a \(g \in G\) such that \(gPg^{-1} = P'\);
        \item Let the number of Sylow \(p\)-subgroups be \(n\),
            then \(n \mid [G:P] = N\) and \(n \equiv 1 \pmod{p}\); and
        \item If \(K \subseteq G\) is a subgroup
            with order \(\abs{K} = p^t\), \(0 \leq t \leq s\),
            then \(K \subseteq P\) a subgroup of a Sylow \(p\)-subgroup.
    \end{enumerate}
\end{theorem}
\begin{proof}
    Let \(\Pi\) be the set of all Sylow \(p\)-subgroups,
    that is, the set of all subgroups of order \(p^s\).
    By \hyperref[thm:sylow-1]{Sylow's first theorem},
    this set is not empty \(\Pi \neq \emptyset\).
    Allow us to let \(G\) act on \(\Pi\) by conjugation.

    % To prove the first statement,
    Let us fix a singular Sylow subgroup \(P \in \Pi\).
    We can clearly see that the stabilizer here is also the normalizer,
    since \(\Normalizer(P) = \stab(P) = \{g \in G : gPg^{-1} = P\}\).
    Then the \hyperref[cor:orbit-stabilizer]{orbit-stabilizer theorem}
    gives us that \(\abs{\orb_G(P)} = \abs{G}/\abs{\Normalizer(P)}\).
    By Proposition~\ref{prop:normalizer-subgroup}
    and Lemma~\ref{lem:prime-power-sylow-subgroup}
    we know that \(P \subseteq \Normalizer(P) \subseteq G\),
    so \hyperref[thm:lagrange]{Lagrange's theorem} tells us
    \(p^s \mid \abs{\Normalizer(P)}\),
    and \(\abs{\Normalizer(P)} \mid Np^s\),
    and hence we can write \(\abs{\Normalizer(P)} = mp^s\)
    where \(m \mid N\).
    If we now were to assume that statement (a) is true,
    then \(G\)-conjugation on \(\Pi\) is transitive,
    so \(\orb_G(P) = X\), and we have \(\abs{X} = (Np^s)/(mp^s) = N/m\),
    and therefore the number of Sylow \(p\)-subgroups
    is \(\abs{X} \mid N\),
    proving the first part of statement (b).

    The orbit of \(P\) under \(G\)-conjugation is a set of groups,
    which we will denote \(\Sigma = \orb_G(P) \subseteq \Pi\) below.
    Restricting our action such that \(P\) acts on \(\Sigma\) by conjugation,
    by closure of group multiplication,
    we can see that any \(g \in P\) gives us \(gPg^{-1} = P\),
    so the orbit of \(P\) under \(P\)-conjugation is
    \(\orb_P(P) = \{P\} \subseteq \Sigma\),
    which must only contain itself.
    The \hyperref[thm:class-equation]{class equation} now tells us that
    \(\abs{\Sigma} = \abs{\orb_P(P)} + \sum_i \abs{\orb_P(X_i)}\),
    where \(X_i \subseteq \Sigma\),
    and \(\abs{\orb_P(P)} = 1\) as seen above.

    From our assumptions, we know that the terms in the summation
    must divide the order \(\abs{P}\), so they must be powers of \(p\),
    that is, \(\orb_P(X_i) = p^r\), \(r \geq 0\).
    We claim that the order of those are at least \(p\),
    that is, \(\orb_P(X_i) = p^r\) where \(r \geq 1\).
    Suppose, by way of contradiction,
    that there exists some other \(P' \in \Sigma\), \(P' \neq P\)
    such that \(\orb_P(P') = \{P'\}\).
    Then going back to the definition of an orbit,
    we have for all \(g \in P\), \(gP'g^{-1} = P'\),
    which implies \(g \in \Normalizer(P')\),
    i.e.\ \(P \subseteq \Normalizer(P')\).
    As \(\abs{P} = p^s\),
    we can invoke the \hyperref[lem:prime-power-sylow-subgroup]{above lemma}
    and conclude that \(P \subseteq P' \subseteq \Normalizer(P')\).
    In particular, since \(\abs{P} = \abs{P'}\),
    we can see that \(P = P'\),
    which contradicts our premise,
    implying that the orbits \(\abs{\orb_P(X_i)} > 1\).
    This allows us to say that all terms in the summation
    must be divisible by \(p\),
    allowing us to write the class equation as
    \(\abs{\Sigma} = 1 + \sum_i r_i p \equiv 1 \pmod{p}\);
    again, if we assume statement (a) is true,
    then \(\abs{\Pi} = \abs{\Sigma}\),
    which gives us \(\abs{\Pi} \equiv 1 \pmod{p}\),
    proving the second part of statement (b).

    Going back to statement (a),
    it is sufficient to prove that
    there exists only one orbit under conjugation.
    Suppose, by way of contradiction,
    that there is some Sylow \(p\)-subgroup \(Q \in (\Pi-\Sigma)\).
    Then we will restrict the action by conjugation
    such that \(Q\) acts on \(\Sigma\).
    Following a similar argument as \(P'\) in the previous paragraph,
    we know each individual \(Q\)-orbit has order \(p^r\), \(r \geq 0\).
    We again claim that each orbit is at least order \(p\).
    If not, there exists some \(Q' \in \Sigma\)
    such that \(\orb_{Q}(Q') = \{Q'\}\),
    which implies for all \(g \in Q\), \(gQ'g^{-1} = Q'\),
    giving us \(g \in \Normalizer(Q')\),
    and hence \(Q \subseteq \Normalizer(Q')\).
    Invoking the \hyperref[lem:prime-power-sylow-subgroup]{above lemma},
    we get \(Q \subseteq Q' \subseteq \Normalizer(Q')\),
    which since \(\abs{Q} = \abs{Q'}\), we have \(Q = Q'\),
    contradicting the fact that \(Q' \in \Sigma\),
    but \(Q \notin \Sigma\).
    Hence we are forced to conclude that all \(Q\)-orbits
    have cardinalities that are multiples of \(p\),
    implying \(p \mid \abs{\Sigma}\).
    But that itself contradicts the class equation
    \(\abs{\Sigma} = 1 + \sum_i r_i p \equiv 1 \pmod{p}\),
    which tells us \(p \nmid \abs{\Sigma}\).
    Therefore we are forced to conclude that \(Q\) does not exist,
    and \(\Pi-\Sigma = \emptyset\).
    There is only one \(G\)-orbit,
    so \(G\) acts by conjugation
    on the set of Sylow \(p\)-subgroups transitively,
    which implies we can send any two \(\{P,P'\} \subset \Pi\)
    to each other via conjugation, \(gPg^{-1} = P'\) for some \(g \in G\).
    This proves statement (a).

    Armed with statements (a) and (b),
    we now let \(K\) act on \(\Pi\) via conjugation.
    The \hyperref[thm:class-equation]{class equation} tells us that
    \(\abs{\Pi} = \sum_i \abs{X_i}\),
    where the orbits \(X_i\)
    must have orders \(\abs{X_i} \mid \abs{K} = p^t\),
    so all orbits are powers of \(p\).
    But statement (b) tells us that \(\abs{\Pi} \equiv 1 \pmod{p}\),
    so there must be at least one \(K\)-orbit that is of size 1.
    Suppose \(P \in \Pi\) has \(\abs{\orb_K(P)} = 1\).
    Then we see that for all \(k \in K\), \(kPk^{-1} = P\),
    so \(k \in \Normalizer(P)\), and \(K \subseteq \Normalizer(P)\).
    We can use the \hyperref[lem:prime-power-sylow-subgroup]{lemma above}
    one last time
    to see that \(K \subseteq P \subseteq \Normalizer(P)\).
    This proves statement (c).
\end{proof}
