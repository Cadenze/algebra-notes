\section{Prerequisite Material}

\begin{remark}
    There is a huge amount of prerequisite material that is needed,
    and it is never possible to study algebra independently of other subjects.
    We need skills like doing proofs
    by induction, contradiction, and contraposition,
    and we also need a basic understanding of numbers in general,
    and perhaps a good deal of geometric intuition would be nice;
    occasionally some proofs might also involve a non-trivial amount of analysis.
    But we need a starting point,
    and for me that starting point
    is the material that perhaps was never formally covered in Science One math
    and second year linear algebra and calculus,
    or maybe was a topic that was last taught in high school.
\end{remark}


\subsection{Relations}

\begin{definition}
    A relation is a denotion on two elements of a set \(X\).
    More specifically,
    if a relation \(R \subset X \cross X\),
    then we write \(aRb\) if \((a,b) \in R\).
\end{definition}

\begin{definition}
    An equivalence relation \(\sim\)
    is a relation that has three properties:
    \begin{enumerate}[label={(\roman*)}, itemsep=0mm]
        \item reflexive \(\forall x \in X, x \sim x\);
        \item symmetric \(\forall x,y \in X, x \sim y \implies y \sim x\); and
        \item transitive \(\forall x,y,z \in X,
            x \sim y \land y \sim z \implies x \sim z\).
    \end{enumerate}
    We say \(x\) is related to \(y\) if \(x \sim y\).
\end{definition}

\begin{definition}
    Suppose we have an equivalence relation \(\sim\).
    The equivalence class of \(x \in X\) is
    \([x] = \{y \in X : y \sim x\}\).
\end{definition}
\begin{lemma}
    An element is in its own equivalence class; \(x \in [x]\).
\end{lemma}
\begin{proof}
    By reflexivity.
\end{proof}
\begin{theorem}
    \(x \sim y \iff [x] = [y]\);
    two related elements must have the same equivalence class.
\end{theorem}
\begin{proof}
    If \(x \sim y\),
    then \([x]\) consists of all elements \(z\) that are related to \(x\),
    so \(z \sim x\), and transitivity gives us \(z \sim y\),
    so \(z \in [y]\), and hence \([x] \subset [y]\).
    Reversing the argument gives us \([y] \subset [x]\),
    so \([x] = [y]\).

    If \([x] = [y]\), then \(y \in [x]\),
    so by definition \(y \sim x\).
\end{proof}
\begin{corollary}
    For all elements \(\{x,y\} \in X\),
    either \([x] = [y]\) or \([x] \cap [y] = \emptyset\).
\end{corollary}
\begin{proof}
    Suppose \([x] \cap [y] \neq \emptyset\).
    Then there exists some element \(z \in [x]\) and \(z \in [y]\).
    But then \(z \sim x\) and \(z \sim y\),
    so \(x \sim y\) by transitivity,
    which by our theorem gives \([x] = [y]\).
\end{proof}

\begin{theorem}\label{thm:equiv-class-partition}
    The set of equivalence classes \(X/{\sim}\) partition \(X\).
\end{theorem}
\begin{proof}
    From the lemma,
    each element must belong in an equivalence class.
    The above corollary tells us equivalence classes do not overlap.
\end{proof}


\subsection{Sets}

\begin{axiom}[Axiom of Choice]\label{ax:choice}
    Suppose we have a collection of nonempty sets \({\{S_i\}}_{i \in I}\)
    and \(I\) an index set;
    it is possible to form a set \({\{x_i\}}_{i \in I}\)
    such that \(x_i \in S_i\).
    In other words,
    we can choose an element from every set
    no matter how many sets we have.
\end{axiom}

\begin{definition}
    A partial order relation \(\leq\) is a relation that has three properties:
    \begin{enumerate}[label={(\roman*)}, itemsep=0mm]
        \item reflexive \(x \leq x\);
        \item antisymmetric \(x \leq y \land y \leq x \implies x = y\); and
        \item transitive \(x \leq y \land y \leq z \implies x \leq z\).
    \end{enumerate}
    Note that there possibly exists elements
    that are neither \(x \leq y\) or \(x \geq y\).
\end{definition}
\begin{definition}
    A partially ordered set, or a poset,
    is a tuple \((S,\leq)\),
    such that \(S\) is a set,
    and \(\leq\) is a partial order relation.
    A totally ordered set is a poset \((S,\leq)\)
    with the additional condition that every pair of element \(x,y\) is comparable,
    so it can be \(x \leq y\) or \(x \geq y\) (or both).
\end{definition}
\begin{definition}
    A chain is a subset of a poset \((S,\leq)\)
    such that under the same order, the subset is totally ordered.
\end{definition}
\begin{axiom}[Zorn's Lemma]\label{ax:zorn}
    A partially ordered set containing an upper bound for every chain
    must contain at least one maximal element.
\end{axiom}

\begin{axiom}[Well-Ordering Principle]\label{ax:well-order}
    Suppose \(S \neq \emptyset\) is a set.
    There exists a well ordering on \(S\),
    i.e.\ there exists a total order \(\leq\)
    such that every nonempty subset \(A \subseteq S\)
    has a smallest element \(a \in A\) where \(a \leq b\) for all \(b \in A\).
\end{axiom}

\begin{theorem}
    The above three axioms
    \begin{enumerate}[label={(\alph*)}, itemsep=0mm]
        \item the \hyperref[ax:choice]{axiom of choice};
        \item \hyperref[ax:zorn]{Zorn's lemma}; and
        \item the \hyperref[ax:well-order]{well-ordering principle}
    \end{enumerate}
    are equivalent in Zermelo-Fraenkel set theory,
    and are independent from any other ZF axiom.
\end{theorem}
\begin{proof}
    This is a result in set theory, and hence we shall omit the proof,
    and use the result as is.
\end{proof}


\subsection{Functions}

\begin{definition}
    A function \(f\) that maps elements in \(A\) to \(B\)
    is denoted \(\func{f}{A}{B}\).
    \(A\) is the domain, and \(B\) is the codomain.
    The range \(f(A)\) is the set of all possible outputs of the function,
    namely \(f(A) = \{b \in B : \exists a \in A, f(a) = b\}\).
\end{definition}

\begin{definition}
    Suppose \(A' \subset A\) and \(B' \subset B\).
    The image of \(A'\) is the set of all possible outputs
    given elements in \(A'\),
    denoted \(f(A') = \{f(a) \in B : a \in A'\}\).
    The preimage of \(B'\) is the set of all possible inputs
    that result in an element in \(B'\),
    denoted \(f^{-1}(B') = \{x \in A : f(x) \in B'\}\).
\end{definition}

\begin{definition}
    A function is 1-to-1 or injective
    when each point in the range (or codomain)
    only corresponds to (at most) one point in the domain.
    That is, \(\forall x,y \in A, f(x) = f(y) \implies x = y\).
\end{definition}
\begin{definition}
    A function is onto or surjective
    when every point in the codomain
    corresponds to at least one point in the domain,
    i.e.\ the codomain is the range.
    That is, \(\forall y \in B, \exists x \in A, f(x) = y\).
\end{definition}
\begin{definition}
    A function is bijective if it is both injective and surjective.
\end{definition}

% \newpage

\begin{theorem}[Pigeonhole Principle]
    Suppose two finite sets \(A,B\) have the same cardinality
    \(\abs{A} = \abs{B}\).
    Then \(\func{f}{A}{B}\) is injective
    if and only if it is surjective.
\end{theorem}
\begin{proof}
    Suppose \(f\) is injective but not surjective.
    Then if \(\abs{A} = n\),
    then there are \(n\) elements in \(B\) that are in the range.
    Lack of surjectivity implies that
    there exists at least one element that is not in the range.
    Hence \(\abs{B} > n\), which is a contradiction.

    Suppose \(f\) is surjective but not injective.
    Then if \(\abs{B} = n\),
    then there are at least \(n\) elements in \(A\).
    But no injectivity implies there must be some two elements
    \(x,y \in A\) that map to the same point \(f(x) = f(y)\),
    which so forces us to conclude \(\abs{A} \geq n+1\),
    which is a contradiction.
\end{proof}

\begin{theorem}\label{thm:composite-injection}
    Suppose \(f = f_n \circ f_{n-1} \circ \cdots \circ f_2 \circ f_1\).
    If \(f\) is injective, then \(f_1\) is injective.
\end{theorem}
\begin{proof}
    % Without loss of generality, let \(f = f_2 \circ f_1\).
    Suppose, by way of contradiction, that \(f_1\) is not injective.
    Then there exists \(x,y\) in the domain of \(f_1\)
    such that \(f_1(x) = f_1(y)\).
    But that implies \(f(x) = f(y)\),
    which contradicts that \(f\) is injective.
\end{proof}
\begin{theorem}\label{thm:composite-surjective}
    Suppose \(f = f_n \circ f_{n-1} \circ \cdots \circ f_2 \circ f_1\).
    If \(f\) is surjective, then \(f_n\) is surjective.
\end{theorem}
\begin{proof}
    % Without loss of generality, let \(f = f_2 \circ f_1\).
    Suppose, by way of contradiction, that \(f_n\) is not surjective.
    Then there exists \(y\) in the codomain of \(f_n\)
    such that for all \(x\) in the domain of \(f_n\),
    \(f_n(x) \neq y\).
    But then there exists \(y\) in the codomain of \(f\)
    such that \(f(x) \neq y\),
    which contradicts that \(f\) is surjective.
\end{proof}


\subsection{Integers}

\begin{definition}
    Suppose we have integers \(a,b\).
    The greatest common denominator, or gcd
    is the largest integer \(n\) such that \(n \mid a\) and \(n \mid b\).
    This is often denoted \(\gcd(a,b)\) or simply \((a,b)\).
    The least common multiple, or lcm
    is the smallest integer \(m\) such that \(a \mid m\) and \(b \mid m\).
    This is often denoted \(\lcm(a,b)\) or simply \([a,b]\).
\end{definition}

\begin{theorem}[Euclid's Lemma]\label{lem:euclid}
    Suppose we have prime \(p\) and integers \(a,b\).
    If \(p \mid ab\), then \(p \mid a\) or \(p \mid b\).
\end{theorem}
\begin{proof}
    Suppose \(p \nmid a\).
    Since \(p \mid ab\),
    there exists an integer \(q\) such that \(pq = ab\).

    We first prove the base case,
    supposing that \(ab = 2\).
    Then the only prime that divides it is 2,
    and we know that \(a = 1\) and \(b = 2\),
    so clearly \(p \mid b\).

    Now proceeding by induction,
    suppose that all values smaller than \(ab\) are proven.
    If \(p < a\),
    then \(pq - pb = ab - pb\),
    which gives us \(p(q-b) = (a-p)b\),
    i.e.\ \(p \mid (a-p)b\).
    Notice that \(p \nmid a-p\),
    and that \((a-p)b < ab\),
    so Euclid's lemma holds by the induction hypothesis.
    If \(p > a\),
    then \(pb - npq = pb - nab\),
    which gives us \(p(b-nq) = (p-na)b\),
    i.e.\ \(p \mid (p-na)b\).
    Notice that \(p \nmid p-a\),
    and that there exists an \(n\) such that \((p-na)b < ab\),
    which completes our proof.
\end{proof}
\begin{remark}
    This is essentially how the Euclidean algorithm for finding gcd works.
    We know that by subtracting off the other number,
    the gcd does not change,
    which allows us to reduce the problem to a smaller case.
\end{remark}

\begin{theorem}[B\'{e}zout's Identity]\label{thm:bezout}
    Suppose \(\gcd(a,b) = d\).
    Then there exists \(m,n \in \bZ\) such that \(ma + nb = d\).
    Moreover, for any \(p,q \in \bZ\)
    we have \(d \mid pa + qb\).
\end{theorem}
\begin{proof}
    Without loss of generality, let \(a \leq b\).
    If \(a = b\), then clearly \(d = a = b\) which is in our desired form.
    If not, then since we know that we can find some \(b - na \leq a\),
    which reduces the case down to a smaller number
    while always keeping the numbers that we are taking the gcd of
    in the form \(ma + nb\).
    Eventually, \(d\) must equal one of these numbers.

    Then we write \(pa + qb = (p-m)a + (q-n)b + ma + nb
    = (p-m)a + (q-n)b + d\).
    Since \(d \mid a\) and \(d \mid b\),
    we have our desired result.
\end{proof}
\begin{remark}
    One might use B\'{e}zout's identity to prove Euclid's lemma,
    but both proofs essentially come from the Euclidean algorithm.
\end{remark}

\begin{theorem}[Fundamental Theorem of Arithmetic]
    Every integer greater than 1 factors uniquely into a product of primes.
    That is, we can write any integer \(n > 1\) as
    \begin{equation*}
        n = \prod_{i=1}^k p_i^{n_i}
    \end{equation*}
    where \(p_i\) are distinct primes.
\end{theorem}
\begin{proof}
    We first prove such a prime factorization exists.
    We can see that 2 is prime.
    Proceeding by induction, assume all integers between 2 and \(n\)
    have a prime decomposition.
    If \(n\) is prime, we are done.
    If \(n\) is not prime,
    then it must be represented by \(n = ab\),
    where \(a,b\) must both be smaller,
    and by the inductive hypothesis,
    have prime decompositions.

    Now we prove that prime factorization is unique.
    Assume the contrary,
    and let \(n\) be the smallest such integer without unique factorization.
    Then we write \(n = \prod_{i=1}^k p_i = \prod_{i=1}^{k'} q_i\),
    two distinct factorizations.
    By Euclid's lemma we see that \(p_1\) divides some \(q_i\),
    which without loss of generality say this is \(q_1\).
    Then \(p_1 = q_1\).
    So \(n/p_1 = \prod_{i=2}^k p_i = \prod_{i=2}^{k'} q_i\)
    is also an integer without unique factorization,
    which contradicts our assumption that \(n\) is the smallest.
\end{proof}
